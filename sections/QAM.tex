
\section*{Modulation QAM}

\begin{align*}
    x(t)   & = x_c(t) + j \cdot x_s(t)                \\
    x_c(t) & = \Re\{p(t)\} \quad x_s(t) = \Im\{p(t)\}
\end{align*}

% Qui viene rappresentato il diagramma di flusso della modulazione QAM
\begin{tikzpicture}
    % Definire i nodi e i percorsi qui
\end{tikzpicture}

\[
    x(t) = x_c(t) \cos(2\pi f_ct) - x_s(t) \sin(2\pi f_ct)
\]

% Diagramma dei punti della costellazione QAM
\begin{tikzpicture}
    \draw [<->] (0,2) node (yaxis) [above] {$\Im$}
    |- (2,0) node (xaxis) [right] {$\Re$};
    \draw (-1,0) -- (1,0);
    \draw (0,-1) -- (0,1);
    \foreach \x/\y in {-1/1, 1/1, -1/-1, 1/-1}
    \draw[fill=orange] (\x,\y) circle (2pt);
\end{tikzpicture}

\[
    \Gamma = \Gamma_c \cap \Gamma_s
\]

\begin{align*}
    A^\Gamma & = \{\alpha^\Gamma_c, \overline{\alpha^\Gamma_c}\} \\
    A^\Gamma & = \{\alpha^\Gamma_s, \overline{\alpha^\Gamma_s}\}
\end{align*}

\[
    A^c \rightarrow \alpha^c = 2c - 1 - \Gamma_c \quad A^s \rightarrow \alpha^s = 2c - 1 - \Gamma_s
\]
