
\section*{Sistemi di Comunicazione Numerici}
Per definizione un sistema di comunicazione numerico è tale quando \( m_s(t) \) e \( \hat{m}_s(t)\) sono sequenze numeriche (\( m_s[n] \) e \( m_r[n] \)) e \( m(t) \) e \( \hat{m}(t) \) sono segnali numerici.

N.B. \( s(t) \) e \( r(t) \) continuano ad essere segnali analogici.

\begin{center}
    \begin{tikzpicture}[auto, node distance=2cm,>=latex']
        \node [input, name=input] {};
        \node [block, right of=input] (modulator) {\( p(t) \)};
        \node [output, right of=modulator] (output) {};

        \draw [->] (input) -- node {\( m_s[n] \)} (modulator);
        \draw [->] (modulator) -- node {\( m(t) \)} (output);
    \end{tikzpicture}
\end{center}
Dove
\( T_s \) è il periodo di segnalazione della sorgente e
\( m_s[n] \) è la sequenza generata dalla sorgente con periodo di sequenza \( T_s \),
Ogni simbolo di \( m_s[n] \) appartiene ad un alfabeto predefinito
\[ m_s[n] \in A_s ,  A_s = \alpha_1, \alpha_2, \ldots, \alpha_M\} ,  M \geq 2  \]La sequenza \( m_s[n] \) può essere vista come il risultato (realizzazione) del campionamento di un processo aleatorio.



\( M(t) \) è un processo aleatorio e \( M_s[n] \) è una V.A. estratta dal processo. \\
Si definisce la sequenza aleatoria di sorgente come
\[ \{ M_s[n] \in A_s, n = 0, \pm 1, \pm 2, \ldots \} \]
N.B. \( m_s[n] \) è una realizzazione di \( M[n] \).

Alla destinazione si definisce in maniera analoga la sequenza di destinazione
\[ \{ \hat{M}_s[n] \in A_s, n = 0, \pm 1, \pm 2, \ldots \} \]

Il segnale \( m(t) \) è ottenuto dalla sequenza aleatoria di sorgente tramite una operazione di modulazione che è del tutto equivalente alla operazione di interpolazione
\[ m(t) = \sum_{n=-\infty}^{+\infty} m_s[n]\cdot p(t - nT_s) \]
dove \( p(t) \) è l'impulso in trasmissione.

Ad esempio \( p(t) = \text{rect}\left(\frac{t}{T_s}\right) \).

Il tasso di erogazione della sorgente è definito come
\[ R_s \coloneqq \frac{1}{T_s} \]
che è il rate con cui escono i simboli apparentemente dall'alfabeto \( A_s \).

Se è presente una codifica binaria per rappresentare un simbolo dell'alfabeto \( A_s \) occorrono \( \log_2 M \) simboli binari,
per cui si definisce il tasso di errore binario come:
\[
    R_b \coloneqq \frac{\log_2 M}{T_s}
\]

\subsection*{Schema funzionale di un sistema di comunicazione numerico}
\begin{adjustwidth*}{-2.3cm}{-2cm}
    \begin{center}
        \begin{tikzpicture}[>=Stealth, block/.style={draw, rectangle}, scale=0.85]
            % Blocks
            \tikzstyle{block} = [rectangle, draw, text centered, minimum height=4em, align=center]

            \node[block] (mod) {Modulatore \\ numerico};
            \node[block, right=1cm of mod] (channel) {Canale di \\ comunicazione};
            \node[block, right=1cm of channel] (demod) {Demodulatore \\ numerico};

            % Nodes for connecting lines
            \coordinate[left=2cm of mod] (input);
            \coordinate[right=2cm of demod] (output);

            % Lines
            \draw[->] (input) -- node[above] {$m_s[n]$} (mod);
            \draw[->] (mod) -- node[above] {$s(t)$} (channel);
            \draw[->] (channel) -- node[above] {$r(t)$} (demod);
            \draw[->] (demod) -- node[above] {$\hat{m}_s[n]$} (output);

            % Dashed box
            \begin{scope}
                \draw[dashed, red] ($(mod.north west)+(-0.5,0.5)$) rectangle ($(demod.south east)+(0.5,-0.5)$);
            \end{scope}

            % Annotations
            \node[align=center, red, above right= -1cm and -6cm of demod.south east] (channel-label) {Canale numerico};

            % Circles
            \draw (input) ++(-0.5cm,0) circle (0.5cm) node {$S$};
            \draw (output) ++(0.5cm,0) circle (0.5cm) node {$D$};
        \end{tikzpicture}
    \end{center}
\end{adjustwidth*}
\begin{itemize}
    \item La coppia modulatore/demodulatore numerico può essere interpretata come un livello di trasduzione. Questo ci permette di vedere il sistema di comunicazione come un canale numerico tra \( S \) e \( D \).
    \item La sorgente genera una sequenza di simboli appartenenti all'alfabeto \( A_s \), che vogliamo trasferire al nodo di destinazione.
    \item Il modulatore numerico può essere visto come l'insieme del trasduttore di sorgente e del modulatore. Questo genera quindi il segnale \( s(t) \) analogico che può essere di tipo passa-basso (in banda base) o di tipo passa-banda (in banda passante).
    \item Il canale di comunicazione è sempre lo stesso (per segnali numerici e analogici).



    \item Il demodulatore numerico può essere visto come l'insieme del demodulatore e del trasformatore di destinazione. Esso produce la sequenza $\hat{m}_s[n]$ dal segnale $v(t)$.

\end{itemize}

Un canale numerico ideale produce:
\begin{equation}
    \hat{m}_s[n] = m_s[n] \quad \forall n
\end{equation}

In casi pratici un canale numerico non è mai ideale e per cui ha senso definire il suo comportamento e quindi le sue prestazioni.

\subsection*{Probabilità di transizione}
\begin{equation}
    P\{i|j\} \coloneqq P\{\hat{m}_s[n]=\alpha_i
    \ |\  m_s[n]=\alpha_j\}
\end{equation}

Un canale numerico è statisticamente caratterizzato quando sono note le $P\{i|j\}$ $\forall i,j$

Se le $P\{i|j\}$ non dipendono da $n$ allora il canale numerico si dice \textbf{stazionario}.

L'insieme delle $P\{i|j\}$ è parte di $M^2$ dove $M$ è la cardinalità dell'alfabeto $A_s$.

Per un canale ideale quindi:
\begin{align}
    P\{i|j\} & = 1 \quad \text{se } i=j      \\
    P\{i|j\} & = 0 \quad \text{se } i \neq j
\end{align}

N.B. Le $P\{i|j\}$ non dipendono solo dai disturbi introdotti dal canale di comunicazione, ma anche dalla modulazione impiegata, per cui esse legano contro delle prestazioni di tutto il sistema numerico.

% Qui potresti includere il disegno utilizzando l'ambiente tikzpicture se necessario
\subsection*{Misure delle prestazioni di un sistema di comunicazione numerico}

Le prestazioni di un sistema di comunicazione numerico sono associabili alla probabilità di errore di simbolo \(M\)-ario

\[
    P_E(M) \coloneqq P\{\hat{m}_s[n] \neq m_s[n]\}
\]

Se la \(P_E(M)\) non dipende da \(n\), allora il sistema di comunicazione è stazionario.

\subsection*{Quality of Service (QoS)}

La qualità del servizio per un sistema di comunicazione numerico è associabile alla probabilità di errore \(P_E(M)\). È ragionevole quindi pensare che si debba fissare una \(P_E(M)\) massima al di là della quale la QoS non è accettabile.

\[
    P_E(M) \leq P_{\text{max}}
\]

Esempio: per i servizi voce \(P_{\text{max}} = 10^{-3}\), mentre per i servizi dati \(P_{\text{max}} = 10^{-7}\).

\subsection*{Dualismo banda e potenza}

Aumentando la potenza del segnale in trasmissione si può fare in modo che la componente utile del segnale prevalga sulla componente di rumore. Questo intuitivamente tende a migliorare le prestazioni del sistema. La potenza però costa e comunque esistono dei limiti fisici o imposti che la limitano.

Lo stesso si può dire per la banda, si può dimostrare che all'aumentare della banda del segnale trasmesso si migliorano le prestazioni del sistema. Anche la banda però è una risorsa limitata.


\subsection*{Efficienza di potenza e efficienza spettrale}

Una modulazione si dice:
\begin{itemize}
    \item \textbf{efficiente in potenza}: quando la potenza trasmessa è bassa a fronte di un certo livello di prestazioni;
    \item \textbf{efficiente spettralmente}: quando la banda utilizzata è piccola a fronte di un certo livello di prestazioni.
\end{itemize}

Sfortunatamente i sistemi efficienti spettralmente non sono anche efficienti in potenza.

Vantaggi di un sistema di comunicazione rispetto ad un altro:
\begin{enumerate}
    \item Basso costo
    \item Sicurezza nella trasmissione di un messaggio
    \item Trasferimento aggregato di messaggi di natura diversa (multiplazione, audio, video, dati)
    \item Possibilità di utilizzare modulazioni e codifiche che rendono il sistema efficiente in potenza e/o spettralmente.
\end{enumerate}

Svantaggi:
\begin{enumerate}
    \item Generalmente la banda occupata da un segnale numerico è maggiore del corrispondente analogico
    \item Complessità, soprattutto per la sincronizzazione
\end{enumerate}

