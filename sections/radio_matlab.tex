\subsection*{Example: Creating a Spectrum Analyzer in MATLAB}

% Example of creating a spectrum analyzer object
A spectrum analyzer object in MATLAB can be created using the \texttt{dsp.SpectrumAnalyzer} class:
\begin{verbatim}
    obj = dsp.SpectrumAnalyzer;
\end{verbatim}
This command initializes a new object of the \texttt{dsp.SpectrumAnalyzer} class.

% Setting properties of an object
Properties of the object can be defined during creation or by direct assignment:
\begin{verbatim}
    obj = dsp.SpectrumAnalyzer('Name', 'Spectrum Analyzer');
    % or
    obj.Name = 'Spectrum Analyzer';
\end{verbatim}
In this example, the \texttt{'Name'} property of the spectrum analyzer object is set to \texttt{'Spectrum Analyzer'}.


\subsection*{Example: Creating a Spectrum Analyzer in MATLAB (Continued)}

% Continued example of creating a spectrum analyzer with property settings
To create a spectrum analyzer with specific properties, use the following syntax:
\begin{verbatim}
    scope = dsp.SpectrumAnalyzer(...
        'Name', 'Spectrum Analyzer', ...
        'Title', 'Spectrum', ...
        'SpectrumType', 'Power', ...
        'FrequencySpan', 'Span and center frequency', ...
        'CenterFrequency', 0, ...
        'Span', 600, ...
        'ShowLegend', true, ...
        'SampleRate', Fs);
\end{verbatim}
This code snippet sets various properties of the spectrum analyzer object, such as the display name, title, type of spectrum, frequency span, and sampling rate.

% Using the step method to display the spectrum
The \texttt{step} function is used to display the frequency spectrum of the input signal:
\begin{verbatim}
    step(obj, X)
\end{verbatim}
Here, \( X \) represents the input signal to the spectrum analyzer, and it can be a matrix where each column is treated as an independent channel.

\section*{Before Starting with RTL-SDR in MATLAB}

% Preparatory steps for using RTL-SDR with MATLAB
Before working with RTL-SDR in MATLAB, the following preparatory steps should be taken:

\begin{enumerate}
    \item Install the support package for RTL-SDR Radio by navigating to the MATLAB Home tab, then clicking on \texttt{Add-Ons} > \texttt{Get Hardware Support Packages}.
    \item In the Add-On Explorer, search for the \texttt{Communications Toolbox Support Package for RTL-SDR Radio}.
    \item Select the support package and click \texttt{Install}. Follow the prompts to install the necessary drivers for the RTL-SDR radio software.
\end{enumerate}
This setup ensures that MATLAB is properly configured to interface with RTL-SDR hardware and perform signal analysis tasks.

\section*{How to Install Drivers for RTL-SDR in MATLAB}

% Steps to install the RTL-SDR drivers after the MATLAB support package installation
After installing the Communications Toolbox Support Package for RTL-SDR Radio:

\begin{enumerate}
    \item Use File Explorer on your PC and navigate to the following path:
    \begin{verbatim}
    C:\ProgramData\MATLAB\SupportPackages\<version>\3P.instrset\zadig.instrset\zadig-XX.exe
    \end{verbatim}
    Here, \( <version> \) is the version of your MATLAB, and \( XX \) is the Zadig version number.
    
    \item Connect the RTL-SDR device to your PC.
    
    \item Run the Zadig application, select the RTL device from the list, and install the driver.
\end{enumerate}

\section*{Hardware Setup}

% Instructions for setting up the RTL-SDR hardware
To set up the RTL-SDR hardware:

\begin{enumerate}
    \item Connect the RTL-SDR into your computer's USB port.
    
    \item Start MATLAB and at the MATLAB command prompt, initiate the hardware setup by executing:
    \begin{verbatim}
    sdrsetup
    \end{verbatim}
    
    \item To obtain information about all radios connected to your computer, use the \texttt{sdrinfo} command:
    \begin{verbatim}
    hwinfo = sdrinfo
    \end{verbatim}
    This command provides details about the connected RTL-SDR radios, such as radio name, address, tuner name, manufacturer, and gain values.
\end{enumerate}

These steps are crucial for ensuring that the RTL-SDR hardware is properly recognized by MATLAB and is ready for signal processing tasks.

\section*{Load RTL-SDR Driver}

% Constructing the RTL-SDR System object
To interface with the RTL-SDR hardware, create a System object in MATLAB:
\begin{verbatim}
    obj_rtlsdr = comm.SDRRTLReceiver
\end{verbatim}
which results in an object with properties such as center frequency, sample rate, and samples per frame:
\begin{verbatim}
    obj_rtlsdr = 
        comm.SDRRTLReceiver with properties:
        RadioAddress: '0'
        CenterFrequency: 102500000
        EnableTunerAGC: true
        SampleRate: 2500000
        OutputDataType: 'int16'
        SamplesPerFrame: 1024
        FrequencyCorrection: 0
        EnableBurstMode: false
\end{verbatim}
These properties configure the RTL-SDR receiver to properly receive and process the RF signals.

\section*{FM Receiver — Practical Implementation}

% Block diagram description of an FM receiver using RTL-SDR
The practical implementation of an FM receiver using RTL-SDR follows a specific signal processing flow:
\begin{itemize}
    \item The analog RF signal \( S_{FM}(t) \) is first received by the SDR hardware.
    \item It is then digitally demodulated into \( \tilde{v}(k) \) at a sample rate of 228 kS/s.
    \item The demodulated signal is filtered through a low-pass filter (LPF) and then subjected to rate conversion to reduce the sample rate to 48 kS/s.
    \item Finally, the processed digital signal is analyzed using a spectrum analyzer.
\end{itemize}
This process outlines the steps from receiving the analog signal to the digital analysis, detailing the journey of the signal through the FM receiver system.

\subsection*{Example MATLAB Code}
\begin{verbatim}
    % Define System object for RTL-SDR Receiver with specified properties
    obj_rtlsdr = comm.SDRRTLReceiver('RadioAddress', '0', ...
                                     'CenterFrequency', 102500000, ...
                                     'EnableTunerAGC', true, ...
                                     'SampleRate', 2500000, ...
                                     'OutputDataType', 'int16', ...
                                     'SamplesPerFrame', 1024, ...
                                     'FrequencyCorrection', 0, ...
                                     'EnableBurstMode', false);
\end{verbatim}
This MATLAB code snippet sets up the RTL-SDR receiver with predefined settings to start receiving signals.


\section*{FM Receiver}

% Description of FM signal reception and demodulation
The complex envelope of the received FM signal is given by:
\[
\tilde{v}(t) = A_c e^{j2\pi k_f \int_{-\infty}^{t} m(\tau) d\tau}
\]
Ignoring noise and channel effects, the modulating signal \( m(t) \) can be recovered by differentiating the phase of \( \tilde{v}(t) \):
\[
\hat{m}(t) = \frac{1}{2\pi k_f} \frac{d}{dt} \arg(\tilde{v}(t))
\]
where \( \arg(\tilde{v}(t)) \) denotes the phase of \( \tilde{v}(t) \).

% Conceptual diagram of an FM baseband receiver
The conceptual diagram of an FM baseband receiver includes the differentiator \( \frac{d}{dt} \) followed by a division by \( 2\pi k_f \) to retrieve the original modulating signal \( m(t) \).

\section*{FM Receiver — Practical Implementation}

% Practical implementation steps for FM signal demodulation
In the practical implementation of an FM receiver using SDR:

\begin{itemize}
    \item The SDR output provides the complex envelope of the signal sampled at frequency \( f_s = \frac{1}{T_s} \).
    \item The signal \( \tilde{v}(k) \) at discrete times \( kT_s \) can be expressed as:
    \[
    \tilde{v}(k) = \tilde{v}(t)|_{t=kT_s} \approx A_c e^{j2\pi k_f \sum_{\ell=-\infty}^{k} m(\ell T_s) T_s}
    \]
    \item The product of two consecutive baseband samples is:
    \[
    \tilde{v}(k) \tilde{v}^*(k - 1) \approx A_c^2 e^{j2\pi k_f m(k) T_s}
    \]
    \item An estimate of \( m(k) \), the \( k \)-th sample of \( m(t) \), is then given by:
    \[
    \hat{m}(k) = \frac{1}{T_s 2\pi \Delta f} \arg(\tilde{v}(k) \tilde{v}^*(k - 1))
    \]
\end{itemize}

% Explanation of the signal processing terms and operations
These equations represent the discrete-time processing of an FM signal, demonstrating how the original modulating signal is extracted from the received waveform in a digital domain.

\section*{FM Mono}

% Information on the first FM mono transmissions
The initial FM transmissions from 1945 to 1960 were mono, meaning they consisted of a single audio channel. The modulating signal \( m(t) \) for mono transmissions was the sum of the left (\( L(t) \)) and right (\( R(t) \)) audio channels:
\[ m(t) = L(t) + R(t) \]

% Normalization and frequency deviation for FM mono
This signal was then normalized to a maximum value of 1, with a frequency sensitivity \( k_f \) of 75 kHz/V. This resulted in a maximum frequency deviation \( \Delta f \) for the FM signal given by:
\[ \Delta f = k_f \max|m(t)| = 75 \text{ kHz} \]

% Power spectrum of FM mono transmissions
The power spectrum of these mono FM transmissions would be a single peak, representing the audio content, spanning up to 15 kHz in frequency:
Here, the figure illustrates the frequency range allocated for mono FM audio transmissions.

