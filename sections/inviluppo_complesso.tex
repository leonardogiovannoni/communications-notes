\section*{Inviluppo complesso}

\paragraph*{Trasformata di Hilbert}
Si definisce la trasformata di Hilbert di un segnale \( x(t) \) come:
\[
    \mathcal{H}\{x(t)\} = x_H(t) \coloneqq \frac{1}{\pi} \int_{-\infty}^{\infty} \frac{x(\tau)}{t - \tau} \, d\tau
\]


questo intergrale può essere riscritto come: 
\[
    x_H(t) = x(t) \ast \frac{1}{\pi t} = x(t) \ast h_H(t)
\]

dove \( h_H(t) \) è la risposta all'impulso del filtro di Hilbert, definito come:
\[
    % inser also fourier transform of h_H(t)
    h_H(t) = \frac{1}{\pi t} \ \xrightarrow{\mathcal{F}} \ H_H(f) = -j\cdot \text{sgn}(f)
\]



definiamo anche il segnale analitico \( x_a(t) \) come:
\[
    x_a(t) = x(t) + j \cdot x_H(t)
\]
\[
    X_a(f) = X(f) \cdot 2\text{u}(f)   
\]
\[
    X(f) = \frac{1}{2} \left[ X_a(f) + X_a^*(-f) \right]    
\]

se la banda del segnale \( x(t) \) è prevalentemente concentrata attorno ad una certa frequenze $f_0$, come avviene per i segnali passa banda,
torna utile definire il segnale detto inviluppo complesso \( \tilde{x}(t) \) come:
\[
    \tilde{x}(t) = x_a(t) e^{-j2\pi f_0 t}
\]