\section*{Modulazioni numeriche in banda base}

\begin{tikzpicture}[
    block/.style={rectangle, draw, minimum height=1cm, minimum width=2.5cm},
    node distance=1cm and 2cm,
    auto
]
    % Blocks
    \node[block] (source) {Sorgente binaria};
    \node[block, right=of source] (encoder) {Codificatore};
    \node[block, right=of encoder] (interp) {Interpolatore};
    \node[block, right=of interp] (channel) {Canale di comunicazione};
    \node[block, below=of channel] (filter) {Filtro di ricezione};
    \node[block, left=of filter] (sampler) {Campionatore};
    \node[block, left=of sampler] (decisor) {Decisore};
    \node[block, left=of decisor] (destination) {Destinazione binaria};

    % Arrows
    \draw[->] (source) -- (encoder) node[midway,above] {$b[n]$}  node[midway,below] {$T_b$};
    \draw[->] (encoder) -- (interp) node[midway,above] {$x[n]$}  node[midway,below] {$T_s$};
    \draw[->] (interp) -- (channel) node[midway,above] {$s(t)$};
    \draw[->] (channel) -- (filter) node[midway,right] {$r(t)$};
    %\draw[->] (filter) -- (sampler) node[midway,above] {$y(t)$};
    \draw[->] (sampler) -- (decisor) node[midway,above] {$\hat{x}[n]$};
    \draw[->] (decisor) -- (destination) node[midway,above] {$\hat{b}[n]$};
    
        % Draw the diagonal line
    % Dashed Boxes
    \draw ([xshift=0]filter.west) -- ([xshift=-0.75cm]filter.west) node[midway,above] {$y(t)$};
    \draw ([xshift=-0.75cm]filter.west) -- ([xshift=-1.25cm,yshift=0.5cm]filter.west) node[midway,below] {$T_s$};

    \draw[->] ([xshift=-1.25cm,yshift=0cm]filter.west) -- ++(sampler) node[midway,above] {$y[n]$};

    \draw[dashed, red, thick] ([xshift=-0.5cm,yshift=0.5cm]sampler.north west) rectangle ([xshift=0.5cm,yshift=-0.5cm]filter.south east);

     \node[align=center, red, above right= -1cm and -6cm of filter.south east] (channel-label) {Demodulatore numerico};

  
   

\end{tikzpicture}

La modulazione numerica è necessaria per poter trasmettere sequenze binarie attraverso un mezzo trasmissivo. In particolare, per adesso ci concentreremo su canali trasmissivi in banda base, come ad esempio il doppino telefonico o il cavo coassiale.

\begin{tikzpicture}
% Drawing the block diagram using TikZ package
% Note: You will need to adjust the positions of the blocks (nodes) and the arrows (paths) to match your notes.
\end{tikzpicture}

% Here you should draw the block diagram as per your notes using the TikZ package.
% Since the TikZ diagram could be quite complex, I will only outline the structure here.
\begin{enumerate}
    
\item \textbf{Codificatore}: trasforma sequenze di bit in simboli \( M \)-ari appartenenti a un alfabeto \( A_s \).

\item \textbf{Interpolatore}: modula impulsi tramite i simboli \( x[n] \) in ingresso per produrre una sequenza di impulsi \( s(t) \).
\[
s(t) = \sum_{k=-\infty}^{\infty} x[k] \cdot p(t - kT_s)
\]

\item \textbf{Canale di comunicazione}: sono assenti i trasduttori in quanto la propagazione nel mezzo trasmissivo è elettrica.

\item \textbf{Filtro di ricezione}: filtra componenti di rumore generate nel canale e compensa eventuali distorsioni.

\item \textbf{Campionatore}: preleva campioni dal segnale filtrato \( y(t) \).

\item \textbf{Decisore}: associa un simbolo dell'alfabeto \( A_s \) ad ogni campione.

\item \textbf{Decodificatore}: trasforma i simboli in sequenze binarie.
 
\end{enumerate}
\paragraph*{Codificatore}

\begin{itemize}
\item Deve essere sincrono con la sorgente
\[
T_s = T_b \log_2 M
\]
\[
R_s = \frac{1}{T_s} = \frac{1}{T_b \log_2 M}
\]


\item La sequenza di simboli generati dal codificatore viene considerata come un processo stazionario. Spesso i simboli trasmessi sono considerati equiprobabili:
\begin{equation*}
    P\{x[n]=\alpha_i\} = \frac{1}{M} \quad \forall i
\end{equation*}
\end{itemize}

\paragraph*{Interpolatore}

In un sistema di comunicazione numerico in banda base, l'interpolatore da solo svolge il compito di modulatore numerico, in quanto effettua la sagomatura, mentre non è prevista nessuna traslazione in frequenza. Il filtro sagomatore è realizzato tramite la generazione dell'impulso \( p(t) \). Infatti, si può pensare a \( P(f) \) come allo spettro del singolo impulso.
\begin{itemize}
    \item \(B_p\): banda dell'impulso \( p(t) \):
    \item \(E_p\): energia dell'impulso \( p(t) \):
\begin{equation*}
    E_p = \int_{-\infty}^{\infty} p(t)^2 \, dt = \int_{-\infty}^{\infty} |P(f)|^2 \, df
\end{equation*}
\end{itemize}





Data l'aleatorietà della sequenza di simboli \( x[k] \), \( s(t) \) deve essere interpretato come la realizzazione di un processo aleatorio \( S(t) \) stazionario.

Il processo \( S(t) \) ha una autocorrelazione \( R_s(\tau) \) ed una densità spettrale di potenza \( S_s(f) \).

Per cui è definita una potenza \( P_s \) ed una banda \( B_T \).


     L'energia media per bit può essere calcolata come:
 \begin{equation*}
    E_b = T_s P_b = \frac{E_s}{\log_2 M} = \frac{E_s}{B_t}
\end{equation*}
\begin{equation*}
    E_s \text{— energia media per simbolo}
\end{equation*}


\paragraph*{Demodulatore numerico}


Il demodulatore numerico produce una sequenza $\hat{x}[k]$ in modo tale da minimizzare la probabilità di errore. La probabilità di errore sul simbolo e sul bit sono:

\begin{align*}
    P_{E_s} &= P\{ \hat{x}[k] \neq x[k] \} \\
    P_{E_b} &= P\{ \hat{b}[n] \neq b[n] \} \quad \text{BEP (bit error probability)}
\end{align*}

\paragraph*{Canale numerico e prestazioni}
\begin{center}
\begin{tikzpicture}
\node (s) at (0,0) {\(x[k]\)};
\node[block] (encoder) at (3,0) {Canale numerico};
\node (r) at (6,0) {\(\hat{x}[k]\)};

\draw[->] (s) -- (encoder);
\draw[->] (encoder) -- (r);
\end{tikzpicture}
\end{center}




\begin{align*}
    x[k] &\in A_s & A_s &= \{\alpha_1, \alpha_2, \ldots, \alpha_M\}
\end{align*}

\begin{equation*}
    P_E(M) \coloneqq P\{\hat{x}[k] \neq x[k]\} = \sum_{i=1}^{M} \sum_{\substack{j=1 \\ j \neq i}}^{M} P\{\hat{x} = \alpha_i, x = \alpha_j\} = \sum_{i=1}^{M} \sum_{\substack{j=1 \\ j \neq i}}^{M} P\{\hat{x} = \alpha_i \ |\ x = \alpha_j\}\cdot P\{x=\alpha_j\}
\end{equation*}

Nel caso di simboli equiprobabili:

\begin{equation*}
    P_E(M) = \frac{1}{M} \sum_{i=1}^{M} \sum_{\substack{j=1 \\ j \neq i}}^{M} P\{\hat{x} = \alpha_i \ | \ x = \alpha_j\}
\end{equation*}

\paragraph*{Formato di modulazione equienergia}

\begin{equation*}
    E_s(i) = \int_{-\infty}^{\infty} s_i(t)^2 dt \quad s_i(t) \text{ è il segnale trasmesso in corrispondenza del simbolo } \alpha_i
\end{equation*}

Il formato di modulazione è equienergetico se
\begin{equation*}
    E_{s}(i) = E_{s} \quad \forall i
\end{equation*}

\paragraph*{Ortogonalità}

Il formato di modulazione è detto ortogonale se
\begin{equation*}
    \int_{-\infty}^{\infty} s_{i}(t) s_{j}(t) \, dt = 0 \quad \forall i \neq j
\end{equation*}

\paragraph*{Efficienza energetica}

Fissata una \( P_{E_{b}} \), l'efficienza energetica è definita come il valore
\begin{equation*}
    \eta_{P} \coloneqq \frac{1}{SNR} , \quad SNR \coloneqq \frac{P_{s}}{P_{n}}
\end{equation*}
che permette di ottenere tale \( BEP \).

Quando tanto maggiore è \( \eta_{P} \), tanto minore deve essere il \( SNR \) che garantisce una data \( BEP \).

\paragraph*{Efficienza spettrale}

L'efficienza spettrale è definita con il rapporto tra il tasso di erogazione binario e la banda di trasmissione
\begin{equation*}
    \eta_{b} \coloneqq \frac{R_{b}}{B_{T}} \quad \text{[bit/s/Hz]}
\end{equation*}

Quindi l'efficienza cresce quando a parità tasso di erogazione la banda utilizzata in trasmissione si riduce.

In termini di \( T_{s} \) e \( M \):
\begin{equation*}
    \eta_{b} = \frac{\log_2 M}{B_{T} T_{s}}
\end{equation*}

