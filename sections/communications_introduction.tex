\section*{Introduction}


Le comunicazioni wireless rappresentano la quota più importante in termini di utenti oggigiorno. Lo spettro delle comunicazioni radio risulta molto affollato, soprattutto nella banda dell'ordine dei GHz in cui si trovano molteplici applicazioni.

\begin{table}[h!]
\centering
\begin{tabular}{ | m{2cm} | m{3cm} | m{2.5cm} | m{4cm} | m{2.5cm} | }
\hline
Banda di frequenza & Intervallo di frequenza & Lunghezza d'onda & Servizi & Propagazione \\
\hline
LF & 30 -- 300 kHz & \(10^{4} - 10^{3}\) m & Orologio radio, navigazione (LORAN), militare (marina) & Onda di terra \\
\hline
MF & 0.3 -- 3 MHz & \(10^{3} - 10^{2}\) m & Radio AM (522--1600 kHz), radiofari & Onda di terra, onda ionosferica \\
\hline
HF & 3 -- 30 MHz & \(10^{2} - 10\) m & Comunicazioni aeronautiche, radar oltre l'orizzonte, radioamatori & Onda ionosferica \\
\hline
VHF & 30 -- 300 MHz & \(10 - 1\) m & Comunicazioni avioniche, radio FM (88 -- 108 MHz), DVB-T (RAI \(177.5\) MHz) & Onda spaziale \\
\hline
UHF & 0.3 -- 3 GHz & \(1 - 10^{-1}\) m & DVB-T (470-860 MHz), Cellulare (\(900,1800,2200\) MHz), Wi-Fi, GPS & Onda spaziale \\
\hline
SHF & 3 -- 30 GHz & \(10^{-1} - 10^{-2}\) m & Wi-Fi, 5G, DVB-S, Radar, SatCom & Onda spaziale \\
\hline
EHF & 30 -- 300 GHz & \(10^{-2} - 10^{-3}\) m & Wi-Fi, 5G, DVB-S, Radar, SatCom & Onda spaziale \\
\hline
\end{tabular}
\end{table}

\begin{itemize}
    \item \textbf{LF (Low Frequency)}: Frequenze molto basse, utilizzate per orologi radio, sistemi di navigazione a lunga distanza come LORAN, e comunicazioni militari navali. La propagazione è principalmente per onda di terra.
    \item \textbf{MF (Medium Frequency)}: Queste frequenze includono la banda di trasmissione AM. Sono utilizzate anche per i radiofari. Le onde possono viaggiare come onde di terra o riflettersi nell'ionosfera (onde ionosferiche).
    \item \textbf{HF (High Frequency)}: Utilizzate per comunicazioni aeronautiche, radar a lungo raggio e dai radioamatori. Queste onde si propagano principalmente attraverso l'ionosfera.
    \item \textbf{VHF (Very High Frequency)}: Coprono servizi come le comunicazioni avioniche e le trasmissioni radio FM. La propagazione è principalmente diretta, conosciuta anche come onda spaziale o visiva.
    \item \textbf{UHF (Ultra High Frequency)}: Include servizi come la televisione digitale terrestre (DVB-T), comunicazioni cellulari, Wi-Fi e GPS. La propagazione è di tipo spaziale, e queste onde richiedono generalmente una linea di vista libera tra trasmettitore e ricevitore.
    \item \textbf{SHF (Super High Frequency) e EHF (Extremely High Frequency)}: Utilizzate per servizi avanzati come il Wi-Fi, il 5G, le trasmissioni satellitari (DVB-S), i radar e le comunicazioni satellitari (SatCom). Anche queste si propagano attraverso onda spaziale o visica.
\end{itemize}

Incrementando la frequenza si ha una banda di trasmissione maggiore, tuttavia la distanza di trasmissione diminuisce.

\begin{itemize}
    \item \textbf{Onde di terra}: Hanno una bassa frequenza e si propagano vicino alla superficie terrestre. Sono utilizzate per comunicazioni a lunga distanza.
    \item \textbf{Onde ionosferiche}: Hanno una frequenza media, sono riflesse dalla ionosfera e possono viaggiare a lunghe distanze.
    \item \textbf{Onde spaziali}: Hanno una frequenza elevata e si propagano in linea retta. Richiedono una linea di vista libera tra trasmettitore e ricevitore.
\end{itemize}

La questione dell'occupazione della banda è un argomento molto trattato attualmente in quanto trasmissioni alla solita frequenza generano collisioni impedendo la ricostruzione del segnale trasmesso. Per ridurre le collisioni si utilizzano tecniche di modulazione e codifica del segnale.


