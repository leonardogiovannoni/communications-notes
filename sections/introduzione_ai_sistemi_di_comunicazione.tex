\section*{Introduzione ai sistemi di comunicazione}

I sistemi di comunicazione sono sistemi che permettono di trasferire informazione da una o più sorgenti ad una o più destinazioni. Il sistema di comunicazione basilare è quello che usa un nodo sorgente ed un nodo destinazione. Tale sistema di comunicazione si denomina come sistema di comunicazione punto-punto.


\begin{figure}[ht]
    \centering
    \begin{adjustwidth*}{-2cm}{-2cm} % Regola questi valori a seconda delle necessità
        \centering
        % Here you start drawing the diagram with TikZ
        \begin{tikzpicture}[>=latex',font=\sffamily,semithick,scale=0.82, every node/.style={scale=0.82}]
            % Define the style for the blocks
            %\tikzstyle{block} = [draw, rectangle, minimum height=3em, minimum width=4em]
            \tikzstyle{block} = [rectangle, draw, minimum height=3em, minimum width=4em, text centered, align=center]

            % Nodes position
            \node [block] (source) {Sorgente};
            \node [block, right=1cm of source] (strasducer) {Trasduttore \\ di sorgente};
            \node [block, right=1cm of strasducer] (transmitter) {Modulatore};
            \node [block, right=1cm of transmitter] (channel) {Canale};
            \node [block, right=1cm of channel] (receiver) {Demodulatore};
            \node [block, right=1cm of receiver] (dtrasducer) {Trasduttore \\ di destinazione};
            \node [block, right=1cm of dtrasducer] (destination) {Destinazione};

            % Arrows
            \draw[->] (source) --   (strasducer) node[midway,above] {$m_s(t)$};
            \draw[->] (strasducer) --            (transmitter) node[midway,above] {$m(t)$};
            \draw[->] (transmitter) -- (channel) node[midway,above] {$s(t)$};
            \draw[->] (channel) -- (receiver) node[midway,above] {$r(t)$};
            \draw[->] (receiver) --  (dtrasducer) node[midway,above] {$\hat{m}(t)$};
            \draw[->] (dtrasducer) --               (destination) node[midway,above] {$\hat{m}_s(t)$};

            % Labels for modulation and demodulation
            \node[below of=transmitter, node distance=1cm] (modulation) {Trasmettitore (TX)};
            \node[below of=receiver, node distance=1cm] (demodulation) {Ricevitore (RX)};

            % Arrows for modulation and demodulation
            %$ \draw[->] (source) |- (modulation);
            %\draw[->] (modulation) -| (receiver);
            %\draw[->] (destination) |- (demodulation);
            %\draw[->] (demodulation) -| (transmitter);
        \end{tikzpicture}
    \end{adjustwidth*}
\end{figure}
% Now the explanation of the symbols in a table format
\begin{tabular}{ll}
    $m_s(t)$       & segnale fisico prodotto dalla sorgente (es. voce)                                      \\
    $m(t)$         & segnale elettrico ottenuto per trasduzione del segnale di sorgente (es. segnale audio) \\
    $s(t)$         & segnale modulato (trasmesso)                                                           \\
    $r(t)$         & segnale ricevuto                                                                       \\
    $\hat{m}(t)$   & segnale demodulato                                                                     \\
    $\hat{m}_s(t)$ & segnale trasdotto                                                                      \\
\end{tabular}
\\
\begin{enumerate}
    \item \textbf{Nodo di sorgente}: genera il messaggio (informazione) da comunicare.
    \item \textbf{Nodo di destinazione}: riceve il messaggio trasmesso.
    \item \textbf{Trasduttore di sorgente}: può convertire, ove necessario, il supporto fisico del segnale $m_s(t)$ così da generare un segnale elettrico opportuno per la trasmissione.



    \item \textbf{Trasmettitore}: converte il segnale elettrico \( m(t) \) in un nuovo segnale elettrico adatto per la trasmissione attraverso il canale di comunicazione a disposizione. Effettua sostanzialmente le seguenti operazioni:
          \begin{enumerate}
              \item Traslazione in frequenza (modulazione): fa sì che la occupazione di banda del segnale sia attorno a un opportuna frequenza centrale.
              \item Sagomatura: garantisce che la occupazione di banda sia quella ottimale e che non disturbi eventuali altre comunicazioni presenti nello stesso canale di comunicazione.
          \end{enumerate}

    \item \textbf{Canale di comunicazione}: permette il trasferimento del segnale \( s(t) \) dal nodo sorgente a quello di destinazione. Il canale di comunicazione prevede:
          \begin{enumerate}
              \item Un trasduttore di ingresso: converte il segnale elettrico \( s(t) \) in un segnale con supporto fisico compatibile con il mezzo trasmissivo (es. onda elettromagnetica per la trasmissione in aria libera per la trasmissione su fibra ottica, ecc.).
              \item Un mezzo trasmissivo: rappresenta il mezzo fisico su cui si propaga il segnale trasmesso (es. onda e vuoto per le onde elettromagnetiche o la fibra ottica per segnali luminosi).
              \item Un trasduttore di uscita: converte il segnale ricevuto tramite il mezzo trasmissivo in un segnale elettrico \( r(t) \).
          \end{enumerate}

    \item \textbf{Ricevitore}: trasforma il segnale \( r(t) \) in un segnale \( \hat{m}(t) \) traslando il segnale in frequenza (operazione di demodulazione) e operando un filtraggio. L'operazione di filtraggio si rende necessaria in quanto il segnale è generalmente una versione modificata del segnale \( s(t) \) per effetto della presenza di:
          \begin{itemize}
              \item rumore introdotto dal canale di comunicazione e dai dispositivi elettronici che compongono il ricevitore;
              \item distorsioni introdotte dal canale di comunicazione.
          \end{itemize}
          In ogni caso, il ricevitore deve effettuare una operazione di sagomatura inversa per ottenere il segnale \( \hat{m}(t) \) in modo che esso sia il più simile possibile al segnale \( m(t) \).

          Una condizione necessaria sul ricevitore è la seguente:
          \[
              \text{Se } r(t) = s(t) \text{ allora } \hat{m}(t) = m(t)
          \]

          Questa condizione deve essere interpretata nel seguente modo: se il segnale ricevuto è identico a quello trasmesso (assenza di disturbi introdotti dal canale di comunicazione) allora il segnale demodulato \( \hat{m}(t) \) deve essere identico a \( m(t) \).

          \subsection*{Banda Passante e Larghezza di Banda di un Canale di Comunicazione}
          Si può applicare il concetto di banda di un filtro.
          \pgfmathdeclarefunction{gauss}{2}{%
              \pgfmathparse{(1-exp(-((exp(-(abs((#1-x)/#2)*3-3)))^2)))}%
          }

          \begin{center}

              \begin{tikzpicture}
                  \begin{axis}[
                          title={Banda passante},
                          xlabel={Frequenza},
                          ylabel={Ampiezza},
                          xmin=0, xmax=30,
                          ymin=0, ymax=1,
                          xtick={10,15,20},
                          xticklabels={\( f_{\text{min}} \),\( f_0 \),\( f_{\text{max}} \)},
                          ytick={0.707,1},
                          yticklabels={-3dB,0dB},
                          legend pos=north east,
                          ymajorgrids=true,
                          grid style=dashed,
                      ]

                      \addplot[
                          color=red,
                          domain=0:30,
                          samples=100,
                          smooth,
                          thick,
                      ]
                      {gauss(15, 5)};

                      %\legend{Banda passante}

                  \end{axis}
              \end{tikzpicture}
          \end{center}

\end{enumerate}
\begin{itemize}
    \item $\text{Banda passante} \left\{ f \vert f_{\text{min}} \leq f \leq f_{\text{max}} \right\}$
    \item $\text{Larghezza di banda: } B = f_{\text{max}} - f_{\text{min}} $
    \item $\text{Frequenza centrale: } f_0 = \frac{f_{\text{max}} + f_{\text{min}}}{2}$
\end{itemize}



\subsection*{Canali a banda larga e a banda stretta}
Il concetto di banda larga o stretta è relativo alla frequenza centrale
\begin{itemize}
    \item \textbf{Banda larga}: $f_0 \leq 2B$
    \item \textbf{Banda stretta}: $f_0 > 2B$
\end{itemize}


Esempi:
\begin{enumerate}
    \item Doppino telefonico (banda larga):
          \begin{itemize}
              \item $f_0 = 2.45 KHz$
              \item $B = 3.7 KHz$
          \end{itemize}
    \item Canale radio (DVB-T, banda stretta)
          \begin{itemize}
              \item $f_0 = 400 MHz$
              \item $B = 8 MHz$
          \end{itemize}
\end{enumerate}

NB: Nonostante la larghezza di banda nel secondo caso sia maggiore che nel primo caso si ha che nel primo caso la banda è larga e nel secondo caso è stretta.

\subsection*{Il canale radio}
Il canale radio merita un po' di attenzione in quanto rappresenta il canale per le comunicazioni "wireless".



% Here we define the structure of the radio channel features
Le caratteristiche principali di un canale radio sono:
\begin{enumerate}
    \item È sempre un canale passa-banda.
    \item È tipicamente un canale a banda stretta.
    \item I trasduttori di canale sono delle \textbf{antenne} e queste devono avere delle dimensioni non inferiori a $\frac{\lambda}{10}$. Questo comporta dei limiti inferiori alle frequenze utilizzabili per la trasmissione radio.
\end{enumerate}

% The formula provided in the notes
\[
    \lambda = \frac{c}{f}
\]

% Radio channel frequency bands
\begin{tabular}{lll}
    LF (Low Frequency)     & 30 - 300 kHz   & Radiolocalizzazione marittima \\
                           & (1 - 10 km)    & e aeronautica                 \\
    MF (Medium Frequency)  & 300 - 3000 kHz & Radionavigazione e            \\
                           & (100 - 1000 m) & radiodiffusione               \\
    HF (High Frequency)    & 3 - 30 MHz     & Collegamenti a lunga distanza \\
                           & (10 - 100 m)   & (riflessione ionosferica)     \\
    VHF (Very High Freq.)  & 30 - 300 MHz   & Radiodiffusione               \\
                           & (1 - 10 m)     &                               \\
    UHF (Ultra High Freq.) & 300 - 3000 MHz & Servizi TV, telefonia mobile  \\
                           & (0.1 - 1 m)    &                               \\
    SHF (Super High Freq.) & 3 - 30 GHz     & TV satellitare, ponti radio   \\
                           & (1 - 10 cm)    &                               \\
\end{tabular}

% Notes on disturbances introduced in the communication channel

\subsection*{
    Disturbi introdotti dal canale di comunicazione
}
Il canale di comunicazione introduce distorsioni sul segnale trasmesso che possono essere descritte da trasformazioni deterministiche. In prima approssimazione tali distorsioni si possono assumere lineari e stazionarie.


Quindi, sotto questa ipotesi, il canale di comunicazione può essere visto come un filtro lineare e stazionario, che può introdurre quindi distorsioni lineari. La sua modellizzazione può essere a questo punto definita come segue:
\tikzset{
    block/.style = {draw, fill=white, rectangle,
            minimum height=3em, minimum width=3em},
    input/.style = {coordinate},
    output/.style = {coordinate},
    sum/.style = {draw, fill=white, circle, node distance=1cm}
}
% Block diagram for communication channel
\begin{center}
    \begin{tikzpicture}[auto, node distance=2cm,>=latex']
        \node [input, name=input] {};
        \node [block, right of=input] (channel) {c(t)};
        \node [output, right of=channel] (output) {};

        \draw [->] (input) -- node {s(t)} (channel);
        \draw [->] (channel) -- node {r(t)} (output);
        node [near end] {} (output);
    \end{tikzpicture}
\end{center}


dove \( c(t) \) è la risposta impulsiva del canale di comunicazione.

\bigskip
Il canale di comunicazione introduce inoltre un disturbo di tipo aleatorio noto come \textbf{rumore}. Tale rumore si può quindi modellare come un processo aleatorio di tipo additivo che quindi si somma alla componente utile del segnale ricevuto.
Quindi la modellizzazione completa di un canale di comunicazione può essere, in prima approssimazione, definita come segue:

% Complete block diagram with noise
\begin{center}
    \begin{tikzpicture}[auto, node distance=2cm,>=latex']
        \node [input, name=input] {};
        \node [block, right of=input] (channel) {c(t)};
        \node [draw, circle, right of=channel] (sum) {\(+\)};
        \node [output, right of=sum] (output) {};
        \node [block, below of=sum] (noise) {n(t)};

        \draw [->] (input) -- node {s(t)} (channel);
        \draw [->] (channel) -- node {} (sum);
        \draw [->] (noise) -- node {} (sum);
        \draw [->] (sum) -- node {r(t)} (output);
    \end{tikzpicture}
\end{center}

dove

\[
    r(t) = s(t) \circledast c(t) + n(t)
\]

% Ideal channel conditions
\textbf{Canale ideale}
\begin{align*}
                     & c(t) = \delta(t) \\
                     & n(t) = 0         \\
    \therefore \quad & r(t) = s(t)
\end{align*}


\subsection*{Sistemi di Comunicazione Analogici}
Un sistema di comunicazione si dice analogico quando sia \( m(t) \) che \( \hat{m}(t) \) sono segnali analogici.

Da notare che:
\begin{enumerate}
    \item Ovviamente anche \( m_s(t) \) e \( \hat{m}_s(t) \) sono segnali analogici in quanto differiscono rispettivamente da \( m(t) \) e \( \hat{m}(t) \) per un cambio del supporto fisico.
    \item \( s(t) \) e \( r(t) \) sono sempre segnali analogici in quanto il primo deve essere trasdotto in un segnale fisico e immesso nel mezzo trasmissivo ed il secondo si ottiene per trasduzione di un segnale fisico.
\end{enumerate}
