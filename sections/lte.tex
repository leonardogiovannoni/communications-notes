\section*{LTE (Long Term Evolution)}


\begin{center}
    \begin{threeparttable}
        \begin{tabular}{|l|l|} 
            \hline
            \textbf{Parametro} & \textbf{Valore} \\
            \hline
             Frequenza di campionamento\tnote{1 } \  ($f_s$) & 30.72 MHz \\
            \hline
            Subcarrier disponibili (Dimensione FFT, $N$) & 2048 \\
            \hline
            Subcarrier utilizzati & 1200 \\
            \hline
            Subcarrier di guardia\tnote{2} \ di livello 0 & 848 \\
            \hline
            Subcarrier per blocco risorsa (RB) & 12 \\
            \hline
            Tasso di codifica ($R$) & $[0.0762, 0.9258]$ \\
            \hline
            Ordine di modulazione ($M$) & 256 (8 bit/simbolo) \\
            \hline
            Numero blocchi OFDM per slot & 7 \\
            \hline
            Durata dello slot & 0.5 ms \\
            \hline
        \end{tabular}
        \begin{tablenotes}
            \item[1] Per OFDM il tempo di campionamento è $T_s = \frac{1}{B}$.
            \item[2] Subcarrier non utilizzati per evitare interferenze.
        \end{tablenotes}
    \end{threeparttable}
\end{center}
Lo standard LTE è il primo ad adottare una modulazione OFDM, con i seguenti parametri:
\[
    \begin{array}{ll}
        f_s = \frac{1}{T_s} = 30.72 \text{ MHz} & \text{(banda totale/sampling frequency, in OFDM $T_s=\frac{1}{B}$)} \\
        N = 2048 & \text{(numero di subcarrier disponibili)} \\
        \Delta f = \frac{f_s}{N} = 15 \text{ kHz} & \text{(banda di un singolo subcarrier)} \\
    \end{array}    
\]
L'allocazione delle risorse di banda agli utenti è indicata come OFDMA, ovvero FDMA basata su OFDM, per cui la banda assegnata è un multiplo intero della banda di un canale.
L'allocazione minima per ciascun utente è un blocco, composto da 12 subcarrier, chiamato \textbf{Resource Block} (RB), che quindi occupa $12 \times 15 \text{ kHz} = 180 \text{ kHz}$.
Inoltre l'allocazione delle risorse richiede anche l'assegnamente di slot temporali, ovvero il tempo entro cui i blocchi di banda sono utilizzabili.
L'unità minima è un slot composta da 7 blocchi OFDM (simboli) della durata di 0.5 ms.
\begin{itemize}
    \item slot temporale: 0.5 ms (7 simboli OFDM) 
    \item blocco di banda: 180 kHz (12 subcarrier)
\end{itemize}
Delle 2048 subcarrier solo 1200 sono utilizzate, le altre rappresentano virtual subcarrier. Per cui:
\[
    B_{\text{raw}} = 1200 \times 15 \text{ kHz} = 18 \text{ MHz}  
\]
Sebbene possa sembrare un numero molto alto di frequenze non utilizzate, la scelta deriva dall'utilizzo di un $f_s$ molto alto, tuttavia lo spettro realmente occupato è circa 20 MHz.
Inoltre un'altra porzione di spettro è utilizzata per informazioni di controllo e sincronizzazione, riducendo la banda disponibile di 4 MHz.
\[
    B_{\text{av}} = B_{\text{raw}} - 4 \text{ MHz} = 1200 \cdot 15 \text{ kHz} - 4 \text{ MHz} = 14 \text{ MHz}
\]

Per quanto riguarda il coding rate e modulation order la scelta avvviene in base alla qualità del link, valutata in termini di CQI (Channel Quality Indicator). Le modulazioni possibili sono quelle comprese tra 4-QAM e 256-QAM, mentre il range di coding rate varia da 0.0762 a 0.9258.
Il massimo rate su un canale SISO può quindi essere calcolato:
\[
    R_{1 \times 1} = 0.9258 \cdot 8 \cdot 14 \text{ MHz} \approx 100 \text{ Mbps}
\]
Lo standard prevedere il supporto per canali MIMO in diverse configurazioni, fino a 4$\times$4, per cui si ottiene un rate potenzialmente quadruplicato:
\[
    R_{4 \times 4} = 4 \cdot R_{1 \times 1} = 400 \text{ Mbps}
\]
I terminali avanzati possono combinare più bande di 20 MHz per raggiungere velocità fino a 2 Gbps.
Tuttavia, in pratica è difficile ottenere tali velocità poiché richiederebbero condizioni ottimali del canale, vicinanza alle antenne, e l'allocazione di tutte le risorse a un solo utente.
Inoltre, gli operatori di solito impongono limiti alle prestazioni massime. 
Con il carrier aggregation, gli operatori acquistano blocchi di spettro più larghi di 20 MHz che possono essere aggregati per aumentare la banda disponibile per un singolo utente, per esempio 3 con $4 \times 4$ MIMO e 2 con $2 \times 2$ MIMO, ottenendo un rate di:
\[
    R_{\text{aggregated}} = 3 \times 400 \text{ Mb/s} + 2 \times 200 \text{ Mb/s} = 1.6 \text{ Gb/s}
\]


