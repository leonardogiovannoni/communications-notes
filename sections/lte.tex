\section*{LTE Overview}

\textbf{Long Term Evolution (LTE)} represents the fourth generation (\textbf{4G}) of wireless communication standards, preceded by:
\begin{itemize}
    \item \textbf{1G}: Analog FDMA-based systems
    \item \textbf{2G}: GSM, the first digital standard supporting data rates up to 9.6 kb/s
    \item \textbf{3G}: UMTS (CDMA)
    \item \textbf{5G}: NR, currently being deployed
\end{itemize}
LTE is primarily utilized for data transmission due to its flexibility and high capacity.

\section*{Physical Layer LTE Numerology}

\begin{itemize}
    \item LTE employs \textbf{Orthogonal Frequency-Division Multiplexing (OFDM)} as its core technology, similar to Wi-Fi systems.
    \item \textbf{Sampling time} (\(f_s\)) is 30.72 MHz, which is a critical factor in determining the FFT size and subcarrier bandwidth.
    \item \textbf{FFT size} (\(N\)) is 2048, playing a role in the frequency resolution of the system.
    \item \textbf{Subcarrier bandwidth} (\(\Delta f\)) is defined as \(\frac{30.72 \text{ MHz}}{2048} = 15 \text{ kHz}\), dictating the spacing between individual OFDM subcarriers.
    \item An \textbf{LTE slot} consists of 7 OFDM symbols, with a total duration of 0.5 ms, thus determining the time-domain structure of LTE transmission.
\end{itemize}

The figure below illustrates a \textbf{resource block} in LTE, showing the subcarrier spacing and the time-frequency structure of the LTE slot and resource blocks.

% Insert image of LTE numerology

\section*{Key Mathematical Relationships}

Given the LTE numerology, we can define some key relationships:
\begin{align*}
    \text{Subcarrier spacing} (\Delta f) &= \frac{f_s}{N} \\
    \text{Slot duration} &= 0.5 \text{ ms (for 7 OFDM symbols)} \\
    \text{Resource block width} &= 180 \text{ kHz} = 12 \text{ subcarriers} \times 15 \text{ kHz}
\end{align*}

These relationships are crucial for understanding LTE's capacity to handle various data rates and modulations, balancing the need for speed with the practical limitations of the wireless medium.



\section*{Physical Layer LTE Numerology and Peak Data Rate}

LTE employs \textbf{Orthogonal Frequency-Division Multiple Access (OFDMA)} for its multiple access technique. This involves allocating groups of subcarriers to different users, which enables efficient spectrum utilization.

The \textbf{minimum allocation unit} in LTE is a \textbf{resource block (RB)}. Each RB:
\begin{itemize}
    \item Contains \(12\) subcarriers
    \item Spans \(180\) kHz in frequency for the duration of a slot (\(0.5\) ms)
\end{itemize}

Regarding the peak data rate for high-end mobile phones:
\begin{itemize}
    \item These often fall in the category list that supports very high data rates (e.g., 19-20 category).
    \item The \textbf{raw bandwidth} calculation is as follows:
          \begin{align*}
              B_{\text{raw}} &= \text{Number of RBs} \times \text{Subcarrier Spacing} \\
                             &= 100 \times 180\text{ kHz} \\
                             &= 18\text{ MHz}
          \end{align*}
    \item Considering control and synchronization overheads, the \textbf{available bandwidth} \(B_{av}\) is:
          \begin{align*}
              B_{av} &= B_{\text{raw}} - 4\text{ MHz for control and synchronization} \\
                     &= 14\text{ MHz}
          \end{align*}
\end{itemize}

\subsection*{Additional Notes}

\begin{itemize}
    \item In practice, \(1200\) of the \(2048\) available subcarriers are used, which accounts for the data rate calculations.
    \item A total of \(848\) are left as guard subcarriers, which are not used for transmission to avoid interference.
    \item The effective channel utilization and data rate can be significantly influenced by the actual number of subcarriers allocated for user data.
\end{itemize}

% An illustration or table showing LTE numerology and data rate could be included here
% \includegraphics{path_to_image}







\section*{5G New Radio (NR) and Latest Chipsets}

\subsection*{5G New Radio (NR)}
5G NR is the latest global wireless standard after 1G, 2G, 3G, and 4G networks, enabling a new kind of network designed to connect virtually everyone and everything together including machines, objects, and devices.

\begin{itemize}
    \item It is characterized by \textbf{Enhanced Mobile Broadband (eMBB)} that aims to deliver peak data rates in the range of gigabytes per second.
    \item It supports a variety of services from high-quality video streaming and virtual reality to reliable low latency communications required for autonomous vehicles.
    \item \textbf{Massive Machine-Type Communications (mMTC)} and \textbf{Ultra-Reliable Low Latency Communications (URLLC)} are central to 5G capabilities, supporting the wide-scale IoT and mission-critical applications, respectively.
\end{itemize}

\subsection*{Most Recent 5G Chip}
The \textbf{Qualcomm\textsuperscript{\textregistered} Snapdragon X65} is the latest 5G chipset that includes:
\begin{itemize}
    \item Support for 5G mmWave and sub-6 GHz frequencies with a peak download speed of 10 Gbps.
    \item 3GPP Release 16 support and a 5G mmWave-sub6 aggregation for expanded range and bandwidth.
    \item Advanced features such as Qualcomm\textsuperscript{\textregistered} Smart Transmit technology, Qualcomm\textsuperscript{\textregistered} 545 mmWave Antenna Module for enhanced signal quality, and AI-based signal boost functionalities.
\end{itemize}

\subsection*{Chipset Specifications}
\begin{itemize}
    \item The modem supports dynamic spectrum sharing (DSS), SA (standalone), NSA (non-standalone) modes, and global multi-SIM.
    \item Designed for compatibility across various technologies such as GSM, CDMA, WCDMA, and others to ensure broad spectrum usage and coverage.
\end{itemize}

