\documentclass{standalone}

\usepackage{tikz,pgf} %and any other packages or tikzlibraries your picture needs

\usepackage{pgfplots}

\usepackage[utf8]{inputenc}
\usepackage[colorlinks=true, allcolors=blue]{hyperref}
\usetikzlibrary{positioning, arrows.meta, fit, shapes}
\begin{document}
\begin{tikzpicture}[
        block/.style={rectangle, draw, minimum height=1cm, minimum width=1.5cm},
        node distance=1cm and 1cm,
        auto
    ]

    \tikzstyle{tri} = [draw, isosceles triangle, isosceles triangle apex angle=60, shape border rotate=360, minimum height=2em]
    \node[inner sep=0pt, minimum size=0pt] (source) {};


    \node[tri, right=of source] (encoder) {};
    \node[block, right=of encoder] (interp) {BPF};
    \node[above=of interp, inner sep=0pt, minimum size=0pt] (dummy1) {};
    \node[below=of interp, inner sep=0pt, minimum size=0pt] (dummy2) {};
    \node[draw, circle, right=1cm of interp] (m1) {\(\times\)};
    \draw[->] (interp) -- (m1);
    \node[below=of m1] (cos) {};
    \draw[->] (cos) -- (m1) node[midway, right] {$\cos(2\pi(f_{LO} - f_{IF})t)$};
    \node[tri, right=of m1] (dummy3) {};
    \draw[->] (m1) -- (dummy3) node[midway, above] {};
    \node[right=1cm of dummy3, inner sep=0pt, minimum size=0pt] (dummy4) {};
    \draw[->] (dummy3) -- (dummy4) node[midway, above] {};
    \draw[->] (source) -- (encoder) node[midway,above] {};
    \draw[->] (encoder) -- (interp) node[midway,above] {};

\end{tikzpicture}
\end{document}
