
\documentclass{article}

\usepackage{amsfonts}
\usepackage{amsmath}
\usepackage{amssymb}
\usepackage{caption}
\usepackage{changepage}
\usepackage[italian]{babel}
\usepackage[letterpaper,top=2cm,bottom=2cm,left=2cm,right=2cm,marginparwidth=1.75cm]{geometry}
\usepackage{mathtools} 
\usepackage{pgfplots}
\usepackage{graphicx}
\usepackage{svg}
\usepackage{array}
\usepackage{pgfplots}
\usepackage{tikz}
\usepackage{tikz} 
\usepackage[utf8]{inputenc}
\usepackage[colorlinks=true, allcolors=blue]{hyperref}
\pgfplotsset{compat=newest}
\usetikzlibrary{arrows.meta, positioning, patterns}
\usepgfplotslibrary{fillbetween}
\usetikzlibrary{arrows.meta, positioning, shapes.geometric, calc}
\usetikzlibrary{arrows,positioning,shapes.geometric}
\pgfplotsset{compat=newest}
\usetikzlibrary{positioning, arrows.meta, fit, calc, shapes}

\title{Appunti di sistemi di comunicazione}
\author{Leonardo Giovannoni}

\begin{document}
\maketitle

%\section*{Introduzione ai sistemi di comunicazione}

I sistemi di comunicazione sono sistemi che permettono di trasferire informazione da una o più sorgenti ad una o più destinazioni. Il sistema di comunicazione basilare è quello che usa un nodo sorgente ed un nodo destinazione. Tale sistema di comunicazione si denomina come sistema di comunicazione punto-punto.


\begin{figure}[ht]
    \centering
    \begin{adjustwidth*}{-2cm}{-2cm} % Regola questi valori a seconda delle necessità
        \centering
        % Here you start drawing the diagram with TikZ
        \begin{tikzpicture}[>=latex',font=\sffamily,semithick,scale=0.82, every node/.style={scale=0.82}]
            % Define the style for the blocks
            %\tikzstyle{block} = [draw, rectangle, minimum height=3em, minimum width=4em]
            \tikzstyle{block} = [rectangle, draw, minimum height=3em, minimum width=4em, text centered, align=center]

            % Nodes position
            \node [block] (source) {Sorgente};
            \node [block, right=1cm of source] (strasducer) {Trasduttore \\ di sorgente};
            \node [block, right=1cm of strasducer] (transmitter) {Modulatore};
            \node [block, right=1cm of transmitter] (channel) {Canale};
            \node [block, right=1cm of channel] (receiver) {Demodulatore};
            \node [block, right=1cm of receiver] (dtrasducer) {Trasduttore \\ di destinazione};
            \node [block, right=1cm of dtrasducer] (destination) {Destinazione};

            % Arrows
            \draw[->] (source) --   (strasducer) node[midway,above] {$m_s(t)$};
            \draw[->] (strasducer) --            (transmitter) node[midway,above] {$m(t)$};
            \draw[->] (transmitter) -- (channel) node[midway,above] {$s(t)$};
            \draw[->] (channel) -- (receiver) node[midway,above] {$r(t)$};
            \draw[->] (receiver) --  (dtrasducer) node[midway,above] {$\hat{m}(t)$};
            \draw[->] (dtrasducer) --               (destination) node[midway,above] {$\hat{m}_s(t)$};

            % Labels for modulation and demodulation
            \node[below of=transmitter, node distance=1cm] (modulation) {Trasmettitore (TX)};
            \node[below of=receiver, node distance=1cm] (demodulation) {Ricevitore (RX)};

            % Arrows for modulation and demodulation
            %$ \draw[->] (source) |- (modulation);
            %\draw[->] (modulation) -| (receiver);
            %\draw[->] (destination) |- (demodulation);
            %\draw[->] (demodulation) -| (transmitter);
        \end{tikzpicture}
    \end{adjustwidth*}
\end{figure}
% Now the explanation of the symbols in a table format
\begin{tabular}{ll}
    $m_s(t)$       & segnale fisico prodotto dalla sorgente (es. voce)                                      \\
    $m(t)$         & segnale elettrico ottenuto per trasduzione del segnale di sorgente (es. segnale audio) \\
    $s(t)$         & segnale modulato (trasmesso)                                                           \\
    $r(t)$         & segnale ricevuto                                                                       \\
    $\hat{m}(t)$   & segnale demodulato                                                                     \\
    $\hat{m}_s(t)$ & segnale trasdotto                                                                      \\
\end{tabular}
\\
\begin{enumerate}
    \item \textbf{Nodo di sorgente}: genera il messaggio (informazione) da comunicare.
    \item \textbf{Nodo di destinazione}: riceve il messaggio trasmesso.
    \item \textbf{Trasduttore di sorgente}: può convertire, ove necessario, il supporto fisico del segnale $m_s(t)$ così da generare un segnale elettrico opportuno per la trasmissione.



    \item \textbf{Trasmettitore}: converte il segnale elettrico \( m(t) \) in un nuovo segnale elettrico adatto per la trasmissione attraverso il canale di comunicazione a disposizione. Effettua sostanzialmente le seguenti operazioni:
          \begin{enumerate}
              \item Traslazione in frequenza (modulazione): fa sì che la occupazione di banda del segnale sia attorno a un opportuna frequenza centrale.
              \item Sagomatura: garantisce che la occupazione di banda sia quella ottimale e che non disturbi eventuali altre comunicazioni presenti nello stesso canale di comunicazione.
          \end{enumerate}

    \item \textbf{Canale di comunicazione}: permette il trasferimento del segnale \( s(t) \) dal nodo sorgente a quello di destinazione. Il canale di comunicazione prevede:
          \begin{enumerate}
              \item Un trasduttore di ingresso: converte il segnale elettrico \( s(t) \) in un segnale con supporto fisico compatibile con il mezzo trasmissivo (es. onda elettromagnetica per la trasmissione in aria libera per la trasmissione su fibra ottica, ecc.).
              \item Un mezzo trasmissivo: rappresenta il mezzo fisico su cui si propaga il segnale trasmesso (es. onda e vuoto per le onde elettromagnetiche o la fibra ottica per segnali luminosi).
              \item Un trasduttore di uscita: converte il segnale ricevuto tramite il mezzo trasmissivo in un segnale elettrico \( r(t) \).
          \end{enumerate}

    \item \textbf{Ricevitore}: trasforma il segnale \( r(t) \) in un segnale \( \hat{m}(t) \) traslando il segnale in frequenza (operazione di demodulazione) e operando un filtraggio. L'operazione di filtraggio si rende necessaria in quanto il segnale è generalmente una versione modificata del segnale \( s(t) \) per effetto della presenza di:
          \begin{itemize}
              \item rumore introdotto dal canale di comunicazione e dai dispositivi elettronici che compongono il ricevitore;
              \item distorsioni introdotte dal canale di comunicazione.
          \end{itemize}
          In ogni caso, il ricevitore deve effettuare una operazione di sagomatura inversa per ottenere il segnale \( \hat{m}(t) \) in modo che esso sia il più simile possibile al segnale \( m(t) \).

          Una condizione necessaria sul ricevitore è la seguente:
          \[
              \text{Se } r(t) = s(t) \text{ allora } \hat{m}(t) = m(t)
          \]

          Questa condizione deve essere interpretata nel seguente modo: se il segnale ricevuto è identico a quello trasmesso (assenza di disturbi introdotti dal canale di comunicazione) allora il segnale demodulato \( \hat{m}(t) \) deve essere identico a \( m(t) \).

          \subsection*{Banda Passante e Larghezza di Banda di un Canale di Comunicazione}
          Si può applicare il concetto di banda di un filtro.
          \pgfmathdeclarefunction{gauss}{2}{%
              \pgfmathparse{(1-exp(-((exp(-(abs((#1-x)/#2)*3-3)))^2)))}%
          }

          \begin{center}

              \begin{tikzpicture}
                  \begin{axis}[
                          title={Banda passante},
                          xlabel={Frequenza},
                          ylabel={Ampiezza},
                          xmin=0, xmax=30,
                          ymin=0, ymax=1,
                          xtick={10,15,20},
                          xticklabels={\( f_{\text{min}} \),\( f_0 \),\( f_{\text{max}} \)},
                          ytick={0.707,1},
                          yticklabels={-3dB,0dB},
                          legend pos=north east,
                          ymajorgrids=true,
                          grid style=dashed,
                      ]

                      \addplot[
                          color=red,
                          domain=0:30,
                          samples=100,
                          smooth,
                          thick,
                      ]
                      {gauss(15, 5)};

                      %\legend{Banda passante}

                  \end{axis}
              \end{tikzpicture}
          \end{center}

\end{enumerate}
\begin{itemize}
    \item $\text{Banda passante} \left\{ f \vert f_{\text{min}} \leq f \leq f_{\text{max}} \right\}$
    \item $\text{Larghezza di banda: } B = f_{\text{max}} - f_{\text{min}} $
    \item $\text{Frequenza centrale: } f_0 = \frac{f_{\text{max}} + f_{\text{min}}}{2}$
\end{itemize}



\subsection*{Canali a banda larga e a banda stretta}
Il concetto di banda larga o stretta è relativo alla frequenza centrale
\begin{itemize}
    \item \textbf{Banda larga}: $f_0 \leq 2B$
    \item \textbf{Banda stretta}: $f_0 > 2B$
\end{itemize}


Esempi:
\begin{enumerate}
    \item Doppino telefonico (banda larga):
          \begin{itemize}
              \item $f_0 = 2.45 KHz$
              \item $B = 3.7 KHz$
          \end{itemize}
    \item Canale radio (DVB-T, banda stretta)
          \begin{itemize}
              \item $f_0 = 400 MHz$
              \item $B = 8 MHz$
          \end{itemize}
\end{enumerate}

NB: Nonostante la larghezza di banda nel secondo caso sia maggiore che nel primo caso si ha che nel primo caso la banda è larga e nel secondo caso è stretta.

\subsection*{Il canale radio}
Il canale radio merita un po' di attenzione in quanto rappresenta il canale per le comunicazioni "wireless".



% Here we define the structure of the radio channel features
Le caratteristiche principali di un canale radio sono:
\begin{enumerate}
    \item È sempre un canale passa-banda.
    \item È tipicamente un canale a banda stretta.
    \item I trasduttori di canale sono delle \textbf{antenne} e queste devono avere delle dimensioni non inferiori a $\frac{\lambda}{10}$. Questo comporta dei limiti inferiori alle frequenze utilizzabili per la trasmissione radio.
\end{enumerate}

% The formula provided in the notes
\[
    \lambda = \frac{c}{f}
\]

% Radio channel frequency bands
\begin{tabular}{lll}
    LF (Low Frequency)     & 30 - 300 kHz   & Radiolocalizzazione marittima \\
                           & (1 - 10 km)    & e aeronautica                 \\
    MF (Medium Frequency)  & 300 - 3000 kHz & Radionavigazione e            \\
                           & (100 - 1000 m) & radiodiffusione               \\
    HF (High Frequency)    & 3 - 30 MHz     & Collegamenti a lunga distanza \\
                           & (10 - 100 m)   & (riflessione ionosferica)     \\
    VHF (Very High Freq.)  & 30 - 300 MHz   & Radiodiffusione               \\
                           & (1 - 10 m)     &                               \\
    UHF (Ultra High Freq.) & 300 - 3000 MHz & Servizi TV, telefonia mobile  \\
                           & (0.1 - 1 m)    &                               \\
    SHF (Super High Freq.) & 3 - 30 GHz     & TV satellitare, ponti radio   \\
                           & (1 - 10 cm)    &                               \\
\end{tabular}

% Notes on disturbances introduced in the communication channel

\subsection*{
    Disturbi introdotti dal canale di comunicazione
}
Il canale di comunicazione introduce distorsioni sul segnale trasmesso che possono essere descritte da trasformazioni deterministiche. In prima approssimazione tali distorsioni si possono assumere lineari e stazionarie.


Quindi, sotto questa ipotesi, il canale di comunicazione può essere visto come un filtro lineare e stazionario, che può introdurre quindi distorsioni lineari. La sua modellizzazione può essere a questo punto definita come segue:
\tikzset{
    block/.style = {draw, fill=white, rectangle,
            minimum height=3em, minimum width=3em},
    input/.style = {coordinate},
    output/.style = {coordinate},
    sum/.style = {draw, fill=white, circle, node distance=1cm}
}
% Block diagram for communication channel
\begin{center}
    \begin{tikzpicture}[auto, node distance=2cm,>=latex']
        \node [input, name=input] {};
        \node [block, right of=input] (channel) {c(t)};
        \node [output, right of=channel] (output) {};

        \draw [->] (input) -- node {s(t)} (channel);
        \draw [->] (channel) -- node {r(t)} (output);
        node [near end] {} (output);
    \end{tikzpicture}
\end{center}


dove \( c(t) \) è la risposta impulsiva del canale di comunicazione.

\bigskip
Il canale di comunicazione introduce inoltre un disturbo di tipo aleatorio noto come \textbf{rumore}. Tale rumore si può quindi modellare come un processo aleatorio di tipo additivo che quindi si somma alla componente utile del segnale ricevuto.
Quindi la modellizzazione completa di un canale di comunicazione può essere, in prima approssimazione, definita come segue:

% Complete block diagram with noise
\begin{center}
    \begin{tikzpicture}[auto, node distance=2cm,>=latex']
        \node [input, name=input] {};
        \node [block, right of=input] (channel) {c(t)};
        \node [draw, circle, right of=channel] (sum) {\(+\)};
        \node [output, right of=sum] (output) {};
        \node [block, below of=sum] (noise) {n(t)};

        \draw [->] (input) -- node {s(t)} (channel);
        \draw [->] (channel) -- node {} (sum);
        \draw [->] (noise) -- node {} (sum);
        \draw [->] (sum) -- node {r(t)} (output);
    \end{tikzpicture}
\end{center}

dove

\[
    r(t) = s(t) \ast c(t) + n(t)
\]

\textbf{Canale ideale}
\begin{align*}
                     & c(t) = \delta(t) \\
                     & n(t) = 0         \\
                    & r(t) = s(t)
\end{align*}


\subsection*{Sistemi di Comunicazione Analogici}
Un sistema di comunicazione si dice analogico quando sia \( m(t) \) che \( \hat{m}(t) \) sono segnali analogici.

Da notare che:
\begin{enumerate}
    \item Ovviamente anche \( m_s(t) \) e \( \hat{m}_s(t) \) sono segnali analogici in quanto differiscono rispettivamente da \( m(t) \) e \( \hat{m}(t) \) per un cambio del supporto fisico.
    \item \( s(t) \) e \( r(t) \) sono sempre segnali analogici in quanto il primo deve essere trasdotto in un segnale fisico e immesso nel mezzo trasmissivo ed il secondo si ottiene per trasduzione di un segnale fisico.
\end{enumerate}

%
\section*{Sistemi di Comunicazione Numerici}
Per definizione un sistema di comunicazione numerico è tale quando \( m_s(t) \) e \( \hat{m}_s(t)\) sono sequenze numeriche (\( m_s[n] \) e \( m_r[n] \)) e \( m(t) \) e \( \hat{m}(t) \) sono segnali numerici.

N.B. \( s(t) \) e \( r(t) \) continuano ad essere segnali analogici.

\begin{center}
    \begin{tikzpicture}[auto, node distance=2cm,>=latex']
        \node [input, name=input] {};
        \node [block, right of=input] (modulator) {\( p(t) \)};
        \node [output, right of=modulator] (output) {};

        \draw [->] (input) -- node {\( m_s[n] \)} (modulator);
        \draw [->] (modulator) -- node {\( m(t) \)} (output);
    \end{tikzpicture}
\end{center}
Dove
\( T_s \) è il periodo di segnalazione della sorgente e
\( m_s[n] \) è la sequenza generata dalla sorgente con periodo di sequenza \( T_s \),
Ogni simbolo di \( m_s[n] \) appartiene ad un alfabeto predefinito
\[ m_s[n] \in A_s ,  A_s = \alpha_1, \alpha_2, \ldots, \alpha_M\} ,  M \geq 2  \]La sequenza \( m_s[n] \) può essere vista come il risultato (realizzazione) del campionamento di un processo aleatorio.



\( M(t) \) è un processo aleatorio e \( M_s[n] \) è una V.A. estratta dal processo. \\
Si definisce la sequenza aleatoria di sorgente come
\[ \{ M_s[n] \in A_s, n = 0, \pm 1, \pm 2, \ldots \} \]
N.B. \( m_s[n] \) è una realizzazione di \( M[n] \).

Alla destinazione si definisce in maniera analoga la sequenza di destinazione
\[ \{ \hat{M}_s[n] \in A_s, n = 0, \pm 1, \pm 2, \ldots \} \]

Il segnale \( m(t) \) è ottenuto dalla sequenza aleatoria di sorgente tramite una operazione di modulazione che è del tutto equivalente alla operazione di interpolazione
\[ m(t) = \sum_{n=-\infty}^{+\infty} m_s[n]\cdot p(t - nT_s) \]
dove \( p(t) \) è l'impulso in trasmissione.

Ad esempio \( p(t) = \text{rect}\left(\frac{t}{T_s}\right) \).

Il tasso di erogazione della sorgente è definito come
\[ R_s \coloneqq \frac{1}{T_s} \]
che è il rate con cui escono i simboli apparentemente dall'alfabeto \( A_s \).

Se è presente una codifica binaria per rappresentare un simbolo dell'alfabeto \( A_s \) occorrono \( \log_2 M \) simboli binari,
per cui si definisce il tasso di errore binario come:
\[
    R_b \coloneqq \frac{\log_2 M}{T_s}
\]

\subsection*{Schema funzionale di un sistema di comunicazione numerico}
\begin{adjustwidth*}{-2.3cm}{-2cm}
    \begin{center}
        \begin{tikzpicture}[>=Stealth, block/.style={draw, rectangle}, scale=0.85]
            % Blocks
            \tikzstyle{block} = [rectangle, draw, text centered, minimum height=4em, align=center]

            \node[block] (mod) {Modulatore \\ numerico};
            \node[block, right=1cm of mod] (channel) {Canale di \\ comunicazione};
            \node[block, right=1cm of channel] (demod) {Demodulatore \\ numerico};

            % Nodes for connecting lines
            \coordinate[left=2cm of mod] (input);
            \coordinate[right=2cm of demod] (output);

            % Lines
            \draw[->] (input) -- node[above] {$m_s[n]$} (mod);
            \draw[->] (mod) -- node[above] {$s(t)$} (channel);
            \draw[->] (channel) -- node[above] {$r(t)$} (demod);
            \draw[->] (demod) -- node[above] {$\hat{m}_s[n]$} (output);

            % Dashed box
            \begin{scope}
                \draw[dashed, red] ($(mod.north west)+(-0.5,0.5)$) rectangle ($(demod.south east)+(0.5,-0.5)$);
            \end{scope}

            % Annotations
            \node[align=center, red, above right= -1cm and -6cm of demod.south east] (channel-label) {Canale numerico};

            % Circles
            \draw (input) ++(-0.5cm,0) circle (0.5cm) node {$S$};
            \draw (output) ++(0.5cm,0) circle (0.5cm) node {$D$};
        \end{tikzpicture}
    \end{center}
\end{adjustwidth*}
\begin{itemize}
    \item La coppia modulatore/demodulatore numerico può essere interpretata come un livello di trasduzione. Questo ci permette di vedere il sistema di comunicazione come un canale numerico tra \( S \) e \( D \).
    \item La sorgente genera una sequenza di simboli appartenenti all'alfabeto \( A_s \), che vogliamo trasferire al nodo di destinazione.
    \item Il modulatore numerico può essere visto come l'insieme del trasduttore di sorgente e del modulatore. Questo genera quindi il segnale \( s(t) \) analogico che può essere di tipo passa-basso (in banda base) o di tipo passa-banda (in banda passante).
    \item Il canale di comunicazione è sempre lo stesso (per segnali numerici e analogici).



    \item Il demodulatore numerico può essere visto come l'insieme del demodulatore e del trasformatore di destinazione. Esso produce la sequenza $\hat{m}_s[n]$ dal segnale $v(t)$.

\end{itemize}

Un canale numerico ideale produce:
\begin{equation}
    \hat{m}_s[n] = m_s[n] \quad \forall n
\end{equation}

In casi pratici un canale numerico non è mai ideale e per cui ha senso definire il suo comportamento e quindi le sue prestazioni.

\subsection*{Probabilità di transizione}
\begin{equation}
    P\{i|j\} \coloneqq P\{\hat{m}_s[n]=\alpha_i
    \ |\  m_s[n]=\alpha_j\}
\end{equation}

Un canale numerico è statisticamente caratterizzato quando sono note le $P\{i|j\}$ $\forall i,j$

Se le $P\{i|j\}$ non dipendono da $n$ allora il canale numerico si dice \textbf{stazionario}.

L'insieme delle $P\{i|j\}$ è parte di $M^2$ dove $M$ è la cardinalità dell'alfabeto $A_s$.

Per un canale ideale quindi:
\begin{align}
    P\{i|j\} & = 1 \quad \text{se } i=j      \\
    P\{i|j\} & = 0 \quad \text{se } i \neq j
\end{align}

N.B. Le $P\{i|j\}$ non dipendono solo dai disturbi introdotti dal canale di comunicazione, ma anche dalla modulazione impiegata, per cui esse legano contro delle prestazioni di tutto il sistema numerico.

% Qui potresti includere il disegno utilizzando l'ambiente tikzpicture se necessario
\subsection*{Misure delle prestazioni di un sistema di comunicazione numerico}

Le prestazioni di un sistema di comunicazione numerico sono associabili alla probabilità di errore di simbolo \(M\)-ario

\[
    P_E(M) \coloneqq P\{\hat{m}_s[n] \neq m_s[n]\}
\]

Se la \(P_E(M)\) non dipende da \(n\), allora il sistema di comunicazione è stazionario.

\subsection*{Quality of Service (QoS)}

La qualità del servizio per un sistema di comunicazione numerico è associabile alla probabilità di errore \(P_E(M)\). È ragionevole quindi pensare che si debba fissare una \(P_E(M)\) massima al di là della quale la QoS non è accettabile.

\[
    P_E(M) \leq P_{\text{max}}
\]

Esempio: per i servizi voce \(P_{\text{max}} = 10^{-3}\), mentre per i servizi dati \(P_{\text{max}} = 10^{-7}\).

\subsection*{Dualismo banda e potenza}

Aumentando la potenza del segnale in trasmissione si può fare in modo che la componente utile del segnale prevalga sulla componente di rumore. Questo intuitivamente tende a migliorare le prestazioni del sistema. La potenza però costa e comunque esistono dei limiti fisici o imposti che la limitano.

Lo stesso si può dire per la banda, si può dimostrare che all'aumentare della banda del segnale trasmesso si migliorano le prestazioni del sistema. Anche la banda però è una risorsa limitata.


\subsection*{Efficienza di potenza e efficienza spettrale}

Una modulazione si dice:
\begin{itemize}
    \item \textbf{efficiente in potenza}: quando la potenza trasmessa è bassa a fronte di un certo livello di prestazioni;
    \item \textbf{efficiente spettralmente}: quando la banda utilizzata è piccola a fronte di un certo livello di prestazioni.
\end{itemize}

Sfortunatamente i sistemi efficienti spettralmente non sono anche efficienti in potenza.

Vantaggi di un sistema di comunicazione rispetto ad un altro:
\begin{enumerate}
    \item Basso costo
    \item Sicurezza nella trasmissione di un messaggio
    \item Trasferimento aggregato di messaggi di natura diversa (multiplazione, audio, video, dati)
    \item Possibilità di utilizzare modulazioni e codifiche che rendono il sistema efficiente in potenza e/o spettralmente.
\end{enumerate}

Svantaggi:
\begin{enumerate}
    \item Generalmente la banda occupata da un segnale numerico è maggiore del corrispondente analogico
    \item Complessità, soprattutto per la sincronizzazione
\end{enumerate}


%\section*{Modulazioni numeriche in banda base}

\begin{tikzpicture}[
    block/.style={rectangle, draw, minimum height=1cm, minimum width=2.5cm},
    node distance=1cm and 2cm,
    auto
]
    % Blocks
    \node[block] (source) {Sorgente binaria};
    \node[block, right=of source] (encoder) {Codificatore};
    \node[block, right=of encoder] (interp) {Interpolatore};
    \node[block, right=of interp] (channel) {Canale di comunicazione};
    \node[block, below=of channel] (filter) {Filtro di ricezione};
    \node[block, left=of filter] (sampler) {Campionatore};
    \node[block, left=of sampler] (decisor) {Decisore};
    \node[block, left=of decisor] (destination) {Destinazione binaria};

    % Arrows
    \draw[->] (source) -- (encoder) node[midway,above] {$b[n]$}  node[midway,below] {$T_b$};
    \draw[->] (encoder) -- (interp) node[midway,above] {$x[n]$}  node[midway,below] {$T_s$};
    \draw[->] (interp) -- (channel) node[midway,above] {$s(t)$};
    \draw[->] (channel) -- (filter) node[midway,right] {$r(t)$};
    %\draw[->] (filter) -- (sampler) node[midway,above] {$y(t)$};
    \draw[->] (sampler) -- (decisor) node[midway,above] {$\hat{x}[n]$};
    \draw[->] (decisor) -- (destination) node[midway,above] {$\hat{b}[n]$};
    
        % Draw the diagonal line
    % Dashed Boxes
    \draw ([xshift=0]filter.west) -- ([xshift=-0.75cm]filter.west) node[midway,above] {$y(t)$};
    \draw ([xshift=-0.75cm]filter.west) -- ([xshift=-1.25cm,yshift=0.5cm]filter.west) node[midway,below] {$T_s$};

    \draw[->] ([xshift=-1.25cm,yshift=0cm]filter.west) -- ++(sampler) node[midway,above] {$y[n]$};

    \draw[dashed, red, thick] ([xshift=-0.5cm,yshift=0.5cm]sampler.north west) rectangle ([xshift=0.5cm,yshift=-0.5cm]filter.south east);

     \node[align=center, red, above right= -1cm and -6cm of filter.south east] (channel-label) {Demodulatore numerico};

  
   

\end{tikzpicture}

La modulazione numerica è necessaria per poter trasmettere sequenze binarie attraverso un mezzo trasmissivo. In particolare, per adesso ci concentreremo su canali trasmissivi in banda base, come ad esempio il doppino telefonico o il cavo coassiale.

\begin{tikzpicture}
% Drawing the block diagram using TikZ package
% Note: You will need to adjust the positions of the blocks (nodes) and the arrows (paths) to match your notes.
\end{tikzpicture}

% Here you should draw the block diagram as per your notes using the TikZ package.
% Since the TikZ diagram could be quite complex, I will only outline the structure here.
\begin{enumerate}
    
\item \textbf{Codificatore}: trasforma sequenze di bit in simboli \( M \)-ari appartenenti a un alfabeto \( A_s \).

\item \textbf{Interpolatore}: modula impulsi tramite i simboli \( x[n] \) in ingresso per produrre una sequenza di impulsi \( s(t) \).
\[
s(t) = \sum_{k=-\infty}^{\infty} x[k] \cdot p(t - kT_s)
\]

\item \textbf{Canale di comunicazione}: sono assenti i trasduttori in quanto la propagazione nel mezzo trasmissivo è elettrica.

\item \textbf{Filtro di ricezione}: filtra componenti di rumore generate nel canale e compensa eventuali distorsioni.

\item \textbf{Campionatore}: preleva campioni dal segnale filtrato \( y(t) \).

\item \textbf{Decisore}: associa un simbolo dell'alfabeto \( A_s \) ad ogni campione.

\item \textbf{Decodificatore}: trasforma i simboli in sequenze binarie.
 
\end{enumerate}
\paragraph*{Codificatore}

\begin{itemize}
\item Deve essere sincrono con la sorgente
\[
T_s = T_b \log_2 M
\]
\[
R_s = \frac{1}{T_s} = \frac{1}{T_b \log_2 M}
\]


\item La sequenza di simboli generati dal codificatore viene considerata come un processo stazionario. Spesso i simboli trasmessi sono considerati equiprobabili:
\begin{equation*}
    P\{x[n]=\alpha_i\} = \frac{1}{M} \quad \forall i
\end{equation*}
\end{itemize}

\paragraph*{Interpolatore}

In un sistema di comunicazione numerico in banda base, l'interpolatore da solo svolge il compito di modulatore numerico, in quanto effettua la sagomatura, mentre non è prevista nessuna traslazione in frequenza. Il filtro sagomatore è realizzato tramite la generazione dell'impulso \( p(t) \). Infatti, si può pensare a \( P(f) \) come allo spettro del singolo impulso.
\begin{itemize}
    \item \(B_p\): banda dell'impulso \( p(t) \):
    \item \(E_p\): energia dell'impulso \( p(t) \):
\begin{equation*}
    E_p = \int_{-\infty}^{\infty} p(t)^2 \, dt = \int_{-\infty}^{\infty} |P(f)|^2 \, df
\end{equation*}
\end{itemize}





Data l'aleatorietà della sequenza di simboli \( x[k] \), \( s(t) \) deve essere interpretato come la realizzazione di un processo aleatorio \( S(t) \) stazionario.

Il processo \( S(t) \) ha una autocorrelazione \( R_s(\tau) \) ed una densità spettrale di potenza \( S_s(f) \).

Per cui è definita una potenza \( P_s \) ed una banda \( B_T \).


     L'energia media per bit può essere calcolata come:
 \begin{equation*}
    E_b = T_s P_b = \frac{E_s}{\log_2 M} = \frac{E_s}{B_t}
\end{equation*}
\begin{equation*}
    E_s \text{— energia media per simbolo}
\end{equation*}


\paragraph*{Demodulatore numerico}


Il demodulatore numerico produce una sequenza $\hat{x}[k]$ in modo tale da minimizzare la probabilità di errore. La probabilità di errore sul simbolo e sul bit sono:

\begin{align*}
    P_{E_s} &= P\{ \hat{x}[k] \neq x[k] \} \\
    P_{E_b} &= P\{ \hat{b}[n] \neq b[n] \} \quad \text{BEP (bit error probability)}
\end{align*}

\paragraph*{Canale numerico e prestazioni}
\begin{center}
\begin{tikzpicture}
\node (s) at (0,0) {\(x[k]\)};
\node[block] (encoder) at (3,0) {Canale numerico};
\node (r) at (6,0) {\(\hat{x}[k]\)};

\draw[->] (s) -- (encoder);
\draw[->] (encoder) -- (r);
\end{tikzpicture}
\end{center}




\begin{align*}
    x[k] &\in A_s & A_s &= \{\alpha_1, \alpha_2, \ldots, \alpha_M\}
\end{align*}

\begin{equation*}
    P_E(M) \coloneqq P\{\hat{x}[k] \neq x[k]\} = \sum_{i=1}^{M} \sum_{\substack{j=1 \\ j \neq i}}^{M} P\{\hat{x} = \alpha_i, x = \alpha_j\} = \sum_{i=1}^{M} \sum_{\substack{j=1 \\ j \neq i}}^{M} P\{\hat{x} = \alpha_i \ |\ x = \alpha_j\}\cdot P\{x=\alpha_j\}
\end{equation*}

Nel caso di simboli equiprobabili:

\begin{equation*}
    P_E(M) = \frac{1}{M} \sum_{i=1}^{M} \sum_{\substack{j=1 \\ j \neq i}}^{M} P\{\hat{x} = \alpha_i \ | \ x = \alpha_j\}
\end{equation*}

\paragraph*{Formato di modulazione equienergia}

\begin{equation*}
    E_s(i) = \int_{-\infty}^{\infty} s_i(t)^2 dt \quad s_i(t) \text{ è il segnale trasmesso in corrispondenza del simbolo } \alpha_i
\end{equation*}

Il formato di modulazione è equienergetico se
\begin{equation*}
    E_{s}(i) = E_{s} \quad \forall i
\end{equation*}

\paragraph*{Ortogonalità}

Il formato di modulazione è detto ortogonale se
\begin{equation*}
    \int_{-\infty}^{\infty} s_{i}(t) s_{j}(t) \, dt = 0 \quad \forall i \neq j
\end{equation*}

\paragraph*{Efficienza energetica}

Fissata una \( P_{E_{b}} \), l'efficienza energetica è definita come il valore
\begin{equation*}
    \eta_{P} \coloneqq \frac{1}{SNR} , \quad SNR \coloneqq \frac{P_{s}}{P_{n}}
\end{equation*}
che permette di ottenere tale \( BEP \).

Quando tanto maggiore è \( \eta_{P} \), tanto minore deve essere il \( SNR \) che garantisce una data \( BEP \).

\paragraph*{Efficienza spettrale}

L'efficienza spettrale è definita con il rapporto tra il tasso di erogazione binario e la banda di trasmissione
\begin{equation*}
    \eta_{b} \coloneqq \frac{R_{b}}{B_{T}} \quad \text{[bit/s/Hz]}
\end{equation*}

Quindi l'efficienza cresce quando a parità tasso di erogazione la banda utilizzata in trasmissione si riduce.

In termini di \( T_{s} \) e \( M \):
\begin{equation*}
    \eta_{b} = \frac{\log_2 M}{B_{T} T_{s}}
\end{equation*}


%\section*{Pulse-amplitude modulation (PAM)}

La PAM nel caso generico è detta anche \( M \)-PAM o PAM \( M \)-aria dove con \( M \) si indica il numero di simboli presenti nell'alfabeto \( A_s \).

\begin{center}
    \begin{tikzpicture}
        \node (s) at (0,0) {\(x[k]\)};
        \node[block] (encoder) at (3,0) {\(p(t)\)};
        \node (r) at (6,0) {\(s(t)\)};

        \draw[->] (s) -- (encoder);
        \draw[->] (encoder) -- (r);
    \end{tikzpicture}
\end{center}

Proprietà che definiscono una \( M \)-PAM:

\begin{enumerate}
    \item \( s(t) \coloneqq \sum_{k=-\infty}^{+\infty} x[k]\cdot p(t - kT_s) \)
    \item Gli \( M \) valori \( (M \geq 2) \) che costituiscono l'alfabeto \( A_s = \{ \alpha_1, \alpha_2, \ldots, \alpha_M \} \) sono definiti come:
          \[ \alpha_i = 2i - 1 - M, \quad i = 1, 2, \ldots, M \]
\end{enumerate}

Esempio:

\( M=4 \) $\Rightarrow$ \( \alpha_1 = -3, \alpha_2 = -1, \alpha_3 = 1, \alpha_4 = 3 \)

\begin{center}
    \begin{tikzpicture}
        \begin{axis}[
                title={\( (M=4) \)},
                xlabel={},
                ylabel={},
                xmin=-5, xmax=5,
                ymin=-4, ymax=4,
                grid=both,
                grid style={line width=.1pt, draw=gray!10},
                major grid style={line width=.2pt,draw=gray!50},
                minor tick num=5,
                axis lines=middle,
                minor tick style={draw=none},
                xtick={-3, -1, 1, 3},
                ytick=\empty,
            ]

            \addplot+[only marks] coordinates {
                    (-3, 0)
                    (-1, 0)
                    (1, 0)
                    (3, 0)
                };
        \end{axis}
    \end{tikzpicture}
\end{center}
\[
    E_s(i) = \int_{-\infty}^{+\infty} s_i^2(t) \, dt =  \int_{-\infty}^{+\infty} \alpha_i^2 \cdot p^2(t - kT_s) \, dt = \int_{-\infty}^{+\infty} (2i - 1 - M)^2 p^2(t) \, dt = (2i - 1 - M)^2 E_p
\]

Per \( M \) pari si ha \( A_s = \{ \pm 1, \pm 3, \ldots, \pm (M-1) \} \)

Per \( M \) dispari si ha \( A_s = \{ 0, \pm 2, \pm 4, \ldots, \pm (M-1) \} \)


I formati M-PAM di pi\`u largo impiego sono quelli dove $M$ è una potenza di 2.

Esempio:

$4$-PAM

sequenza = $\left\{ \begin{array}{l}
        x[-2] = -1, \\
        x[-1] = 1,  \\
        x[0] = -3,  \\
        x[1] = 1,   \\
        x[2] = 3    \\
    \end{array} \right.$

$p(t) = \text{rect}\left(\frac{t - T_s/2}{T_s}\right)$

\begin{center}
    \begin{tikzpicture}
        % Assi
        \draw[thick,->] (-3,0) -- (4,0) node[anchor=north west] {$t$};
        \draw[thick,->] (0,-4) -- (0,4) node[anchor=south east] {$s(t)$};

        % Grafico a gradini
        \draw[ultra thick, red] (-2,0) -- (-2,-1) -- (-1,-1) -- (-1,1) -- (0,1) -- (0,-3) -- (1,-3) -- (1,1) -- (2,1) -- (2,3) -- (3,3) -- (3,0);

        % Tacche e etichette sull'asse delle x
        \foreach \x in {-2,2,3}
        \draw (\x cm,1pt) -- (\x cm,-3pt)
        node[anchor=north] {$\x T_s$};

        \draw (-1 cm,1pt) -- (-1 cm,-3pt)
        node[anchor=north] {$-T_s$};
        \draw (1 cm,1pt) -- (1 cm,-3pt)
        node[anchor=north] {$T_s$};

        % Linee tratteggiate
        \draw[dashed] (-1,-1) -- (0,-1);
        \draw[dashed] (2,3) -- (0,3);

        % Etichette
        \node at (0,1) [right] {$1$};
        \node at (0,-1) [right] {$-1$};
        \node at (0,3) [left] {$3$};
        \node at (0,-3) [left] {$-3$};
    \end{tikzpicture}
\end{center}

\subsection*{Proprietà derivate della M-PAM}
\begin{enumerate}
    \item Il valor medio di \( s(t) \) è zero per ogni \( t \):
          \begin{equation*}
              \mathbb{E}\left[ s(t) \right] = 0 \quad \forall t
          \end{equation*}
          \begin{equation*}
              \mathbb{E} \left[ \sum_{k=-\infty}^{+\infty} x[k] \ p(t-kT_s) \right] = \sum_{k=-\infty}^{+\infty} \mathbb{E}\left[x[k]\right]\ p(t-kT_s) = 0
          \end{equation*}
          \begin{equation*}
              \mathbb{E}\left[x[k]\right] = \frac{1}{M} \sum_{i=1}^{M} \alpha_i \mathbb{P}\{\alpha_i\} = \frac{1}{M} \sum_{i=1}^{M} (2i - 1 - M)
          \end{equation*}
          \begin{equation*}
              = \frac{2}{M} \sum_{i=1}^{M} i - 1 - M = \frac{2}{M} \frac{M(M+1)}{2} - (M+1) = 0
          \end{equation*}

    \item La densità spettrale di potenza invece è: \( S_s(f) \):
          \begin{equation*}
              S_s(f) = \frac{1}{T_s} S_x(f) \ |P(f)|^2
          \end{equation*}
          \begin{equation*}
              \text{dove } \sigma_x^2 = \mathbb{E}\left[ x\left[k\right]^2 \right] = \frac{(M-1)(M+1)}{3}
          \end{equation*}
\end{enumerate}

\[
    R_s(t,\tau) = \mathbb{E}\{s(t) \ s^*(t-\tau)\}
\]
\[
    = \mathbb{E} \left\{ \sum_{n=-\infty}^{+\infty} x\left[n\right] p(t - nT_s) \sum_{k=-\infty}^{+\infty} x^*\left[k\right] p^*(t - \tau - kT_s) \right\}
\]
\[
    = \sum_{n=-\infty}^{+\infty} \sum_{k=-\infty}^{+\infty} E\{x\left[n\right] x^*\left[k\right]\} \cdot p(t - nT_s) \cdot p^*(t - \tau - kT_s)
\]
\[
    = \sum_{n=-\infty}^{+\infty} \sum_{k=-\infty}^{+\infty} R_x\left[n-k\right] \cdot p(t - nT_s) \cdot p^*(t - \tau - kT_s)
\]

Imponendo \( n-k = m \) abbiamo che \( k = n-m \), quindi:

\[
    = \sum_{m=-\infty}^{+\infty} R_x[m] \sum_{n=-\infty}^{+\infty} p(t - nT_s) \cdot p^*(t - \tau - nT_s + mT_s)
\]

\paragraph*{Autocorrelazione media}

La funzione di autocorrelazione media \( \overline{R}_s(\tau) \) è:
\begin{align*}
    \overline{R}_s(\tau) & = \lim_{T\to\infty} \frac{1}{T} \int_{-\frac{T}{2}}^{\frac{T}{2}} R_s(t, \tau) dt                                                                   \\
    \overline{R}_s(\tau) & = \frac{1}{T_0} \int_{-\frac{T_0}{2}}^{\frac{T_0}{2}} R_s(t,\tau)dt \quad \text{se} \quad R_s(t,\tau) \quad \text{è periodico in} \quad t           \\
    \overline{R}_s(\tau) & = \sum_{m=-\infty}^{\infty} R_x[m] \frac{1}{T_s} \sum_{n=-\infty}^{\infty} \int_{-\frac{T_s}{2}}^{\frac{T_s}{2}} p(t-nT_s)p^*(t-\tau-nT_s+mT_s)dt   \\
                         & = \sum_{m=-\infty}^{\infty} R_x[m] \frac{1}{T_s} \sum_{n=-\infty}^{\infty}\int_{-\frac{T_s}{2}+nT_s}^{\frac{T_s}{2}+nT_s} p(t')p^*(t'-\tau+mT_s)dt' \\
                         & = \sum_{m=-\infty}^{\infty} R_x[m] \frac{1}{T_s} \int_{-\infty}^{\infty} p(t')p^*(t'-\tau+mT_s)dt'                                                  \\
                         & = \sum_{m=-\infty}^{\infty} R_x[m] \frac{1}{T_s} \int_{-\infty}^{\infty} P(f)[P(f)e^{-j2\pi f\tau}e^{j2\pi fmT_s}]^*df                              \\
                         & = \int_{-\infty}^{\infty} P(f)P^*(f)\frac{1}{T_s} \sum_{m=-\infty}^{\infty} R_x[m] e^{-j2\pi fmT_s}e^{j2\pi f\tau}df                                \\
                         & = \frac{1}{T_s} \int_{-\infty}^{\infty} |P(f)|^2 S_x(f)e^{j2\pi f\tau}df
\end{align*}


\[
    \overline{R}_s(\tau) = \frac{1}{T_s} TCF^{-1} \left[ |P(f)|^2 S_x(f) \right]
\]
\[
    \Rightarrow S_s(f) = \frac{1}{T_s} S_x(f) |P(f)|^2
\]

Nel caso in cui:
\begin{enumerate}
    \item $\mathbb{E} \{ x[n] \} = 0$
    \item $R_x[m] = \sigma_x^2 \delta[m]$
\end{enumerate}

Si ha che:
\[
    S_s(f) = \frac{\sigma_x^2}{T_s} |P(f)|^2
\]

In questo caso la $B_T$ coincide con quella del sagomatore $P(f)$.

Calcolo di $\sigma_x^2$:
\[
    \sigma_x^2 = \mathbb{E} \left[ (x - \mu_x)^2 \right] = \int_{-\infty}^{\infty} (x - \mu_x)^2 f_x(x) dx
\]
\[
    = \frac{1}{M} \sum_{i=1}^{M} (2i - 1 - M)^2
\]
\[
    = \frac{1}{M} \left[ 2 \sum_{i=1}^{M} i^2 + (1+M)^2 M - 4(1+M) \sum_{i=1}^{M} i \right]
\]

Sfruttando i seguenti risultati noti:
\[
    \sum_{i=1}^{n} i^2 = \frac{n(n+1)(2n+1)}{6}, \quad \sum_{i=1}^{n} i = \frac{n(n+1)}{2}
\]

Si ottiene:
\[
    \sigma_x^2 = \frac{M^2 - 1}{3}
\]

\[
    P_s = \frac{\sigma_x^2 E_p}{T_s} = \frac{M^2 - 1}{3} \frac{E_p}{T_s}
\]

\paragraph*{Efficienza Spettrale di una M-PAM}
\[
    \beta = \frac{R_b}{B_T} = \frac{\log_2 M}{T_s B_P}
\]
essendo \( B_T = B_P \),

L'efficienza spettrale aumenta con l'aumentare del numero di livelli. Sfortunatamente, come verrà dimostrato più avanti, l'efficienza in potenza diminuisce all'aumentare di \( M \).

\subsection*{PAM Binaria o BPSK (Binary Phase Shift Keying)}


\[
    s(t) = \sum_{k=-\infty}^{\infty} x[n] p(t - kT_s)
\]
con \( x[n] \in A_s = \{\pm 1\} \) (M = 2)

\[
    \Rightarrow T_B = T_s \quad (\log_2 2 = 1)
\]

Esempio con \( p(t) = \text{rect}\left(\frac{t-T_s/2}{T_s}\right) \)

% Drawing the binary PAM signal
\begin{center}

    \begin{tikzpicture}
        \begin{axis}[
                xlabel = \( t \),
                ylabel = \( s(t) \),
                xtick={-3, -2, ..., 3},
                ytick={-1, 1},
                yticklabels={-1, 1},
                xticklabels={\(-3T_s\), \(-2T_s\), \(-T_s\), 0, \(T_s\), \(2T_s\), \(3T_s\)},
                ymin=-2, ymax=2,
                xmin=-4, xmax=4,
                axis lines=middle,
                enlargelimits=true,
                clip=false,
                grid=both
            ]
            \addplot+[const plot, no marks, thick] coordinates {(-3, 0) (-3,-1) (-2,-1) (-2,1) (0,-1) (1,1) (2,-1) (3,0)};
        \end{axis}
    \end{tikzpicture}
\end{center}
\[
    \text{Segnale PAM binario (o BPSK)}
\]

Il PAM Binario è un formato equiprobabile

% Energy calculation
\[
    E_{s_1} = \int_{-\infty}^{\infty} s_1^2(t) dt = \int_{-\infty}^{\infty} \left( +1 \right)^2 p^2(t) dt =  \int_{-\infty}^{\infty} \left( -1 \right)^2 p^2(t) dt = \int_{-\infty}^{\infty}  p^2(t) dt  = E_{s_2}
\]

\[
    E_{s_1} = E_{s_2} = \int_{-\infty}^{\infty} p^2(t) dt = E_p
\]

% Drawing the constellation diagram
\begin{center}

    \begin{tikzpicture}
        \begin{axis}[
                title={Simboli antipodali},
                xlabel={Re},
                ylabel={Im},
                xmin=-2, xmax=2,
                ymin=-2, ymax=2,
                grid=both,
                axis lines=middle,
                enlargelimits=true
            ]
            \addplot+[only marks, mark size=3, very thick] coordinates {(1, 0) (-1, 0)};
            \node at (axis cs:1,0) [anchor=west] {};
            \node at (axis cs:-1,0) [anchor=east] {};
        \end{axis}
    \end{tikzpicture}
\end{center}



\begin{enumerate}
    \item $E\left[s(t)\right] = \dfrac{1}{2} \sum_{k=-\infty}^{+\infty} E[x_k] p(t-kT_s)
              = \sum_{k=-\infty}^{+\infty} \left(\dfrac{1}{2} (+1) + \dfrac{1}{2} (-1)\right) p(t-kT_s) = 0 \quad \text{se i simboli sono equiprobabili}$

    \item $S_s(t) = \dfrac{1}{T_b} |P(f)|^2 \quad \text{Densità spettrale di potenza}$

    \item $P_s = \dfrac{E_p}{T_b} \quad \text{Potenza media}$

    \item $B_T = B_P \quad \text{Banda}$

    \item $M_P = \dfrac{1}{T_bB_p}$
\end{enumerate}

\paragraph{Segnalazione ON-OFF}

È un tipo di PAM binaria con simboli appartenenti ad $A_s = \{0, 1\}$

e impulsi rettangolari  $p(t) = \text{rect}\left(\dfrac{t-T_b/2}{T_b}\right)$

\[
    s(t) = \sum_{k=-\infty}^{+\infty} x\left[k\right] \text{rect}\left(\dfrac{t-T_b/2-kT_b}{T_b}\right)
\]

\[
    \begin{cases}
        S_1(t) = 0                                            & \Rightarrow E_{S_1} = 0   \\
        S_2(t) = \text{rect}\left(\dfrac{t-T_b/2}{T_b}\right) & \Rightarrow E_{S_2} = T_b
    \end{cases}
\]


\begin{itemize}
    \item $
              E[S(t)] = \sum_{k=-\infty}^{+\infty} E[x[k]] p(t-kT_b) = \dfrac{1}{2}$
    \item $
              E_s = \dfrac{1}{2}E_{S_1} + \dfrac{1}{2}E_{S_2} = \dfrac{T_b}{2}
          $
    \item
          $
              P_s = \dfrac{E_s}{T_b} = \dfrac{1}{2}$
    \item $R_x[m] = C_x[m] + {\eta}_x^2 = \frac{1}{4} {\delta}[m] + \frac{1}{4}$

    \item $S_x(f) = TFS\left[ R_x[m] \right] = \frac{1}{4} + \frac{1}{4T_b} \sum_{m=-\infty}^{\infty} \delta\left(f - \frac{m}{T_b}\right)$
    \item $S_s(f) = \frac{1}{T_b} S_x(f)|P(f)|^2 = \frac{1}{T_b} \left( \frac{1}{4} + \frac{1}{4T_b} \delta(f) \right) T_b^2 \operatorname{sinc}^2(T_b f) = \frac{T_b}{4} \left( 1 + \frac{\delta(f)}{T_b} \right)\operatorname{sinc}^2(T_b f)$

    \item ${\eta}_b = \frac{\log_2{2}}{T_b B_p} = 2 \quad \text{e} \quad B_p = \frac{1}{2T_b} \text{ per la sinc}$
\end{itemize}



% Considerazioni
\subsection*{Considerazioni}
\begin{enumerate}
    \item L'efficienza spettrale della modulazione on-off è unitaria come nel caso della 2-PAM.
    \item Essendo la modulazione on-off di tipo unipolare (solo a valori positivi) questa può essere utilizzata su canali di comunicazione che, per loro natura, non possono sovrapporre segnali bipolari.
    \item La densità spettrale della on-off presenta un impulso di dirac nell'origine delle frequenze (alla continua) per cui il canale di trasmissione deve avere una risposta in frequenza che non sia nulla nell'origine.
\end{enumerate}

%\section*{Prestazioni dei Sistemi di Comunicazione Numerici in Banda Base}
Nel valutare le prestazioni dei sistemi di comunicazioni numerici in banda base considereremo due fenomeni peggiorativi:
\begin{enumerate}
    \item Interferenza intersimbolo (ISI)
    \item Presenza di rumore
\end{enumerate}

Per il momento ignoriamo il rumore e ci concentriamo sul primo problema.

\subsection*{Interferenza intersimbolica (ISI)}
Il primo fenomeno è causato dalla non perfetta risposta in frequenza del canale di trasmissione e quindi dalle distorsioni lineari introdotte da questo.

\noindent
\begin{minipage}{.5\textwidth}
    \centering
    \begin{tikzpicture}[scale=0.6]
        % Canale ideale
        \draw[->] (-1,0) -- (4,0) node[right] {$f$}; % asse x
        \draw[->] (0,-1) -- (0,4) node[above] {$S_s(f)$}; % asse y
        \draw[thick, red] (-2.5,0) -- (-2.5,2.8) -- (2.5,2.8) -- (2.5,0); % rettangolo

        \draw[->] (-4,0) -- (4,0) node[right] {$f$}; % asse x
        \draw[->] (0,-1) -- (0,4) node[above] {$S_s(f)$}; % asse y
        \draw[thick, black] (-2,0) arc (180:0:2);

        \node[red, above] at (1.5,2.8) {$C(f)$};

        % Canale non ideale
        \begin{scope}[shift={(6,0)}] % sposta tutto a destra

        \end{scope}
    \end{tikzpicture}
    \\
    \textbf{Assenza di ISI:}
    \[
        y[k] = f(x[k])
    \]
    \\
\end{minipage}%
\begin{minipage}{.5\textwidth}
    \centering

    \begin{tikzpicture}[scale=0.6]
        % Canale ideale
        \draw[->] (-1,0) -- (4,0) node[right] {$f$}; % asse x
        \draw[->] (0,-1) -- (0,4) node[above] {$S_s(f)$}; % asse y
        \draw[thick, red] (-2.5,0) -- (-2.5,2.8) -- (0, 2.2) -- (2.5,2.8) -- (2.5,0); % rettangolo

        \draw[->] (-4,0) -- (4,0) node[right] {$f$}; % asse x
        \draw[->] (0,-1) -- (0,4) node[above] {$S_s(f)$}; % asse y
        \draw[thick, black] (-2,0) arc (180:0:2);

        \node[red, above] at (1.5,2.8) {$C(f)$};

        \begin{scope}[shift={(6,0)}] % sposta tutto a destra
        \end{scope}
    \end{tikzpicture}
    \\
    \textbf{Presenza di ISI:}
    \[
        y[k] = f(\dots, x[k-1], x[k], x[k+1], \dots)
    \]
    \\
\end{minipage}


Il risultato è che il campione estratto al ricevitore dal segnale ricevuto al k-esimo istante non dipende solo dal k-esimo simbolo.

\begin{center}
    \begin{tikzpicture}[
            block/.style={rectangle, draw, minimum height=1cm, minimum width=2.5cm},
            node distance=1cm and 2cm,
            auto
        ]

        \node[block] (filter) {$h_R(t)$};
        \node[left=of filter] (channel) {};
        \node[right=of filter] (sampler) {};

        \draw[->] (channel) -- (filter) node[midway,above] {$r(t)$};

        \draw ([xshift=0]filter.east) -- ([xshift=1cm]filter.east) node[midway,above] {$y(t)$};
        \draw ([xshift=1cm]filter.east) -- ([xshift=1.5cm,yshift=0.5cm]filter.east) node[midway,below, yshift=-0.2cm] {$T_s$};

        \draw[->] ([xshift=1.5cm,yshift=0cm]filter.east) -- ++(1.5cm,0) node[midway,above] {$y[n]$};


    \end{tikzpicture}
\end{center}





Per ridurre gli effetti dell'ISI si devono considerare:
\begin{enumerate}
    \item il sagomatore in trasmissione \( p(t) \)
    \item la risposta impulsiva del canale \( c(t) \)
    \item il filtro in ricezione
\end{enumerate}

\begin{center}
    \begin{tikzpicture}[
            block/.style={rectangle, draw, minimum height=1cm, minimum width=2.5cm},
            node distance=1cm and 2cm,
            auto
        ]
        \node[block] (interpolatore) {$p(t)$};
        \node[left=of interpolatore] (tmp) {};
        \node[block, right= of interpolatore] (chan) {$c(t)$};
        \node[block, right= of chan] (filter) {$h_R(t)$};
        \node[right=of filter] (sampler) {};

        \draw[->] (channel) -- (interpolatore) node[midway,above] {$x[k]$};
        \draw[->] (interpolatore) -- (chan) node[midway,above] {$s(t)$};
        \draw[->] (chan) -- (filter) node[midway,above] {$r(t)$};

        \draw ([xshift=0]filter.east) -- ([xshift=1cm]filter.east) node[midway,above] {$y[k]$};
        \draw ([xshift=1cm]filter.east) -- ([xshift=1.5cm,yshift=0.5cm]filter.east) node[midway,below, yshift=-0.2cm] {$T_s$};

        \draw[->] ([xshift=1.5cm,yshift=0cm]filter.east) -- ++(1.5cm,0) node[midway,above] {$y[n]$};

        \draw[dashed, red, thick] ([xshift=-0.5cm,yshift=0.5cm]interpolatore.north west) rectangle ([xshift=0.5cm,yshift=-0.5cm]filter.south east);

        \node[align=center, red, above right= -1cm and -6cm of filter.south east] (channel-label) {$h(t)$};


    \end{tikzpicture}
\end{center}
\[
    h(t)=p(t) \ast c(t) \ast h_R(t)
\]

\paragraph*{Dimostrazione}
\[ Y(f) = R(f) \cdot H_R(f) = S(f) \cdot C(f) \cdot H_R(f) = \overline{X}(f) \cdot P(f) \cdot C(f) \cdot H_R(f) \]
\[ = \overline{X}(f) \cdot H(f) \quad  \text{dove} \quad H(f) = P(f) \cdot C(f) \cdot H_R(f) \]

\[ y(t) = \sum_{n=-\infty}^{+\infty} x[n] \cdot h(t - nT_s) \]

\[ y[k] = y(kT_s) = \sum_{n=-\infty}^{+\infty} x[n] \cdot h((k - n) \cdot T_s)  \]
\[ = x[k] \cdot h(0) + \sum_{\substack{n=-\infty \\ n \neq k}}^{+\infty} x[n]\cdot h((k - n) \cdot T_s) \]

Il secondo termine rappresenta la componente ISI.




\subsection*{Canale con ISI}

Un canale con banda \( B_c \) in generale introduce ISI. Ci sono due aspetti di cui ci occuperemo:

\begin{enumerate}
    \item Determinazione del \( T_s \) minimo che può essere adottato al fine di ottenere una sequenza campionata priva di ISI.
    \item Determinare le condizioni sotto le quali è possibile trasmettere un segnale M-PAM attraverso un canale non ideale in modo che non vi sia ISI nella sequenza campionata.
\end{enumerate}

Nel risolvere i due problemi riterremo \( c(t) \) fissata, e \( p(t) \) e \( h_R(t) \) variabili, in quanto determinabili dal progettista.

Un approccio non perseguibileconsiste nel trasmettere impulsi di durata finita e quindi con banda illimitata. Questo è in contrasto con la limitatezza messa a disposizione dal canale di trasmissione \( (B_c < \infty) \).

\(\Rightarrow\) Gli impulsi \( p(t) \) devono avere durata infinita.

\subsection*{Primo criterio di Nyquist per la trasmissione priva di ISI}

\[ h(kT_s) =
    \begin{cases}
        1, & \text{se } k=0      \\
        0, & \text{se } k \neq 0
    \end{cases}
    \quad \text{(Dominio del tempo)}
\]

\[ \sum_{k=-\infty}^{+\infty} H\left(f-\frac{k}{T_s}\right) = T_s \quad \forall f \quad \text{(Dominio della frequenza)} \]




\paragraph*{Dimostrazione}
Il primo criterio di Nyquist nel dominio del tempo garantisce l'assenza di ISI in quanto
\[ y[k] = x[k] \cdot h(0) + \sum_{\substack{n=-\infty \\ n \neq k}}^{+\infty} x[n] \cdot h((n-k)T_s) = x[k] \cdot h(0) \]
dove il secondo termine è nullo e non vi è ISI se \( h[n] = \delta[n] \).

La relazione in frequenza si ottiene come trasformazione
\[ h[k] = \delta[k] \quad \Longleftrightarrow \quad \overline{H}(f) = 1 \quad \forall f \]
\[ \overline{H}(f) = \frac{1}{T_s} \sum_{k=-\infty}^{+\infty} H\left(f - \frac{k}{T_s}\right) = 1 \quad \forall f \]
\[ \sum_{k=-\infty}^{+\infty} H\left(f - \frac{k}{T_s}\right) = T_s \quad \forall f \]

\subsection*{Trasmissione priva di ISI}
Supponiamo sia assegnato un canale a banda rigorosamente limitata con banda \( B_c \).
\[ C(f) = 0 \quad \text{per} \quad |f| > B_c \]
e supponiamo che \( B_T = B_c \), ovvero che il segnale trasmesso occupa tutta la banda messa a disposizione dal canale.
Allora si verificano le seguenti:
\begin{enumerate}
    \item Non è possibile in alcun modo eliminare l'ISI quando \( T_s < \frac{1}{2B_c} \).


          \paragraph*{Dimostrazione:}

          Quando \( T_s < \frac{1}{2B_c} \)

          \begin{tikzpicture}[scale=0.5]
              \draw[->] (-15,0) -- (15,0) node[right] {\( f \)};
              \draw[->] (0,-1) -- (0,7) node[above] {\( \overline{H}(f) \)};

              % Triangles
              \draw (-11,0) -- (-8,3) -- (-5,0);
              \draw[dashed, red] (-5,-1) -- (-5,4);
              \draw[dashed, red] (-3,-1) -- (-3,4);
              \draw (-3,0) -- (0,3) -- (3,0);
              \draw[dashed, red] (3,-1) -- (3,4);
              \draw[dashed, red] (5,-1) -- (5,4);
              \draw (5,0) -- (8,3) -- (11,0);

              \node at (-12,1.5) {\( \cdots \)};
              \node at (12,1.5) {\( \cdots \)};
          \end{tikzpicture}

          Esistono degli intervalli di frequenza dove \( \overline{H}(f) = 0 \) per cui non può mai accadere che \( \overline{H}(f) = 1 \) \( \forall f \)

          \bigskip

    \item Il più piccolo valore di \( T_s \) che permette di eliminare l'\( ISI \) è

          \( T_s^{(min)} = \frac{1}{2B_c} \)

          \bigskip

          \( f_s^{(max)}\) = \( \frac{1}{T_s^{(min)}} = 2B_c = f_N \) \quad (frequenza di Nyquist)

          \bigskip

          \begin{tikzpicture}[scale=0.5]
              \draw[->] (-15,0) -- (15,0) node[right] {\( f \)};
              \draw[->] (0,-1) -- (0,7) node[above] {\( \overline{H}(f) \)};

              % Triangles
              \draw (-9,0) -- (-6,3) -- (-3,0);
              \draw (-3,0) -- (0,3) -- (3,0);
              \draw (3,0) -- (6,3) -- (9,0);

              \node at (-10,1.5) {\( \cdots \)};
              \node at (10,1.5) {\( \cdots \)};
              \
          \end{tikzpicture}

          Non esistono intervalli di frequenza dove \( \overline{H}(f) = 0 \)

          \bigskip

    \item Nel caso valga la condizione \( T_s = \frac{1}{2B_c} \), allora l'unica funzione di trasferimento che permette di eliminare completamente l'\( ISI \) è

          \[ H(f) = \frac{1}{2B_c} \text{rect}\left(\frac{f}{2B_c}\right) \quad \Leftrightarrow \quad h(t) = \text{sinc}(2B_c t) \]





          \paragraph*{Dimostrazione:}

          \begin{center}

              \begin{tikzpicture}[scale=1]
                  \begin{axis}[
                          axis lines=middle,
                          xlabel={$f$},
                          ylabel={$H(f)$},
                          xtick={-4, -2, 2, 4},
                          xticklabels={$-2B_c$, $-B_c$, $B_c$, $2B_c$},
                          ytick={100},
                          yticklabels={},
                          ymin=-0.2, ymax=2,
                          xmin=-8, xmax=8,
                          %every axis x label/.style={at={(ticklabel* cs:1.05)}, anchor=west,},
                          %every axis y label/.style={at={(ticklabel* cs:1.05)}, anchor=south,},
                          xmajorgrids=false,
                          ymajorgrids=false,
                          clip=false
                      ]

                      % Draw the rectangle
                      \draw [thick] (axis cs:-2,0) rectangle (axis cs:2,0.4);

                      % Add the summation formula
                      \node [red]at (axis cs:7,0.6) {$\sum_{k=-\infty}^{\infty} H(f - \frac{k}{T_s})$};

                      % Draw the dashed lines
                      \draw [dashed, red] (axis cs:-6,0.4) -- (axis cs:-6,0);

                      \draw [dashed, red] (axis cs:-8,0.4) -- (axis cs:-2,0.4);
                      \draw [dashed, red] (axis cs:2,0.4) -- (axis cs:8,0.4);

                      \draw [dashed, red] (axis cs:6,0.4) -- (axis cs:6,0);

                  \end{axis}
                  \
              \end{tikzpicture}
          \end{center}

          Si nota anche che la funzione $\text{sinc}(2Bt)$ si annulla quando $t = \frac{k}{2B}$ con $k \neq 0$ per cui
          \[
              h[kT_s] = \text{sinc} \left(2B_c\cdot\frac{k}{2B_c}\right) = \text{sinc}(k) = \left\{
              \begin{array}{ll}
                  1 & \text{se } k=0    \\
                  0 & \text{se } k\neq0
              \end{array}
              \right.
          \]
\end{enumerate}


\textbf{Limiti di applicabilità della funzione di trasferimento rettangolare:}

\begin{enumerate}
    \item Realizzabilità di una funzione di trasferimento rettangolare: risposte in frequenza ideali come quella rettangolare non sono fisicamente realizzabili (Criterio di Paley-Wiener).
    \item Piccoli errori di campionamento provocano un ISI molto grande poiché la funzione $\text{sinc}(2B_ct)$ decresce molto lentamente.
\end{enumerate}

Un errore è nel campionatore induce un ISI grande in quanto si sommano molte contributi!




\begin{figure}[ht]
    \centering
    \begin{center}
        \begin{tikzpicture}
            \begin{axis}[
                    axis lines=middle,
                    xlabel={$t$},
                    ylabel={$\text{sinc}(2B_c t)$},
                    xtick={-3, -2, -1, 0, 1, 2, 3},
                    xticklabels={$-3T_s$, $-2T_s$, $-T_s$, $0$, $T_s$, $2T_s$, $3T_s$},
                    ytick={1},
                    ymin=-0.5, ymax=1.5,
                    xmin=-4, xmax=4,
                    every axis x label/.style={at={(ticklabel* cs:1.05)}, anchor=west,},
                    every axis y label/.style={at={(ticklabel* cs:1.05)}, anchor=south,},
                    xmajorgrids=true,
                    ymajorgrids=true,
                    grid style=dashed,
                    clip=false,
                    no markers,
                    samples=1000,
                    domain=-3.5:3.5
                ]


                \draw [thick, red] (axis cs:-2.75,0) -- (axis cs:-2.75,0.082);
                \draw [thick, red] (axis cs:-1.75,0) -- (axis cs:-1.75,-0.13);
                \draw [thick, red] (axis cs:-0.75,0) -- (axis cs:-0.75,0.30);

                \draw [thick, red] (axis cs:0.15,0) -- (axis cs:0.15,0.965);

                \draw [thick, red] (axis cs:1.15,0) -- (axis cs:1.15,-0.125);
                \draw [thick, red] (axis cs:2.15,0) -- (axis cs:2.15,0.07);
                \draw [thick, red] (axis cs:3.15,0) -- (axis cs:3.15,-0.045);


                % Define sinc function
                \addplot+[thick, black, smooth, unbounded coords=jump] {sin(deg(pi*x))/(pi*x)};
                \addplot+[thick, black, smooth] coordinates {(0, 1)};

                % Add the red vertical line at t=0

            \end{axis}
        \end{tikzpicture}
    \end{center}
    \caption*{Un errore $\epsilon$ nel campionatore induce un ISI grande in quanto si sommano molti contributi. In rosso l'errore $\epsilon$ del compionatore.}
    %\label{fig:my_label} % Optional, for referencing the figure
\end{figure}



Rilassando la condizione $T_s > \frac{1}{2B_c}$, ovvero ammettendo

\[ T_s > \frac{1}{2B_c} \]

si ottiene il seguente effetto:

\begin{center}

    \begin{tikzpicture}[scale=1]
        \begin{axis}[
                axis lines=middle,
                xlabel={$f$},
                ylabel={$H(f)$},
                xtick={-6, -4, -2, 2, 4, 6},
                xticklabels={$-3B_c$, $-2B_c$, $-B_c$, $B_c$, $2B_c$, $3B_c$},
                ytick={100},
                yticklabels={},
                ymin=-0.2, ymax=2,
                xmin=-8, xmax=8,
                %every axis x label/.style={at={(ticklabel* cs:1.05)}, anchor=west,},
                %every axis y label/.style={at={(ticklabel* cs:1.05)}, anchor=south,},
                xmajorgrids=false,
                ymajorgrids=false,
                clip=false
            ]

            \node[red] at (-7,0.25) {\( \cdots \)};

            \draw[red] (-6,0) -- (-4,0.5) -- (-2,0);
            \draw[red] (-4,0) -- (-2,0.5) -- (0,0);

            \draw[red] (0,0) -- (2,0.5) -- (4,0);
            \draw[red] (2,0) -- (4,0.5) -- (6,0);

            \node[red] at (7,0.25) {\( \cdots \)};

            \draw[blue] (-6,0.5) -- (6,0.5);

            \draw (-2,0) -- (0,0.5) -- (2,0);


        \end{axis}
    \end{tikzpicture}
\end{center}

La sovrapposizione permette di definire una classe di infinite funzioni di trasferimento che soddisfano il primo criterio di Nyquist.

In questo caso però $B_c > \frac{1}{2T_s}$, per cui al punto di $T_s$ c'è bisogno di una banda disponibile nel canale che è maggiore di quella che occorre con la funzione di trasferimento rettangolare.

\subsection*{Filtro a coseno rialzato}

% Define the piecewise function
\[ H_{rc}(f) =
    \begin{cases}
        T_s                                                                                                          & \text{if } 0 \leq |f| \leq \frac{1-\alpha}{2T_s}                  \\
        \frac{T_s}{2} \left[ 1 - \sin\left(\frac{\pi T_s}{\alpha} \left( |f| - \frac{1}{2T_s} \right)\right) \right] & \text{if } \frac{1-\alpha}{2T_s} < |f| \leq \frac{1+\alpha}{2T_s} \\
        0                                                                                                            & \text{if } |f| > \frac{1+\alpha}{2T_s}
    \end{cases}
\]

con $0 < \alpha < 1$.

\begin{center}

    \definecolor{myblue}{RGB}{30,144,255}
    \definecolor{myred}{RGB}{178,34,34}
    \begin{tikzpicture}
        \begin{axis}[
                axis lines=middle,
                xlabel={$f$},
                ylabel={$H_{RC}(f)$},
                xtick={-0.5, 0.5},
                xticklabels={$\frac{-1}{2T_s}$, $\frac{1}{2T_s}$},
                ytick={0.5},
                yticklabels={$T_s/2$},
                ymin=0, ymax=1.5,
                xmin=-1, xmax=1,
                every axis x label/.style={at={(ticklabel* cs:1.05)}, anchor=west,},
                every axis y label/.style={at={(ticklabel* cs:1.05)}, anchor=south,},
                xmajorgrids=false,
                ymajorgrids=false,
                clip=false,
                no markers,
            ]

            % Draw the ideal filter response (black box)
            \draw [thick] (axis cs:-0.5,0) -- (axis cs:-0.5,1) -- (axis cs:0.5,1) -- (axis cs:0.5,0);

            % Draw the realistic filter response for alpha = 0.5 (blue line)
            % \addplot [myblue, thick, smooth, domain=-1:1] {0.5+0.5*cos(deg(pi*x))};
            \addplot [myblue, thick, smooth, domain=-0.75:0.-0.25] {0.5 * (1 + cos(deg(pi*(abs(x)-0.25)/0.5)))};
            \addplot [myblue, thick, smooth, domain=0.25:0.75] {0.5 * (1 + cos(deg(pi*(abs(x)-0.25)/0.5)))};


            % Draw the realistic filter response for alpha = 1 (red dashed line)
            \addplot [myred, thick, dashed, smooth, domain=-1:1] {0.5-0.5*sin(deg(pi*(abs(x)-0.5))};

            % Add annotations for alpha values
            \node[myblue] at (axis cs:0.75,0.8) {$\alpha=0.5$};
            \node at (axis cs:-0.75,0.8) {$\alpha=0$};
            \node[myred] at (axis cs:0.9,0.2) {$\alpha=1$};

            % Add black dot at intersection
            \node[circle,fill,inner sep=1.5pt] at (axis cs:0,0.5) {};

        \end{axis}
    \end{tikzpicture}
\end{center}
\subsection*{Propriet\`a}
\begin{enumerate}
    \item Quando \( \alpha = 0 \) il coseno rialzato coincide con la funzione di trasferimento rettangolare
    \item La banda \( B_H \) \`e direttamente ottenibile da \( B_H = \frac{1+\alpha}{2T_S} \)
\end{enumerate}

La \( h_{RC}(t) \) \`e calcolabile in forma chiusa:

\[ h_{RC}(t) = \sin\left(\frac{t}{T_S}\right) \frac{\cos\left(\frac{\alpha \pi t}{T_S}\right)}{\left(1- \frac{2\alpha t}{T_S}\right)^2}  \]

\[ h_{RC}(kT_S) = \delta[k] \]

\begin{itemize}
    \item Soddisfa il criterio di Nyquist nel tempo, per cui garantisce l'assenza di ISI
    \item Decresce per \( t \rightarrow \infty \) come \( \frac{1}{|t|^3} \) per \( \alpha > 0 \) quindi molto pi\`u velocemente del caso \( \alpha = 0 \) (rettangolare)
\end{itemize}




\subsection*{Eccesso di banda e efficienza spettrale dei sistemi \( M-PAM \) con coseno rialzato}

Dato:
\[ p(t) \otimes c(t) \otimes h_R(t) = h_{RC}(t) \]
dove \( \otimes \) indica la convoluzione, $p(t)$ il sagomatore in trasmissione, $c(t)$ la risposta impulsiva del canale , $h_R(t)$ il filtro in ricezione e \( h_{RC}(t) \) la risposta impulsiva del filtro a coseno rialzato, l'efficienza spettrale del canale di comunicazione numerico è:
\[ \eta_{B} = \frac{\log_2M}{B_T T_s} = \frac{\log_2M}{B_H T_s} = \frac{\log_2M}{ T_s} \frac{2T_s}{1+\alpha} = \frac{2\log_2M}{1+\alpha}\]

Considerazioni:
\begin{itemize}
    \item L'efficienza spettrale, a parità di \( M \), decresce al crescere del coefficiente di roll-off (\( \alpha \)).
    \item La robustezza del sistema di comunicazione numerico all'ISI aumenta al crescere di \( \alpha \).
\end{itemize}

C'è quindi un trade-off tra robustezza all'ISI e efficenza spettrale, e i valori ottimali si trovano in corrispondenza di \( \alpha \simeq 0.4 \) .

\[ \text{Eccesso di banda richiesto dall'adozione del coseno rialzato:} \]
\[ \Delta B_{H} = B_{H} - \frac{1}{2T_s} = \frac{\alpha}{2T_s} \]



\subsection*{Prestazioni di un sistema di comunicazione numerico in banda base in presenza di rumore}

\paragraph*{Capacità di canale}

La capacità \( C \) di un canale di comunicazione è definita come il massimo valore che può assumere il tasso binario di segnalazione \( R_b = \frac{1}{T_b} \) al variare di tutte le possibili coppie modulatore/demodulatore, sotto il vincolo che la probabilità di errore sia esattamente nulla.


\[
    \begin{cases}
        C \coloneqq \max_{\{R_b\}} \\
        P_E(b) = P\{\hat{b}[n] \neq b[n]\} = 0
    \end{cases}
\]

La capacità di canale \( C \) si misura in bit/s ed è un numero non-negativo. Ovviamente, più è grande la capacità del canale e migliori sono le sue prestazioni.

\paragraph*{Capacità di canale con rumore gaussiano bianco additivo (AWGN)}

\begin{center}
    \begin{tikzpicture}
        \node (s) at (0,0) {\(s(t)\)};
        \node[draw, circle] (plus) at (2,0) {\(+\)};
        \node (n) at (2,-1.5) {\(n(t)\)};
        \node (r) at (4,0) {\(r(t)\)};

        \draw[->] (s) -- (plus);
        \draw[->] (n) -- (plus);
        \draw[->] (plus) -- (r);
        \node[below of=n, node distance=1cm] {\(n(t)\) Gaussiano Bianco};
    \end{tikzpicture}
\end{center}

In questo caso la capacità di canale può essere espressa in forma chiusa ed è in dipendenza dei parametri caratteristici del segnale trasmesso e del rumore.

\[
    C = B_T \log_2 \left( 1 + \frac{P_s}{N_0 B_T} \right) \quad \text{(Shannon)}
\]

Dove:
\begin{itemize}
    \item \( B_T \) = banda del segnale \( s(t) \)
    \item \( P_s \) = potenza media di \( s(t) \)
    \item \( \frac{N_0}{2} \) = DSP del rumore \( n(t) \) (costante essendo bianco)
\end{itemize}

Considerazioni:

\begin{enumerate}
    \item Fissato \( B_T \)
          \begin{align*}
              \lim_{\frac{P_s}{N_0} \to \infty} C & = 0       \\
              \lim_{\frac{P_s}{N_0} \to \infty} C & = +\infty
          \end{align*}

    \item Fissato \( \frac{P_s}{N_0} \)
          \begin{align*}
              \lim_{B_T \to \infty} C & = 0                              \\
              \lim_{B_T \to \infty} C & = \log_2 e \cdot \frac{P_s}{N_0}
          \end{align*}

    \item Riscrivendo la formula di Shannon utilizzando \( P_s = E_b R_b \)
          \[
              \frac{C}{B_T} = \log_2 \left( 1 + \frac{E_b}{N_0} \cdot \frac{R_b}{B_T} \right)
          \]
          Dove:
          \begin{itemize}
              \item \( E_b \) = energia per bit
              \item \( R_b \leq C \) (data la definizione di \( C \) come valore massimo di \( R_b \))
          \end{itemize}
\end{enumerate}

\paragraph*{Sistema di comunicazione numerico ideale}
Un sistema di comunicazione numerico è detto ideale se soddisfa le seguenti condizioni:
\begin{enumerate}
    \item \( R_b = C \)
    \item \( P_E(b) = 0 \)
\end{enumerate}

In queste condizioni è possibile mettere in relazione l'efficienza spettrale con il rapporto \( \frac{E_b}{N_0} \) (legato all
efficienza in potenza)

\[
    \eta_B = \log_2 \left( 1 + \frac{E_b}{N_0} \eta_B  \right) \quad \text{soggetto a} \quad  \eta_B = \frac{R_b}{B_T} = \frac{C}{B_T}
\]

\[
    \Rightarrow \frac{E_b}{N_0} = \frac{2^{\eta_B} - 1}{\eta_B} \quad \text{con} \quad \eta_B \geq 0
\]


\paragraph*{Considerazioni}

\begin{enumerate}
    \item
          $\lim_{\eta_B \to +\infty} \frac{2^{\eta_B } - 1}{\eta_B } = +\infty$


    \item $\lim_{\eta_B \to +\infty} \frac{2^{\eta_B } - 1}{\eta_B } = \ln{2} \quad (-1.6 \, \text{dB})$

\end{enumerate}


\subsection*{Ricezione ottima in presenza di rumore bianco}

Per il momento consideriamo solo gli effetti relativi al rumore.

\begin{center}
    \begin{tikzpicture}[
            block/.style={rectangle, draw, minimum height=1cm, minimum width=2cm},
            node distance=3cm and 3cm,
            auto
        ]

        \node[block] (filter) {$h_R(t)$};
        \node[left=of filter] (channel) {};
        \node[right=of filter] (sampler) {};
        \draw[->] ([xshift=-3cm]filter.west) -- (filter.west) node[midway,above] {$r(t)=s(t)+n(t) \quad$};


        \draw ([xshift=0]filter.east) -- ([xshift=3.5cm]filter.east) node[midway,above] {$y(t)=s_u(t)+n_u(t)$};
        \draw ([xshift=3.5cm]filter.east) -- ([xshift=4cm,yshift=0.5cm]filter.east) node[midway,below, yshift=-0.2cm] {$T_s$};

        \draw[->] ([xshift=4cm,yshift=0cm]filter.east) -- ++(1.5cm,0) node[midway,above] {$y[k]$};
    \end{tikzpicture}
\end{center}
Dove \( s(t) \) è un segnale di forma nota e \( n(t) \) è un rumore additivo bianco

\[
    s_u(t) = s(t) \ast h_R(t), \quad n_u(t) = n(t) \ast h_R(t)
\]

\[
    y(T_s) = s_u(T_s) + n_u(T_s)
\]

Si definisce il rapporto segnale-rumore in uscita al filtro \( h_R(t) \) all'istante \( t = T_s \) come

\[
    SNR \coloneqq \frac{s_u^2(T_s)}{E[n_u^2(T_s)]}
\]

Si definisce ricevitore ottimo il filtro \( h_R(t) \) che massimizza l'SNR in uscita al filtro.

Nel caso di rumore bianco in ingresso il filtro ottimo prende il nome di \textbf{filtro adattato}

Problema:
\begin{enumerate}
    \item Derivare il filtro \( h_R(t) \) che massimizza l'SNR all'uscita.
    \item Determinare il valore massimo dell'SNR all'uscita.
\end{enumerate}


\paragraph*{Derivazione del filtro adattato}

L'SNR viene espressa come:
\[
    \text{SNR} = \frac{s_u^2(T_s)}{E[n_u^2(T_s)]}
\]

Calcoliamo i termini al numeratore e al denominatore della SNR.
\[
    s_u^2(T_s) = \left( \int_{-\infty}^{+\infty} s(\tau) h_R(T_s - \tau) d\tau \right)^2 = \left( \int_{-\infty}^{+\infty} S(f) H_R(f) e^{j2\pi fT_s} df \right)^2
\]

E l'energia del rumore all'uscita del filtro ricevitore sarà:
\[
    E[n_u^2(T_s)] = R_{n_u}(0)
\]

Dove:
\[
    R_{n_u}(\tau) = R_n(\tau) \ast h_R(\tau) \ast h_R(-\tau)
\]

E quindi per la potenza del rumore:
\[
    S_{n_u}(f) = S_n(f) \left| H_r(f) \right|^2 = \frac{N_0}{2} \left| H_R(f) \right|^2
\]

\[
    R_{n_u}(0) = \int_{-\infty}^{+\infty} S_{n_u}(f) df
\]

Sostituendo otteniamo l'SNR in funzione della risposta in frequenza del filtro ricevitore \( h_R(f) \):
\[
    \text{SNR} = \frac{\left( \int_{-\infty}^{+\infty} S(f) H_R(f) e^{j2\pi fT_s} df \right)^2}{\frac{N_0}{2} \int_{-\infty}^{+\infty} \left| H_R(f) \right|^2 df} = \frac{2}{N_0 E_{h_R}} \left( \int_{-\infty}^{+\infty} S(f) H_R(f) e^{j2\pi fT_s} df \right)^2
\]

Utilizzando la disuguaglianza di Schwarz si può dimostrare che $\int_{-\infty}^{+\infty} S(f) H_R(f) e^{j2\pi fT_s} df$ raggiunge il massimo valore quando:
\[
    H_R(f) e^{j2\pi fT_s} = S^*(f)
\]

Quindi:
\[
    H_R(f) = S^*(f) e^{-j2\pi fT_s}
\]

\[
    \boxed{h_R(t) = s(T_s - t)}
\]

Dalla espressione della risposta impulsiva \( h_R(t) \) si deduce il nome di \textbf{filtro adattato}, in quanto la sua risposta impulsiva è adattata al segnale in ingresso al filtro stesso.



Studiando il modulo della risposta in frequenza \( H_R(f) \) si deduce che il filtro tende ad amplificare le componenti frequenziali dove è presente il segnale e ad attenuare (o eliminare) le componenti frequenziali dove il contributo di segnale è scarso (o addirittura assente).

\[
    \left| H_R(f) \right| = \left| S(f) \right|
\]

La simbologia per indicare un filtro adattato è \( h_{FA}(t) \) o \( H_{FA}(f) \).

Esempio:

% Drawing the example signals using TikZ
\begin{tikzpicture}
    \draw[->] (-1,0) -- (5,0) node[right] {\(t\)};
    \draw[->] (0,-1) -- (0,2) node[above] {\(s(t)\)};
    \draw[dashed] (0,1) -- (1,1);
    \node at (0,1) [left] {\(1\)};
    \draw (0,0) -- (1,1) -- (1,0);
    \node at (1,0) [below] {\(T_s\)};
    \draw[->] (6,0) -- (12,0) node[right] {\(t\)};
    \draw[->] (7,-1) -- (7,2) node[above] {\(h_{FA}(t)\)};
    \draw (7,1) -- (8,0);
    \node at (8,0) [below] {\(T_s\)};
\end{tikzpicture}

Calcolo del \( SNR_{max} \)

Il valore del \( SNR_{max} \) si ottiene per definizione quando si utilizza il \( FA \).

\[
    SNR = \frac{2}{N_0 E_s} \cdot E_s^2 = \frac{2E_s}{N_0}
\]

Da notare che:

\begin{itemize}
    \item Il \( SNR \) non dipende dalla forma del segnale, ma solo dalla sua energia. Questo da spazio alla progettazione della forma del segnale indipendentemente dai risultati in termini di \( SNR \).
    \item Il massimo del \( SNR \) si ottiene per qualunque \( h_{FA}(t) = k \cdot s(T_s-t) \) con \( k \in \mathbb{R} \), infatti basta calcolare il $SNR$:
          \[ SNR = \frac{2}{N_0K^2 E_s} \cdot K^2 E_s^2 = \frac{2 E_s}{N_0} \]
\end{itemize}

Quindi, fattori di amplificazione e/o attenuazione non cambiano il risultato. Questo è abbastanza intuitivo in quanto un fattore costante di amplificazione opera allo stesso modo sul segnale utile e sul rumore per cui nel rapporto i contributi si elidono.

\paragraph*{Schema del ricevitore con filtro adattato}

Consideriamo la trasmissione di un simbolo:

\begin{center}
    \begin{tikzpicture}[
            block/.style={rectangle, draw, minimum height=1cm, minimum width=2.5cm},
            node distance=1cm and 2cm,
            auto
        ]




        \node[block] (filter) {$h_{FA}(t)$};
        %\node[left=of filter] (channel) {};
        \node[right=of filter] (sampler) {};




        \node (s) at (-5,0) {\(s(t)\)};
        \node[draw, circle] (plus) at (-3,0) {\(+\)};
        \node (n) at (-3,-1.5) {\(n(t)\)};

        \draw[->] (s) -- (plus);
        \draw[->] (n) -- (plus);
        %\draw[->] (plus) -- (channel);

        \draw[->] (plus) -- (filter) node[midway,above] {$r(t)$};

        \draw ([xshift=0]filter.east) -- ([xshift=1cm]filter.east) node[midway,above] {$y(t)$};
        \draw ([xshift=1cm]filter.east) -- ([xshift=1.5cm,yshift=0.5cm]filter.east) node[midway,below, yshift=-0.2cm] {$kT_s$};

        \draw[->] ([xshift=1.5cm,yshift=0cm]filter.east) -- ++(1.5cm,0) node[midway,above] {$y[k]$};


    \end{tikzpicture}
\end{center}

\[ s(t) = \alpha \cdot p(t-nTs) \]
\[ h_{FA}(t) = k \cdot p(T_s - t) \]
\[ y(t) = s_u(t) + n_u(t) \]

Caratteristiche di \( s_u(t) \) e \( n_u(t) \):

\[ s_u(t) = s_u(t) \ast h_{FA}(t) = k \cdot \alpha \cdot p(t) \ast p(T_s - t) \]
\[ = k \cdot \alpha \int_{-\infty}^{\infty} p(\tau) p(\tau - (t - T_s)) d\tau = k \cdot \alpha \cdot C_p (t - T_s) \]
\[ C_p(t) = \int_{-\infty}^{\infty} P(\tau) P(\tau - t) d\tau \quad \text{(Autocorrelazione dell'impulso sagomatore)}
\]

\textbf{Esempio:}\\
La funzione rettangolare:
\[ s(t) = \text{rect}\left(\frac{t - \frac{T_s}{2}}{T_s}\right) \]

La risposta unitaria:
\[ s_u(t) = C_s(t - T_s) = T_s\cdot \left( 1 - \frac{\left| t - T_s \right|}{T_s} \right) \text{rect}\left(\frac{t - T_s}{2T_s}\right) \]
\noindent
\begin{minipage}{.5\textwidth}
    \centering

    % Diagrammi usando TikZ
    \begin{tikzpicture}
        \begin{axis}[
                axis lines = middle,
                xlabel = \( t \),
                ylabel = \( s(t) \),
                xtick = {1},
                xticklabels={$T_s$},
                ytick = {1},
                ymin = 0, ymax = 3.5,
                xmin = 0, xmax = 2.5,
                every axis x label/.style={at={(current axis.right of origin)},anchor=west},
                every axis y label/.style={at={(current axis.above origin)},anchor=south}
            ]
            \addplot+[const plot, no marks, thick] coordinates {(0,1) (1,1) (1,0)};
        \end{axis}
    \end{tikzpicture}
\end{minipage}%
\begin{minipage}{.5\textwidth}
    \centering
    \begin{tikzpicture}
        \begin{axis}[
                axis lines = middle,
                xlabel = \( t \),
                ylabel = \( s_u(t) \),
                xtick = {1, 2},
                xticklabels={$T_s$, $2T_s$},
                ytick = {1},
                yticklabels={$T_s$},
                ymin = 0, ymax = 2.5,
                xmin = 0, xmax = 2.5,
                every axis x label/.style={at={(current axis.right of origin)},anchor=west},
                every axis y label/.style={at={(current axis.above origin)},anchor=south}
            ]
            \draw[dashed] (0, 1) -- (1, 1) -- (1,0);
            \addplot+[sharp plot, no marks, thick] coordinates {(0,0) (1,1) (2,0)};
        \end{axis}
    \end{tikzpicture}
    \\
\end{minipage}


\[ n_u(t) = n(t) \ast h_{FA}(t) \]
\[ n(t) = \text{rumore bianco Gaussiano additivo (AWGN)} \]
\[ E[n(t)] = 0 \]
\[ R_n(\tau) = \sigma_n^2 \delta(\tau) = \frac{N_0}{2} \delta(\tau) \]
\[ n(\overline{t}) = \text{variabile aleatoria con densità di probabilità } f_N(n) = \frac{1}{\sqrt{2\pi\sigma_n^2}} e^{-\frac{n^2}{2\sigma_n^2}} = \frac{1}{\sqrt{N_0}} e^{-\frac{n^2}{N_0}} \]
Essendo il filtro in ricezione un filtro lineare e stazionario, \( n_u(t) \) è un rumore Gaussiano, additivo e stazionario.
\[ E[n_u(t)] = 0 \]
\[ R_{n_u}(\tau) = R_n(\tau) \ast h_{FA}(\tau) \ast h_{FA}(-\tau) = \frac{N_0}{2} C_{h_{FA}}(\tau)  \]
\[
    C_{h_{FA}}(\tau) = \text{autocorrelazione di } h_{FA}(t)
\]
\[ S_{n_u}(f) = \frac{N_0}{2} |H_{FA}(f)|^2 \]
\[ P_{n_u} = \frac{N_0}{2} E_{H_{FA}} = \frac{N_0}{2} E_p k^2 \]

È importante capire se i campioni di rumore sono tutti loro correlati o meno.
\[
    E[n_u[k]n_u[n]] = 0 \quad \forall k \neq n \quad \text{(incorrelazione)}
\]
N.B. la si può scrivere così poiché
\(E[n_u[k]] = 0\)

Questo vuol dire che
\[
    R_{n_u}[kT_s] = 0 \quad \forall k \neq 0
\]

\[
    R_{n_u}[nT_s] = \frac{N_0}{2} C_p[kT_s] = 0 \Rightarrow C_p[kT_s] = 0
\]


Dobbiamo ricordare che il segnale utile in ingresso al filtro adattato è ottenuto tramite il modulatore in trasmissione per cui è la funzione \( p(t) \) che determina la sagoma (forma) del segnale \( s(t) \).

\begin{enumerate}
    \item Impulso rettangolare
          \[
              p(t) = \text{rect}\left(\frac{t-T_s/2}{T_s}\right)
          \]

          \[
              C_p(\tau) = T_s \left(1 - \frac{|\tau|}{T_s}\right) \text{rect}\left(\frac{\tau}{2T_s}\right)
          \]

          \[
              R_{n_u}(\tau) = k \frac{N_0}{2} C_p(\tau)
          \]

          \[
              R_{n_u}(nT_s) =
              \begin{cases}
                  \frac{k^2 N_0 T_s}{2} & \text{se } k = 0    \\
                  0                     & \text{se } k \neq 0
              \end{cases}
              \Rightarrow
              \text{campioni di rumore incorrelati} \Rightarrow \text{indipendenti (Gaussiani)}
          \]
          \begin{center}
              \begin{tikzpicture}
                  \draw[->] (-3.5,0) -- (3.5,0) node[right] {$\tau$};
                  \draw[->] (0,-0.5) -- (0,2) node[above] {$C_p(\tau)$};
                  \draw[scale=1,domain=-1.5:1.5,smooth,variable=\x] plot ({\x},{1.5-abs(\x)});
                  \draw (1.5,0.1) -- (1.5,-0.1) node[below] {$T_s$};
                  \draw (-1.5,0.1) -- (-1.5,-0.1) node[below] {$-T_s$};
                  \draw (0.1,1.5) -- (-0.1,1.5) node[left] {$T_s$};
              \end{tikzpicture}

          \end{center}
    \item Impulso a radice di coseno rialzato
          \[
              P(f) = \sqrt{H_{RC}(f)}
          \]

          \begin{align*}
              S_{n_u}(f)    & = k^2 \frac{N_0}{2} |P(f)|^2 = k^2 \frac{N_0}{2} H_{RC}(f) \\
              R_{n_u}(\tau) & = k^2 \frac{N_0}{2} h_{RC}(\tau)                           \\
              R_{n_u}(nT_s) & = \begin{cases}
                                    \frac{k}{2} N_0, & k = 0    \\
                                    0,               & k \neq 0
                                \end{cases} \quad
              \text{N.B. la $h_{RC}(\tau)$ ha la sinc che si annulla in multipli di $T_s$.}
          \end{align*}

\end{enumerate}


\paragraph{SNR per bit all'ingresso del ricevitore}

Il SNR per bit è un parametro utile per determinare le prestazioni di una ricevitore in quanto tiene in considerazione quantità energetiche sia del segnale utile che del rumore
\[
    SNR_b = \frac{E_b}{N_0}
\]
\[
    E_b \coloneqq P_s T_b = E\left[x^2[k]\right]T_b \quad \text{Energia per bit} \]
\[
    \frac{N_0}{2} = S_n(f)
    \Rightarrow SNR_b = \frac{E\left[ x[k]\right]}{N_0 R_b}, \quad R_b = \frac{1}{T_b}
\]

\paragraph{Decisore ottimo e criterio della massimo verosimiglianza}

Il decisore deve mappare i campioni $y[k]$ in simboli dell'alfabeto. I campioni $y[k]$ sono statisticamente indipendenti l'uno dall'altro. Questo è dimostrato dal fatto che:

\[
    y[k] = s_u[k] + n_u[k]
    \quad
    \text{dove } s_u(k) \text{ e } n_u(k) \text{ sono indipendenti}
\]


%\draw[->] (source) -- (encoder) node[midway,above] {$b[n]$}  node[midway,below] {$T_b$};
\begin{center}
    \begin{tikzpicture}[
            block/.style={rectangle, draw, minimum height=1cm, minimum width=2.5cm},
            node distance=1cm and 2cm,
            auto
        ]
        \node[block] (filter) {Decisore};
        \node[left=of filter] (channel) {};
        \node[right=of filter] (sampler) {};

        \draw[->] (channel) -- (filter) node[midway,above] {$y[k]$}  node[midway,below] {$T_b$};
        \draw[->] (filter) -- (sampler) node[midway,above] {$\hat{x}[k]$}  node[midway,below] {$T_b$};
    \end{tikzpicture}
\end{center}

Quindi si può concludere che $\hat{x}[k] \in A_s $ può essere deciso in base alla sola conoscenza di $y[k]$. Questa decisione si dice di tipo "ad un sol colpo" (one-shot detector).

\paragraph{Decisione a minima probabilità di errore}

\begin{itemize}
    \item Probabilità di errore sul simbolo
          \(
          P_E(M) \coloneqq P\{\hat{x}[k] \neq x[k]\}
          \)
    \item
          Criterio di ottimalità: minimizzazione della $P_E(M)$

\end{itemize}

Derivazione del decisore ottimo:
\[
    x \coloneqq \hat{x}[k], \quad y \coloneqq y[k], \quad n_u \coloneqq n_u[k], \quad \hat{x} = \hat{x}[k]
\]

\paragraph{Criterio a massima probabilità a posteriori e minima probabilità di errore}
Con
MAP si intende l'acronmo di Maximum A-posteriori Probability
\[
    \hat{x} = \arg\max_{i=1,\ldots,M} P(x=\alpha_i|y)
\]

Viene associato ad un osservato $y$ il simbolo dell'alfabeto $\hat{x}$ tale che sia massima la probabilità a posteriori (condizionata) che quel simbolo sia stato trasmesso.

Se il decisore adotta il criterio MAP allora la probabilità di errore sul simbolo è minima.


\paragraph{Dimostrazione}

Definiamo
\(
R(i) \coloneqq \left\{ y \in \mathbb{R} : \hat{x} = \alpha_i \right\}, \quad i = 1,\ldots,M
\)
come la "zona di decisione" del simbolo $\alpha_i$, ovvero l'insieme dei valori di $y$ per cui si decide per il simbolo $\alpha_i$.

\[
    P\{x = \alpha_i | y\} = \frac{f_Y(y | \hat{x} = \alpha_i) P\{x = \alpha_i\}}{f_Y(y)} \quad (\text{Bayes})
\]

\[
    P_E(M) = P\{\hat{x} \neq x\} = 1 - P\{\hat{x} = x\} = 1 - \sum_{i=1}^M P\{\hat{x} = \alpha_i, x = \alpha_i\}
\]

\[
    = 1 - \sum_{i=1}^M P\{\hat{x} = \alpha_i | x = \alpha_i\} P\{x = \alpha_i\} =
\]

\[
    = 1 - \sum_{i=1}^M P\{x = \alpha_i\} P\{y \in R(i) | x = \alpha_i\}
\]

\[
    = 1 - \sum_{i=1}^M P\{x = \alpha_i\} \int_{y \in R(i)} f_Y(y | x = \alpha_i) dy
\]

\[
    = 1 - \sum_{i=1}^M \int_{y \in R(i)} P\{x = \alpha_i\} f_Y(y | x = \alpha_i) dy
\]

\[
    = 1 - \sum_{i=1}^M \int_{y \in R(i)} f_Y(y) P\{x = \alpha_i | y\} dy
\]

Per minimizzare la $P_E(M)$ devo scegliere le $R(i)$ in modo tale che osservato $y$ sia massima la probabilità a posteriori relativa al simbolo $i$-esimo.

Si osserva che se le probabilità a priori sono identiche
\[
    P\{x = a_i\} = \frac{1}{M}, \quad i=1,\ldots,M
\]
allora, dato che $f_Y(y)$ non dipende da $i$:
\[
    \hat{x} = \arg \max_{i=1,\ldots,M} \left\{ \frac{1}{M} \frac{f_Y(y | x = \alpha_i)}{f_Y(y)} \right\} = \arg \max_{i=1,\ldots,M} f_Y(y | x = \alpha_i)
\]


La funzione $f_Y(y|x=\alpha_i)$ viene detta anche \textbf{funzione di verosimiglianza}

In pratica il criterio di minima probabilità di errore (o massima probabilità a posteriori) coincide con il criterio di massima verosimiglianza quando le probabilità a priori $P\{x=\alpha_i\}$ sono identiche.
Nel caso di AGWN
\[
    y = s_u + n_u = \alpha_i + n_u, \quad n_u \in \mathcal{N}(0,\sigma_{n_u}^2)
\]
\[
    f_Y(y|x = \alpha_i) = f_{n_u}(y - \alpha_i)
\]
\[
    = \frac{1}{\sqrt{2\pi\sigma_{n_u}^2}} e^{-\frac{(y-\alpha_i)^2}{2\sigma_{n_u}^2}}
\]

\[
    \hat{x} = \underset{i=1,\ldots,M}{\mathrm{argmax}} \frac{1}{\sqrt{2\pi\sigma_{n_u}^2}} e^{-\frac{(y-\alpha_i)^2}{2\sigma_{n_u}^2}} = \underset{i=1,\ldots,M}{\mathrm{argmin}}  (y - \alpha_i)^2
\]

\[
    \boxed{
        \hat{x} = \underset{i=1,\ldots,M}{\mathrm{argmin}} \{ |y-\alpha_i| \}
    } \quad \text{(minimo della distanza euclidea)}
\]


Il decisore ottimo coincide con la scelta del simbolo a distanza euclidea minima dall'osservato.

Le zone di decisione sono quindi stabilite dalla regola di quantizzazione uniforme.

Questo significa che il decisore può essere realizzato con un quantizzatore uniforme.


\paragraph{Ricevitore ottimo per un sistema di comunicazione PAM}

Per un sistema di comunicazione PAM con simboli equiprobabili, il ricevitore ottimo secondo il criterio a minima probabilità di errore è il seguente:

\begin{center}
    \begin{tikzpicture}[
            block/.style={rectangle, draw, minimum height=1cm, minimum width=2.5cm},
            node distance=3cm and 3cm,
            auto
        ]

        \node[block] (filter) {$p(T_s - t)$};
        \node[left=of filter] (channel) {};
        \node[right=of filter] (sampler) {};
        \node[block, right=of filter](quantizzatore) {Quantizzatore uniforme M livelli};

        \draw[->] (channel) -- (filter) node[midway,above] {$r(t)$};

        \draw ([xshift=0]filter.east) -- ([xshift=1cm]filter.east) node[midway,above] {$y(t)$};
        \draw ([xshift=1cm]filter.east) -- ([xshift=1.5cm,yshift=0.5cm]filter.east) node[midway,below, yshift=-0.2cm] {$T_s$};

        \draw[->] ([xshift=1.5cm,yshift=0cm]filter.east) -- ++(quantizzatore) node[midway,above] {$y[k]$};

        \draw[->] (quantizzatore.east) -- ++(1.5cm,0) node[midway,above] {$\hat{x}[k]$};

    \end{tikzpicture}
\end{center}

\begin{figure}
    \centering
    \begin{center}
        \begin{tikzpicture}[scale=0.75]
            % Assi
            \draw[thick,->] (-6,0) -- (6,0) node[anchor=north west] {};
            \draw[thick,->] (0,-5) -- (0,5) node[anchor=south east] {};

            % Grafico a gradini
            \draw[ultra thick, red] (-4,-3) -- (-2,-3) -- (-2,-1) -- (0,-1) -- (0,1) -- (2,1) -- (2,3) -- (4,3);

            % Tacche e etichette sull'asse delle x
            \foreach \x in {-4,-2,2,4}
            \draw (\x cm,1pt) -- (\x cm,-3pt)
            node[anchor=north] {$\x$};


            % Linee tratteggiate
            \draw[dashed, red] (-4,0) -- (-4,-3);
            \draw[dashed, red] (-2,-1) -- (-2,0);

            \draw[dashed, red] (4,0) -- (4,3);
            \draw[dashed, red] (2,1) -- (2,0);
            \draw[dashed, red] (2,3) -- (0,3);
            \draw[dashed, red] (-2,-3) -- (0,-3);

            % Etichette
            \node at (0,1) [left] {$1$};
            \node at (0,-1) [right] {$-1$};
            \node at (0,3) [left] {$3$};
            \node at (0,-3) [right] {$-3$};
        \end{tikzpicture}
    \end{center}
    \caption*{Esempio di una 4-Pam, con quantizzatore uniforme a 4 livelli}
\end{figure}

\paragraph{Probabilità di errore di bit e di simbolo}

\[
    P_E(b) = P\{\hat{b}[k] \neq b[k]\} \quad \text{bit}
\]

\[
    P_E(M) = P\{\hat{x}[k] \neq x[k]\} \quad \text{simbolo}
\]

\[
    P_E(M) = P_E(b) \quad \text{solo quando l'alfabeto } A_s \text{ è composto da soli due simboli}
\]
Vale però sempre che:
\[
    \frac{P_E(M)}{\log_2 M} \leq P_E(b) \leq \frac{M/2}{M-1} P_E(M)
\]

\paragraph{Codifica di Gray}

Sia \( A_s = \{\alpha_1, \ldots, \alpha_M\} \) dove
\[
    \alpha_i = 2i - M - 1 \quad i = 1, \ldots, M
\]

La codifica di Gray associa stringhe di bit ai simboli dell'alfabeto in modo che le stringhe di bit relative a due simboli consecutivi differiscano al più per un bit.

Nel caso di SNR sufficientemente elevato \( (> 10 dB) \), l'evento errore consiste generalmente nel decidere per uno dei simboli dell'alfabeto adiacenti a quello trasmesso.

Utilizzando quindi la codifica di Gray e in condizioni di SNR elevato, un errore su un simbolo \( M-ario \) ogni \( N \) simboli  \( M-ari \) si traduce in un errore su una sola cifra binaria ogni \( N \) simboli \( M-ari \), cioè ogni \( N \log_2 M \) cifre binarie, quindi:

\[
    P_E(b) \approx \frac{P_E(M)}{\log_2 M}
\]


Esempi:

\begin{enumerate}
    \item 4-PAM
          \begin{itemize}
              \item $-3 \rightarrow 00$
              \item $-1 \rightarrow 01$
              \item $+1 \rightarrow 11$
              \item $+3 \rightarrow 10$
          \end{itemize}

    \item 8-PAM
          \begin{itemize}
              \item $-7 \rightarrow 000$
              \item $-5 \rightarrow 001$
              \item $-3 \rightarrow 011$
              \item $-1 \rightarrow 010$
              \item $+1 \rightarrow 110$
              \item $+3 \rightarrow 111$
              \item $+5 \rightarrow 101$
              \item $+7 \rightarrow 100$
          \end{itemize}
\end{enumerate}
\paragraph{Prestazioni di un M-PAM in presenza di rumore:}

\begin{center}
    \begin{tikzpicture}[
            block/.style={rectangle, draw, minimum height=1cm, minimum width=2.5cm},
            node distance=2.5cm and 2.5cm,
            auto
        ]




        \node[block] (filter) {$h_{FA}(t)$};
        \node[right=of filter] (sampler) {};
        \node[block, right=of filter](decisore) {Decisore};

        \node (s) at (-5,0) {\(s(t)\)};
        \node[draw, circle] (plus) at (-3,0) {\(+\)};
        \node (n) at (-3,-1.5) {\(n(t)\)};

        \draw[->] (s) -- (plus);
        \draw[->] (n) -- (plus);
        %\draw[->] (plus) -- (channel);

        \draw[->] (plus) -- (filter) node[midway,above] {$r(t)$};

        \draw ([xshift=0]filter.east) -- ([xshift=1cm]filter.east) node[midway,above] {$y(t)$};
        \draw ([xshift=1cm]filter.east) -- ([xshift=1.5cm,yshift=0.5cm]filter.east) node[midway,below, yshift=-0.2cm] {$T_s$};

        \draw[->] ([xshift=1.5cm,yshift=0cm]filter.east) -- (decisore) node[midway,above] {$y[k]$};
        \draw[->] (decisore.east) -- ++(1.5,0) node[midway,above] {$\hat{x}[k]$};

    \end{tikzpicture}
\end{center}

\[
    s(t) = \sum_{k=-\infty}^{\infty} x[k] P(t - kT_s)
\]

\[
    P_E(M) = \frac{M-1}{M} \text{erfc} \left( \sqrt{\frac{3\cdot SNR \cdot \log_2 M}{M^2-1}} \right)
\]

\begin{center}

    \begin{tikzpicture}
        \begin{axis}[
                title={Errore di probabilità $P_E(M)$ vs SNR (dB)},
                xlabel={SNR (dB)},
                ylabel={$P_E(M)$},
                xmin=-5, xmax=20,
                ymin=1e-6, ymax=1,
                ymode=log,
                legend pos=north east,
                ymajorgrids=true,
                grid style=dashed,
            ]

            % approximate erf(x) with tanh(1.2*x)
            \addplot[
                color=blue,
                mark=none,
                thick,
                domain=-5:20,
                samples=100,
            ] {0.5*(1-tanh(1.2*sqrt(x)))}; \addlegendentry{$M=2$}

            \addplot[
                color=red,
                mark=none,
                thick,
                domain=-5:20,
                samples=100,
            ] {0.75*(1-tanh(1.2*sqrt(2*x/5)))}; \addlegendentry{$M=4$}

            \addplot[
                color=green,
                mark=none,
                thick,
                domain=-5:20,
                samples=100,
            ] {0.875*(1-tanh(1.2*sqrt(x/7)))}; \addlegendentry{$M=8$}

        \end{axis}
    \end{tikzpicture}
\end{center}

Per SNR $> 10$ dB, utilizzando la codifica di Gray
\[
    P_E(b) \approx \frac{P_E(M)}{\log_2 M}
\]

Per una BPSK (2-PAM)
\[
    P_E(M) = P_E(b) = \frac{1}{2} \text{erfc}\left(\sqrt{\text{SNR}}\right)
\]

\paragraph{Dimostrazione}

Considerando le densità di probabilità $f_Y(y|x=1)$ e $f_Y(y|x=-1)$:

La probabilità di errore binario $P_E(b)$ è data dalla somma delle aree sotto le curve di $f_Y(y|x=1)$ e $f_Y(y|x=-1)$ dove queste si sovrappongono.



La densità di probabilità condizionata dato $x$ è:
\[
    P\{ \hat{x} = -1 | x = 1 \} = \int_{-\infty}^{\infty} f_Y(y | x = 1) \, dy
\]
\[
    f_Y(y | x = 1) = \frac{1}{\sqrt{2\pi \sigma_{n_u}^2}} \exp \left( -\frac{(y - h(0)x)^2}{2\sigma_{n_u}^2} \right) \quad \text{per} \quad x = 1
\]

Dopo il campionatore:
\[
    y[k] = x[k]\cdot h(0) + n_u[k]
\]

Dove la varianza del rumore è data da:
\[
    \sigma_{n_u}^2 = \frac{N_0}{2} E_{h_R} = \frac{N_0}{2} h(0)
\]

La densità spettrale di potenza del rumore bianco è:
\[
    S_w(f) = \frac{N_0}{2} \quad \Rightarrow \quad \text{DSP sul processo del rumore in ingresso è uguale in distribuzione } h_R(t)
\]

Il rapporto segnale-rumore (SNR) è:
\[
    SNR = \frac{h(0)^2}{\frac{N_0}{2} h(0)} = \frac{2 h(0)}{N_0}
\]

La probabilità condizionata dato $x$ è:
\[
    P\{ \hat{x} = -1 | x = 1 \} = 1 - Q\left( \frac{0 - h(0)}{\sqrt{\frac{N_0}{2} h(0)}} \right)
\]
\[
    = Q\left( \sqrt{\frac{2 h(0)}{N_0}} \right) = Q\left( \sqrt{SNR} \right) = \frac{1}{2} \text{erfc}\left( \sqrt{SNR} \right)
\]

Si può dimostrare per simmetria che:
\[
    P\{ \hat{x} = 1 | x = -1 \} = P\{ \hat{x} = -1 | x = 1 \} = \frac{1}{2} \text{erfc}\left( \sqrt{SNR} \right)
\]


Quindi:
\[
    P_E(b) = \frac{1}{2} \cdot \frac{1}{2} \text{erfc}\left( \frac{\sqrt{SNR}}{2} \right) + \frac{1}{2} \cdot \frac{1}{2} \text{erfc}\left( \frac{\sqrt{SNR}}{2} \right) = \frac{1}{2} \text{erfc}\left( \frac{\sqrt{SNR}}{2} \right)
\]

\paragraph{Presenza di ISI e rumore}

Il segnale ricevuto è:
\[
    y(t) = \sum_{k=-\infty}^{\infty} x[k] h(t - kT_s) + n_u(t)
\]

Dove:
\[
    h(t) = p(t) * c(t) * h_R(t)
\]
\[
    n_u(t) = n(t) * h_R(t)
\]

E il segnale al campionato è:
\[
    y[k] = x[k] \cdot h(0) + I[k] + n_u[k]
\]

Dove il termine di interferenza è dato da:
\[
    I[k] = \sum_{\substack{n=-\infty \\ n \neq k}}^{+\infty} x[n] h\left( (k-n)T_s \right)
\]

L'approccio da seguire è il seguente:
il filtro $h_R(t)$ deve essere allo stesso tempo quello che elimina l'ISI e che massimizza l'$SNR$. Questo problema può essere risolto progettando opportunamente $p(t)$ e $h_R(t)$.

\paragraph{Equalizzatore zero forcing}

Massimizza il SNR vincolando \( I[k] = 0 \) per ogni \( k \).  È un problema di massimizzazione vincolata per cui la soluzione non porta alla realizzazione del filtro adattato.

\begin{itemize}
    \item \( I[k] = 0 \) quando \( h(t) = h_{RC}(t) \), allora
          \[
              P(f) C(f) H_R(f) = H(f) = H_{RC}(f) e^{-j 2 \pi f T_s}
          \]
          per la causalità.

          Si pone quindi il problema di massimizzare il $SNR$ con il vincolo
          \[
              P(f) C(f) H_R(f) = H_{RC}(f) e^{-j 2 \pi f T_s}
          \]
\end{itemize}

Si può scrivere:
\[
    \left| P(f) \right| = \left| H_R(f) \right| = \sqrt{\frac{\left| H_{RC}(f) \right|}{\left| C(f) \right|}}
\]
\[
    \angle P(f) = \angle H_R(f) = - \pi f T_s - \frac{\angle C(f)}{2}
\]

N.B. Se \( C(f) = 1 \) (canale ideale) allora
\[
    P(f) = H_R(f) = \sqrt{H_{RC}(f)} e^{-j 2 \pi f T_s}
\]

Il valore di SNR in tal caso è:



\[
    SNR = \frac{E\left\{\left|X[k]\right|^2\right\} h^2(0)}{\frac{N_0}{2}
    \int_{-B_c}^{B_c} \frac{H_{RC}(f)}{C(f)}\, df }
\]

dove \( B_c \) è la banda del canale \( C(f) \), \( h(t) \) è la risposta all'impulso del canale ideale e $h(0)$ può essere visto come $
    \int_{-\infty}^{+\infty} H(f)\, df$.

\[
    E_s = P_s T_s = T_s \frac{M^2-1}{3}
\]

Se \( C(f) = 1 \) allora \( SNR = \frac{2 E_s}{N_0} \) e il filtro è adattato.

Il segnale ricevuto è:
\[
    r(t) = \sum_{k=-\infty}^{\infty} x[k] p(t - kT_s) + n(t)
\]
dove \( n(t) \) è il rumore bianco additivo gaussiano (AWGN).

La funzione di trasferimento del filtro ricevitore è:
\[
    h_{R}(f) = p (T_s - t)
\]
\[
    H_{R}(f) = P(f) e^{-j2\pi fT_s}
\]
\[
    H(f) = P^2(f) e^{-j2\pi fT_s} = H_{RC}(f) e^{-j2\pi fT_s}
\]

Riassumendo, il problema di eliminare l'ISI e massimizzare il SNR si risolve utilizzando il filtro sagomatore \( p(t) \) e quello di ricezione, \( h_R(t) \), realizzati con la radice di coseno rialzato.

Nel caso di canale ideale, la soluzione coincide con il filtro adattato.

%\section*{Inviluppo complesso}

\paragraph*{Trasformata di Hilbert}
Si definisce la trasformata di Hilbert di un segnale \( x(t) \) come:
\[
    \mathcal{H}\{x(t)\} = x_H(t) \coloneqq \frac{1}{\pi} \int_{-\infty}^{\infty} \frac{x(\tau)}{t - \tau} \, d\tau
\]


questo intergrale può essere riscritto come: 
\[
    x_H(t) = x(t) \ast \frac{1}{\pi t} = x(t) \ast h_H(t)
\]

dove \( h_H(t) \) è la risposta all'impulso del filtro di Hilbert, definito come:
\[
    % inser also fourier transform of h_H(t)
    h_H(t) = \frac{1}{\pi t} \ \xrightarrow{\mathcal{F}} \ H_H(f) = -j\cdot \text{sgn}(f)
\]



definiamo anche il segnale analitico \( x_a(t) \) come:
\[
    x_a(t) = x(t) + j \cdot x_H(t)
\]
\[
    X_a(f) = X(f) \cdot 2\text{u}(f)   
\]
\[
    X(f) = \frac{1}{2} \left[ X_a(f) + X_a^*(-f) \right]    
\]

se la banda del segnale \( x(t) \) è prevalentemente concentrata attorno ad una certa frequenze $f_0$, come avviene per i segnali passa banda,
torna utile definire il segnale detto inviluppo complesso \( \tilde{x}(t) \) come:
\[
    \tilde{x}(t) = x_a(t) e^{-j2\pi f_0 t}
\]
%\section*{Modulazione Numerica in Banda Passante}

\subsection*{Segnale Passa Banda}

Il segnale passa banda può essere espresso come:
\[
s(t) = a(t) \cos[2\pi f_0 t + \phi(t)]
\]
dove \( a(t) \) è l'inviluppo reale di \( s(t) \) (segnale passa-basso) e \( \phi(t) \) la fase di \( s(t) \).



\begin{tikzpicture}[font=\small,>=stealth',thick, node distance=2cm]
\tikzstyle{block} = [rectangle, draw,
    text width=5em, text centered, minimum height=2.5em]
\tikzstyle{signal} = [draw,->]
\tikzstyle{modulator} = [circle, draw, inner sep=0pt, minimum size=1cm]

\node[block] (filter) {p(t)};
\node[modulator, right of=filter, node distance=3cm] (modulator) {\Large$\times$};
\node[above of=filter, node distance=1cm] (input) {x(t)};
\node[right of=modulator, node distance=3cm] (output) {s(t)};

\draw[signal] (input) -- (filter);
\draw[signal] (filter) -- (modulator);
\draw[signal] (modulator) -- (output);
\draw[signal] (filter) -- node[above] {baseband} (modulator);
\draw[signal] (modulator.east) -- node[above] {passband} (output);

\node[above of=modulator, node distance=1cm] (carrier) {cos(2$\pi$f0t)};
\draw[signal] (carrier) -- (modulator);

\end{tikzpicture}


Espandendo l'inviluppo complesso di \( s(t) \), otteniamo:
\[
s(t) = \Re\left\{ a(t) e^{j[2\pi f_0 t + \phi(t)]} \right\}
\]
\[
= \Re\left\{ a(t) \cos[2\pi f_0 t + \phi(t)] + j \cdot a(t) \sin[2\pi f_0 t + \phi(t)] \right\}
\]
\[
= a(t) \cos[2\pi f_0 t + \phi(t)] = \Re\left\{ \tilde{a}(t) e^{j2\pi f_0 t} \right\}
\]
dove \( \tilde{a}(t) \) è l'inviluppo complesso di \( s(t) \).



















\begin{tikzpicture}[>=Stealth, 
    block/.style={draw, rectangle, minimum height=2em, minimum width=3em},
    sum/.style={draw, circle, node distance=1cm},
    node distance=2cm and 3cm
]
    % Nodes
    \node[block] (pfilter) {$p(t)$};
    \node[sum, right of=pfilter] (mixer) {$\times$};
    \node[right= of mixer] (output) {$s(t)$};
    \node[above= of mixer] (cosine) {$\cos(2\pi f_0 t)$};
    \node[left= of pfilter] (input) {$x(t)$};
    
    % Lines
    \draw[->] (input) -- (pfilter);
    \draw[->] (pfilter) -- (mixer);
    \draw[->] (mixer) -- (output);
    \draw[->] (cosine) -- (mixer);

    % Lower part with s tilde
    \node[block, below= of pfilter] (pfilter2) {$p(t)$};
    \node[sum, right of=pfilter2] (mixer2) {$\times$};
    \node[right= of mixer2] (output2) {$\tilde{s}(t)$};
    \node[above= of mixer2] (exponential) {$e^{j2\pi f_0 t}$};
    \node[left= of pfilter2] (input2) {$\tilde{x}(t)$};
    
    % Lines for lower part
    \draw[->] (input2) -- (pfilter2);
    \draw[->] (pfilter2) -- (mixer2);
    \draw[->] (mixer2) -- (output2);
    \draw[->] (exponential) -- (mixer2);
    
    % Draw the constellation diagram
    \begin{scope}[shift={($(output2.south east)+(3cm,-1cm)$)},scale=0.5]
        \draw[->] (-4,0) -- (4,0) node[below] {$\Re$};
        \draw[->] (0,-3) -- (0,3) node[left] {$\Im$};
        
        % Points in the constellation
        \foreach \x in {-3,-1,1,3}
            \fill[orange] (\x,0) circle (5pt);
            
        % Label for n=4
        \node[below] at (3,0) {$n=4$};
    \end{scope}
\end{tikzpicture}








% Definizione di M-PAM
\[
M-PAM \quad \Rightarrow \quad \text{simboli:} \quad A_s = \{\alpha_1, \dots, \alpha_M\}
\]
\[
\alpha = 2c - 1 - M
\]
\[
p(t) = \text{impulso in TX}
\]

% Definizione del segnale s(t)
\[
s(t) = \sum_{n=-\infty}^{\infty} x(n) p(t - nT_s) \cos(2\pi f_c t)
\]

% Diagramma segnale
\begin{tikzpicture}
\node (x) at (0,0) {$x(n)$};
\node[right=2cm of x] (p) {$p(t)$};
\node[right=2cm of p] (s) {$s(t)$};
\draw[->] (x) -- (p);
\draw[->] (p) -- (s) node[midway,above] {$\times$} node[midway,below] {$\cos(2\pi f_c t)$};
\end{tikzpicture}

% Altre espressioni del segnale
\[
\tilde{S}(t) = \sum_{n=-\infty}^{\infty} x(n) p(t - nT_s)
\]
\[
\tilde{s}(t) = \Re\{\tilde{S}(t) e^{j2\pi f_c t}\}
\]
\[
= \Re \left\{ \sum_{n=-\infty}^{\infty} x(n) p(t - nT_s) e^{j2\pi f_c t} \right\}
\]
\[
= \sum_{n=-\infty}^{\infty} x(n) p(t - nT_s) \cos(2\pi f_c t) + j \sin(2\pi f_c t)
\]

% Diagramma dei simboli
\begin{tikzpicture}
\draw[->] (-3.5,0) -- (3.5,0) node[anchor=west] {$\Re$};
\draw[->] (0,-2) -- (0,2) node[anchor=south] {$\Im$};
\foreach \x in {-3,...,3}
  \draw (\x,0.1) -- (\x,-0.1) node[anchor=north] {$\x$};
\foreach \y in {-1,1}
  \draw (0.1,\y) -- (-0.1,\y) node[anchor=east] {$\y$};
% Aggiungere punti qui se necessario
\end{tikzpicture}
\noindent Modulazione di fase PSK (Phase Shift Keying):

% Definizione del segnale s(t)
\begin{flalign}
& s_c(t) = p(t) \cos(2\pi f_0 t + \theta_i) && \text{(simbolo $i$-esimo)} & \\
& \theta_i = \frac{2\pi}{M}(i-1) && \text{per $i = 1, \ldots, M$} &
\end{flalign}

\noindent Espansione del segnale s(t) come sommatoria:

\begin{flalign}
& s(t) = \sum_{n=-\infty}^{\infty} p(t - nT_s) \cos(2\pi f_0 t + \theta_i(n)) & \\
& \theta_i(n) \in A_s = \{\theta_1, \ldots, \theta_M\} &
\end{flalign}

\noindent Definizione del segnale modulato in fase $\tilde{s}_c(t)$:

\begin{flalign}
& \tilde{s}_c(t) = p(t) e^{j\theta_i} & \\
& \tilde{s}_c(t) = \Re\{p(t) e^{j\theta_i} e^{j2\pi f_0 t}\} & \\
& \phantom{\tilde{s}_c(t)} = \Re\{p(t) e^{j(2\pi f_0 t + \theta_i)}\} & \\
& \phantom{\tilde{s}_c(t)} = p(t) \cos(2\pi f_0 t + \theta_i) &
\end{flalign}

\noindent Definizione di $x(n)$ basata su $\theta_n$:

\begin{flalign}
& x(n) = e^{j\theta_n} &
\end{flalign}
% Equations arranged in a row
\noindent
\begin{minipage}{.5\linewidth}
\begin{equation*}
    s(t) = \Re \{ s(t) e^{j 2\pi f_0 t} \}
\end{equation*}
\begin{equation*}
    s(t) \approx \sum_{n=-\infty}^{\infty} s_n e^{j \Omega_n}
\end{equation*}
\end{minipage}%
\begin{minipage}{.5\linewidth}
\begin{equation*}
    s(t) = \Re \left\{ \sum_{n=-\infty}^{\infty} p(t-nT_s) e^{j (2\pi f_0 t + \Omega_n)} \right\}
\end{equation*}
\end{minipage}

% Another row of equations
\noindent
\begin{minipage}{.5\linewidth}
\begin{equation*}
    \approx \sum_{n=-\infty}^{\infty} p(t+nT_s) \cos(2\pi f_0 t + \Theta_n)
\end{equation*}
\end{minipage}%
\begin{minipage}{.5\linewidth}
\begin{equation*}
    S_i(t) = A_{I_i} p(t) \cos(2\pi f_0 t) - A_{Q_i} p(t) \sin(2\pi f_0 t)
\end{equation*}
\end{minipage}

% TikZ Diagram in a row with QAM explanation
\noindent
\begin{minipage}[c]{0.3\linewidth}
\begin{tikzpicture}[scale=0.8]
    \draw[->] (-1.5,0) -- (1.5,0) node[right] {$\Re$};
    \draw[->] (0,-1.5) -- (0,1.5) node[above] {$\Im$};
    \draw (0,0) circle (1cm);
    \draw[->] (1,0) arc (0:90:1cm);
    \foreach \angle/\label in {0/\theta_1, 90/\theta_2, 180/\theta_3, 270/\theta_4}{
      \draw[fill=green] (\angle:1cm) circle (2pt);
      \node at (\angle:1.2cm) {$\label$};
    }
    \foreach \angle in {45,135,225,315}{
      \draw[fill=orange] (\angle:1cm) circle (2pt);
    }
\end{tikzpicture}
\end{minipage}%
\begin{minipage}[c]{0.7\linewidth}
\begin{equation*}
    \frac{A_{I_i}}{A_{Q_i}} \text{ -- verso e ampiezza della modulazione}
\end{equation*}
\begin{equation*}
    i \text{ -- indice del simbolo } x(t)
\end{equation*}
\end{minipage}




% Start of the content
\[
A^c_i = \text{componente in fase di } x(t)
\]
\[
A^s_i = \text{componente in quadratura di } x(t)
\]

\[
s^c_i(t) = (A^c_i + j A^s_i) p(t)
\]

\[
\Re\{s^c_i(t) e^{j2\pi f_0t}\} = \Re\{(A^c_i + A^s_i) p(t) e^{j2\pi f_0t}\}
\]

\[
= \Re\{A^c_i p(t) e^{j2\pi f_0t} + j A^s_i p(t) e^{j2\pi f_0t}\}
\]

\[
= \Re\{A^c_i p(t) \cos(2\pi f_0t) + j A^c_i p(t) \sin(2\pi f_0t) + A^s_i p(t) \cos(2\pi f_0t) - A^s_i p(t) \sin(2\pi f_0t)\}
\]

\[
= A^c_i p(t) \cos(2\pi f_0t) - A^s_i p(t) \sin(2\pi f_0t)
\]

\[
s(t) = \sum_{n=-\infty}^{+\infty} x_n p(t-nT_s) \cos(2\pi f_0t) - x_s p(t-nT_s) \sin(2\pi f_0t)
\]

\[
s(t) = \Re\{\sum_{n=-\infty}^{+\infty} x_n p(t-nT_s)e^{j2\pi f_0 t}\}
\]

\[
    x[n] = x_c[n] + jx_s[n]
\]


%
\section*{Modulazione QAM}



\paragraph*{Modulazione QAM (Quadrature Amplitude Modulation)}


\begin{equation*}
    S_i(t) = A_{I_i} p(t) \cos(2\pi f_0 t) - A_{Q_i} p(t) \sin(2\pi f_0 t)
\end{equation*}

% TikZ Diagram in a row with QAM explanation
\noindent
\begin{minipage}[c]{0.3\linewidth}

\end{minipage}%
\begin{minipage}[c]{0.7\linewidth}
    \begin{equation*}
        \frac{A_{I_i}}{A_{Q_i}} \text{ -- verso e ampiezza della modulazione}
    \end{equation*}
    \begin{equation*}
        i \text{ -- indice del simbolo } x(t)
    \end{equation*}
\end{minipage}




% Start of the content
\[
    A^c_i = \text{componente in fase di } x(t)
\]
\[
    A^s_i = \text{componente in quadratura di } x(t)
\]

\[
    s^c_i(t) = (A^c_i + j A^s_i) p(t)
\]

\[
    \Re\{s^c_i(t) e^{j2\pi f_0t}\} = \Re\{(A^c_i + A^s_i) p(t) e^{j2\pi f_0t}\}
\]

\[
    = \Re\{A^c_i p(t) e^{j2\pi f_0t} + j A^s_i p(t) e^{j2\pi f_0t}\}
\]

\[
    = \Re\{A^c_i p(t) \cos(2\pi f_0t) + j A^c_i p(t) \sin(2\pi f_0t) + A^s_i p(t) \cos(2\pi f_0t) - A^s_i p(t) \sin(2\pi f_0t)\}
\]

\[
    = A^c_i p(t) \cos(2\pi f_0t) - A^s_i p(t) \sin(2\pi f_0t)
\]

\[
    s(t) = \sum_{n=-\infty}^{+\infty} x_n p(t-nT_s) \cos(2\pi f_0t) - x_s p(t-nT_s) \sin(2\pi f_0t)
\]

\[
    s(t) = \Re\{\sum_{n=-\infty}^{+\infty} x_n p(t-nT_s)e^{j2\pi f_0 t}\}
\]

\[
    x[n] = x_c[n] + jx_s[n]
\]



\begin{align*}
    x(t)   & = x_c(t) + j \cdot x_s(t)                \\
    x_c(t) & = \Re\{p(t)\} \quad x_s(t) = \Im\{p(t)\}
\end{align*}

% Qui viene rappresentato il diagramma di flusso della modulazione QAM
\begin{tikzpicture}
    % Definire i nodi e i percorsi qui
\end{tikzpicture}

\[
    x(t) = x_c(t) \cos(2\pi f_ct) - x_s(t) \sin(2\pi f_ct)
\]

% Diagramma dei punti della costellazione QAM
\begin{tikzpicture}
    \draw [<->] (0,2) node (yaxis) [above] {$\Im$}
    |- (2,0) node (xaxis) [right] {$\Re$};
    \draw (-1,0) -- (1,0);
    \draw (0,-1) -- (0,1);
    \foreach \x/\y in {-1/1, 1/1, -1/-1, 1/-1}
    \draw[fill=orange] (\x,\y) circle (2pt);
\end{tikzpicture}

\[
    \Gamma = \Gamma_c \cap \Gamma_s
\]

\begin{align*}
    A^\Gamma & = \{\alpha^\Gamma_c, \overline{\alpha^\Gamma_c}\} \\
    A^\Gamma & = \{\alpha^\Gamma_s, \overline{\alpha^\Gamma_s}\}
\end{align*}

\[
    A^c \rightarrow \alpha^c = 2c - 1 - \Gamma_c \quad A^s \rightarrow \alpha^s = 2c - 1 - \Gamma_s
\]

\section*{Introduction}


Le comunicazioni wireless rappresentano la quota più importante in termini di utenti oggigiorno. Lo spettro delle comunicazioni radio risulto molto affollato, soprattutto nella banda dell'ordine dei GHz in cui si trovano molteplici applicazioni.

\begin{table}[h!]
\centering
\begin{tabular}{ | m{2cm} | m{3cm} | m{2.5cm} | m{4cm} | m{2.5cm} | }
\hline
Banda di frequenza & Intervallo di frequenza & Lunghezza d'onda & Servizi & Propagazione \\
\hline
LF & 30 -- 300 kHz & \(10^{4} - 10^{3}\) m & Orologio radio, navigazione (LORAN), militare (marina) & Onda di terra \\
\hline
MF & 0.3 -- 3 MHz & \(10^{3} - 10^{2}\) m & Radio AM (522--1600 kHz), radiofari & Onda di terra, onda ionosferica \\
\hline
HF & 3 -- 30 MHz & \(10^{2} - 10\) m & Comunicazioni aeronautiche, radar oltre l'orizzonte, radioamatori & Onda ionosferica \\
\hline
VHF & 30 -- 300 MHz & \(10 - 1\) m & Comunicazioni avioniche, radio FM (88 -- 108 MHz), DVB-T (RAI \(177.5\) MHz) & Onda spaziale \\
\hline
UHF & 0.3 -- 3 GHz & \(1 - 10^{-1}\) m & DVB-T (470-860 MHz), Cellulare (\(900,1800,2200\) MHz), Wi-Fi, GPS & Onda spaziale \\
\hline
SHF & 3 -- 30 GHz & \(10^{-1} - 10^{-2}\) m & Wi-Fi, 5G, DVB-S, Radar, SatCom & Onda spaziale \\
\hline
EHF & 30 -- 300 GHz & \(10^{-2} - 10^{-3}\) m & Wi-Fi, 5G, DVB-S, Radar, SatCom & Onda spaziale \\
\hline
\end{tabular}
\end{table}

\begin{itemize}
    \item \textbf{LF (Low Frequency)}: Frequenze molto basse, utilizzate per orologi radio, sistemi di navigazione a lunga distanza come LORAN, e comunicazioni militari navali. La propagazione è principalmente per onda di terra.
    \item \textbf{MF (Medium Frequency)}: Queste frequenze includono la banda di trasmissione AM. Sono utilizzate anche per i radiofari. Le onde possono viaggiare come onde di terra o riflettersi nell'ionosfera (onde ionosferiche).
    \item \textbf{HF (High Frequency)}: Utilizzate per comunicazioni aeronautiche, radar a lungo raggio e dai radioamatori. Queste onde si propagano principalmente attraverso l'ionosfera.
    \item \textbf{VHF (Very High Frequency)}: Coprono servizi come le comunicazioni avioniche e le trasmissioni radio FM. La propagazione è principalmente diretta, conosciuta anche come onda spaziale o visuale.
    \item \textbf{UHF (Ultra High Frequency)}: Include servizi come la televisione digitale terrestre (DVB-T), comunicazioni cellulari, Wi-Fi e GPS. La propagazione è di tipo spaziale, e queste onde richiedono generalmente una linea di vista libera tra trasmettitore e ricevitore.
    \item \textbf{SHF (Super High Frequency) e EHF (Extremely High Frequency)}: Utilizzate per servizi avanzati come il Wi-Fi, il 5G, le trasmissioni satellitari (DVB-S), i radar e le comunicazioni satellitari (SatCom). Anche queste si propagano attraverso onda spaziale o visuale.
\end{itemize}

Incrementando la frequenza di ha una banda di trasmissione maggiore, tuttavia la distanza di trasmissione diminuisce.

\begin{itemize}
    \item \textbf{Onde di terra}: Hanno una bassa frequenza e si propagano vicino alla superficie terrestre. Sono utilizzate per comunicazioni a lunga distanza.
    \item \textbf{Onde ionosferiche}: Hanno una frequenza media, sono riflesse dalla ionosfera e possono viaggiare a lunghe distanze.
    \item \textbf{Onde spaziali}: Hanno una frequenza elevata e si propagano in linea retta. Richiedono una linea di vista libera tra trasmettitore e ricevitore.
\end{itemize}

La questione dell'occupazione della banda è un argomento molto trattato attualmente in quanto trasmissioni alla solita frequnza generano collisioni non promuovendo la ricostruzione del segnale trasmesso. Per ridurre le collisioni si utilizzano tecniche di modulazione e codifica del segnale.



\section*{Comunicazioni analogiche}

Amplitude Modulation Dual Side Band (AM-DSB) is one of the simplest forms of modulation, generalmente utilizzata per trasmettere solo la voce. The signal \( s_{DSB}(t) \) is 

\begin{equation}
s_{DSB}(t) = A_c m(t) \cos(2\pi f_c t)
\end{equation}
where:
\begin{itemize}
  \item \( m(t) \) is the modulating signal or the message.
  \item \( A_c \) is the amplitude of the carrier signal.
  \item \( \cos(2\pi f_c t) \) is the carrier wave at frequency \( f_c \).
\end{itemize}

The frequency spectrum of AM-DSB shows two sidebands, each located at \( f_c \pm W \) ans \( -f_c \pm W \), where \( W \) is the bandwidth of the message signal. These sidebands carry the same information, hence the term 'dual sideband'.

Taking the Fourier transform of \( s_{DSB}(t) \), we obtain the frequency domain representation \( S_{DSB}(f) \):
\begin{equation}
S_{DSB}(f) = \frac{1}{2} A_c [M(f - f_c) + M(f + f_c)]
\end{equation}
where \( M(f) \) is the Fourier transform of \( m(t) \).
Sebbene la componente negativa possa essere trascurata adesso la traslazione del segnale alla frequenza $f_c >> $ comporta un'occupazione di banda di $2W$ per ogni lato, ma la ciò comporta uno spreco in quanto si ha duplicazione di informazione. 
\subsection*{Properties}
\begin{itemize}
  \item AM-DSB is a linear modulation technique since the modulated signal is a linear combination of \( m(t) \) and \( \cos(2\pi f_c t) \).
  \item Due to the modulation, the message signal is translated to higher frequencies, which can be efficiently transmitted over long distances.
  \item AM-DSB is power inefficient since the carrier contains no information but consumes power.
\end{itemize}



\subsection*{Applications}
AM-DSB is widely used in traditional AM radio broadcasting. Despite its simplicity, it is not the most power-efficient mode of transmission. In modern communications, variations like Single Sideband (SSB) or Suppressed Carrier (SC) are more frequently used to conserve bandwidth and power.
\begin{tikzpicture}
\begin{axis}[
    title={AM-DSB Modulation},
    xlabel={Time},
    ylabel={Amplitude},
    axis lines=middle,
    ymax=2,
    ymin=-2,
    xtick=\empty,
    ytick=\empty,
    clip=false,
    no markers,
    width=10cm,
    height=5cm
]
% Modulating signal m(t)
\addplot [domain=0:2*pi, samples=100, name path=m] {sin(deg(x))};
\addlegendentry{$m(t)$}

% Carrier signal Ac*cos(2*pi*f*t)
\addplot [domain=0:2*pi, samples=100, name path=carrier] {1.5*cos(deg(5*x))};
\addlegendentry{$A_c \cos(2\pi f_c t)$}

% Modulated signal Ac*m(t)*cos(2*pi*f*t)
\addplot [domain=0:2*pi, samples=100, red, name path=am] {1.5*sin(deg(x))*cos(deg(5*x))};
\addlegendentry{$s_{DSB}(t)$}

% Adding dashed lines for envelops
\addplot [domain=0:2*pi, samples=100, dashed, name path=upper] {1.5*sin(deg(x))};
\addplot [domain=0:2*pi, samples=100, dashed, name path=lower] {-1.5*sin(deg(x))};

% Filling the envelope
\addplot[fill=gray!30] fill between[of=upper and lower];

\end{axis}
\end{tikzpicture}

\begin{tikzpicture}[>=Stealth]
    % Blocks
    \node (m) at (0,0) {$m(t)$};
    \node[draw, right=2cm of m] (multiplier) {Multiplier};
    \node[draw, right=2cm of multiplier] (carrier) {$A_c\cos(2\pi f_ct)$};
    \node[right=2cm of carrier] (s_dsb) {$s_{DSB}(t)$};
    
    % Arrows
    \draw[->] (m) -- (multiplier);
    \draw[->] (multiplier) -- (carrier);
    \draw[->] (carrier) -- (s_dsb);

    % Labels
    \node[above=0.5mm of multiplier] {Modulating signal};
    \node[above=0.5mm of carrier] {Carrier frequency};
\end{tikzpicture}

\begin{tikzpicture}
    \begin{axis}[
        title={Frequency Spectrum of $s_{DSB}(t)$},
        xlabel={Frequency},
        ylabel={Magnitude},
        xtick={-2,-1,0,1,2},
        ytick=\empty,
        ymin=0,
        ymax=2,
        grid=major,
        axis lines=middle,
        width=10cm,
        height=4cm,
        xticklabels={$-f_m$,$-f_c$,$0$,$f_c$,$f_m$},
        no markers,
        every axis plot/.append style={ultra thick}
    ]
    
    % Lower sideband
    \addplot+[sharp plot] coordinates {(-1.5,0) (-1,1) (-0.5,0)};
    \node at (axis cs:-1,1.2) {$M(f-f_c)$};
    
    % Upper sideband
    \addplot+[sharp plot] coordinates {(0.5,0) (1,1) (1.5,0)};
    \node at (axis cs:1,1.2) {$M(f+f_c)$};
    \end{axis}
\end{tikzpicture}

The DSB signal \( s_{DSB}(t) \) after being filtered by the bandpass filter (BPF) retains its form. The coherent detection is implemented by multiplying \( s_{DSB}(t) \) with a synchronized carrier \( 2\cos(2\pi f_c t) \). This results in a signal that contains the original message signal \( m(t) \) and a high-frequency component at \( 2f_c \). The low-pass filter (LPF) then removes the high-frequency component, leaving only the message signal \( m(t) \).
Using Fourier analysis, the multiplication in the time domain corresponds to a convolution in the frequency domain. Therefore, the message signal spectrum \( M(f) \) is convolved with a delta function at \( f_c \) due to the coherent carrier signal. The BPF ensures that only frequencies near the carrier frequency are passed through. After demodulation and filtering, the spectrum is centered back to baseband, recovering the message signal.
La duplicazione dello spettro giustifica il termine con cui si definisce la modulazione DSB.
Possiamo definire $B = f_c + W$ come la banda di trasmissione se consideriamo il massimo o $B = 2W$ se consideriamo quanto occupa.
Nella radio AM si ha una carrier frequency tra i 500 e i 1600 kHz e una $W$ tra i 5 e i 10 kHz.
Il segnale è pari (in modulo) a causa della simmetria hermitiana, infatti essendo $m(t)$ reale, $M(-f) = M^*(f)$ e quindi $|M(-f)| = |M(f)|$.

\begin{tikzpicture}
    \begin{axis}[
        width=15cm,
        height=5cm,
        axis lines=center,
        xlabel={$f$},
        ylabel={$V(f)$},
        xlabel style={at=(current axis.right of origin), anchor=west},
        ylabel style={at=(current axis.above origin), anchor=south},
        xtick={-2,0,2},
        xticklabels={$-2f_c$, $0$, $2f_c$},
        ytick=\empty,
        ymin=0,
        ymax=2,
        xmin=-3,
        xmax=3,
        no markers,
        every axis plot/.append style={ultra thick}
    ]
    % Draw the triangle peaks
    \addplot+[sharp plot, red] coordinates {(-2.5,0) (-2,1.5) (-1.5,0)};
    \addplot+[sharp plot, red] coordinates {(1.5,0) (2,1.5) (2.5,0)};

    % Draw the message signal spectrum
    \addplot+[sharp plot, green] coordinates {(-0.5,0) (0,1) (0.5,0)};
    
    % Annotations
    \node at (axis cs:-2,1.7) {$-M(f)$};
    \node at (axis cs:0,1.2) {$A_cM(f)$};
    \node at (axis cs:2,1.7) {$M(f)$};
    
    % Draw the bandwidth lines and label them
    \draw[dashed] (axis cs:-0.5,0) -- (axis cs:-0.5,1);
    \draw[dashed] (axis cs:0.5,0) -- (axis cs:0.5,1);
    \node at (axis cs:0.5,-0.2) {$W$};
    \node at (axis cs:-0.5,-0.2) {$-W$};
    \end{axis}
\end{tikzpicture}


La ricostruzione del segnale originario richiede tre step:
\begin{itemize}
  \item Filtraggio passa-banda: il segnale ricevuto è filtrato tramite un filtro passa-anda, centrato in \( f_c \), per rimuovere le componenti fuori banda.
  \item Demodulazione: il segnale modulato viene moltiplicato per un secondo coseno, ottenendo oltre al segnale originario anche due contributi in \(\pm 2f_c \)
  \item Filtraggio passa-basso: il segnale demodulato viene filtrato con un filtro passa basso di ampiezza W, per rimuovere la componente ad alta frequenza.
\end{itemize}

\begin{equation}
    v(t) = s_{DSB}(t) \cdot 2\cos(2\pi f_c t) = A_c m(t) \cdot 2\cos^2(2\pi f_c t) = A_c m(t) + A_c m(t) \cos(4\pi f_c t)
\end{equation}

L'ipotesi $f_c^{(t)} = f_c^{(r)}$ garantisce la possibilità di effettuare il passaggio trigonometrico e dunque riscostruire il segnale originario senza alcuna distorsione. Tuttavia a prescindere dalla bontà dell'oscillatore utilizzato non è possibile ottenere una sincronizzazione perfetta senza alcuna logica aggiuntiva, un reale sistema di modulazione AM prevede tale meccanismo.

\section*{Analog Quadrature Amplitude Modulation}

Per raddoppiare la quantità di informazioni trasmesse all'interno della stessa banda si può sfruttare l'ortogonalità della funzione seno e coseno, in modo da trasmettere contemporaneamente due segnali, occupando in modo ottimale la banda a disposizione. Tale modulazione è detta QAM, il termine quadrature indica lo shift di $\pi/2$ tra le due funzioni portant (seno e coseno), mentre amplitude ricorda che si tratta comunique di una modulazione in cui l'informazione è trasportata interamente dall'ampiezza del segnale. L'ampiezza del segnale portante è infatti modulata in modo proporzionale all'ampiezza del segnale modulante, contente l'informazione da trasmettere.





The composite QAM signal \( s(t) \) can be represented as the sum of two DSB-modulated signals, one modulated with a cosine (the in-phase component) and the other with a sine (the quadrature-phase component):
\begin{equation}
s_{QAM}(t) = A_{c} m_1(t) \cos(2\pi f_c t) - A_{c} m_2(t) \sin(2\pi f_c t)
\end{equation}

Where:
\begin{itemize}
    \item \( m_1(t) \) and \( m_2(t) \) are the message signals for the in-phase and quadrature-phase components, respectively.
    \item \( f_c \) is the carrier frequency.
\end{itemize}


The demodulation of the QAM signal is performed by multiplying the composite signal \( s(t) \) with \( 2\cos(2\pi f_c t) \) and \( -2\sin(2\pi f_c t) \) to retrieve the in-phase \( s_I(t) \) and quadrature-phase \( s_Q(t) \) components, respectively. These products are then passed through low-pass filters to extract the original message signals \( m_1(t) \) and \( m_2(t) \).

La demodulazione è fatta nella seguente maniera:
\begin{equation}
    v_I(t) = s(t) \cdot 2\cos(2\pi f_c t) = A_{c} m_1(t) + A_{c} m_1(t) \cos(4\pi f_c t) - A_{c} m_2(t) \sin(4\pi f_c t)
\end{equation}
\begin{equation}
    v_Q(t) = s(t) \cdot -2\sin(2\pi f_c t) = A_{c} m_2(t) - A_{c} m_1(t) \cos(4\pi f_c t) - A_{c} m_2(t) \sin(4\pi f_c t)
\end{equation}

Le componenti ad alta frequenza sono filtrate dal filtro passa-basso, ottenendo così i segnali originali \( m_1(t) \) e \( m_2(t) \).
Lo spettro del segnale ottenuto tramite moduazione non risulta più simmetrico rispetto a $f_c$, tuttavia essendo un segnale reale resta valida la simmetria rispetto all'asse delle ordinate.
La mancanza di simmetria rispetto a $f_c$ implica la non esistenza di un segnale reale in grado di rappresentare lo spettro in banda base, in quanto verrebbe meno la simmetria rispetto alla frequenza 0.
Sebbene questa mancanza non precluda l'utilizzo della modulazione QAM, nella pratica risulta più semplice lo studio di un segnale in banda base.
\begin{itemize}
    \item La carrier frequency \( f_c \) non aggiunge informazione al segnale, ma viene utilizzata per traslare il segnale in frequenza.
    \item Il criterio di Nyquist richiede un comportamente ad una frequnza doppia rispetto alla banda del segnale, quindi nel caso di segnale passa-banda tale valore può essere molto elevato. Nel caso di segnale in banda base la frquenza di campionamente può essere nettamente inferiore.
\end{itemize}

Il criterio di Nyquist stabilisce che \( T \leq \frac{1}{2B} \), dove \( T \) è il periodo di campionamento e \( B \) è la banda del segnale, e quindi \( f_s \geq 2B \). Quindi nel nostro caso, $f_s \geq 2(f_c + W)$.
Un segnale è definito in passa banda quando la propria energia è concentrata all'interno di una banda $2W$ centrata attorno a una carrier frequency $f_c$, con il vincolo $f_c >> W$.  


\subsection*{Baseband and Passband Signal Spectra}
In passband systems, a message signal with a certain bandwidth \( B \) Hz is modulated onto a carrier signal with a frequency of \( f_c \) Hz. The baseband signal spectrum represents the frequency content of the message signal, and it is usually centered at zero frequency. When the signal is modulated onto the carrier frequency, the spectrum shifts to be centered around \( f_c \), creating a passband signal spectrum.

The equation for the modulated passband signal can be expressed as:
\begin{equation}
s(t) = A_{c1} m_1(t) \cos(2\pi f_c t) - A_{c2} m_2(t) \sin(2\pi f_c t)
\end{equation}
where \( m_1(t) \) and \( m_2(t) \) represent the in-phase and quadrature components of the message signal, and \( A_{c1} \), \( A_{c2} \) are the amplitudes of the carrier.

The process of demodulation involves retrieving the original baseband signal from the modulated passband signal. The demodulated signal \( v(t) \) can be approximated by the original message signal when the carrier frequency is significantly higher than the bandwidth of the message signal, which allows for the use of simple demodulation techniques.




The received signal can be represented by the equation:
\begin{equation}
    v(t) = s(t) \cdot \cos(2\pi f_c t) + n(t)
\end{equation}

Where \( s(t) \) is the transmitted signal and \( n(t) \) is the noise.

The transmitted signal \( s(t) \) is a modulated signal with the form:
\begin{equation}
    s(t) = A_{c1} m_1(t) \cos(2\pi f_c t) - A_{c2} m_2(t) \sin(2\pi f_c t)
\end{equation}

The terms in this equation are defined as follows:
\begin{itemize}
    \item \( A_{c1} \) and \( A_{c2} \) are the amplitudes of the carrier.
    \item \( m_1(t) \) and \( m_2(t) \) are the baseband signals that modulate the carrier.
    \item \( f_c \) is the carrier frequency.
\end{itemize}

The term \( v(t) \) can also be expanded and approximated, assuming that \( m_1(t) \) and \( m_2(t) \) are limited to a bandwidth \( B \), and the carrier frequency \( f_c \) is much greater than \( B \). This approximation leads to:
\begin{align}
    v(t) & = \frac{A_{c1}}{2} m_1(t) + \frac{A_{c1}}{2} m_1(t) \cos(4\pi f_c t) + \frac{A_{c2}}{2} m_2(t) \sin(4\pi f_c t) + n(t) \\
    & = \frac{A_{c1}}{2} m_1(t) + \underbrace{A_{c1} m_1(t) \cos(4\pi f_c t)}_{\text{High frequency terms}} + \underbrace{A_{c2} m_2(t) \sin(4\pi f_c t)}_{\text{High frequency terms}} + n(t)
\end{align}

The high frequency terms can be removed using a low-pass filter, which leaves the term \( \frac{A_{c1}}{2} m_1(t) \) and the noise \( n(t) \). This is the essence of coherent demodulation, where the original message signal is recovered from the received signal.


\section*{Inviluppo Complesso di un Segnale in Banda Passante}

La maggior parte dei sistemi di comunicazione opera in banda passante. Il segnale trasmesso \( s(t) \), concentrato in una banda di frequenza \( 2B \) e centrato attorno a una frequenza portante \( f_c \), risiede ben al di sopra della corrente continua (DC) o frequenza zero. Per un segnale in banda passante, vale la condizione \( f_c \gg 2B \), indicando che la frequenza portante è molto maggiore del doppio della larghezza di banda del segnale.

L'assenza di un segnale in banda base ha portato all'adozione dell'inviluppo complesso, definito come:
\begin{equation}
    \tilde{s}(t) = s_I(t) + js_Q(t) = A(t)e^{j\phi(t)}
\end{equation}
\begin{equation}
    s(t) = \Re\{\tilde{s}(t)e^{j2\pi f_c t}\} = s_I(t)\cos(2\pi f_c t) - s_Q(t)\sin(2\pi f_c t)
\end{equation}

Questo inviluppo complesso permette di rappresentare \( s(t) \) come la parte reale del prodotto del suo inviluppo complesso per un esponenziale complesso alla frequenza portante, facilitando l'analisi e la modulazione del segnale nei sistemi di comunicazione digitale. Per esempio, nella modulazione AM si ottiene:
\begin{equation}
    \tilde{s}_{AM}(t) = m(t) = A(t)e^{j\phi(t)}
\end{equation}
dove \( \phi(t) \) assume valore \( 0 \) o \( \pi \), essendo assente la componente immaginaria.

La sincronizzazione del clock tra trasmettitore e ricevitore è cruciale per una ricostruzione perfetta dell'informazione trasmessa, mantenendo la caratteristica di ortogonalità perfetta tra seno e coseno, necessaria per separare i due messaggi una volta modulati.

L'inviluppo complesso rappresenta un modello equivalente in banda base che facilita l'analisi e l'elaborazione dei segnali in banda passante come se fossero segnali in banda base, offrendo numerosi vantaggi:
\begin{itemize}
    \item Il modello in banda base è più semplice da studiare, poiché elimina gli effetti della frequenza portante dal modello del segnale, semplificando l'analisi matematica e la comprensione delle proprietà del segnale.
    \item La simulazione numerica di un modello in banda base richiede meno risorse computazionali rispetto a un modello in banda passante, grazie alla minor larghezza di banda necessaria, riducendo quindi i costi computazionali.
    \item Di conseguenza, anche la frequenza di campionamento è inferiore per i modelli in banda base, il che si traduce in tassi di trasmissione dati ridotti, vantaggioso per l'elaborazione e lo stoccaggio del segnale digitale.
    \item Il modello in banda base è spesso la base per un'implementazione digitale di un sistema di comunicazione in banda passante, facilitando l'applicazione delle tecniche di elaborazione del segnale digitale, fondamentali nelle comunicazioni moderne.
\end{itemize}

\section*{Comunicazioni Analogiche: Modulazione di Frequenza (FM)}

Nella modulazione di frequenza (FM), il messaggio è incorporato nella fase del segnale \( \phi(t) \). Il segnale FM \( S_{FM}(t) \) può essere espresso come:
\begin{equation}
    S_{FM}(t) = A_c \cos\left(2\pi f_c t + 2\pi k_f \int_{-\infty}^{t} m(\tau) d\tau \right)
\end{equation}

La fase \( \phi(t) \) del segnale FM è quindi data dall'integrale del segnale del messaggio \( m(t) \), scalato dalla sensibilità alla deviazione di frequenza \( k_f \):
\begin{equation}
    \phi(t) = 2\pi f_c t + 2\pi k_f \int_{-\infty}^{t} m(\tau) d\tau
\end{equation}

L'indice di modulazione \( m_f \) è definito come il rapporto tra la deviazione di frequenza e la larghezza di banda del messaggio \( B_m \):
\begin{equation}
    m_f = \frac{\Delta f}{B_m}
\end{equation}

La rappresentazione complessa del segnale FM può essere ottenuta utilizzando la formula di Eulero:
\begin{equation}
    \tilde{s}_{FM}(t) = A_c e^{j\phi(t)}
\end{equation}

Espandendo \( \phi(t) \) otteniamo:
\begin{equation}
    \tilde{s}_{FM}(t) = A_c e^{j\left(2\pi f_c t + 2\pi k_f \int_{-\infty}^{t} m(\tau) d\tau\right)}
\end{equation}

Questa rappresentazione è particolarmente utile per l'analisi dei segnali FM nel contesto del trattamento digitale dei segnali.

I vantaggi della modulazione in frequenza includono:
\begin{itemize}
    \item Minore sensibilità ai disturbi, migliorando così la qualità del segnale ricevuto.
    \item Maggiore efficienza energetica, poiché il segnale informativo non richiede potenza aggiuntiva a quella della portante.
    \item L'ampliezza costante permette l'uso di amplificatori semplificati in fase di trasmissione, evitando la necessità di mantenere l'amplificatore nella zona lineare, come sarebbe necessario con le modulazioni di ampiezza variabile (AM).
    \item Possibilità di configurare la modulazione per ottimizzare il compromesso tra qualità della trasmissione e banda occupata.
\end{itemize}

\begin{tikzpicture}
\begin{axis}[
    axis lines=middle,
    xlabel={$t$},
    ylabel={Amplitude},
    xtick=\empty,
    ytick=\empty,
    xmin=0, xmax=10,
    ymin=-2, ymax=2,
    domain=0:10,
    samples=500,
    smooth,
    width=15cm, height=8cm
]


% Carrier Signal
\addplot [red, thick] {sin(2 * pi * 50 * x)};

% Baseband Input Signal
\addplot [green, thick, domain=0:10] {sin(pi * 10*x)};

% FM Waveform (simplified representation)
\addplot [blue, thick] {sin(2 * pi * 50 * x + 1000*sin(pi * 10*x))};

\end{axis}
\end{tikzpicture}

\subsection*{FM Signal with a Modulating Sinusoid}

% Modulating sinusoid
Sia \( m(t) \) una sinusoide data da \( m(t) = V_m \cos(2\pi f_m t) \). Il segnale FM diventa:
\[
s_{FM}(t) = A_c \cos \left( 2\pi f_c t + 2\pi k_f \int_{-\infty}^{t} V_m \cos(2\pi f_m \tau) d\tau \right)
\]
che si semplifica in:
\[
s_{FM}(t) = A_c \cos \left(2\pi f_c t + m_f \sin(2\pi f_m t) \right)
\]
% Complex envelope
L'inviluppo complesso sarà:
\[
\hat{s}_{FM}(t) = A_c e^{j m_f \sin(2\pi f_m t)}
\]


Per ottenere il coefficiente di Foruier di deve risolvere l'integrale:





To compute the Fourier series coefficients for the complex envelope of the FM signal given by \( \hat{s}_{FM}(t) = A_c e^{j m_f \sin(2\pi f_m t)} \) and using your specified Fourier series integral formula, we will follow these steps:

%The integral formula to find the Fourier series coefficients \( c_n \) of the signal \( x(t) \) over one period \( T_0 \) is given by:
%\[ c_n = \frac{1}{T_0} \int_{-\frac{T_0}{2}}^{\frac{T_0}{2}} x(t) e^{-j 2\pi f_0 n t} dt \]

La formula dell'integrale per trovare i coefficienti della serie di Fourier \( c_n \) del segnale \( x(t) \) su un periodo \( T_0 \) è data da:
\[ c_n = \frac{1}{T_0} \int_{-\frac{T_0}{2}}^{\frac{T_0}{2}} x(t) e^{-j 2\pi f_0 n t} dt \]

Per la modulazione FM, dove \( f_0 = f_m \) (la frequenza di modulazione) e \( T_0 = \frac{1}{f_m} \), i coefficienti per l'inviluppo complesso \( \hat{s}_{FM}(t) \) diventano:
\[ c_n = \frac{1}{T_0} \int_{-\frac{T_0}{2}}^{\frac{T_0}{2}} A_c e^{j m_f \sin(2\pi f_m t)} e^{-j 2\pi f_m n t} dt \]

Applicando l'espansione di Jacobi-Anger, sappiamo che:
\[ e^{j z \sin \theta} = \sum_{k=-\infty}^\infty J_k(z) e^{j k \theta} \]

Quindi, con \( z = m_f \) e \( \theta = 2\pi f_m t \), otteniamo:
\[ e^{j m_f \sin(2\pi f_m t)} = \sum_{k=-\infty}^\infty J_k(m_f) e^{j k 2\pi f_m t} \]


Sostituendo l'espansione nell'integrale per \( c_n \):
\[ c_n = \frac{A_c}{T_0} \int_{-\frac{T_0}{2}}^{\frac{T_0}{2}} \sum_{k=-\infty}^\infty J_k(m_f) e^{j k 2\pi f_m t} e^{-j 2\pi f_m n t} dt \]

Scambiando la somma e l'integrale (supponendo convergenza uniforme), e notando che \( e^{j (k-n) 2\pi f_m t} \) è periodico con periodo \( T_0 \), otteniamo:
\[ c_n = \frac{A_c}{T_0} \sum_{k=-\infty}^\infty J_k(m_f) \int_{-\frac{T_0}{2}}^{\frac{T_0}{2}} e^{j (k-n) 2\pi f_m t} dt \]



L'integrale:
\[ \int_{-\frac{T_0}{2}}^{\frac{T_0}{2}} e^{j (k-n) 2\pi f_m t} dt \]
è non nullo solo quando \( k = n \), in tal caso è uguale a \( T_0 \). Altrimenti è nullo a causa dell'ortogonalità delle funzioni esponenziali su un periodo completo.


Quindi:
\[ c_n = A_c J_n(m_f) \]





La rappresentazione in serie di Fourier dello spettro dell'inviluppo complesso del segnale FM è quindi data da:
\[
\hat{S}_{FM}(t) = A_c \sum_{n=-\infty}^{\infty} J_n(m_f) e^{j 2\pi n f_m t}
\]
% FM signal spectrum representation
Il segnale FM può essere rappresentato come:
\[
s_{FM}(t) = \text{Re}\{ \hat{S}_{FM}(t) e^{j 2\pi f_c t} \} = A_c \sum_{n=-\infty}^{\infty} J_n(m_f) \cos(2\pi (f_c + n f_m) t)
\]
mostrando come la frequenza della portante sia alterata dalle frequenze del segnale di modulazione, con l'ampiezza di ciascuna componente data dai valori della funzione di Bessel.

Ogni componente sinusoidale ha una frequenza che è un multiplo intero della frequenza di modulazione \( f_m \).
Queste componenti sono chiamate "sidebands", e la loro ampiezza è determinata dalle funzioni di Bessel \( J_n(m_f) \).
La frequenza della portante appare come il picco centrale nello spettro (per \( n = 0 \)), con la sua ampiezza modulata da \( J_0(m_f) \).

In generale, l'ampiezza delle funzioni di Bessel (e quindi delle bande laterali) diminuisce all'aumentare di \( |n| \), anche se questo decadimento non è necessariamente monotono.
Il pattern esatto dipende dal valore dell'indice di modulazione \( m_f \), un indice di modulazione più alto distribuisce più energia nelle bande laterali di ordine superiore, allargando la banda del segnale FM,
infatti si può vedere graficamente come più è l'indice di modulazione, più è grande il numero di funzioni di Bessel di cui devo tenere conto quando rappresento lo spettro del segnale FM.
Lo spettro è simmetrico rispetto alla frequenza della portante perché \( J_{-n}(m_f) = (-1)^n J_n(m_f) \). Quindi, per ogni componente di frequenza positiva, c'è una corrispondente componente di frequenza negativa con la stessa ampiezza ma potenzialmente fase diversa (a seconda che \( n \) sia dispari o pari).
La banda di Carson approssima la larghezza di banda del segnale FM a:
\[
    B_{FM} \approx 2(\Delta f + f_{m}) = 2(m_f + 1) f_{m}
\]
\[
    \Delta f \coloneqq \text{max} \{ | f_d(t) | \} = k_f \cdot \text{max} \{ | m(t) | \}
\]
% TODO: qual è la differenza tra f_{max} e f_m e B?
Dove \( \Delta f \) è la deviazione massima della frequenza, \( f_{m} \) è la frequenza massima del signal modulante e \( m_f \) è l'indice di modulazione.
Questa regola stima la banda in cui è concentrata la maggior parte dell'energia del segnale FM.
Ogni segnale modulato in frequenza ha un numero infinito di bande laterali e quindi una banda infinita. Ma la maggior parte dell'energia (98\% o più) è concentrata all'interno della banda definita dalla regola di Carson. Per la radio FM mono commerciale:
\begin{align*}
f_m &= 15 \text{ kHz (high quality audio)}, \\
\Delta f &= 75 \text{ kHz}, \\
m_f &= 5, \\
B_{FM} &\approx 180 \text{ kHz}.
\end{align*}

\subsection*{FM Receiver}

% Complex envelope of the received signal
Neglecting the effect of noise and channel, the complex envelope of the received signal is
\[ \hat{v}(t) = A_c e^{j2\pi k_f \int_{-\infty}^{t} m(\tau) d\tau} \]

% Recovering the modulating signal
The modulating signal can be recovered by differentiating the phase of \( \hat{v}(t) \):
\[ \hat{m}(t) = \frac{1}{2\pi k_f} \frac{d}{dt} \Delta \hat{v}(t) \]
where \( \Delta \hat{v}(t) \) is the phase deviation of the received signal.

% Conceptual FM baseband receiver block diagram
The conceptual block diagram for an FM baseband receiver can be illustrated using a low-pass filter (L), followed by a differentiator and a demodulator to recover the message signal \( m(t) \).

% Including the block diagram in TikZ would be the next step if you want a graphical representation of the FM baseband receiver as well.
\section*{SDR (Software-Defined Radio) Concept}

% SDR concept explained
Software-defined radio (SDR) leverages modern computing power and analog-to-digital converters (ADCs) to implement radio components such as modulators, demodulators, and tuners in software, which traditionally were implemented in hardware.

\section*{SDR Paradigm}

% Details of SDR processing
In an SDR system, the incoming signal is converted to a digital format and then processed digitally. This allows for the radio hardware to be largely programmable and configurable by software, making it possible to update and adapt the system to new modulation formats, algorithms, and applications easily.

% Benefits of SDR
SDR platforms offer versatility across various products and applications, potentially reducing costs and either maintaining or enhancing performance.



\begin{tikzpicture}[
    block/.style={rectangle, draw, minimum height=2em, minimum width=3em},
    line/.style={draw, -Latex}
]


% create the node antenna with image imgs/antenna.pdf

% Define nodes
\node[block, align=center] (rfFrontEnd) {Flexible RF \\ Front End};
\node[block, right=0.5cm of rfFrontEnd, align=center] (adConverter) {A/D Converter};
\node[block, right=0.5cm of adConverter, align=center] (demod) {Demodulation \\ to Bband};
\node[block, right=0.5cm of demod, align=center] (decimation) {Decimation \\ Filtering};
\node[block, right=0.5cm of decimation, align=center] (carrierSync) {Carrier \& Timing \\ Synchronisation};
\node[block, right=0.5cm of carrierSync, align=center] (baseband) {Baseband \\ Processing};
% before rfFrontEnd, and upper of it of 0.5cm
\node[inner sep=0pt, outer sep=0pt, above left=0.25cm and 0.05cm of rfFrontEnd] (antenna) {\includegraphics[width=1cm]{imgs/antenna.pdf}};

% Connect blocks with arrows
\draw[line] (rfFrontEnd) -- (adConverter);
\draw[line] (adConverter) -- (demod);
\draw[line] (demod) -- (decimation);
\draw[line] (decimation) -- (carrierSync);
\draw[line] (carrierSync) -- (baseband);
% Draw antennae
\draw (rfFrontEnd.west) -- ++(-0.5, 0) -- ++(0,0.5);


% Group hardware and software parts
\node[draw, dashed, fit=(rfFrontEnd) (adConverter) (demod) (decimation), inner sep=0.2cm, label=above:RTL-SDR Hardware] (hardware) {};
\node[draw, dashed, fit=(carrierSync) (baseband), inner sep=0.2cm, label=above:MATLAB / Simulink design (running on computer)] (software) {};

% Add frequency range and sampling frequency annotations
\node[align=left, below=0.5cm of hardware.south] (freqRange) {Range: \\ 25MHz -- 1.75GHz};
\node[align=left, below=0.5cm of software.south] (samplingFreq) {Sampling frequency (\(f_s\)): \\ up to around 2.8MHz};

% Baseband output symbol
%\node[right=0.5cm of baseband] (output) {\(\musicalnote\)};
% create a node with a background image imgs/notes.svg
\node[inner sep=0pt, outer sep=0pt, right=0.5cm of baseband] (output) {\includegraphics[width=1cm]{imgs/notes.pdf}};
\draw[line] (baseband) -- (output);
\end{tikzpicture}
\











\section*{SDR – Historical Milestones}

% Timeline of SDR development
Notable historical milestones in the development of SDR include:
\begin{itemize}
    \item \textbf{1984:} E-Systems Inc. (Raytheon) coined the term 'software radio.'
    \item \textbf{1991:} SPEAKeasy, the first military radio program specifically requiring software radios, is initiated.
    \item \textbf{1992:} J. Mitola publishes a paper on 'Software Radios: Survey, Critical Analysis and Future Directions.'
    \item \textbf{2005:} Ettus Research introduces the USRP, the first commercial SDR, setting a standard for open-source, hardware GNU radio.
    \item \textbf{2011:} A Finnish hacker discovers that the Realtek RTL2832U chip could be forced to operate in test mode, outputting raw I/Q data, marking a significant point in SDR accessibility.
\end{itemize}


\section*{Programmable Hardware for SDR}

% Types of programmable hardware used in SDR
There are various types of programmable hardware that facilitate the development of SDR:
\begin{enumerate}
    \item \textbf{Reconfigurable Radio:} Typically FPGA-based, these devices offer a high degree of flexibility and are often used for prototyping and research.
    \item \textbf{Programmable Radio:} Devices like DSPs or hybrids of FPGA/DSP are used for more specific applications where dedicated processing is required.
    \item \textbf{Fully Software Radio:} Utilizes General-Purpose Processors (GPP) like the Intel Core i7 for all radio functions, leveraging the processing power of modern CPUs.
\end{enumerate}

\section*{Telecom Services in the 0.1-2.5 GHz Band}

% Telecom frequency bands
The frequency spectrum from 0.1 to 2.5 GHz is populated with various telecom services, each occupying a specific band:
\begin{itemize}
    \item \textbf{AM Radio} is found in the lower frequencies, typically below 1 MHz.
    \item \textbf{FM Radio and TV Broadcasts} generally occupy the spectrum from 88 to 108 MHz and 470 to 862 MHz, respectively.
    \item \textbf{Mobile Services} such as 2G, 3G, and 4G span across multiple bands, from 700 MHz to over 2 GHz.
    \item \textbf{Wi-Fi} operates in bands around 2.4 GHz and 5 GHz.
    \item \textbf{Satellite Communications} and \textbf{GPS} also have designated bands within this range.
\end{itemize}
This band is essential for communication services and is carefully regulated to prevent interference between different services.

\section*{RTL-SDR Architecture}

% Detailing the components of RTL-SDR
The RTL-SDR is a popular SDR device that comprises several key components:
\begin{itemize}
    \item \textbf{R820T DTV Tuner:} Responsible for selecting the desired signal frequency band from the RF spectrum.
    \item \textbf{28.8MHz Clock Crystal:} Provides a stable reference frequency for the tuner and demodulator.
    \item \textbf{RTL2832U COFDM Demodulator:} Demodulates the digital signal from the tuner.
    \item \textbf{MCX Antenna Connector:} Allows for the connection of various types of antennas.
    \item \textbf{Serial EEPROM:} Stores the configuration data for the USB device.
    \item \textbf{USB 2.0 Interface:} Facilitates the connection with computers for further processing.
    \item Additional components such as a \textbf{Power LED}, \textbf{ESD Diode}, and \textbf{IR Sensor} for user interfacing and device protection.
\end{itemize}

\section*{SDR Block Diagram for an FM Receiver}

% Explanation of the SDR block diagram for FM reception
The block diagram for an FM receiver using RTL-SDR hardware typically involves the following processing stages:
\begin{enumerate}
    \item \textbf{RF Antenna:} Captures the FM signal from the air.
    \item \textbf{Flexible RF Front End:} Filters and amplifies the RF signal from the antenna.
    \item \textbf{Analogue to Digital Converter (ADC):} Converts the analog RF signal to a digital format.
    \item \textbf{Demodulation to IQ Band:} The digital signal is demodulated into I (In-phase) and Q (Quadrature) components.
    \item \textbf{Decimation Filtering:} Reduces the sampling rate and noise, retaining the signal of interest.
    \item \textbf{Carrier \& Timing Synchronization:} Synchronizes the received signal with the local reference for accurate demodulation.
    \item \textbf{Baseband Processing:} This may involve MATLAB/Simulink running on a computer to process the demodulated signal.
    \item \textbf{Baseband Output:} The final audio signal is output, ready for listening.
\end{enumerate}
The signal processing range for this system is from 25MHz to 1.75GHz with a sampling frequency (IF) up to around 2.8MHz.


\section*{RTL-SDR Receiving Steps}

% Detailed steps of the RTL-SDR superheterodyne receiver chain
The RTL-SDR device functions as a superheterodyne receiver, processing signals through the following steps:
\begin{enumerate}
    \item \textbf{Mixing to Intermediate Frequency:} Incoming signals are mixed down to an intermediate frequency (IF) of 3.57 MHz, which allows for easier signal processing and better selectivity and sensitivity.
    \item \textbf{Sampling by ADC:} The IF signal is then sampled by a two-channel (I/Q) 28.8 MS/s 8-bit analog-to-digital converter (ADC), turning the analog signal into a digital format.
    \item \textbf{Digital Downconversion:} The digitized I/Q data passes through parallel paths in a digital downconversion process that includes mixing, filtering, and decimating the input signal to a user-specified rate. The maximum sampling rate after decimation can reach approximately 2.8 MS/s.
    \item \textbf{Data Transfer to Host:} Finally, the downconverted samples are transferred to the host computer via a standard USB connection for further processing and analysis.
\end{enumerate}
This process outlines the steps involved in the digital reception and conversion of RF signals in the RTL-SDR system.


\section*{RTL-SDR Block Diagram}
The process of converting an RF signal to a baseband signal in an RTL-SDR involves several steps, illustrated in the block diagram and detailed equations:
\begin{enumerate}
    \item The RF signal undergoes amplification and filtering, including active gain control and rejection of RF image frequencies.
    \item The analog signal is mixed down to an intermediate frequency (IF) of 3.57 MHz to suit the ADC sampling rate.
    \item The ADC samples the IF at 28.8 MS/s and then the signal is filtered and decimated.
    \item Digital downconversion is performed using quadrature numerically-controlled oscillators (NCOs) to generate in-phase (I) and quadrature (Q) components.
    \item The digitized I/Q data is then resampled and synchronized, outputting to the USB port at a lower rate (up to 2.8 MS/s) and passed to software like MATLAB/Simulink for further processing.
\end{enumerate}
This process effectively moves the signal from the RF spectrum into the digital domain, where it can be manipulated and analyzed with much greater flexibility.

The RF signal is converted to baseband through the following mathematical steps:

\begin{enumerate}
    \item The received RF signal at frequency \( f_c \) is mixed with a local oscillator signal to shift its frequency down to the intermediate frequency (IF):
    \[ S_{IF}(t) = S_{RF}(t) \cdot \cos(2\pi f_{LO} t) \]
    where \( f_{LO} \) is the local oscillator frequency chosen such that \( f_{IF} = |f_{LO} - f_c| \).

    \item The resulting IF signal is then sampled by an ADC at a sampling frequency \( f_s \):
    \[ S_{ADC}(n) = S_{IF}(nT_s) \]
    where \( T_s = \frac{1}{f_s} \) is the sampling period, and \( n \) is the discrete sample index.

    \item The digitized signal undergoes digital downconversion (DDC), which includes mixing, filtering, and decimation to reduce the sample rate to a lower intermediate frequency:
    \[ S_{DDC}(t) = S_{ADC}(t) \cdot e^{-j2\pi f_{IF} t} \]
    followed by low-pass filtering to isolate the baseband component.

    \item The complex baseband signal \( S_{BB}(t) \) is obtained:
    \[ S_{BB}(t) = \text{LPF}\{S_{DDC}(t)\} \]
    which represents the in-phase (I) and quadrature (Q) components of the original RF signal.
\end{enumerate}
These steps translate the high-frequency RF signal into a lower frequency baseband signal that can be further processed by software applications for various communication tasks.
\section*{Classes and Objects in MATLAB}

% Explanation of OOP in MATLAB
Object-Oriented Programming (OOP) in MATLAB allows for the creation of classes, which are blueprints for objects that encapsulate both data structure and behavior. A class defines properties (data fields) and methods (functions) for the objects created from the class.

% Definition of an object in MATLAB
In MATLAB, an object is an instance of a class. To use an object effectively, one must understand the properties and methods of its class. Operations performed on objects are limited to the methods defined in their respective classes.

% The 'step' method in MATLAB classes
MATLAB provides the \texttt{step} method for many of its system objects, which executes the primary algorithm of the object it is called upon. The behavior of \texttt{step} varies with each class and is designed to work on the data stored within the object.

\subsection*{Example: Creating a Spectrum Analyzer in MATLAB}

% Example of creating a spectrum analyzer object
A spectrum analyzer object in MATLAB can be created using the \texttt{dsp.SpectrumAnalyzer} class:
\begin{verbatim}
    obj = dsp.SpectrumAnalyzer;
\end{verbatim}
This command initializes a new object of the \texttt{dsp.SpectrumAnalyzer} class.

% Setting properties of an object
Properties of the object can be defined during creation or by direct assignment:
\begin{verbatim}
    obj = dsp.SpectrumAnalyzer('Name', 'Spectrum Analyzer');
    % or
    obj.Name = 'Spectrum Analyzer';
\end{verbatim}
In this example, the \texttt{'Name'} property of the spectrum analyzer object is set to \texttt{'Spectrum Analyzer'}.


\subsection*{Example: Creating a Spectrum Analyzer in MATLAB (Continued)}

% Continued example of creating a spectrum analyzer with property settings
To create a spectrum analyzer with specific properties, use the following syntax:
\begin{verbatim}
    scope = dsp.SpectrumAnalyzer(...
        'Name', 'Spectrum Analyzer', ...
        'Title', 'Spectrum', ...
        'SpectrumType', 'Power', ...
        'FrequencySpan', 'Span and center frequency', ...
        'CenterFrequency', 0, ...
        'Span', 600, ...
        'ShowLegend', true, ...
        'SampleRate', Fs);
\end{verbatim}
This code snippet sets various properties of the spectrum analyzer object, such as the display name, title, type of spectrum, frequency span, and sampling rate.

% Using the step method to display the spectrum
The \texttt{step} function is used to display the frequency spectrum of the input signal:
\begin{verbatim}
    step(obj, X)
\end{verbatim}
Here, \( X \) represents the input signal to the spectrum analyzer, and it can be a matrix where each column is treated as an independent channel.

\section*{Before Starting with RTL-SDR in MATLAB}

% Preparatory steps for using RTL-SDR with MATLAB
Before working with RTL-SDR in MATLAB, the following preparatory steps should be taken:

\begin{enumerate}
    \item Install the support package for RTL-SDR Radio by navigating to the MATLAB Home tab, then clicking on \texttt{Add-Ons} > \texttt{Get Hardware Support Packages}.
    \item In the Add-On Explorer, search for the \texttt{Communications Toolbox Support Package for RTL-SDR Radio}.
    \item Select the support package and click \texttt{Install}. Follow the prompts to install the necessary drivers for the RTL-SDR radio software.
\end{enumerate}
This setup ensures that MATLAB is properly configured to interface with RTL-SDR hardware and perform signal analysis tasks.

\section*{How to Install Drivers for RTL-SDR in MATLAB}

% Steps to install the RTL-SDR drivers after the MATLAB support package installation
After installing the Communications Toolbox Support Package for RTL-SDR Radio:

\begin{enumerate}
    \item Use File Explorer on your PC and navigate to the following path:
    \begin{verbatim}
    C:\ProgramData\MATLAB\SupportPackages\<version>\3P.instrset\zadig.instrset\zadig-XX.exe
    \end{verbatim}
    Here, \( <version> \) is the version of your MATLAB, and \( XX \) is the Zadig version number.
    
    \item Connect the RTL-SDR device to your PC.
    
    \item Run the Zadig application, select the RTL device from the list, and install the driver.
\end{enumerate}

\section*{Hardware Setup}

% Instructions for setting up the RTL-SDR hardware
To set up the RTL-SDR hardware:

\begin{enumerate}
    \item Connect the RTL-SDR into your computer's USB port.
    
    \item Start MATLAB and at the MATLAB command prompt, initiate the hardware setup by executing:
    \begin{verbatim}
    sdrsetup
    \end{verbatim}
    
    \item To obtain information about all radios connected to your computer, use the \texttt{sdrinfo} command:
    \begin{verbatim}
    hwinfo = sdrinfo
    \end{verbatim}
    This command provides details about the connected RTL-SDR radios, such as radio name, address, tuner name, manufacturer, and gain values.
\end{enumerate}

These steps are crucial for ensuring that the RTL-SDR hardware is properly recognized by MATLAB and is ready for signal processing tasks.

\section*{Load RTL-SDR Driver}

% Constructing the RTL-SDR System object
To interface with the RTL-SDR hardware, create a System object in MATLAB:
\begin{verbatim}
    obj_rtlsdr = comm.SDRRTLReceiver
\end{verbatim}
which results in an object with properties such as center frequency, sample rate, and samples per frame:
\begin{verbatim}
    obj_rtlsdr = 
        comm.SDRRTLReceiver with properties:
        RadioAddress: '0'
        CenterFrequency: 102500000
        EnableTunerAGC: true
        SampleRate: 2500000
        OutputDataType: 'int16'
        SamplesPerFrame: 1024
        FrequencyCorrection: 0
        EnableBurstMode: false
\end{verbatim}
These properties configure the RTL-SDR receiver to properly receive and process the RF signals.

\section*{FM Receiver — Practical Implementation}

% Block diagram description of an FM receiver using RTL-SDR
The practical implementation of an FM receiver using RTL-SDR follows a specific signal processing flow:
\begin{itemize}
    \item The analog RF signal \( S_{FM}(t) \) is first received by the SDR hardware.
    \item It is then digitally demodulated into \( \tilde{v}(k) \) at a sample rate of 228 kS/s.
    \item The demodulated signal is filtered through a low-pass filter (LPF) and then subjected to rate conversion to reduce the sample rate to 48 kS/s.
    \item Finally, the processed digital signal is analyzed using a spectrum analyzer.
\end{itemize}
This process outlines the steps from receiving the analog signal to the digital analysis, detailing the journey of the signal through the FM receiver system.

\subsection*{Example MATLAB Code}
\begin{verbatim}
    % Define System object for RTL-SDR Receiver with specified properties
    obj_rtlsdr = comm.SDRRTLReceiver('RadioAddress', '0', ...
                                     'CenterFrequency', 102500000, ...
                                     'EnableTunerAGC', true, ...
                                     'SampleRate', 2500000, ...
                                     'OutputDataType', 'int16', ...
                                     'SamplesPerFrame', 1024, ...
                                     'FrequencyCorrection', 0, ...
                                     'EnableBurstMode', false);
\end{verbatim}
This MATLAB code snippet sets up the RTL-SDR receiver with predefined settings to start receiving signals.


\section*{FM Receiver}

% Description of FM signal reception and demodulation
The complex envelope of the received FM signal is given by:
\[
\tilde{v}(t) = A_c e^{j2\pi k_f \int_{-\infty}^{t} m(\tau) d\tau}
\]
Ignoring noise and channel effects, the modulating signal \( m(t) \) can be recovered by differentiating the phase of \( \tilde{v}(t) \):
\[
\hat{m}(t) = \frac{1}{2\pi k_f} \frac{d}{dt} \arg(\tilde{v}(t))
\]
where \( \arg(\tilde{v}(t)) \) denotes the phase of \( \tilde{v}(t) \).

% Conceptual diagram of an FM baseband receiver
The conceptual diagram of an FM baseband receiver includes the differentiator \( \frac{d}{dt} \) followed by a division by \( 2\pi k_f \) to retrieve the original modulating signal \( m(t) \).

\section*{FM Receiver — Practical Implementation}

% Practical implementation steps for FM signal demodulation
In the practical implementation of an FM receiver using SDR:

\begin{itemize}
    \item The SDR output provides the complex envelope of the signal sampled at frequency \( f_s = \frac{1}{T_s} \).
    \item The signal \( \tilde{v}(k) \) at discrete times \( kT_s \) can be expressed as:
    \[
    \tilde{v}(k) = \tilde{v}(t)|_{t=kT_s} \approx A_c e^{j2\pi k_f \sum_{\ell=-\infty}^{k} m(\ell T_s) T_s}
    \]
    \item The product of two consecutive baseband samples is:
    \[
    \tilde{v}(k) \tilde{v}^*(k - 1) \approx A_c^2 e^{j2\pi k_f m(k) T_s}
    \]
    \item An estimate of \( m(k) \), the \( k \)-th sample of \( m(t) \), is then given by:
    \[
    \hat{m}(k) = \frac{1}{T_s 2\pi \Delta f} \arg(\tilde{v}(k) \tilde{v}^*(k - 1))
    \]
\end{itemize}

% Explanation of the signal processing terms and operations
These equations represent the discrete-time processing of an FM signal, demonstrating how the original modulating signal is extracted from the received waveform in a digital domain.

\section*{FM Mono}

% Information on the first FM mono transmissions
The initial FM transmissions from 1945 to 1960 were mono, meaning they consisted of a single audio channel. The modulating signal \( m(t) \) for mono transmissions was the sum of the left (\( L(t) \)) and right (\( R(t) \)) audio channels:
\[ m(t) = L(t) + R(t) \]

% Normalization and frequency deviation for FM mono
This signal was then normalized to a maximum value of 1, with a frequency sensitivity \( k_f \) of 75 kHz/V. This resulted in a maximum frequency deviation \( \Delta f \) for the FM signal given by:
\[ \Delta f = k_f \max|m(t)| = 75 \text{ kHz} \]

% Power spectrum of FM mono transmissions
The power spectrum of these mono FM transmissions would be a single peak, representing the audio content, spanning up to 15 kHz in frequency:
Here, the figure illustrates the frequency range allocated for mono FM audio transmissions.



\section*{Comunicazioni digitali}


\begin{adjustwidth*}{-2.3cm}{-2cm}
    \begin{center}
        \tikzsetnextfilename{digital_channel}
        \documentclass{standalone}

\usepackage{tikz,pgf} %and any other packages or tikzlibraries your picture needs

\usepackage{pgfplots}
\usepackage[letterpaper,top=2cm,bottom=2cm,left=2cm,right=2cm,marginparwidth=1.75cm]{geometry}
\usepackage{mathtools} 
\usepackage{forest}
\usepackage{standalone}
\usepackage{pgfplots}
\usepackage{graphicx}
\usepackage{svg}
\usepackage{array}
\usepackage{pgfplots}
\usepackage{tikz}
\usepackage[utf8]{inputenc}
\usepackage[colorlinks=true, allcolors=blue]{hyperref}
\usetikzlibrary{positioning, arrows.meta, fit, shapes}
\begin{document}


 \begin{tikzpicture}[>=Stealth, block/.style={draw, rectangle}, scale=0.85]
            % Blocks
            \tikzstyle{block} = [rectangle, draw, text centered, minimum height=4em, align=center]

            \node[block] (mod) {Modulatore \\ numerico};
            \node[block, right=1cm of mod] (channel) {Canale di \\ comunicazione};
            \node[block, right=1cm of channel] (demod) {Demodulatore \\ numerico};

            % Nodes for connecting lines
            \coordinate[left=2cm of mod] (input);
            \coordinate[right=2cm of demod] (output);

            % Lines
            \draw[->] (input) -- node[above] {$m_s[n]$} (mod);
            \draw[->] (mod) -- node[above] {$s(t)$} (channel);
            \draw[->] (channel) -- node[above] {$r(t)$} (demod);
            \draw[->] (demod) -- node[above] {$\hat{m}_s[n]$} (output);

            % Dashed box
            \begin{scope}
                \draw[dashed, red] ($(mod.north west)+(-0.5,0.5)$) rectangle ($(demod.south east)+(0.5,-0.5)$);
            \end{scope}

            % Annotations
            \node[align=center, red, above right= -1cm and -6cm of demod.south east] (channel-label) {Canale numerico};

            % Circles
            \draw (input) ++(-0.5cm,0) circle (0.5cm) node {$S$};
            \draw (output) ++(0.5cm,0) circle (0.5cm) node {$D$};
        \end{tikzpicture}












\end{document}

    \end{center}
\end{adjustwidth*}


In una comunicazione digitale l'informazione è codificata in sequenze di bit. Se rappresentassimo ogni bit come un funzione delta otterremmo un treno di delta:
\[
    \sum_{k=-\infty}^{+\infty} d_k \delta(t - kT_b)
\]
che occupa però una banda infinita.

La modulazione PAM consiste nell'effettuare la mappatura dei bit ed effettuare un filtraggio con un filtro passa basso avente risposta all'impulso \( g_T(t) \), ottenendo:
\[ s_{PAM}(t) = \sum_{k=-\infty}^{+\infty} a_k\cdot g_T(t - kT_s) \]

Una sequenza di bit può essere trattata come una variabile aleatoria con distribuzione uniforme:
\[
    \mathbb{P}(d_k = 0) = \mathbb{P}(d_k = 1) = \frac{1}{2} 
\]
\[
    \mathbb{E}\{d_k\} = 0\cdot \mathbb{P}(d_k = 0) + 1\cdot \mathbb{P}(d_k = 1) = \frac{1}{2}
\]

\[
    E_{s_{PAM}}(i) = \int_{-\infty}^{+\infty} s_{PAM, i}^2(t) \, dt =  \int_{-\infty}^{+\infty} \alpha_i^2 \cdot g_T^2(t - kT_s) \, dt = \int_{-\infty}^{+\infty} (2i - 1 - M)^2 g_T^2(t) \, dt = (2i - 1 - M)^2 E_{g_T}
\]

Scegliendo una mappatura simmetrica dei bit, possiamo trasmettere simboli con media nulla e quindi con energia a media nulla:
\[
    a_i = 
    \begin{cases}
        -1 & \text{se } d_k = 0 \\
        1 & \text{se } d_k = 1
    \end{cases}
\]
Per \( M \) pari si ha \( A_s = \{ \pm 1, \pm 3, \ldots, \pm (M-1) \} \)

Per \( M \) dispari si ha \( A_s = \{ 0, \pm 2, \pm 4, \ldots, \pm (M-1) \} \)


Gli \( M \) valori \( (M \geq 2) \) che costituiscono l'alfabeto \( A_s = \{ \alpha_1, \alpha_2, \ldots, \alpha_M \} \) sono definiti come:
\[ \alpha_i = 2i - 1 - M, \quad i = 1, 2, \ldots, M \]




\subsection*{Proprietà derivate della M-PAM}
\begin{enumerate}
    \item Il valor medio di \( s(t) \) è zero per ogni \( t \):
          \begin{equation*}
              \mathbb{E}\left[ s(t) \right] = 0 \quad \forall t
          \end{equation*}
          \begin{equation*}
              \mathbb{E} \left[ \sum_{k=-\infty}^{+\infty} a[k] \ g_T(t-kT_s) \right] = \sum_{k=-\infty}^{+\infty} \mathbb{E}\left[a[k]\right]\ g_T(t-kT_s) = 0
          \end{equation*}
          \begin{equation*}
              \mathbb{E}\left[a[k]\right] = \frac{1}{M} \sum_{i=1}^{M} \alpha_i \mathbb{P}\{\alpha_i\} = \frac{1}{M} \sum_{i=1}^{M} (2i - 1 - M)
          \end{equation*}
          \begin{equation*}
              = \frac{2}{M} \sum_{i=1}^{M} i - 1 - M = \frac{2}{M} \frac{M(M+1)}{2} - (M+1) = 0
          \end{equation*}

    \item La densità spettrale di potenza invece è:
          \begin{equation*}
              S_s(f) = \frac{1}{T_s} S_a(f) \ |G_T(f)|^2
          \end{equation*}
          \begin{equation*}
              \text{dove } \sigma_a^2 = \mathbb{E}\left[ a\left[k\right]^2 \right] = \frac{(M-1)(M+1)}{3}
          \end{equation*}
\end{enumerate}

\[
    R_s(t,\tau) = \mathbb{E}[s(t) \ s^*(t-\tau)]
\]
\[
    = \mathbb{E} \left[ \sum_{n=-\infty}^{+\infty} a\left[n\right] g_T(t - nT_s) \sum_{k=-\infty}^{+\infty} a^*\left[k\right] g_T^*(t - \tau - kT_s) \right]
\]
\[
    = \sum_{n=-\infty}^{+\infty} \sum_{k=-\infty}^{+\infty} \mathbb{E}[a\left[n\right] a^*\left[k\right]] \cdot g_T(t - nT_s) \cdot g_T^*(t - \tau - kT_s)
\]
\[
    = \sum_{n=-\infty}^{+\infty} \sum_{k=-\infty}^{+\infty} R_a\left[n-k\right] \cdot g_T(t - nT_s) \cdot g_T^*(t - \tau - kT_s)
\]

Imponendo \( n-k = m \) abbiamo che \( k = n-m \), quindi:

\[
    = \sum_{m=-\infty}^{+\infty} R_a[m] \sum_{n=-\infty}^{+\infty} g_T(t - nT_s) \cdot g_T^*(t - \tau - nT_s + mT_s)
\]

\paragraph*{Autocorrelazione media}

La funzione di autocorrelazione media \( \overline{R}_s(\tau) \) è:
\begin{align*}
    \overline{R}_s(\tau) & = \lim_{T\to\infty} \frac{1}{T} \int_{-\frac{T}{2}}^{\frac{T}{2}} R_s(t, \tau) dt                                                                   \\
    \overline{R}_s(\tau) & = \frac{1}{T_0} \int_{-\frac{T_0}{2}}^{\frac{T_0}{2}} R_s(t,\tau)dt \quad \text{se} \quad R_s(t,\tau) \quad \text{è periodico in} t           \\
    \overline{R}_s(\tau) & = \sum_{m=-\infty}^{\infty} R_a[m] \frac{1}{T_s} \sum_{n=-\infty}^{\infty} \int_{-\frac{T_s}{2}}^{\frac{T_s}{2}} g_T(t-nT_s)g_T^*(t-\tau-nT_s+mT_s)dt   \\
                         & = \sum_{m=-\infty}^{\infty} R_a[m] \frac{1}{T_s} \sum_{n=-\infty}^{\infty}\int_{-\frac{T_s}{2}+nT_s}^{\frac{T_s}{2}+nT_s} g_T(t')g_T^*(t'-\tau+mT_s)dt' \\
                         & = \sum_{m=-\infty}^{\infty} R_a[m] \frac{1}{T_s} \int_{-\infty}^{\infty} g_T(t')g_T^*(t'-\tau+mT_s)dt'                                                  \\
                         & = \sum_{m=-\infty}^{\infty} R_a[m] \frac{1}{T_s} \int_{-\infty}^{\infty} G_T(f)[G_T(f)e^{-j2\pi f\tau}e^{j2\pi fmT_s}]^*df                              \\
                         & = \int_{-\infty}^{\infty} G_T(f)G_T^*(f)\frac{1}{T_s} \sum_{m=-\infty}^{\infty} R_a[m] e^{-j2\pi fmT_s}e^{j2\pi f\tau}df                                \\
                         & = \frac{1}{T_s} \int_{-\infty}^{\infty} |G_T(f)|^2 S_a(f)e^{j2\pi f\tau}df
\end{align*}


\[
    \overline{R}_s(\tau) = \frac{1}{T_s} TCF^{-1} \left[ |G_T(f)|^2 S_a(f) \right]
\]
\[
    \Rightarrow S_s(f) = \frac{1}{T_s} S_a(f) |G_T(f)|^2
\]

Nel caso in cui:
\begin{enumerate}
    \item $\mathbb{E} [ a_n ] = 0$
    \item $R_a[m] = \sigma_a^2 \delta[m]$
\end{enumerate}

Si ha che:
\[
    S_s(f) = \frac{\sigma_a^2}{T_s} |G_T(f)|^2
\]

In questo caso la $B_T$ coincide con quella del filtro $G_T(f)$.

Calcolo di $\sigma_a^2$:
\[
    \sigma_a^2 = \mathbb{E} \left[ (a - \mu_a)^2 \right] = \int_{-\infty}^{\infty} (a - \mu_a)^2 f_a(a) da
\]
\[
    = \frac{1}{M} \sum_{i=1}^{M} (2i - 1 - M)^2
\]
\[
    = \frac{1}{M} \left[ 4 \sum_{i=1}^{M} i^2 + (1+M)^2 M - 4(1+M) \sum_{i=1}^{M} i \right]
\]

Sfruttando i seguenti risultati noti:
\[
    \sum_{i=1}^{n} i^2 = \frac{n(n+1)(2n+1)}{6}, \quad \sum_{i=1}^{n} i = \frac{n(n+1)}{2}
\]

Si ottiene:
\[
    \sigma_a^2 = \frac{M^2 - 1}{3}
\]


\[
    P_s = \frac{\sigma_a^2 E_{g_T}}{T_s} = \frac{M^2 - 1}{3} \frac{E_{g_T}}{T_s}
\]


In banda passante invece si ha che:
\[
    \sigma_a^2 = \frac{M^2 - 1}{6}
\]
\[
    P_s = \frac{M^2 - 1}{6} \frac{E_{g_T}}{T_s}
\]

Come sarà dimostrato più avanti, $R_s \propto B$, dove $B$ è la banda del segnale $s_{PAM}(t)$. 
Per $R_s$ valgono le seguenti equivalenze:
\[
    R_s = \frac{1}{T_s} = \frac{1}{mT_s} = \frac{R_b}{m} = \frac{R_b}{\log_2 M}
\]

Dove $R_b$ è il bit rate.




\paragraph*{Efficienza Spettrale di una M-PAM}
\[
    \beta = \frac{R_b}{B_T} = \frac{\log_2 M}{T_s B_{g_T}}
\]
essendo \( B_T = B_{g_T} \),

L'efficienza spettrale aumenta con l'aumentare del numero di livelli. Sfortunatamente, come verrà dimostrato più avanti, l'efficienza in potenza diminuisce all'aumentare di \( M \).

\begin{center}
    \tikzsetnextfilename{pam_trasmitter}
    \documentclass{standalone}

\usepackage{tikz,pgf}
\usepackage[utf8]{inputenc}
\usepackage[colorlinks=true, allcolors=blue]{hyperref}
\usetikzlibrary{positioning, arrows.meta, fit, shapes}
\begin{document}


\begin{tikzpicture}[
            block/.style={rectangle, draw, minimum height=1cm, minimum width=2cm},
            node distance=1cm,
            auto
        ]
        \node[draw, circle] (source)  {S};
        \node[block, right= of source] (interpolatore) {PAM tx};
        \node[left=of interpolatore] (tmp) {};
        \node[block, right= of interpolatore] (chan) {$h(t)$};

        \node[draw, circle, right= of chan] (plus)  {\(+\)};
        \node[below=of plus, inner sep=0pt, minimum size=0pt] (n) {};
        \node[block, right=of plus] (sampler) {PAM rx};
        \node[circle, draw, right=of sampler] (dest) {D};
        
        \draw[->] (source) -- (interpolatore) node[midway,above] {$d_k$};
        \draw[->] (interpolatore) -- (chan) node[midway,above] {$\tilde{s}(t)$};
        \draw[->] (chan) -- (plus) node[midway,above] {$y(t)$};
        \draw[->] (plus) -- (sampler) node[midway,above] {$r(t)$};
        \draw[->] (n) -- (plus) node[midway, right] {$w(t)$};
        \draw[->] (sampler) -- (dest) node[midway,above] {$\hat{d}_k$};
    \end{tikzpicture}



\end{document}

\end{center}


Il canale di comunicazione è generalmente modellato come un sistema LTI, caratterizzato da una risposta impusliva $h(t)$ e da un rumore additivo $w(t)$.
Nel caso ideale $h(t) = \delta (t)$.
Il rumore aggiunto al segnale si modella come rumore bianco gaussiano a potenza spettrale costante $N_0/2$, col quale si modella il disturbo proveniente dalle radiazioni solari, dalle radiazioni del big bang e del rumore dei dispositivi.
Il ricevitore è composto da 3 componenti principali:
\begin{enumerate}
    \item \textbf{Filtro passa basso}, che filtra il segnale ricevuto per eliminare le frequenze superiori a $B_T$.
    \item \textbf{Campionatore}, che campiona il segnale filtrato con rate $1/T_s$. 
    \item \textbf{Decisore}, che associa ad ogni campione un simbolo appartenente alla costellazione.
\end{enumerate}
Il segnale ricevuto è il seguente:
\[
    r(t) = s(t) \ast h(t) + w(t)
\]

mentre il segnale filtrato è:
\[
    x(t) = r(t) \ast g_R(t) = \sum_{k=-\infty}^{+\infty} a_k g(t - kT_s) + n(t)
\]
dove $g(t) = g_T(t) \ast h(t) \ast g_R(t)$ e $n(t) = w(t) \ast g_R(t)$. La densità spettrale di energia del rumore diventa:
\[
    S_n(f) = S_w(f) |G_R(f)|^2
\]



\subsection*{Interferenza intersimbolica (ISI)}
\noindent
\begin{minipage}{.5\textwidth}
    \centering
    \textbf{Assenza di ISI:}
    \[
        x[k] = f(a[k])
    \]
    \\
\end{minipage}%
\begin{minipage}{.5\textwidth}
    \centering
    \textbf{Presenza di ISI:}
    \[
        x[k] = f(\dots, a[k-1], a[k], a[k+1], \dots)
    \]
    \\
\end{minipage}


Il risultato è che il campione estratto al ricevitore dal segnale ricevuto al $k$-esimo istante non dipende solo dal $k$-esimo simbolo.

\begin{center}
    \begin{tikzpicture}[
            block/.style={rectangle, draw, minimum height=1cm, minimum width=2.5cm},
            node distance=1cm and 2cm,
            auto
        ]

        \node[block] (filter) {$g_R(t)$};
        \node[left=of filter] (channel) {};
        \node[right=of filter] (sampler) {};

        \draw[->] (channel) -- (filter) node[midway,above] {$r(t)$};

        \draw ([xshift=0]filter.east) -- ([xshift=1cm]filter.east) node[midway,above] {$x(t)$};
        \draw ([xshift=1cm]filter.east) -- ([xshift=1.5cm,yshift=0.5cm]filter.east) node[midway,below, yshift=-0.2cm] {$T_s$};

        \draw[->] ([xshift=1.5cm,yshift=0cm]filter.east) -- ++(1.5cm,0) node[midway,above] {$x[k]$};


    \end{tikzpicture}
\end{center}

dove:
\[
    x(t) = r(t) \ast g_R(t) = \sum_{k=-\infty}^{+\infty} a_k g(t - kT_s) + n(t)
\]
campionato in $t = mT_s$ si ha:
\[
        x(mT_s) = x[m] = \sum_{k=-\infty}^{+\infty} a_k g(mT_s - kT_s) + n(mT_s) = \sum_{k=-\infty}^{+\infty} a_k g[m-k] + n[m] \\ 
 \]
\[
        = a_m g(0) + \sum_{\substack{k \neq m,\\ k=-\infty}}^{+\infty} a_k g[m-k] + n[m]
\]



Un canale con banda \( B_c \) in generale introduce ISI. Ci sono due aspetti di cui ci occuperemo:

\begin{enumerate}
    \item Determinazione del \( T_s \) minimo che può essere adottato al fine di ottenere una sequenza campionata priva di ISI.
    \item Determinare le condizioni sotto le quali è possibile trasmettere un segnale M-PAM attraverso un canale non ideale in modo che non vi sia ISI nella sequenza campionata.
\end{enumerate}

Nel risolvere i due problemi riterremo \( h(t) \) fissata, e \( g_T(t) \) e \( g_R(t) \) variabili, in quanto determinabili dal progettista.

Un approccio non perseguibileconsiste nel trasmettere impulsi di durata finita e quindi con banda illimitata. Questo è in contrasto con la limitatezza messa a disposizione dal canale di trasmissione \( (B_c < \infty) \).

\(\Rightarrow\) Gli impulsi \( g_T(t) \) devono avere durata infinita.

\paragraph*{Criterio di Nyquist}{Primo criterio di Nyquist per la trasmissione priva di ISI}

\[ g(kT_s) =
    \begin{cases}
        1, & \text{se } k=0      \\
        0, & \text{se } k \neq 0
    \end{cases}
    \quad \text{(Dominio del tempo)}
\]

\[ \sum_{k=-\infty}^{+\infty} G\left(f-\frac{k}{T_s}\right) = T_s \quad \text{(Dominio della frequenza)} \]




\paragraph*{Dimostrazione}
Il criterio di Nyquist nel dominio del tempo garantisce l'assenza di ISI in quanto
\[ x[k] = a[k] \cdot g(0) + \sum_{\substack{i=-\infty \\ i \neq k}}^{+\infty} a[i] \cdot g[i-k] = a[k] \cdot g(0) \]
dove il secondo termine è nullo e non vi è ISI se \( g[k] = \delta[k] \).

La relazione in frequenza si ottiene come trasformazione
\[ g[k] = \delta[k] \quad \Longleftrightarrow \quad \overline{G}(f) = 1 \]
\[ \overline{G}(f) = \frac{1}{T_s} \sum_{k=-\infty}^{+\infty} G\left(f - \frac{k}{T_s}\right) = 1 \]
\[ \sum_{k=-\infty}^{+\infty} G\left(f - \frac{k}{T_s}\right) = T_s \]

\paragraph*{Trasmissione priva di ISI}
Supponiamo sia assegnato un canale a banda rigorosamente limitata con banda \( B_c \).
\[ H(f) = 0 \quad \text{per} \quad |f| > B_c \]
e supponiamo che \( B_T = B_c \), ovvero che il segnale trasmesso occupa tutta la banda messa a disposizione dal canale.
Allora si verificano le seguenti:
\begin{enumerate}
    \item Non è possibile in alcun modo eliminare l'ISI quando \( T_s < \frac{1}{2B_c} \).


          \paragraph*{Dimostrazione:}

          Quando \( T_s < \frac{1}{2B_c} \)

          \begin{tikzpicture}[scale=0.5]
              \draw[->] (-15,0) -- (15,0) node[right] {\( f \)};
              \draw[->] (0,-1) -- (0,7) node[above] {\( \overline{G}(f) \)};

              % Triangles
              \draw (-11,0) -- (-8,3) -- (-5,0);
              \draw[dashed, red] (-5,-1) -- (-5,4);
              \draw[dashed, red] (-3,-1) -- (-3,4);
              \draw (-3,0) -- (0,3) -- (3,0);
              \draw[dashed, red] (3,-1) -- (3,4);
              \draw[dashed, red] (5,-1) -- (5,4);
              \draw (5,0) -- (8,3) -- (11,0);

              \node at (-12,1.5) {\( \cdots \)};
              \node at (12,1.5) {\( \cdots \)};
          \end{tikzpicture}

          Esistono degli intervalli di frequenza dove \( \overline{H}(f) = 0 \) per cui non può mai accadere che \( \overline{H}(f) = 1 \) \( \forall f \)

          \bigskip

    \item Il più piccolo valore di \( T_s \) che permette di eliminare l'\( ISI \) è

          \( T_s^{(min)} = \frac{1}{2B_c} \)

          \bigskip

          \( f_s^{(max)}\) = \( \frac{1}{T_s^{(min)}} = 2B_c = f_N \) \quad (frequenza di Nyquist)

          \bigskip

          \begin{tikzpicture}[scale=0.5]
              \draw[->] (-15,0) -- (15,0) node[right] {\( f \)};
              \draw[->] (0,-1) -- (0,7) node[above] {\( \overline{G}(f) \)};

              % Triangles
              \draw (-9,0) -- (-6,3) -- (-3,0);
              \draw (-3,0) -- (0,3) -- (3,0);
              \draw (3,0) -- (6,3) -- (9,0);

              \node at (-10,1.5) {\( \cdots \)};
              \node at (10,1.5) {\( \cdots \)};
              \
          \end{tikzpicture}

          Non esistono intervalli di frequenza dove \( \overline{G}(f) = 0 \)

          \bigskip

    \item Nel caso valga la condizione \( T_s = \frac{1}{2B_c} \), allora l'unica funzione di trasferimento che permette di eliminare completamente l'ISI è

          \[ G(f) = \frac{1}{2B_c} \text{rect}\left(\frac{f}{2B_c}\right) \quad \Leftrightarrow \quad g(t) = \text{sinc}(2B_c t) \]

          \paragraph*{Dimostrazione:}

          \begin{center}

              \begin{tikzpicture}[scale=1]
                  \begin{axis}[
                          axis lines=middle,
                          xlabel={$f$},
                          ylabel={$G(f)$},
                          xtick={-4, -2, 2, 4},
                          xticklabels={$-2B_c$, $-B_c$, $B_c$, $2B_c$},
                          ytick={100},
                          yticklabels={},
                          ymin=-0.2, ymax=2,
                          xmin=-8, xmax=8,
                          xmajorgrids=false,
                          ymajorgrids=false,
                          clip=false
                      ]

                      \draw [thick] (axis cs:-2,0) rectangle (axis cs:2,0.4);

                      \node [red]at (axis cs:7,0.6) {$\sum_{k=-\infty}^{\infty} G(f - \frac{k}{T_s})$};

                      \draw [dashed, red] (axis cs:-6,0.4) -- (axis cs:-6,0);
                      \draw [dashed, red] (axis cs:-8,0.4) -- (axis cs:-2,0.4);
                      \draw [dashed, red] (axis cs:2,0.4) -- (axis cs:8,0.4);
                      \draw [dashed, red] (axis cs:6,0.4) -- (axis cs:6,0);
                  \end{axis}
                  \
              \end{tikzpicture}
          \end{center}

          Si nota anche che la funzione $\text{sinc}(2Bt)$ si annulla quando $t = \frac{k}{2B}$ con $k \neq 0$ per cui
          \[
              g(kT_s) = \text{sinc} \left(2B_c\cdot\frac{k}{2B_c}\right) = \text{sinc}(k) = 
              \begin{cases}
                  1 & \text{se } k=0    \\
                  0 & \text{se } k\neq0
              \end{cases}
          \]
\end{enumerate}


\textbf{Limiti di applicabilità della funzione di trasferimento rettangolare:}

\begin{enumerate}
    \item Realizzabilità di una funzione di trasferimento rettangolare: risposte in frequenza ideali come quella rettangolare non sono fisicamente realizzabili (Criterio di Paley-Wiener).
    \item Piccoli errori di campionamento provocano un ISI molto grande poiché la funzione $\text{sinc}(2B_ct)$ decresce molto lentamente.
\end{enumerate}

Un errore nel campionatore induce un ISI grande in quanto si sommano molte contributi!




\begin{figure}[ht]
    \centering
    \begin{center}
        \begin{tikzpicture}
            \begin{axis}[
                    axis lines=middle,
                    xlabel={$t$},
                    ylabel={$\text{sinc}(2B_c t)$},
                    xtick={-3, -2, -1, 0, 1, 2, 3},
                    xticklabels={$-3T_s$, $-2T_s$, $-T_s$, $0$, $T_s$, $2T_s$, $3T_s$},
                    ytick={1},
                    ymin=-0.5, ymax=1.5,
                    xmin=-4, xmax=4,
                    every axis x label/.style={at={(ticklabel* cs:1.05)}, anchor=west,},
                    every axis y label/.style={at={(ticklabel* cs:1.05)}, anchor=south,},
                    xmajorgrids=true,
                    ymajorgrids=true,
                    grid style=dashed,
                    clip=false,
                    no markers,
                    samples=1000,
                    domain=-3.5:3.5
                ]


                \draw [thick, red] (axis cs:-2.75,0) -- (axis cs:-2.75,0.082);
                \draw [thick, red] (axis cs:-1.75,0) -- (axis cs:-1.75,-0.13);
                \draw [thick, red] (axis cs:-0.75,0) -- (axis cs:-0.75,0.30);

                \draw [thick, red] (axis cs:0.15,0) -- (axis cs:0.15,0.965);

                \draw [thick, red] (axis cs:1.15,0) -- (axis cs:1.15,-0.125);
                \draw [thick, red] (axis cs:2.15,0) -- (axis cs:2.15,0.07);
                \draw [thick, red] (axis cs:3.15,0) -- (axis cs:3.15,-0.045);


                % Define sinc function
                \addplot+[thick, black, smooth, unbounded coords=jump] {sin(deg(pi*x))/(pi*x)};
                \addplot+[thick, black, smooth] coordinates {(0, 1)};

                % Add the red vertical line at t=0

            \end{axis}
        \end{tikzpicture}
    \end{center}
    \caption*{Un errore $\epsilon$ nel campionatore induce un ISI grande in quanto si sommano molti contributi. In rosso l'errore $\epsilon$ del compionatore.}
    %\label{fig:my_label} % Optional, for referencing the figure
\end{figure}



Rilassando la condizione $T_s > \frac{1}{2B_c}$, ovvero ammettendo

\[ T_s > \frac{1}{2B_c} \]

si ottiene il seguente effetto:

\begin{center}

    \begin{tikzpicture}[scale=1]
        \begin{axis}[
                axis lines=middle,
                xlabel={$f$},
                ylabel={$H(f)$},
                xtick={-6, -4, -2, 2, 4, 6},
                xticklabels={$-3B_c$, $-2B_c$, $-B_c$, $B_c$, $2B_c$, $3B_c$},
                ytick={100},
                yticklabels={},
                ymin=-0.2, ymax=2,
                xmin=-8, xmax=8,
                %every axis x label/.style={at={(ticklabel* cs:1.05)}, anchor=west,},
                %every axis y label/.style={at={(ticklabel* cs:1.05)}, anchor=south,},
                xmajorgrids=false,
                ymajorgrids=false,
                clip=false
            ]

            \node[red] at (-7,0.25) {\( \cdots \)};

            \draw[red] (-6,0) -- (-4,0.5) -- (-2,0);
            \draw[red] (-4,0) -- (-2,0.5) -- (0,0);

            \draw[red] (0,0) -- (2,0.5) -- (4,0);
            \draw[red] (2,0) -- (4,0.5) -- (6,0);

            \node[red] at (7,0.25) {\( \cdots \)};

            \draw[blue] (-6,0.5) -- (6,0.5);

            \draw (-2,0) -- (0,0.5) -- (2,0);


        \end{axis}
    \end{tikzpicture}
\end{center}

La sovrapposizione permette di definire una classe di infinite funzioni di trasferimento che soddisfano il primo criterio di Nyquist.

In questo caso però $B_c > \frac{1}{2T_s}$, per cui al punto di $T_s$ c'è bisogno di una banda disponibile nel canale che è maggiore di quella che occorre con la funzione di trasferimento rettangolare.

\subsection*{Filtro a coseno rialzato}

% Define the piecewise function
\[ H_{rc}(f) =
    \begin{cases}
        T_s                                                                                                          & \text{if } 0 \leq |f| \leq \frac{1-\alpha}{2T_s}                  \\
        \frac{T_s}{2} \left[ 1 - \sin\left(\frac{\pi T_s}{\alpha} \left( |f| - \frac{1}{2T_s} \right)\right) \right] & \text{if } \frac{1-\alpha}{2T_s} < |f| \leq \frac{1+\alpha}{2T_s} \\
        0                                                                                                            & \text{if } |f| > \frac{1+\alpha}{2T_s}
    \end{cases}
\]

con $0 < \alpha < 1$.

\begin{center}

    \definecolor{myblue}{RGB}{30,144,255}
    \definecolor{myred}{RGB}{178,34,34}
    \begin{tikzpicture}
        \begin{axis}[
                axis lines=middle,
                xlabel={$f$},
                ylabel={$H_{RC}(f)$},
                xtick={-0.5, 0.5},
                xticklabels={$\frac{-1}{2T_s}$, $\frac{1}{2T_s}$},
                ytick={0.5},
                yticklabels={$T_s/2$},
                ymin=0, ymax=1.5,
                xmin=-1, xmax=1,
                every axis x label/.style={at={(ticklabel* cs:1.05)}, anchor=west,},
                every axis y label/.style={at={(ticklabel* cs:1.05)}, anchor=south,},
                xmajorgrids=false,
                ymajorgrids=false,
                clip=false,
                no markers,
            ]

            % Draw the ideal filter response (black box)
            \draw [thick] (axis cs:-0.5,0) -- (axis cs:-0.5,1) -- (axis cs:0.5,1) -- (axis cs:0.5,0);

            % Draw the realistic filter response for alpha = 0.5 (blue line)
            % \addplot [myblue, thick, smooth, domain=-1:1] {0.5+0.5*cos(deg(pi*x))};
            \addplot [myblue, thick, smooth, domain=-0.75:0.-0.25] {0.5 * (1 + cos(deg(pi*(abs(x)-0.25)/0.5)))};
            \addplot [myblue, thick, smooth, domain=0.25:0.75] {0.5 * (1 + cos(deg(pi*(abs(x)-0.25)/0.5)))};


            % Draw the realistic filter response for alpha = 1 (red dashed line)
            \addplot [myred, thick, dashed, smooth, domain=-1:1] {0.5-0.5*sin(deg(pi*(abs(x)-0.5))};

            % Add annotations for alpha values
            \node[myblue] at (axis cs:0.75,0.8) {$\alpha=0.5$};
            \node at (axis cs:-0.75,0.8) {$\alpha=0$};
            \node[myred] at (axis cs:0.9,0.2) {$\alpha=1$};

            % Add black dot at intersection
            \node[circle,fill,inner sep=1.5pt] at (axis cs:0,0.5) {};

        \end{axis}
    \end{tikzpicture}
\end{center}
\subsection*{Propriet\`a}
\begin{enumerate}
    \item Quando \( \alpha = 0 \) il coseno rialzato coincide con la funzione di trasferimento rettangolare
    \item La banda \( B_H \) \`e direttamente ottenibile da \( B_H = \frac{1+\alpha}{2T_S} \)
\end{enumerate}

La \( h_{RC}(t) \) \`e calcolabile in forma chiusa:

\[ h_{RC}(t) = \sin\left(\frac{t}{T_S}\right) \frac{\cos\left(\frac{\alpha \pi t}{T_S}\right)}{\left(1- \frac{2\alpha t}{T_S}\right)^2}  \]

\[ h_{RC}(kT_S) = \delta[k] \]

\begin{itemize}
    \item Soddisfa il criterio di Nyquist nel tempo, per cui garantisce l'assenza di ISI
    \item Decresce per \( t \rightarrow \infty \) come \( \frac{1}{|t|^3} \) per \( \alpha > 0 \) quindi molto pi\`u velocemente del caso \( \alpha = 0 \) (rettangolare)
\end{itemize}


\paragraph*{Rumore}
Nel caso della banda passante il rumore dell'inviluppo complesso è definito come:
\[
    w(t) = w_I(t) + jw_Q(t) 
\]
e la densità spettrale di potenza è:
\[
    S_w(f) = 2N_0
\]


Dove sia la parte reale che immaginaria danno un contributo $N_0$. Il segnale originale presentava invece un rumore con densità spettrale di potenza $N_0/2$.
Dopo il filtraggio la densià spettrale di potenza del rumore assume la forma:
\[
    S_n(f) = S_w |G_R(f)|^2 = 2N_0 |G_R(f)|^2
\]

Il ricevitore deve quindi essere implementato al fine di minimizzare l'effetto del rumore, ovvero massimizzare l'SNR.

\paragraph*{Filtro adattato}
La massimizzazione dell'SNR si ottiene con un filtro adattato, dove:
\[
    g_R(t) = g_T(-t) \quad \Longleftrightarrow \quad G_R(f) = G_T^*(f)
\]

Volendo quindi soddisfare il criterio di Nyquist col coseno rialzato e volendo utilizzare il filtro adattato, otteniamo:
\[
    G(f) = G_T(f) H(f) G_R(f) = |G_T(f)|^2 H(f)
\]

Nel caso di canale ideale, definendo come $H_{SRRC} = \sqrt{H_{RC}}$ si ottiene:
\[
    g_T(t) = h_{SRRC}(t) \quad \Longleftrightarrow \quad G_T(f) = H_{SRRC}(f) 
\]

\[
    g_R(t) = h_{SRRC}(-t) \quad \Longleftrightarrow \quad H_R(f) = H_{SRRC}^*(f)
\]


In tal caso e considerando i simboli a media nulla ed equiprobabli: 
\[
    S_s(f) = \frac{1}{T_s} S_a |G_T(f)|^2 = \frac{A}{T_s} H_{RC}(f)
\]


Per quanto riguarda la banda invece:
\[
    B^{(BB)} = \frac{1+\alpha}{2T_s} = \frac{1+\alpha}{2} \frac{1}{\log_2(M) T_b} = \frac{1+\alpha}{2} \frac{R_b}{\log_2(M)} \quad \text{banda inviluppo complesso PAM}
\]
\[
    B^{(PB)} = 2 B^{(BB)} = (1+\alpha) \frac{R_b}{\log_2(M)} \quad \text{banda PAM in banda passante}
\]

\[
    P_s^{(BB)} = \int_{-\infty}^{+\infty} S_s(f) df = \frac{A}{T_s} \int_{-\infty}^{+\infty} H_{RC}(f) df = \frac{A}{T_s} h_{RC}(0) = \frac{A}{T_s} \quad \text{potenza media inviluppo complesso}
\]

\[
    P_s^{(PB)} = \frac{1}{2} P_s^{(BB)} = \frac{A}{2T_s} \quad \text{potenza media PAM in banda passante}
\]

\[
    E_s = P_s T_s = \frac{A}{2} = \frac{M^2 - 1}{6} \quad \text{energia media per simbolo}
\]


\section*{Strategia di decisione}


\[
    x[m] = a_m + n[m] = a_m + n_I[m] + jn_Q[m]
\]  

La potenza del segnale di disturbo può essere calcolatora considerando la densità spettrale di potenza:
\[
    P_n = \int_{-\infty}^{+\infty} S_n(f) df = 2N_0 \int_{-\infty}^{+\infty} |G_R(f)|^2 df = 2N_0 \int_{-\infty}^{+\infty} |H_{RC}(f)|^2 df = 2N_0
\]

Essendo il rumore a media nulla, la varianza del rumore è pari alla sua potenza:
\[
    \sigma_n^2 = P_n = 2N_0
\]

Quindi la variabile aleatoria associata al rumore ha distribuzione del tipo $\mathcal{N}(0, 2 N_0)$.

Si può dimostrare che in presenza di filtro RRC e simboli equiprobabili la strategia di dicisione ottima consiste nel massimizzare la probabilità condizionata di aver ricevuto $a^{(i)}$ sapendo di aver ricevuto $x[m]$:

\[
    \hat{a}_m = \underset{i=1,\ldots,M}{\mathrm{argmax}} \ \mathbb{P}(a^{(i)}|x[m])
\]

Usando il criterio di massima verosimiglianza e la formula di Bayes di può dimostrare che:
\[
    \mathbb{P}(a^{(i)} \ | \ x[m]) \propto  \mathbb{P}(x[m] \ | \ a^{(i)})
\]

\[
    x[m] \ | \ a^{(i)}  = a^{(i)} + n[m] \quad \Rightarrow \quad x[m] \ | \ a^{(i)} \sim \mathcal{N}(a^{(i)}, \sigma_n^2)
\]

\[
    f_X(x \ | \ a^{(i)}) = f_{W}(x - a^{(i)}) = \frac{1}{\sqrt{2\pi \sigma_n^2}} e^{-\frac{(x - a^{(i)})^2}{2\sigma_n^2}}
\]


Essendo i simboli reali e quindi anche il segnale PAM, è possibile ignorare la componente in quadratura del segnale ricevuto, in quanto non vi sarà alcuna informazione.
\[
    x[m] \ | \ a^{(i)} \sim \mathcal{N}(a^{(i)}, N_0) \quad \Rightarrow \quad f_X(x \ | \ a^{(i)}) = \frac{1}{\sqrt{2\pi \sigma_n^2}} e^{-\frac{(x - a^{(i)})^2}{2\sigma_n^2}}
\]
Il massimo della densità di probabilità è ottenuto minimizzando l'opposto dell'esponente:
\[
    \hat{a}_m = \underset{i=1,\ldots,M}{\mathrm{argmin}} \left| x[m] - a^{(i)} \right| 
\]

Per adottare il criterio di decisione ottimo si possono definire regioni non sovrapposte, all'interno delle quali il segnale ricevuto è associato a un determinato simbolo.
Ogni regione è composta da due punti il cui simbolo più vicino è $a^{(i)}$, nel caso di più simboli le due regioni agli estremi avranno una larghezza infinita, mentre quelle interne saranno limitate.
Le soglie sono date dal punto medio tra i due simboli adiacenti:
\[
    Z^{(i)} = \left\{ x \mid d(x, a^{(i)}) < d(x, a^{(j)}), j \neq i, j = 1, \ldots, M \right\}
\]

% include image in img/regions.jpg
%\begin{figure}[ht]
%    \centering
%    \includegraphics[width=0.875\textwidth]{imgs/regions.png}
%    \caption*{Regioni di decisione per un sistema PAM}
%\end{figure}

\begin{center}
    \tikzsetnextfilename{2PAM_decisor}
    \documentclass{standalone}
\usepackage{tikz}
\usetikzlibrary{positioning, decorations.pathreplacing}

\begin{document}
\begin{tikzpicture}

% Define colors
\definecolor{myred}{RGB}{255, 210, 210}
\definecolor{myyellow}{RGB}{255, 255, 210}

% Draw the rectangles
\fill[myred] (-2, 0) rectangle (0, 2.5);
\fill[myyellow] (0, 0) rectangle (2, 2.5);

% Vertical line in the middle
\draw[very thick] (0, 0) -- (0, 2.5);
% Draw the dashed lines in red
\draw[red, ultra thick, dashed] (0, 0) -- (0, 0.4);
\draw[red, ultra thick, dashed] (0, 0.8) -- (0, 1.2);
\draw[red, ultra thick, dashed] (0, 1.6) -- (0, 2);

% Draw the horizontal line
\draw[thick] (-2, 1) -- (2, 1);

% Draw the nodes
\fill (0, 1) circle (2pt);
\fill (-1, 1) circle (2pt);
\fill (1, 1) circle (2pt);

% Add labels
\node at (-1, 1.2) {\textnormal{\tiny $0$}};
\node at (1, 1.2) {\textnormal{\tiny $1$}};

\node at (-1, 0.7) {$-1$};
\node at (1, 0.7) {$1$};

\node at (-1, 1.7) {$a^{(0)}$};
\node at (1, 1.7) {$a^{(1)}$};

\node[above right] at (-1.8, 1.9) {$Z^{(0)}$};
\node[above left] at (1.8, 1.9) {$Z^{(1)}$};

% Add the bottom text
\node at (0, -0.5) {$M=2$};
\node at (0, -1) {$m=1$};

\end{tikzpicture}
\end{document}



\end{center}

\begin{center}
    \tikzsetnextfilename{4PAM_decisor}
    \documentclass{standalone}
\usepackage{tikz}
\usetikzlibrary{positioning, decorations.pathreplacing}

\begin{document}
\begin{tikzpicture}

% Define colors
\definecolor{myred}{RGB}{255, 210, 210}
\definecolor{mygreen}{RGB}{210, 255, 210}
\definecolor{myblue}{RGB}{210, 210, 255}
\definecolor{myyellow}{RGB}{255, 255, 210}

% Draw the rectangles
\fill[myred] (-4, 0) rectangle (-2, 2.5);
\fill[mygreen] (-2, 0) rectangle (0, 2.5);
\fill[myblue] (0, 0) rectangle (2, 2.5);
\fill[myyellow] (2, 0) rectangle (4, 2.5);

% Vertical lines in the middle
\draw[very thick] (-2, 0) -- (-2, 2.5);
\draw[very thick] (0, 0) -- (0, 2.5);
\draw[very thick] (2, 0) -- (2, 2.5);

% Draw the dashed lines in red
\draw[red, ultra thick, dashed] (-2, 0) -- (-2, 0.4);
\draw[red, ultra thick, dashed] (-2, 0.8) -- (-2, 1.2);
\draw[red, ultra thick, dashed] (-2, 1.6) -- (-2, 2);

\draw[red, ultra thick, dashed] (0, 0) -- (0, 0.4);
\draw[red, ultra thick, dashed] (0, 0.8) -- (0, 1.2);
\draw[red, ultra thick, dashed] (0, 1.6) -- (0, 2);

\draw[red, ultra thick, dashed] (2, 0) -- (2, 0.4);
\draw[red, ultra thick, dashed] (2, 0.8) -- (2, 1.2);
\draw[red, ultra thick, dashed] (2, 1.6) -- (2, 2);

% Draw the horizontal line
\draw[thick] (-4, 1) -- (4, 1);

% Draw the nodes
\fill (-3, 1) circle (2pt);
\fill (-1, 1) circle (2pt);
\fill (1, 1) circle (2pt);
\fill (3, 1) circle (2pt);

% Add labels
\node at (-3, 1.2) {\textnormal{\tiny $00$}};`'
\node at (-1, 1.2) {\textnormal{\tiny $01$}};
\node at (1, 1.2) {\textnormal{\tiny $11$}};
\node at (3, 1.2) {\textnormal{\tiny $10$}};

\node at (-3, 0.7) {$-3$};
\node at (-1, 0.7) {$-1$};
\node at (1, 0.7) {$1$};
\node at (3, 0.7) {$3$};

\node at (-3, 1.7) {$a^{(0)}$};
\node at (-1, 1.7) {$a^{(1)}$};
\node at (1, 1.7) {$a^{(2)}$};
\node at (3, 1.7) {$a^{(3)}$};

\node[above right] at (-3.8, 1.9) {$Z^{(0)}$};
\node[above left] at (-0.2, 1.9) {$Z^{(1)}$};
\node[above right] at (0.2, 1.9) {$Z^{(2)}$};
\node[above left] at (3.8, 1.9) {$Z^{(3)}$};

% Add the bottom text
\node at (0, -0.5) {$M=4$};
\node at (0, -1) {$m=2$};

\end{tikzpicture}
\end{document}



\end{center}

\begin{center}
    \tikzsetnextfilename{8PAM_decisor}
    \documentclass{standalone}
\usepackage{tikz}
\usetikzlibrary{positioning, decorations.pathreplacing}

\begin{document}
\begin{tikzpicture}

% Define colors
\definecolor{myred}{RGB}{255, 0, 0}

% Draw the horizontal line
\draw[thick] (-8, 1) -- (8, 1);

% Draw the nodes
\fill (-7, 1) circle (2pt);
\fill (-5, 1) circle (2pt);
\fill (-3, 1) circle (2pt);
\fill (-1, 1) circle (2pt);
\fill (1, 1) circle (2pt);
\fill (3, 1) circle (2pt);
\fill (5, 1) circle (2pt);
\fill (7, 1) circle (2pt);

% Add labels for a
\node at (-7, 0.7) {$-7$};
\node at (-5, 0.7) {$-5$};
\node at (-3, 0.7) {$-3$};
\node at (-1, 0.7) {$-1$};
\node at (1, 0.7) {$1$};
\node at (3, 0.7) {$3$};
\node at (5, 0.7) {$5$};
\node at (7, 0.7) {$7$};

% Add binary labels for a
\node at (-7, 1.3) {\textnormal{\tiny $000$}};
\node at (-5, 1.3) {\textnormal{\tiny $001$}};
\node at (-3, 1.3) {\textnormal{\tiny $011$}};
\node at (-1, 1.3) {\textnormal{\tiny $010$}};
\node at (1, 1.3) {\textnormal{\tiny $110$}};
\node at (3, 1.3) {\textnormal{\tiny $111$}};
\node at (5, 1.3) {\textnormal{\tiny $101$}};
\node at (7, 1.3) {\textnormal{\tiny $100$}};

% Add labels for a^(i)
\node at (-7, 1.7) {$a^{(0)}$};
\node at (-5, 1.7) {$a^{(1)}$};
\node at (-3, 1.7) {$a^{(2)}$};
\node at (-1, 1.7) {$a^{(3)}$};
\node at (1, 1.7) {$a^{(4)}$};
\node at (3, 1.7) {$a^{(5)}$};
\node at (5, 1.7) {$a^{(6)}$};
\node at (7, 1.7) {$a^{(7)}$};

% Draw the dashed lines in green
\draw[myred, ultra thick, dashed] (-6, 0) -- (-6, 2.5);
\draw[myred, ultra thick, dashed] (-4, 0) -- (-4, 2.5);
\draw[myred, ultra thick, dashed] (-2, 0) -- (-2, 2.5);
\draw[myred, ultra thick, dashed] (0, 0) -- (0, 2.5);
\draw[myred, ultra thick, dashed] (2, 0) -- (2, 2.5);
\draw[myred, ultra thick, dashed] (4, 0) -- (4, 2.5);
\draw[myred, ultra thick, dashed] (6, 0) -- (6, 2.5);

% Add labels for Z
\node[above] at (-7, 2.2) {$Z^{(0)}$};
\node[above] at (-5, 2.2) {$Z^{(1)}$};
\node[above] at (-3, 2.2) {$Z^{(2)}$};
\node[above] at (-1, 2.2) {$Z^{(3)}$};
\node[above] at (1, 2.2) {$Z^{(4)}$};
\node[above] at (3, 2.2) {$Z^{(5)}$};
\node[above] at (5, 2.2) {$Z^{(6)}$};
\node[above] at (7, 2.2) {$Z^{(7)}$};

% Add the bottom text
\node at (0, -0.5) {$M=8$};
\node at (0, -1) {$m=3$};

\end{tikzpicture}
\end{document}



\end{center}



Sebbene si adotti il criterio di decisione ottimo, la presenza del rumore può portare a una decisione errata, traslando il segnale in una regione non associata al simbolo trasmesso.
La probabilità di errore sul simbolo equivale alla probabilità di sovrastimare $a^{(i)}$ e il campione ricevuto non rientri nella regione $Z^{(i)}$:
\[
    P_e = \frac{1}{M} \sum_{i=0}^{M-1} \mathbb{P}(\text{errore} \ | \ a^{(i)}) = \lim_{N^{(s)} \to +\infty} \frac{N_e^{(s)}}{N^{(s)}} \quad \text{probabilità errore PAM}
\]

Considerando una modulazione 2-PAM, il campione ricevuto in presenza di rumore avrà una distribuzione gaussiana centrata in $a^{(i)}$, ovvero in $\pm 1$. Poiché la soglia è data da 0, nel caso sia stato trasmesso +1 (-1) la probabilità di errore corrisponde all'area della gaussiana a sinistra (destra) della soglia, centrata in 1 (-1).

% include imgs/2pamerror.jpg
\begin{figure}[ht]
    \centering
    \includegraphics[width=0.65\textwidth]{imgs/2pamerror.jpg}
    \caption*{Probabilità di errore per una modulazione 2-PAM}
\end{figure}


Quindi la probabilità di errore in questo caso può essere espressa come:
\[
    P(e | -1) = Q\left(\frac{\text{d}(t_i, m_x)}{\sigma}\right) = Q\left(\frac{1}{\sigma}\right)
\]

Dove l'ultima uguaglianza deriva dal fatto che la soglia $t_i=0$ e il valore medio della gaussiana è $m_x = -1$.

Si ottiene il solito valore anche calcolando la probabilità di errore per il simbolo $+1$, quindi la probabilità di errore della 2-PAM equivale a:
\[
    P^{\text{2-PAM}}_e = \frac{1}{2} Q\left(\frac{1}{\sigma}\right) + \frac{1}{2} Q\left(\frac{1}{\sigma}\right) = Q\left(\frac{1}{\sigma}\right)
\]

Se i simboli scelti fossero stati $a^{(i)} = \{-3, -1, 1, 3\}$, la distanza tra simbolo e soglia sarebbe stata in ogni caso 1.
Per una 4-PAM la probabilità di errore è:

\[ 
    P^{\text{4-PAM}}_e = \frac{1}{4} \sum_{i=1}^{4} \mathbb{P}(e \mid a^{(i)}) = \frac{1}{4} \cdot 2 \left[ \mathbb{P}(e \mid a^{(i)} = 1) + \mathbb{P}(e \mid a^{(i)} = 3) \right] = \frac{1}{2} \left[ 2 Q\left( \frac{1}{\sigma} \right) + Q\left( \frac{1}{\sigma} \right) \right] = \frac{3}{2} Q\left( \frac{1}{\sigma} \right)
\]

Considerando che per la banda passante $E_s = \frac{M^2 - 1}{6}$, possiamo trovare la relazione:
\[
    1 = \frac{6 E_s}{(M^2 - 1)}
\]
Mentre ricordando che la varianza del rumore è $\sigma_n^2 = N_0$, possiamo esprimere la probabilità di errore in funzione della SNR ($\frac{E_s}{N_0}$):

\[
    P_e^{\text{M-PAM}} = \frac{2(M - 1)}{M} Q\left(\sqrt{\frac{6 E_s}{(M^2 - 1) N_0}}\right)
\]


Esprimendo i risultati in funzione della SNR ($\frac{E_s}{N_0}$), si ottiene:
\[
    \text{2-PAM:} \quad E_s = \frac{1}{2}, P^{\text{2-PAM}}_e = Q\left( \sqrt{\frac{2 E_s}{N_0}} \right)
\]
\[
    \text{4-PAM:} \quad E_s = \frac{5}{2}, P^{\text{4-PAM}}_e = \frac{3}{2} Q\left( \sqrt{\frac{2 E_s}{5 N_0}} \right)
\]




\section*{Gray mapping}
La codifica di Gray consiste nel definire la costellazione dei simboli in modo tale che simboli adiacenti differiscano di un solo bit.
Può essere trovata ricorsivamente col seguente codice:


\begin{minted}{python3}
def gray_code(m):
    if m == 1:
        return [0, 1] 
    else:
        lower = gray_code(m - 1)
        msb = 1 << (m - 1)
        return lower + [(msb | x) for x in reversed(lower)]
\end{minted}
Dove a partire dal caso precedente, si specchia la lista e si somma alla seconda metà il bit più significativo settato a 1. 
per $m=3$ si ottiene:
\begin{align*}
a^{(0)} &= -7 \rightarrow 000 \\
a^{(1)} &= -5 \rightarrow 001 \\
a^{(2)} &= -3 \rightarrow 011 \\
a^{(3)} &= -1 \rightarrow 010 \\
a^{(4)} &= 1 \rightarrow 110 \\
a^{(5)} &= 3 \rightarrow 111 \\
a^{(6)} &= 5 \rightarrow 101 \\
a^{(7)} &= 7 \rightarrow 100
\end{align*}



La codifica di Gray permette di assumere che un errore su un simbolo riguardi solo un bit, infatti essendo a il rumore a media nulla è molto probabile che non sia tale per cui un campione finisca in una regione non adiacente a quella del simbolo trasmesso.

\[
    P^{(b)}_E = \lim_{N^{(b)} \to +\infty} \frac{N_e^{(b)}}{N^{(b)}} \approx \lim_{N^{(s)} \to +\infty} \frac{N_e^{(s)}}{ \log_2(M) N^{(s)}} = \frac{1}{\log_2(M)} P^{(s)}_E = \frac{1}{\log_2(M)} \left(\frac{M-1}{M}\right) 2Q\left(\sqrt{\frac{6 \log_2 M \cdot SNR}{M^2-1}}\right)
\]
Per quanto riguardo l'energia per bit invece (TODO da aggiungere qualche commento):
\[
    E_b = \frac{E_s}{\log_2(M)}
\]

Incrementando il numero di bit per simbolo, per ottenere una stessa probabilità di errore, è necessario un SNR superiore, quindi più energia.




\section*{Modulazione QAM}
\begin{center}

    \begin{tikzpicture}[
            block/.style={rectangle, draw, minimum height=1cm, minimum width=1.5cm},
            node distance=1cm and 1cm,
            auto,
            every node/.style={align=center}
        ]
        % Blocks
        \node[block] (source) {Sorgente\\binaria};
        \node[block, right=of source] (encoder) {Mapper};
        \node[block, right=of encoder] (interp) {S/P};

        % create a node which is above interp
        \node[above=of interp, inner sep=0pt, minimum size=0pt] (dummy1) {};
        \node[below=of interp, inner sep=0pt, minimum size=0pt] (dummy2) {};


        \node[block, right=of dummy1] (p1) {$g_T(t)$};

        \node[block, right=of dummy2] (p2) {$g_T(t)$};
        \node[draw, circle, right=1cm of p1] (m1) {\(\times\)};
        \node[draw, circle, right=1cm of p2] (m2) {\(\times\)};

        \draw[->] (p1) -- (m1);
        \node[below=of m1] (cos) {$\cos(2\pi f_c t)$};
        \draw[->] (cos) -- (m1);

        \draw[->] (p2) -- (m2);
        \node[below=of m2] (sin) {$-\sin(2\pi f_c t)$};
        \draw[->] (sin) -- (m2);

        \node[right=2cm of m1, inner sep=0pt, minimum size=0pt] (dummy3) {};
        \node[right=2cm of m2, inner sep=0pt, minimum size=0pt] (dummy4) {};

        \node[draw, circle, right=5.125cm of interp] (sum) {\(+\)};

        \draw[->] (m1) -| (sum);
        \draw[->] (m2) -| (sum);

        \node[right=2cm of sum] (dummy5) {};
        \draw[->] (sum) -- node[midway, above] {$s_{QAM}(t)$} (dummy5) {};

        %\draw[-] (interp) -- (dummy1);
        %\draw[-] (dummy2) -- (interp);

        \draw[->] (interp) |- node[midway, above] {$a_i$} (p1);
        \draw[->] (interp) |- node[midway, below] {$b_i$} (p2);

        \draw[->] (source) -- (encoder) node[midway,above] {$bin_i$};
        \draw[->] (encoder) -- (interp) node[midway,above] {$c_i$};


    \end{tikzpicture}
\end{center}

La modulazione PAM non è utilizzata nella pratica perché non è molto efficiente dal punto di vista energetico.
Trasmettendo due PAM ortogonali si ottiene una modulazione QAM.
\[  
    s_{QAM}(t) = \underbrace{\sum_{i=-\infty}^{+\infty} a_i g_T(t - iT)}_{m_I} + j \underbrace{\sum_{i=-\infty}^{+\infty} b_i g_T(t - iT)}_{m_Q}
\]


I simboli trasmessi sono dunque complessi da un punto di vista matematico:

\[
    c_i = a_i + jb_i
\]

Che nella pratica equivale a due segnali PAM ortogonali.
Sfruttando la notazione complessa si ottiene:
\[
    s_{QAM}(t) = \sum_{i=-\infty}^{+\infty} c_i g_T(t - iT)
\]

La modulazione QAM essendo una combinazione di due PAM ortogonali con $M$ simboli prevede un totale di $M^2$ simboli. 







\begin{minipage}{.5\textwidth}
    \centering
    \begin{tikzpicture}
        \draw[->] (-4,0) -- (4,0) node[right] {$\Re$};
        \draw[->] (0,-4) -- (0,4) node[above] {$\Im$};

        \draw (0,1) -- (0.1,1) node[above] {$\ \ 1$};


        \foreach \point in {
                (-3, 1),
                (-3, -1),
                (3, 1),
                (3, -1),
                (-1, 1),
                (1, 1),
                (-1, -1),
                (1, -1)}{
                \draw[fill=orange] \point circle (1.5pt);
            }

        \node[below=4.5cm] at (0,0) {
            Simboli 8-QAM, $M_c = 4, M_s = 2$
        };

    \end{tikzpicture}

\end{minipage}
\noindent
\begin{minipage}{.5\textwidth}

    \centering
    \begin{tikzpicture}
        \draw[->] (-4,0) -- (4,0) node[right] {$\Re$};
        \draw[->] (0,-4) -- (0,4) node[above] {$\Im$};

        \draw (0,1) -- (0.1,1) node[above] {$\ \ 1$};


        \foreach \point in {
                (1, -3),
                (-1, -3),
                (1, 3),
                (-1, 3),
                (-1, 1),
                (1, 1),
                (-1, -1),
                (1, -1)}{
                \draw[fill=orange] \point circle (1.5pt);
            }


        % insert caption here
        \node[below=4.5cm] at (0,0) {
            Simboli 8-QAM, $M_c = 2, M_s = 4$
        };

    \end{tikzpicture}


\end{minipage}

I simboli possono essere quindi rappresentati nel piano complesso come in figura.

Considerando $M_c = M_s$, simboli indipendenti e a media nulla, il valor quadratico medio è:
\[
    A^{\text{QAM}} = \mathbb{E} \left\{  c_i c^*_i  \right\} = \mathbb{E} \left\{ a_i^2 \right\} + \mathbb{E} \left\{ b_i^2 \right\} = 2 \left( \frac{M^2 - 1}{3} \right)
\]

Mentre l'energia media per simbolo è:
\[
    E_s = \frac{A}{2} = \frac{M^2 - 1}{3}
\]

La probabilità di errore invece:
\begin{align*}
    x\left[ m \right] &= c_m + n \left[ m \right] \\
    &= a_m + j b_m + n_I \left[ m \right] + j n_Q \left[ m \right]
\end{align*}


L'evento di errore è $\mathcal{E}^{(i)} = \mathcal{E}_I^{(i)} \cup \mathcal{E}_Q^{(i)}$ quindi:
\[
    \mathbb{P} \left( e \mid c^{(i)} \right) = \mathbb{P} \left( \mathcal{E}^{(i)} \right) \leq \mathbb{P} \left( \mathcal{E}_I^{(i)} \right) + \mathbb{P} \left( \mathcal{E}_Q^{(i)} \right)
\]

ovvero:
\[
    P_e^{(M, QAM)} \leq 2 \ P_e^{(\sqrt{M}, PAM)}
\]
Facciamo quindi un'approssimazione che si chiama \textit{union bound}.




\[
\begin{array}{ccccc}
\text{4-QAM} & E_s = 1 = \frac{1}{\sigma} = \sqrt{\frac{E_s}{N_0}} & \Rightarrow & P_e^{\text{4-QAM}} < 2 \ Q\left(\sqrt{\frac{E_s}{N_0}}\right) \\
\text{16-QAM} & E_s = 5 = \frac{1}{\sigma} = \sqrt{\frac{E_s}{5N_0}} & \Rightarrow & P_e^{\text{16-QAM}} < 3 \ Q\left(\sqrt{\frac{E_s}{5N_0}}\right)\\
\end{array}
\]

Adesso le regioni associate a un simbolo sono porzioni del piano complesso, lungo 2 direzioni,
Per quanto riguarda l'errore sul bit, adottando la codifica di Gray, che vale quando abbiamo un SNR alto, si ottiene:
\[
    P_e^{(M-QAM, b)} = \lim_{N^{(b)} \to +\infty} \frac{N_e^{(b)}}{N^{(b)}} \approx \lim_{N^{(s)} \to +\infty} \frac{N_e^{(s)}}{\log_2(M) N^{(s)}} = \frac{1}{\log_2(M)} P_e^{(M-QAM)} 
\]

Sapendo che $P_e^{(M-QAM)} \leq 2 \ P_e^{(\sqrt{M}, PAM)}$ e che per la codifica di Gray $P_e^{(M-PAM, b)} \approx \frac{1}{\log_2{M}} P_e^{(M-PAM)}$ si ottiene:
\[
     P_e^{(M-QAM, b)} = \frac{1}{\log_2(M)} P_e^{(M-QAM)} \leq \frac{2}{\log_2(M)} P_e^{(\sqrt{M}, PAM)} = \frac{1}{\log_2(\sqrt{M})} P_e^{(\sqrt{M}, PAM)} \approx P_e^{(\sqrt{M}, PAM, b)}
\]

Da ciò si evince che la probabilità di errore sul bit per una QAM a $M$ simboli è approssimabile a quella di una modulazione PAM a $\sqrt{M}$ simboli.

 
\section*{Propagazione del segnale nell'aria}
La propagazione del segnale nell'atmosfera può essere classificata in base alle bande di frequenza e alle relative caratteristiche. Di seguito vengono elencate le diverse bande di classificazione, le iniziali, i range di frequenza e le principali caratteristiche:
\begin{table}[h!]
\centering
\begin{tabular}{llll}
\hline
\textbf{Classification Band} & \textbf{Initials} & \textbf{Frequency Range} & \textbf{Characteristics} \\
\hline
Extremely low frequency & ELF & $<$ 300 Hz & Ground wave \\
Very low frequency & VLF & 3 kHz - 30 kHz & Ground/Sky wave \\
Low frequency & LF & 30 kHz - 300 kHz & Ground/Sky wave \\
Medium frequency & MF & 300 kHz - 3 MHz & Ground/Sky wave \\
High frequency & HF & 3 MHz - 30 MHz & Sky wave \\
Very high frequency & VHF & 30 MHz - 300 MHz & Space wave \\
Ultra high frequency & UHF & 300 MHz - 3 GHz & Space wave \\
Super high frequency & SHF & 3 GHz - 30 GHz & Space wave \\
\hline
\end{tabular}
\end{table}


Dalla frequenza delle onde, da cui dipende la modalità di propagazione attraverso il canale, si indentificano i seguenti meccanismi di propagazione:

\begin{itemize}
    \item \textbf{Ground wave} (fino a 2 MHz): le onde si propagano seguendo la curvatura terrestre riuscendo a raggiungere un ricevitore oltre l'orizzonte, in certi casi anche a centinaia di chilometri di distanza.
    \item \textbf{Sky wave} (1-10 MHz): le onde sono riflesse dalla ionosfera e si propagano rimbalzando tra quest'ultima e la superficie terrestre, riuscendo a coprire distanze nell'ordine dei migliaia di chilometri.
    \item \textbf{Space wave} (da 30 MHZ): le onde richiedono un line-of-sight per poter essere ricevute correttamente, inoltre bisogna tenere in considerazione che maggiore è la frequenza maggiore sarà l'attenuazione. Il ricevitore oltre alla componente diretta riceve anche componenti aggiuntive, date dalla riflessione delle onde su ostacoli lungo il percorso.
\end{itemize}



Le space wave rappresentano il meccanismo più importante dato che la maggior parte dei sistemi di comunicazione ne fa affidamento. Le space wave sono soggette a diversi fenomeni di propagazione:
\begin{itemize}
    \item \textbf{Riflessione}: quando il segnale impatta con un oggetto liscio molto grande rispetto alla lunghezza d'onda può essere riflesso, ovvero rimbalza e prosegue con angolo differente, oppure passa attraverso.
    \begin{center}
        \begin{tikzpicture}
            \draw[thick] (-2,0) -- (2,0);
            \draw[thick, -latex] (-1,1) -- (-0.025,0.01);
            \draw[thick, -latex, dashed] (0.025,0.01) -- (1,1);
        \end{tikzpicture}
    \end{center}



    
   
    \item \textbf{Diffrazione}: quando il segnale è ostruito da oggetti con superfici taglienti, parte del segnale può modificare l'angolo con cui si propaga. Questo fenomeno permette in certe condizioni di ricevere un segnale anche in assenza di un line-of-sight a causa di un oggetto che crea una zona d'ombra, ovviamente l'energia ricevuta sarà minore rispetto a quella originaria del segnale.
    \begin{center}
        \begin{tikzpicture}
            \draw[thick] (-2,0) -- (-0.0625,0);
            \draw[thick] (0.0625,0) -- (2,0);
            \draw[thick, -latex] (0,1) -- (0,0.1);
           
            \draw[thick, -latex, dashed] (-0.1618033988749895, -0.1175570504584946) -- (-0.8090169943749476, -0.587785252292473);
            \draw[thick, -latex, dashed] (-0.061803398874989514, -0.1902113032590307) -- (-0.30901699437494756, -0.9510565162951535);
            \draw[thick, -latex, dashed] (0.061803398874989444, -0.19021130325903074) -- (0.30901699437494723, -0.9510565162951536);
            \draw[thick, -latex, dashed] (0.16180339887498946, -0.11755705045849467) -- (0.8090169943749473, -0.5877852522924734);
        \end{tikzpicture}
    \end{center}


    
    \item \textbf{Scattering}: quando il segnale impatta con oggetti di piccola dimensione rispetto alla lunghezza d'onda si ha una propagazione in più direzioni del segnale.
    \begin{center}
        \begin{tikzpicture}
            \shade[ball color=black!40, opacity=0.4] (0,0) circle (0.15cm);
            \draw[thick, -latex] (0,2) -- (0,0.1);
            \draw[thick, -latex, dashed] (0.16180339887498948, 0.11755705045849463) -- (0.8090169943749475, 0.5877852522924731);
            \draw[thick, -latex, dashed] (0.06180339887498949, 0.1902113032590307) -- (0.30901699437494745, 0.9510565162951535);
            \draw[thick, -latex, dashed] (-0.061803398874989465, 0.19021130325903074) -- (-0.30901699437494734, 0.9510565162951536);
            \draw[thick, -latex, dashed] (-0.16180339887498946, 0.11755705045849466) -- (-0.8090169943749473, 0.5877852522924732);
            \draw[thick, -latex, dashed] (-0.2, 2.4492935982947065e-17) -- (-1.0, 1.2246467991473532e-16);
            \draw[thick, -latex, dashed] (-0.1618033988749895, -0.1175570504584946) -- (-0.8090169943749476, -0.587785252292473);
            \draw[thick, -latex, dashed] (-0.061803398874989514, -0.1902113032590307) -- (-0.30901699437494756, -0.9510565162951535);
            \draw[thick, -latex, dashed] (0.061803398874989444, -0.19021130325903074) -- (0.30901699437494723, -0.9510565162951536);
            \draw[thick, -latex, dashed] (0.16180339887498946, -0.11755705045849467) -- (0.8090169943749473, -0.5877852522924734);
            \draw[thick, -latex, dashed] (0.2, -4.898587196589413e-17) -- (1.0, -2.4492935982947064e-16);
        \end{tikzpicture}
    \end{center}
 





    \item \textbf{Assorbimento e rifrazione}: si tratta di fenomeni meno importanti, ma che comunque possono modificare la propagazione del segnale. 
\end{itemize}

Gli effetti generati da questi fenomeni possono essere riassunti in due diversi tipi di attenuazione del segnale:
\begin{itemize}
    \item \textbf{Small-scale fading}: modella fluttazioni rapide della potenza del segnale su lunghezze paragonabili a quella dell'onda.
    \item \textbf{Large-scale fading}: modella la variazione della potenza del segnale su lunghe distanze.
\end{itemize}



\begin{center}
    \begin{tikzpicture}[every node/.style={rectangle,draw, minimum size=1.75em, align=center, font=\scriptsize},level/.style={sibling distance = 8.5cm/#1, level distance = 1.5cm}] 
        \node {Fading Types}
          child { node {Large Scale Fading} 
            child { node {Path Loss} }
            child { node {Shadowing} } 
          }
          child { node {Small Scale Fading} 
              child { node {Multipath \\ delay spread} 
                  child { node {Flat Fading} } 
                  child { node {Frequency \\ Selective \\ Fading} } 
              }
              child { node {Doppler \\ spread} 
                  child { node {Fast \\ Fading} } 
                  child { node {Slow \\ Fading} } 
              } 
              };
      \end{tikzpicture}
\end{center}


\subsection*{Large-scale fading}
Il LSF modella la variazione della potenza del segnale in base alla distanza fra trasmettitore e ricevitore, tipicamente variando per distanze nell'ordine del metro. Gli effetti sono modellati attraverso la combinazione di \textbf{path loss} e \textbf{shadowing}:

\paragraph*{Path Loss}
Il path loss rappresenta un'approssimazione delle equazioni di Maxwell.
\begin{equation}
    P_{RX} \approx P_{TX} \cdot \Gamma(f_0, d_0) \cdot \left( \frac{d_0}{d} \right)^n \quad \text{for } d > d_0
\end{equation}

dove \( \Gamma(f_0, d_0) \approx \left( \frac{\lambda}{4 \pi d_0} \right)^2 \) rappresenta il termine di campo vicino, \( P_{TX} \) è la potenza trasmessa, \( d_0 \) è la distanza di riferimento e \( n \) è l'esponente del path loss. Quest'ultimo dipende dall'ambiente in cui ci si trova, ad esempio in un ambiente urbano il valore di \( n \) varia tra 2.7 e 3.5. 

Il valore di \( d_0 \) è scelto in modo da poter misurare o stimare con precisione il path loss in condizioni controllate, in modo da poter avere un punto di riferimento da cui calcolare le perdite aggiuntive a distanze maggiori.
In reti cellulari che coprono aree estese si utilizza una distanza di riferimento di 1 km, mentre in reti progettate per aree più piccole e densamente popolate si utilizzano distanze di riferimento di 100 metri o anche 1 metro.
La formula, deterministica, che fornisce il valore di attenuazione dato dal path loss è:
\[
    A_{PL} = \frac{P_{RX}}{P_{TX}} = \Gamma(f_0, d_0) \cdot \left( \frac{d_0}{d} \right)^n
\] 
\[
    A_{PL}^{dB} = 10 \cdot \log_{10}(A_{PL}) = \Gamma(f_0, d_0)_{dB} + n \cdot 10 \cdot \log_{10}(d_0) - n \cdot 10 \cdot \log_{10}(d)
\]


Nel vuoto l'attenuazione del segnale in base alla distanza è data dalla relazione
\[
    P_{RX} = P_{TX} \cdot \frac{A}{4\pi d^2}  
\]
dove \( A \) è l'area dell'antenna in ricezione, che tipicamente è \( A = \lambda^2 \) dove \( \lambda \) è la lunghezza d'onda del segnale.


Per frequenze inferiori ai 6GHz l'attenuazione dipende dalle frequenze con una relazione quadratica, tuttavia oltre tale scoglio la lunghezza d'onda è sufficientemente piccola per interagire con molecole presenti nell'aria, in particulare ossigeno e vapore acqueo, incrementando ulteriormente l'attenuazione.

Nella progettazione di sistemi di comunicazione wireless è essenziale tenere in considerazione il rapportarto tra frequenza utilizzata e distanza copribile da una singola antenna.

\paragraph*{Shadowing}

Dati due punti alla stessa distanza dal trasmettitore, se si considerasse unicamente il path loss, si avrebbe la stessa attenuazione, tuttavia nella realtà vi è una componente aleatoria da dover considerare, modellabile tramite shadowing. La componente aleatoria è dovuta alla presenza di ostacoli differenti che coprono parzialmente il segnale ricevuto. La componente aleatoria è caratterizzata da una distribuzione log-normale con parametri $\mu = 0$ e $\sigma_S$, espressi in dB, quindi una distribuzione normale con parametri espressi in dB, ovvero la PDF è:
\[
    f_{A_s}(a_S) = \frac{1}{\sqrt{2\pi \sigma_S}} e^{-\frac{a_S^2}{2\sigma_S^2}}
\]




Gli effetti del path loss e dello shadowing sono sommati per ottenere le variazioni dovute al large scale fading. 
\[
    A_{LS} = A_{PL} \cdot A_S \quad \text{scala lineare}
\]

\[
    A_{LS}^{dB} = A_{PL}^{dB} + A_S^{dB} \quad \text{scala logaritmica}
\]


Il path loss è deterministico e dipende dall'esponente scelto in base all'ambiente circostante e dalla distanza tra trasmettitore e ricevitore, mentre lo shadowing è aleatorio e distribuito come una log-normale. Entrambi gli eggetti contribuisoono a variazioni nella potenza media ricevuta, le cui fluttuazioni sono significative solo per grandi distanze, considerando la lughezza d'onda.



\section*{Small scale fading}

Lo small scale fading modella le variazioni aleatorie della potenza istantanea su distanze nell'ordine della lunghezza d'onda, dovute ai vari fenomeni di propagazione delle onde che determinano la ricezione di repliche del segnale, ognuno con un certo ritardo, fase ed attenuazione. 
Il canale di tramissione è modellato come un filtro LTI, la cui risposta impulsiva dipende dagli effetti SSF:

\[
    h(t) = A_{LS} \sum_{\ell=0}^{N_c-1} \alpha_{\ell} e^{j\phi_{\ell}} \delta(t - \tau_{\ell})
\]

Dove \( A_{LS} \) è l'attenuazione dovuta al large scale fading (quindi path loss e shadowing), \( \alpha_{\ell} \) e \( \phi_{\ell} \) sono rispettivamente ampiezza e fase del segnale \(\ell\)-esimo, e \( \tau_{\ell} \) è il ritardo temporale. La risposta impulsiva del canale è la somma delle risposte impulsiva di ogni singolo percorso.

Per quanto riguarda i parametri delle varie repliche:
\begin{itemize}
    \item \textbf{Attenuazioni} ($\alpha_\ell$): sono modellate come variabili aleatorie, tipicamente con distribuzione di Rayleigh, derivante dal fatto che le componenti in fase e quadratura hanno una distribuzione normale.
    \[
        \alpha e^{j\phi_i} = x_i + j y_i \quad \text{con } x_i, y_i \sim \mathcal{N}(0, \sigma^2)
    \]
    \[
        \|\alpha e^{j\phi_i}\| = \sqrt{x_i^2 + y_i^2} \sim \text{Rayleigh}(\sigma)
    \]
    \item \textbf{Fasi} ($\phi_\ell$): sono modellate come variabili distribuite uniformemente nell'intervallo $[0, 2\pi]$.
\end{itemize}

La ricezione di più repliche, soprattutto se notevolemnte distanziate tra loro, genera ISI.
\[
    x(t) = \sum_{i} c_i g(t - iT) \quad \text{uscita filtro in ricezione (senza rumore)}
\]
\[
    g(t) = g_T(t) \ast h(t) \ast g_R(t) = g_N(t) \ast h(t) = \sum_{\ell=0}^{N_c-1} \alpha_{\ell} e^{j\phi_{\ell}} g_N(t - \tau_{\ell})
\]
\[
    g\left[k\right] = \sum_{\ell=0}^{N_c-1} \alpha_{\ell} e^{j\phi_{\ell}} g_N\left(kT - \tau_{\ell}\right)
\]

\[
    x\left[m\right] = \sum_{k=-\infty}^{+\infty} c_k g\left[m - k \right] = c_m g(0) + \sum_{\substack{k \neq m,\\ k=-\infty}}^{+\infty} c_{k} g\left[m-k\right]
\]


Sebbene quindi si scelga un filtro che rispetti la condizione di Nyquist, gli effetti del canale non permettono di rimuovere completamente l'ISI, generata dal fatto che $g\left[m-k\right]$ non si annulla.


Il delay spread permette di misurare la dispersione temporale introdotta dal canale:
\begin{itemize}
    \item $\sigma_\tau \ll T$: delay spread inferiore al symbol time, il canale è detto \textbf{flat fading}  e l'effetto dell'ISI è trascurabile in quanto le varie repliche ricevute fanno tutte riferimento allo stesso simbolo.
    \item $\sigma_\tau > T$: delay spread maggiore del symbol time, il canale è detto \textbf{multipath} e l'effetto dell'ISI non è più trascurabile in quanto le varie repliche, appartenti a simboli differenti, interferiscono tra loro.
\end{itemize}

In entrambi i casi il canale è detto \textbf{multipath} (TODO: non ha senso questo discorso) in quanto la ricezione del segnale avviene attraverso più replichem poi si aggiunge una classificazione ulteriore dovuta al delay spread.
Il delay spread in frequenza è espresso tramite la \textbf{coherence bandwidth}, definita come l'intervallo di frequenze per cui la risposta del canale è circa costante: 
\[
    B_c \approx \frac{1}{5\sigma_\tau}
\]
Se $B_c > B_S \approx \frac{1}{T}$, ovvero la banda del segnale trasmesso, non si avrà alcune distorsione, il canale risulta \textbf{flat fading}. In caso contrario, se $B_c < B_S$ il canale è detto \textbf{frequency selective}. 

La coherence bandwidth può essere determinata considerando la densità spettrale di potenza.
TODO questa parte non è chiara.
Quando $\sigma_{\tau} < T$ vale $B_c > B_s$ e il canale è flat fading, altrimenti se $\sigma_{\tau} > T$ vale $B_c < B_s$ e il canale è frequency selective.
I delay sono modellabili come variabili aleatorie $\tau$, perché per ottenere proprietà statistiche si possono applicare le operazioni standard per il calcolo del valor medio e della varianza, tuttavia ciò risulta particolarmente complesso e serve dunque approssimare tramite un procedimento approssimativo.


\[
   \overline{\tau} = \mathbb{E} \left[\tau\right] = \int_{0}^{+\infty} \tau f_{\tau}(\tau) d\tau \approx \sum_{\ell=0}^{N_c-1} \frac{\alpha_{\ell}^2}{\sum_{m=0}^{N_c-1} \alpha_{m}^2} \tau_{\ell}
\]

Per quanto riguarda la varianza:
\[  
    \sigma_\tau^2 = \mathbb{E}\left[({\tau - \overline{\tau}})^2\right] \approx \overline{\tau^2} - \overline{\tau}^2, \quad \overline{\tau^2} = \mathbb{E} \left[\tau^2\right] = \int_{0}^{+\infty} \tau^2 f_{\tau}(\tau) d\tau \approx \sum_{\ell=0}^{N_c-1} \frac{\alpha_{\ell}^2}{\sum_{m=0}^{N_c-1} \alpha_{m}^2} \tau_{\ell}^2
\]

Per modellare gli effetti di small scale fading si utilizza il parametro $\sigma_\tau$, ovvero il delay spread che dipende dall'ambiente in cui avviene la trasmissione. Il modello ottenuto risulta piuttosto semplice da utilizzare e secgliendo accuratamente i parametri può essere utilizzato per modellare diversi scenari di trasmissione. L'accuratezza ottenuta non è sempre fedele al caso reale, ma nella maggior parte dei casi non è richiesto, considerando anche che il canale è soggetto a cambiamenti repentini.

Il ber a parità di rumore dipende fortemente dalla distorsione introdotta nel canale:
\begin{itemize}
    \item \textbf{No fading}: aumentando il rapporto SNR è possibile ridurre drasticamente il BER con una potenza di trasmission contenuta.
    \item \textbf{Flat fading}: aumentando il rapporto SNR è ancora possibile ridurre il BER, ma sarà necessaria una potenza di trasmissione maggiore.
    \item \textbf{Frequency selective}: il BER non è riducibile oltre una certa soglia nonostante l'increment del SNR, principalemnte a causa delle interferenze generate dall'ISI.
\end{itemize}

Per quanto riguarda le decision variables:
\[
    x\left[m\right] = c_m + n\left[m\right] \quad \text{no fading} \quad P_e = Q(\sqrt{\text{SNR}})
\]

\[
    x\left[m\right] = \alpha c_m + n\left[m\right] \quad \text{flat fading} \quad P_e = \frac{1}{\text{SNR}}
\]

\[
    x\left[m\right] = c_m g(0) + \sum_{\substack{k \neq m,\\ k=-\infty}}^{+\infty} c_{k} g\left[m-k\right] + n\left[m\right] \quad \text{frequency selective}  
\]

Nell'ultimo caso, inizialmente è il rumore a dominare, quindi incrementando la potenza di rasmissione si ottengono miglioramente nel BER, tuttavia così facendo si incrementa anche l'energia dei simboli finendo in una zona in cui è l'ISI a dominare.
Il trend delle nuove generazioni radio è quello di incrementare il symbol-time. Questo comporta il rischio di ottenere con più facilità canali frequency selective anche in ambienti non particolarmente ostili. Per questo le classiche modulazioni PAM o QAM non risultano adatte in certi contesti, ma si fa uso di nuove tipologie di modulazioni multi-carrier come OFDM.


Le misurazioni del canale dipendono dalla frequenza del segnale e dall'ambiente in cui avviene la trasmissione.
%Typical values of delay spread are:
Valori tipici di delay spread sono:

\begin{itemize}
    \item $0.2 \, \mu s$ (area rurale),
    \item $0.5 \, \mu s$ (area suburbana),
    \item $3-8 \, \mu s$ (area urbana),
    \item $<2 \, \mu s$ (microcella urbana),
    \item $50-300 \, \text{ns}$ (picocella indoor).
\end{itemize}

\begin{center}
    \begin{tabular}{|c|c|c|}
        \hline
        \textbf{Ambiente} & \textbf{RMS Delay Spread ($\sigma_\tau$)} & \textbf{Note} \\
        \hline
        Urbano & $1300 \, \text{ns} \, (3500 \, \text{ns max})$ & NYC \\
        LTE ETU & Fino a $5 \, \mu s$ & Caso tipico medio \\
        Suburbano & $1960-2110 \, \text{ns}$ & Caso estremo medio \\
        Indoor & $10-50 \, \text{ns}$ & Edificio d'uffici \\
        Indoor & $70-94 \, \text{ns} \, (1470 \, \text{ns max})$ & Edificio d'uffici \\
        \hline
    \end{tabular}
\end{center}


Per esempio, in un'area urbana con dealy spread $\sigma_\tau = 4 \mu s$, avremo una coherence bandwidth di $B_c \approx \frac{1}{5 \sigma_\tau} = \frac{1}{20 \mu s} = 50 \text{kHz}$.
TODO: two path channels and comparison between awgn, flat fading and frequency selective BER.
\section*{Time varying channel}
Se il ricevitore di un segnale è in movimento il modello di canale risulta più complesso, in quanto è necessario aggiungere una dipendenza dal tempo ai guadagni e alle fasi alle varie repliche. 
Intuitivamente tali dipendenze dipendono dal fatto che gli effetti di small scale fading variano drasticamente anche per distanze paragonabili alla lunghezza d'onda, le quali possono essere nell'ordine del centimetro. 
Tipicamente il canale è descritto come sistema LTI in modo da poter sfruttare la sua risposta impulsive, ma nel caso di ricevitore mobile vi è una dipendenza dal tempo. 
Tuttavia nella maggior parte dei casi pratici l'assunzione LTI risulta valida. 

\[
    h(t, \tau) = A_{LS} \sum_{\ell=0}^{N_c-1} \alpha_{\ell}(t) e^{j\phi_{\ell}(t)} \delta(\tau - \tau_{\ell})
\]


Per il ritardo non si aggiunge dipendenza dal tempo in quanto, rispetto ai guadagni e alla fase, varia molto più lentamente, dato che le repliche viaggiano alla velocità della luce e quindi muoversi di pochi metri non ha un impatto significativo.






\subsection*{Effeto doppler}
L'effetto doppler, considerando un qualsiasi onda, è un fenomeno fisico che consiste nel cambiamento apparente della frequenza d'onda percepita da un osservatore raggiunto da un'onda emessa da una sorgente in movimento rispetto ad esso.
In particolare se la sorgente si avvicina la frequenza apparirà più elevata, mentre se si allontana sembrerà meno elevata, questo deriva dalla compressione (o allargamento) dei tempi in cui l'onda è ricevuta dall'osservatore.
Questo effetto ha delle implicazioni anche per quanto riguarda le onde radio utilizzate per la trasmissioni wireless, introducendo uno shift nelle frequenze del segnale ricevuto indicato come \textbf{doppler shift} ($f_d$).
Lo shift può essere determinato considerando la velocità con cui si sta muovendo il ricevitore:
\[
    d = vt \quad \text{distanza tra $x$ e $y$}
\]  

\[
    \Delta \tau = \frac{d}{c} = \frac{vt}{c} \quad \text{tempo impiegato dal segnale a percorrere $d$}
\]

\[
    y_x(t) = s(t) = \sin(2\pi f_c t) \quad \text{segnale ricevuto in $x$ al tempo $t$}
\]

\[
    y_y(t) = s(t-\Delta \tau) = \sin(2\pi f_c (t-\Delta \tau)) \quad \text{segnale ricevuto in $y$ al tempo $t$}
\]

\[
    = \sin\left(2\pi f_c \left(t - \frac{vt}{c}\right)\right) = \sin\left(2\pi \left(f_c - \frac{f_c v}{c}\right) t\right) = \sin\left(2\pi \left(f_c - f_d \right) t \right) \quad \boxed{f_d = -\frac{f_c v}{c}}
\]
Indicando con $f'$ la frequenza apparente percepita in $y$, ovvero $f_c - f_d$, vale la relazione:
\[
    f' = f_c - f_d = f_c \left(1 - \frac{v}{c}\right) = f_c \left( \frac{c - v}{c} \right)
\]
cioè un caso particolare della formula generale dell'effetto Doppler, ovvero
\[
    f' = f_c \left( \frac{c \pm v}{c \pm v_s} \right)
\]

dove:
\begin{itemize}
    \item $f$ è la frequenza trasmessa,
    \item $c$ è la velocità della luce,
    \item $v$ è la velocità del ricevitore rispetto al mezzo di propagazione,
    \item $v_s$ è la velocità della sorgente rispetto al mezzo di propagazione.
\end{itemize}


Il segno dipende dalla direzione del movimento, ovvero se la sorgente e l'osservatore si stanno avvicinando o allontanando. Nel nostro caso, si assume che il ricevitore si stia muovendo lontano dalla sorgente, quindi il segno è negativo, mentre il trasmettitore è fermo.
(TODO: controllare se effettivamente il segno è corretto e se il ricevitore si sta allontanando)
Il segnale ricevuto in $y$ risulta avere una frequenza inferiore a quello in $x$. Il termine $f_d$ assume un valore significativo solo per $v$ o $f_c$ molto grandi, in quanto a denominatore si ha la velocità della luce. In generale l'effetto introdotto non produce grandi errori, tuttavia nel caso di canale con introduzione di repliche si ha la ricezione di repliche con angoli differenti, generando un fenomeno più difficile da trattare e non deterministico, detto \textbf{doppler spread}.
In generale il doppler shift di ciascuna replica dipende dall'angolo $\theta$, quindi $f_d = f_c \frac{v \cos(\theta)}{c}$.
Esistono diversi modelli utilizzati per descrivere lo spreading in frequenza, 
che si verifica quando le varie repliche del segnale vengono aggregate al ricevitore.

Tra questi modelli, il modello dello spettro Doppler di Jakes è particolarmente noto, il quale presuppone che le repliche del segnale arrivino in maniera uniformemente distribuita da ogni direzione, con ciascuna repliche che apporta la stessa quantità di energia, cosicché nessuna direzione risulti favorita rispetto ad un'altra in termini di intensità del segnale ricevuto (\textit{scattering isotropico}) (TODO: è corretto dire che è scattering isotropico).
L'assunzione non è del tutto realistica, ma più che la forma dello spread è importante conoscere il fenomeno introdotto dal canale.

\begin{figure}[ht]
    \centering
    \includegraphics[width=0.875\textwidth]{imgs/jakes.jpg}
    \captionsetup{width=.5\textwidth}
    \caption*{Spettro doppler di Jakes:
        per il segnale $s$ vale \( S_S(f) = \frac{1}{2} \delta \left(f - f_c \right) + \frac{1}{2} \delta \left(f + f_c\right) \), 
        mentre per il segnale $y$ vale \( S_Y(f) = S_D(f) \ast S_S(f) \).
    }
\end{figure}

% include image imgs/jakes.jpg
\[
  S_D(f) = \frac{1}{\pi f_d \sqrt{1 - \left(\frac{f}{f_d}\right)^2}}
\]
Lo spettro originale è ripetuto su un intervallo maggiore di frequenza. L'effetto prodotto è l'allargamento dello spettro occupato (\textbf{Doppler spread}), derivante dal fatto che il segnale stocastico ricevuto assume la forma:

\[
    y(t) = s(t) d(t)
\]
dove $s(t)$ è il segnale trasmesso e $a(t)$ è il Jokes' doppler spectrum. In frequenza, la convoluzione tra queste due funzioni produce una funzione la cui durata (tempo) e banda (frequenza) è la somma delle due funzioni di partenza, per questo motivo si ha un un incremento della banda occcupata.

\[
    S_Y(f) = S_S(f) \ast S_D(f) 
\]

Lo studio della funzione di autocrrelazione del canale wireless nel caso del modello di Jakes permette di giungere alla definizione del \textbf{coherence time}, ovvero l'intervallo temporale entro cui è possibile considerare il canale costante e dunque rappresentabile come un sistema LTI.


(TODO: tau o t?)
\[
    R_D (\tau) = J_0(2\pi f_d \tau) \quad \text{funzione di autocorrelazione time varying channel}
\]
Per $x=\frac{1}{2}$ si ha che $j_0(2\pi x) \approx 0$, quindi il canale può considerarsi incorrelato per $f_d T_c = \frac{1}{2}$.

dove $J_0$ è la funzione di bessel di prima specie di ordine 0.  Se la funzione si annulla il canale può essere considerato non correlato, ovvero il canale nei due istanti assume valori indipendenti e dunque varia.

\[
    f_d T_c = \frac{1}{2} \quad \Rightarrow \quad  T_c = \frac{1}{2 f_d} = \frac{1}{2} \frac{c}{f_c v} \quad \text{coherence time}
\]

Lo stesso concetto può essere espresso anche in termini di distanza, molto importante per l'utilizzo di antenne direzionali.
\[
    d_c = v T_c = \frac{1}{2} v \frac{c}{f_c v} = \frac{\lambda}{2}
\]
Questo implica che segnali ricevuti a distanze nell'ordine di metà lunghezza d'onda possono essere considerati incorrelati (?).
Se $T < T_c$ si può asssumere che il canale sia modellabile come LTI in quanto la risposta risulta costante nel tempo per almeno $T_c$, inferiore al tempo dei simboli.
Incrementando il rate di riduce il tempo dei simboli, permettendo di modellare il canale come LTI, tuttavia si rischia di incorrere in un canale frequency selective per cui sono necessarie contromisure per contrastare l'ISI generato.  




\begin{center}


\begin{tikzpicture}
    \begin{axis}[
        title={Receiving Power vs. Distance},
        xlabel={Distance (m)},
        ylabel={Receiving power (dBm)},
        xmin=0, xmax=40,
        ymin=-60, ymax=-10,
        xtick={0,10,20,30,40},
        ytick={-50,-40,-30,-20},
        legend pos=north east,
        ymajorgrids=true,
        grid style=dashed,
    ]

    \addplot[
        color=orange,
        mark=none,
        smooth,
        tension=0.5,
    ]
    coordinates {
	   (0.1,-39.94) (0.29,-40.45) (0.48,-39.42) (0.58,-39.87) (0.68,-40.39) (0.77,-41.55) (0.97,-42.0) (1.07,-41.03) (1.26,-42.0) (1.45,-40.45) (1.65,-45.74) (1.74,-44.19) (1.84,-43.42) (1.94,-40.32) (2.03,-39.48) (2.23,-37.87) (2.42,-36.45) (2.52,-36.84) (2.62,-37.68) (2.81,-39.42) (3.0,-39.87) (3.2,-40.13) (3.39,-40.39) (3.58,-40.84) (3.78,-40.71) (3.87,-40.32) (4.16,-41.23) (4.26,-40.19) (4.36,-39.74) (4.46,-40.52) (4.75,-39.74) (4.94,-38.9) (5.04,-37.42) (5.13,-36.58) (5.23,-37.81) (5.33,-40.84) (5.52,-40.39) (5.62,-38.97) (5.71,-38.32) (5.91,-39.68) (6.0,-41.03) (6.2,-40.32) (6.3,-40.58) (6.39,-39.94) (6.49,-39.35) (6.59,-36.84) (6.68,-36.26) (6.88,-38.52) (7.07,-39.74) (7.17,-38.97) (7.26,-39.61) (7.46,-38.9) (7.55,-38.06) (7.65,-37.61) (7.85,-36.45) (8.04,-36.45) (8.23,-38.52) (8.33,-39.35) (8.52,-38.77) (8.62,-39.42) (8.81,-40.45) (8.91,-39.87) (9.01,-39.35) (9.2,-38.19) (9.39,-37.42) (9.59,-37.68) (9.78,-38.45) (9.88,-34.77) (10.07,-34.06) (10.17,-34.26) (10.36,-35.35) (10.56,-34.97) (10.75,-35.35) (10.94,-40.06) (11.14,-38.71) (11.23,-39.48) (11.33,-40.0) (11.43,-40.65) (11.62,-39.61) (11.72,-38.58) (11.91,-37.42) (12.11,-37.81) (12.3,-37.16) (12.49,-37.23) (12.59,-38.13) (12.69,-38.65) (12.88,-40.13) (12.98,-41.23) (13.17,-41.55) (13.27,-41.48) (13.37,-43.48) (13.56,-44.06) (13.66,-43.74) (13.85,-42.32) (14.04,-39.61) (14.14,-39.81) (14.24,-43.35) (14.43,-42.77) (14.62,-41.87) (14.82,-42.9) (15.01,-42.58) (15.21,-44.26) (15.3,-43.61) (15.4,-42.45) (15.59,-44.65) (15.79,-43.35) (15.98,-41.55) (16.17,-42.77) (16.37,-42.52) (16.56,-42.26) (16.76,-41.35) (16.85,-40.45) (16.95,-39.23) (17.05,-38.52) (17.24,-40.58) (17.34,-41.42) (17.53,-42.39) (17.72,-42.71) (17.92,-42.84) (18.11,-42.71) (18.21,-42.19) (18.31,-41.61) (18.4,-42.19) (18.5,-42.97) (18.69,-44.26) (18.89,-42.97) (18.98,-42.52) (19.08,-41.74) (19.27,-40.71) (19.47,-40.84) (19.66,-39.23) (19.95,-39.03) (20.15,-37.48) (20.24,-37.68) (20.82,-39.29) (21.02,-39.94) (21.21,-38.71) (21.31,-39.23) (21.4,-39.74) (21.5,-40.9) (21.69,-41.42) (21.79,-40.45) (21.99,-41.48) (22.18,-39.74) (22.37,-45.03) (22.47,-43.61) (22.57,-42.71) (22.66,-39.68) (22.86,-38.97) (22.95,-39.29) (23.15,-35.74) (23.24,-36.13) (23.34,-36.9) (23.54,-38.71) (23.73,-39.1) (23.92,-39.35) (24.12,-39.81) (24.31,-40.06) (24.41,-41.55) (24.6,-42.71) (24.7,-42.52) (24.89,-42.9) (25.08,-42.45) (25.28,-41.81) (25.38,-40.45) (25.47,-39.81) (25.57,-38.52) (25.67,-41.87) (25.86,-42.97) (25.96,-42.84) (26.05,-41.48) (26.25,-42.58) (26.34,-44.19) (26.44,-43.48) (26.63,-42.71) (26.73,-43.48) (26.83,-42.45) (26.92,-39.94) (27.12,-39.03) (27.31,-41.42) (27.41,-42.97) (27.6,-41.42) (27.8,-41.61) (27.89,-40.9) (27.99,-40.32) (28.18,-39.16) (28.38,-38.84) (28.57,-40.84) (28.77,-41.61) (28.96,-41.35) (29.15,-43.03) (29.35,-42.32) (29.44,-40.97) (29.54,-40.26) (29.64,-40.97) (29.83,-40.32) (30.02,-40.84) (30.12,-41.1) (30.22,-37.94) (30.31,-37.23) (30.51,-36.71) (30.7,-37.74) (30.9,-37.42) (31.09,-37.81) (31.19,-37.94) (31.28,-42.97) (31.48,-41.55) (31.57,-42.45) (31.67,-41.68) (31.77,-43.68) (31.86,-43.1) (32.06,-42.13) (32.25,-40.45) (32.35,-41.23) (32.45,-40.65) (32.64,-39.61) (32.83,-39.74) (32.93,-40.13) (33.03,-40.71) (33.22,-42.0) (33.32,-43.03) (33.51,-44.19) (33.61,-44.06) (33.7,-45.87) (33.9,-46.71) (34.0,-46.71) (34.09,-45.61) (34.19,-45.16) (34.29,-43.68) (34.38,-42.77) (34.48,-41.81) (34.67,-47.16) (34.77,-46.45) (34.87,-45.61) (35.06,-45.23) (35.16,-45.16) (35.35,-45.42) (35.45,-44.97) (35.54,-47.29) (35.64,-46.52) (35.74,-45.68) (35.84,-44.9) (36.03,-46.97) (36.22,-45.87) (36.42,-44.52) (36.61,-45.16) (36.71,-44.84) (36.9,-45.03) (37.09,-44.0) (37.19,-43.29) (37.29,-42.84) (37.38,-41.16) (37.48,-42.06) (37.58,-42.77) (37.68,-44.26) (37.77,-43.68) (37.87,-45.03) (38.06,-45.35) (38.26,-45.16) (38.45,-45.16) (38.64,-44.52) (38.74,-44.19) (38.84,-44.84) (39.03,-46.97) (39.13,-46.19) (39.32,-45.48) (39.42,-45.42) (39.61,-43.16) (39.71,-43.1) (40.0,-41.94) (40.19,-42.13) (40.39,-41.35) (40.48,-40.13)  
    };	
    \addlegendentry{fast fading}

    \addplot[
        color=black,
        mark=none,
        smooth,
        tension=0.7,
    ]
    coordinates {
        (0.0,-40.0) (1.27,-40.22) (2.53,-38.96) (3.92,-38.92) (5.26,-39.14) (6.52,-37.69) (7.79,-37.56) (9.05,-35.95) (10.32,-36.58) (12.94,-41.45) (19.37,-38.56) (21.02,-39.55) (22.29,-39.07) (24.82,-41.29) (26.08,-41.52) (27.35,-40.04) (31.14,-40.54) (32.41,-40.31) (33.67,-44.89) (34.94,-45.58) (36.2,-44.71) (37.47,-44.76) (40, -43.5)
    };

    \addlegendentry{slow fading}
    
    \end{axis}
\end{tikzpicture}


\end{center}
    



\begin{itemize}
    \item \textbf{Slow fading}: l'effetto doppler è trascurabile ($B_S \gg f_d$, $T_c \gg T$)
    \item \textbf{Fast fading}: l'effetto doppler distorce notevolemnte il segnale, rendendo difficile ridurre il BER ($B_S < f_d, T_c < T$)
\end{itemize}
Se la banda del segnale è maggiore del doppler shift la convoluzione genera un comportamento non molto significativo.

In generale gli effeti LSF determinano la dimensione della cella di copertura, considerando anche la frequenza di trasmissione.


Come esempio, consideriamo il caso di trasmissione di un segnale a una frequenza \( f_c = 2.1 \si{GHz} \) in un'area suburbana. L'utente si muove alla velocità di $90 \si{km/h}$, ovvero \( v = 25 \si{m/s} \). Lo scenario è caratterizzato da un delay spread \( \sigma_\tau = 2 \si{\mu s} \).

Per la trasmissione, la banda del segnale è \( B_s = 2 \si{MHz} \), da cui possiamo approssimare \( T \) come \( T \approx \frac{1}{B_s} = 500 \si{ns} \), rappresentando la durata di un simbolo nella trasmissione dei dati.
Un importante aspetto di questo scenario è il Doppler spread, calcolato utilizzando \( f_d = \frac{f_c v}{c} = \frac{2.1 \times 10 ^9 \si{Hz} \cdot 25 \si{m/s}}{3 \times 10^8 \si{m/s}} = 175\si{Hz}\).
Il tempo di coerenza del canale \( T_c \) è quindi derivato da \( T_c = \frac{1}{2f_d} \approx 3 \si{ms}\), descrivendo il periodo di tempo durante il quale le condizioni del canale sono relativamente costanti.
Inoltre, la banda di coerenza del canale \( B_c \) è calcolata dal delay spread \( \sigma_\tau \), risultando in \( B_c = \frac{1}{\sigma_\tau} = 500 \si{kHz} \). Questa banda misura l'intervallo di frequenze su cui la risposta del canale rimane uniforme.
Il canale è quindi definibile come \textit{slow} poiché \( B_s \gg f_d \) (\( T \ll T_c \)), indicando che i cambiamenti nel canale dovuti agli effetti Doppler sono relativamente lenti rispetto alla larghezza di banda del segnale e al tempo di simbolo.
Il canale è anche definibile come \textit{frequency-selective} a causa di \( B_s > B_c \) (\( T < \sigma_\tau \)), implicando che diversi segmenti della larghezza di banda possono subire variazioni delle caratteristiche di fading. 


















\subsection*{Ground Wave Propagation}
Ground wave propagation is a mode of radio wave propagation that enables radio signals to travel across the Earth's surface. Illustrations show how ground waves bend following the curvature of the Earth, allowing the reception of signals over the horizon.

\begin{itemize}
    \item The wave propagates by following the curvature of the Earth, which allows signals to reach receivers located beyond the line of sight, sometimes extending to hundreds of kilometers.
    \item This propagation mode is predominantly valid for frequencies below 2 MHz, encompassing the LF and MF bands.
\end{itemize}


\subsection*{Sky Wave Propagation}

Sky wave propagation involves the reflection of radio waves by the ionosphere back to Earth's surface. This phenomenon is particularly important for high-frequency (HF) signals.

\begin{itemize}
    \item Within certain frequency ranges, specifically around 10 MHz, the ionosphere acts as a reflective layer, bouncing signals back toward the Earth.
    \item The signals can 'hop' between the ionosphere and the Earth, enabling long-distance communication over several thousand kilometers.
    \item This type of propagation is most effective in the HF band.
\end{itemize}


\subsection*{Space Wave Propagation}

As the frequency of the radio waves increases beyond 30 MHz, propagation increasingly occurs via direct line-of-sight paths.

\begin{itemize}
    \item Frequencies above 30 MHz typically utilize space wave propagation, which primarily involves a direct, line-of-sight path.
    \item The received signal is a combination of the direct wave and additional components reflected or refracted by objects in the environment.
    \item Higher frequencies are subject to greater propagation attenuation, meaning the signal weakens more rapidly with distance.
\end{itemize}





\subsection*{The Wireless Propagation Channel (Space Wave)}

Space wave propagation is essential in the context of mobile services operating in the frequency range of 30 MHz to 30 GHz. This propagation is primarily influenced by the following physical phenomena:

\begin{itemize}
    \item \textbf{Reflection, Diffraction, Scattering:} These are the principal physical interactions affecting space wave propagation. Together, they lead to the phenomena of large-scale and small-scale fading, impacting the reliability and quality of the received signal.
\end{itemize}



\subsection*{Propagation Phenomena}

The propagation of wireless signals is governed by three major mechanisms:

\begin{enumerate}
    \item \textbf{Reflection:} Occurs when a signal encounters an object much larger than its wavelength, resulting in the signal being bounced back.
    \item \textbf{Diffraction:} Occurs when the radio path between the transmitter and receiver is obstructed by a sharp edge or object, causing the signal to bend around the obstacle.
    \item \textbf{Scattering:} Caused by the signal hitting irregularities or small objects in the medium, leading to the dispersion of the signal in multiple directions.
\end{enumerate}

Detailed diagrams can illustrate these phenomena more clearly:

% Insert diagrams for transmission, reflection, diffraction, absorption, and scattering here



\subsection*{Large-Scale Fading}

Large-scale fading refers to signal strength variations over large distances, caused by path-loss and shadowing effects. The characteristics of large-scale fading include:

\begin{itemize}
    \item Described by propagation models which estimate average signal strengths based on the distance between the transmitter and receiver.
    \item Takes into account the averaged received power, which notably changes over distances on the order of a meter or more.
    \item It can be mathematically modeled by combining path-loss and shadowing effects.
\end{itemize}

\subsection*{Path-Loss in Large-Scale Fading}

The path-loss component of large-scale fading simplifies the Maxwell equations into models that predict signal decay over distance.

\begin{itemize}
    \item These models may vary in complexity, but generally express the mean power decay as proportional to \(d^n\), where \(d\) is the distance and \(n\) is the path-loss exponent, depending on the environment.
    \item The average received power \(P_{rx}\) at a distance \(d\) from the transmitter is approximately given by the equation:
    \[
    P_{rx} \approx P_{tx} \cdot \left( \frac{d_0}{d} \right)^n \quad \text{for } d > d_0
    \]
    where \(P_{tx}\) is the transmitted power, \(d_0\) is a reference distance, and \(n\) is the path-loss exponent.
    \item The near field term \(F(d_0, \lambda)\) reflects the free space propagation loss at the reference distance \(d_0\) and is approximated by:
    \[
    F(d_0, \lambda) \approx \left( \frac{\lambda_0}{4 \pi d} \right)^2
    \]
    where \(\lambda\) is the wavelength of the transmitted signal.
\end{itemize}

\begin{table}[h!]
\centering
\begin{tabular}{ll}
\hline
\textbf{Environment} & \textbf{Path Loss Exponent (\(n\))} \\
\hline
Free space & 2 \\
Urban area cellular radio & 2.7--3.5 \\
Urban area cellular (obstructed) & 3--5 \\
In-building line-of-sight & 1.6--1.8 \\
Obstructed in-building & 4--6 \\
Obstructed factories & 2--3 \\
\hline
\end{tabular}
\caption{Path loss exponents for different environments.}
\label{table:pathlossexponents}
\end{table}







\subsection*{Large-Scale Fading: Attenuation Due to Frequency}

The attenuation of wireless signals is also dependent on the frequency. The key points are:

\begin{itemize}
    \item For frequencies below 6 GHz, channel attenuation largely follows a square law relative to the carrier frequency.
    \item Above this threshold, the attenuation is influenced more by physical factors, such as absorption by atmospheric constituents like oxygen and water vapor.
    \item Millimeter-wave (mmWave) frequencies experience significant attenuation, making them challenging for long-range communication without the aid of technologies like beamforming.
\end{itemize}
\subsection*{Path-Loss and Cell Size}

The size of cellular network cells and the path-loss are interrelated as follows:

\begin{itemize}
    \item Path-loss attenuation becomes more significant at the edges of a cell, potentially exceeding 100 dB.
    \item Higher carrier frequencies lead to greater attenuation, thereby reducing the effective cell radius.
    \item Consequently, larger cells tend to use lower frequencies to ensure coverage, while smaller cells, which aim to provide high capacity, often operate at higher frequencies, including mmWave bands.
\end{itemize}

The relationship between cell size and frequency can be encapsulated by the inequality \(r \propto \frac{1}{A}\), indicating that cell radius (\(r\)) is inversely proportional to the attenuation (\(A\)).

\subsection*{Large-Scale Fading: Shadowing}

Shadowing is a phenomenon that contributes to variations in received signal strength even when the transmitter-receiver distance remains constant.

\begin{itemize}
    \item Shadowing causes random variations in the average signal attenuation.
    \item It is characterized as a log-normally distributed random variable \( A_S \) with a mean \( \mu \) of 0 and a standard deviation \( \sigma_S \), typically in the range of 0 to 9 dB.
\end{itemize}

The probability density function (pdf) for shadowing \( A_S \) in dB is given by:
\[
p(A_S) = \frac{1}{A_S \sqrt{2\pi \sigma_S}} e^{-\frac{(\ln(A_S))^2}{2\sigma_S^2}}
\]

\subsection*{Modeling Large-Scale Fading with Shadowing}

Considering a channel affected only by path-loss and shadowing, the received power \( P_{RX} \) can be expressed as:

\begin{itemize}
    \item \( P_{RX} \) is the product of the transmitted power \( P_{TX} \), path-loss \( L_{PL} \), and shadowing \( A_S \).
    \item Shadowing makes \( P_{RX} \) a random variable, leading to variations in the received signal level.
\end{itemize}

Given the mean path-loss \( \overline{PL} \) in dBm and the shadowing variable \( A_S \) in dB, the received power in dBm is modeled as:
\[
P_{RX} = P_{TX} + \overline{PL} + A_S
\]

For example, if \( P_{TX} + \overline{PL} = -100 \) dBm and \( \sigma_S = 3 \) dB, the received power \( P_{RX} \) is a random variable distributed around -100 dBm with a standard deviation of 3 dB.

The probability density function of \( P_{RX} \) in dBm is then:
\[
p_{P_{RX},dBm}(P) = \frac{1}{\sqrt{2\pi\sigma_S}} e^{-\frac{(P + 100)^2}{2\sigma_S^2}}
\]

And in linear scale:
\[
p_{P_{RX},mW}(R) = \frac{1}{\sqrt{2\pi\sigma_S \log(10)}} e^{-\frac{(\log_{10}(R) + 13)^2}{2\sigma_S^2}}
\]


\subsection*{Large-Scale Fading: Combined Effects}

The received power in a wireless channel is influenced by multiple fading effects:

\begin{itemize}
    \item The total received power \( P_{RX} \) in dBm is given by:
    \[
    P_{RX} = P_{TX} + A_{PL} + A_{S} + A_{SS}
    \]
    where \( A_{PL} \) is the path-loss (deterministic), \( A_{S} \) is the shadowing (log-normally distributed), and \( A_{SS} \) is the small-scale fading (rapid fluctuations).

    \item The path-loss \( A_{PL} \) is deterministic and typically modeled as a function of distance \( d \).

    \item Shadowing \( A_{S} \) accounts for large-scale variations in signal power due to obstacles in the propagation environment.

    \item Small-scale fading \( A_{SS} \), in contrast, is characterized by rapid fluctuations in signal amplitude, phase, or multipath delays.
\end{itemize}
\subsection*{Understanding Fading Through Superposition}

The superposition of path-loss, shadowing, and small-scale fading creates the observed signal power variation over distance, as illustrated in the accompanying diagrams.

\begin{itemize}
    \item The overall fading profile is a combination of these three effects.
    \item The linear scale plots for path-loss, shadowing, and small-scale fading can be summed to show the composite effect on the received signal strength.
\end{itemize}

The figure below demonstrates how each component contributes to the total fading experienced in a wireless communication channel.


\subsection*{Propagation Channel: Small-Scale Fading}

The characteristics of small-scale fading are critical in defining the behavior of a propagation channel:

\begin{itemize}
    \item A wireless propagation channel can be modeled as a Linear Time-Invariant (LTI) system.
    \item The channel's response \( h(t) \) captures the small-scale fading characteristics that result in rapid fluctuations of the received signal strength.
    \item The output of the channel \( y(t) \), which is the received signal, is the convolution of the input signal \( s(t) \) with the channel's impulse response:
    \[
    y(t) = s(t) \ast h(t)
    \]
    \item Additive White Gaussian Noise (AWGN), denoted as \( w(t) \), is also present at the receiver, affecting the signal.
\end{itemize}

The small-scale fading results from multiple propagation paths, and the received signal is a superposition of numerous copies of the transmitted signal, each affected by reflection, diffraction, and scattering.
\subsection*{Small-Scale Fading}

Small-scale fading has several key aspects:

\begin{itemize}
    \item It accounts for the random variations in the signal's instantaneous power over distances of the order of a wavelength.
    \item The multitude of waves, each carrying a replica of the transmitted signal with varying delays and amplitudes, leads to constructive and destructive interference at the receiver, manifesting as fading.
\end{itemize}

Each path contributes differently to the received signal based on the propagation phenomena it experiences:

\begin{itemize}
    \item Direct waves travel straight from the transmitter to the receiver.
    \item Reflected waves bounce off surfaces before reaching the receiver.
    \item Diffracted waves bend around obstacles.
    \item Scattered waves result from irregularities in the path.
\end{itemize}

The combined effect of these multiple paths can be observed in the resultant signal's amplitude and phase, often summarized as a multipath fading profile.


\subsection*{Mathematical Representation of Small-Scale Fading}

The small-scale fading phenomenon can be described mathematically as follows:

\begin{itemize}
    \item The complex envelope of the received signal \( y(t) \) is the sum of multiple delayed replicas of the transmitted signal \( s(t) \), each with its own attenuation and phase shift, represented by:
    \[
    y(t) = A_{LS} \sum_{\ell=0}^{N_c-1} \alpha_{\ell} e^{j\phi_{\ell}} s(t - \tau_{\ell})
    \]
    where \( A_{LS} \) is the large-scale fading component, \( \alpha_{\ell} \) and \( \phi_{\ell} \) are the amplitude and phase of the \(\ell\)th path, respectively, and \( \tau_{\ell} \) is the time delay.
    
    \item This is equivalent to the convolution of the transmitted signal with the channel's impulse response:
    \[
    y(t) = s(t) \ast h(t)
    \]
    with the impulse response \( h(t) \) given by:
    \[
    h(t) = A_{LS} \sum_{\ell=0}^{N_c-1} \alpha_{\ell} e^{j\phi_{\ell}} \delta(t - \tau_{\ell})
    \]
\end{itemize}

\subsection*{Small-Scale Fading: Rayleigh Distribution}

For the statistical modeling of small-scale fading:

\begin{itemize}
    \item The path gains \( \alpha_{\ell} \) are typically modeled as random variables with a Rayleigh distribution, particularly in non-line-of-sight (NLOS) environments where there is no direct path between transmitter and receiver.
    \item The path phases \( \phi_{\ell} \) are modeled as uniformly distributed variables over the interval \( [0,2\pi] \).
    \item These statistical properties lead to the received signal strength varying rapidly over short distances or short time intervals, characteristic of small-scale fading.
\end{itemize}







\subsection*{Channel Gain Characterization}

The statistical nature of channel gains in small-scale fading is characterized as follows:

\begin{itemize}
    \item The amplitude \( \alpha \) of each multipath component follows a Rayleigh distribution for NLOS propagation:
    \[
    p(\alpha) = \begin{cases} 
    \frac{\alpha}{\sigma^2} e^{-\frac{\alpha^2}{2\sigma^2}} & \alpha \geq 0 \\
    0 & \alpha < 0
    \end{cases}
    \]
    where \( \sigma \) is the scale parameter of the Rayleigh distribution.

    \item The power \( s \) of the channel, defined as \( s = \alpha^2 \), follows an exponential distribution:
    \[
    p(s) = \begin{cases} 
    \frac{1}{2\sigma^2} e^{-\frac{s}{2\sigma^2}} & s \geq 0 \\
    0 & s < 0
    \end{cases}
    \]
\end{itemize}

These distributions describe the variation of signal amplitude and power due to multipath effects in a wireless channel.
The multipath propagation channel introduces time dispersion, which can lead to inter-symbol interference (ISI):

\begin{itemize}
    \item Multipath components arrive at the receiver at different times, creating copies of the signal that can add constructively or destructively.
    \item The impulse response of the channel captures this effect, showing spikes at delays corresponding to the arrival times of the multipath components.
    \item Time dispersion is a key factor that affects the design of communication systems, particularly in terms of equalization and symbol timing.
\end{itemize}
The figures and equations provided illustrate the impact of small-scale fading on the signal's amplitude and power distribution, as well as the time dispersion effect due to multipath.


\subsection*{Signal Processing in Multipath Channels}

The behavior of a transmitted signal \( s(t) \) as it propagates through a multipath channel and is received as \( y(t) \), including noise \( w(t) \), can be described by:

\begin{equation}
    y(t) = \left( \sum_{\ell=0}^{N_c-1} \alpha_{\ell} s(t - \tau_{\ell}) \right) \ast h(t) + w(t)
\end{equation}

where:
\begin{itemize}
    \item \( \alpha_{\ell} \) are the path gains for each multipath component.
    \item \( \tau_{\ell} \) are the time delays for each path.
    \item \( h(t) \) is the channel impulse response.
    \item \( w(t) \) represents the noise.
\end{itemize}

The channel impulse response \( h(t) \) is a summation of impulses delayed by \( \tau_{\ell} \) and scaled by \( \alpha_{\ell} \), reflecting the multipath effects:
\begin{equation}
    h(t) = \sum_{\ell=0}^{N_c-1} \alpha_{\ell} \delta(t - \tau_{\ell})
\end{equation}

The received signal \( y(t) \) thus consists of the sum of delayed and attenuated replicas of the transmitted signal, which interfere with each other, potentially causing inter-symbol interference (ISI).

\subsection*{Superposition of Multipath Components}

The superposition of the multipath components at the receiver can be visualized by their individual contributions, as depicted in the provided graph. The graph illustrates how the delayed replicas of the transmitted signal combine, with their amplitudes and phases, to form the received signal.
\subsection*{Analytical Representation of the Received Signal}

The analytical representation of the received signal, neglecting noise, is given by the convolution of the transmitted signal with the channel's impulse response:
\begin{equation}
    y(t) = s(t) \ast h(t) = \sum_{\ell=0}^{N_c-1} \alpha_{\ell} s(t - \tau_{\ell})
\end{equation}

This equation highlights the impact of each path's gain and delay on the form of the received signal, which is critical in the design of communication systems to mitigate the effects of ISI.


\subsection*{Signal Reception in a Multipath Channel}

In the context of small-scale fading and multipath channels, the received signal is processed as follows:

\begin{itemize}
    \item The output of the receive filter, neglecting noise, is given by:
    \[
    x(t) = \sum_{i} c_i g(t - iT)
    \]
    where \( g(t) \) is the convolution of the received pulse \( g_r(t) \), channel impulse response \( h(t) \), and the transmit pulse \( g_T(t) \), such that:
    \[
    g(t) = g_r(t) \ast h(t) = g_T(t) \ast h(t) = \sum_{\ell=0}^{N_c-1} \alpha_{\ell} e^{j\phi_{\ell}} g_T(t - \tau_{\ell})
    \]
    
    \item The channel's impulse response is modeled as:
    \[
    h(t) = \sum_{\ell=0}^{N_c-1} \alpha_{\ell} e^{j\phi_{\ell}} \delta(t - \tau_{\ell})
    \]
    with each \( \delta(t - \tau_{\ell}) \) representing a path with a delay \( \tau_{\ell} \).
\end{itemize}

\subsection*{Decision Variable for Symbol Detection}

In a multipath channel, the decision variable \( x(m) \) for the \( m \)th symbol period is:

\begin{equation}
    x(m) = x(t) \Big|_{t=mT} = \sum_{k} c_{m-k} g(kT) = c_m g(0) + \sum_{\substack{k \\ k \neq 0}} c_{m-k} g(kT)
\end{equation}

where:
\begin{itemize}
    \item \( g(kT) \) is the sampled channel impulse response, incorporating all multipath components at the symbol rate.
    \item \( c_m \) represents the transmitted symbol coefficients.
    \item The sum over \( k \neq 0 \) represents inter-symbol interference (ISI) from other symbols due to the multipath spread.
\end{itemize}

This formulation shows how multipath components arriving at different times contribute to ISI, making the correct detection of symbols more challenging.

\subsection*{Time Dispersion and ISI}

The time dispersion caused by multipath propagation is illustrated in the impulse response of the channel figure. It shows discrete reflections from multiple paths arriving at different times:
The precise modeling of this effect is essential for designing robust communication systems that can mitigate the adverse impacts of ISI.


\subsection*{Delay Spread}

The concept of delay spread is central to understanding the time dispersion of a channel:

\begin{itemize}
    \item Delay spread \( \sigma_{\tau} \) quantifies the extent of time dispersion in the channel.
    \item A small delay spread \( \sigma_{\tau} < T \), where \( T \) is the symbol time, suggests that the channel has flat fading characteristics with only one resolvable path.
    \item A large delay spread \( \sigma_{\tau} > T \) indicates multiple resolvable paths, causing frequency selective fading due to multipath interference.
\end{itemize}

\subsection*{Coherence Bandwidth}

Coherence bandwidth \( B_c \) is a measure related to the delay spread and characterizes the frequency selectivity of the channel:

\begin{itemize}
    \item \( B_c \) is inversely proportional to the delay spread, with \( B_c \approx \frac{1}{5\sigma_{\tau}} \).
    \item For \( \sigma_{\tau} < T \), \( B_c \) is greater than the signal bandwidth \( B_s \), indicating flat fading.
    \item For \( \sigma_{\tau} > T \), \( B_c \) is less than \( B_s \), indicating a frequency selective (multipath) channel.
\end{itemize}

The figures below show the relationship between the delay spread, coherence bandwidth, and the symbol time \( T \):
Understanding these parameters is crucial for the design of communication systems, especially in determining the required equalization techniques to combat ISI and choosing the appropriate modulation schemes to maximize data throughput while maintaining signal integrity.

\subsection*{Calculating Delay Statistics}

The statistical properties of delay in a multipath channel, such as mean excess delay and root mean square (RMS) delay spread, can be computed as follows:

\begin{itemize}
    \item The mean excess delay \( \bar{\tau} \) is defined as the expected value of the delay, weighted by the power of each path, and can be calculated using:
    \[
    \bar{\tau} = \sum_{\ell=0}^{N_c-1} \frac{\alpha_{\ell}^2}{\sum_{i=0}^{N_c-1} \alpha_{i}^2} \tau_{\ell}
    \]
    
    \item The RMS delay spread \( \sigma_{\tau} \) quantifies the dispersion of delays and is given by:
    \[
    \sigma_{\tau}^2 = \sum_{\ell=0}^{N_c-1} \frac{\alpha_{\ell}^2}{\sum_{i=0}^{N_c-1} \alpha_{i}^2} (\tau_{\ell} - \bar{\tau})^2
    \]
    It measures the spread of multipath delays around the mean excess delay and is critical for determining the coherence bandwidth of the channel.
\end{itemize}
The figure above depicts the RMS delay spread alongside the mean excess delay, indicating the dispersion of signal paths and their delays, which is an important factor in the design and analysis of wireless communication systems.


\subsection*{Typical Values of RMS Delay Spread}

RMS delay spread varies depending on the frequency of the signal and the environment:

\begin{itemize}
    \item In a rural area, the typical RMS delay spread is around \(0.2 \mu s\).
    \item Suburban areas see values around \(0.5 \mu s\).
    \item Urban areas can experience a wider range from \(3 \mu s\) to \(8 \mu s\).
    \item Microcell urban environments have values less than \(2 \mu s\), while picocell indoor environments can range from \(50 \mu s\) to \(300 \mu s\).
\end{itemize}

These values have direct implications on the coherence bandwidth \( B_c \) of the channel, influencing the design of wireless systems in different environments.

\subsection*{The Two-Ray Channel Model}

An example of a basic multipath model is the two-ray channel model, which considers the direct path and a single reflected path from the ground:

\begin{itemize}
    \item The direct path and reflected path create a phase difference due to the difference in path lengths, which can cause constructive or destructive interference at the receiver.
    \item The model is represented by the equation for the delayed signal:
    \[
    r_{2\text{-}ray}(t) = \frac{\sqrt{G_a G_c G_p G_d}}{4\pi l} e^{-j 2\pi \frac{r - D/c}{\lambda}} + \frac{R \sqrt{G_a G_c G_p G_d}}{4\pi (x + x')} e^{-j 2\pi \frac{r' - D/c}{\lambda}}
    \]
    where:
    \begin{itemize}
        \item \( G_a \) is the antenna gain.
        \item \( G_c \) is the cable loss.
        \item \( G_p \) is the polarization loss.
        \item \( G_d \) is the diffraction loss.
        \item \( l \) is the direct path length.
        \item \( D \) is the distance between the transmitter and receiver.
        \item \( R \) is the reflection coefficient.
        \item \( x \) and \( x' \) are the distances from the transmitter to the reflecting surface and from the reflecting surface to the receiver, respectively.
    \end{itemize}
\end{itemize}


\subsection*{The Two-Path Channel Model}

In the two-path channel model, we consider the impulse response as a combination of two delta functions representing the direct path and one reflected path:

\begin{equation}
    h(t) = \alpha_1 \delta(t) + \alpha_2 \delta(t - \tau)
\end{equation}

where:
\begin{itemize}
    \item \( \alpha_1 \) and \( \alpha_2 \) are the path gains of the direct and reflected paths, respectively.
    \item \( \tau \) is the delay of the reflected path relative to the direct path.
\end{itemize}

Given the parameters:
\begin{itemize}
    \item \( \alpha_1 = 1, \alpha_2 = 0.9 \)
    \item \( \tau = 0.1T \) or \( \tau = T \), where \( T \) is the symbol period.
\end{itemize}

The channel's frequency response can be derived from the impulse response and is given by:
\begin{equation}
    H(f) = \alpha_1 + \alpha_2 e^{-j2\pi f\tau}
\end{equation}

This model is illustrative of scenarios where a signal reaches the receiver directly and by reflection, causing multipath interference.
\subsection*{Impact of Delay Spread on Symbol Timing}

The delay spread impacts the timing of new symbols as follows:
\begin{itemize}
    \item With \( \tau = 0.1T \), the delay spread is small, and the new symbol timing is largely unaffected.
    \item With \( \tau = T \), the delay spread is significant, and the timing of the new symbol overlaps with the tail of the previous symbol's timing, leading to inter-symbol interference.
\end{itemize}

Understanding these parameters is critical in designing communication systems to effectively manage and mitigate the effects of multipath propagation.


\subsection*{Extension of the Two-Path Channel Model}

Further exploration of the two-path channel model considers different values of the delay \(\tau\) relative to the symbol period \(T\):

\begin{itemize}
    \item For \(\tau = 1.5T\), the delay spread is larger than the symbol period, indicating significant inter-symbol interference. The impulse and frequency response plots demonstrate this by showing a distinct second path arrival and a more rapid variation in the frequency domain, respectively.
    \item Increasing the delay to \(\tau = 4T\) suggests an even greater multipath effect. A 40 times faster bit rate, implied by a smaller \(T\), means that the multipath components are more spread out in time, as shown in the impulse response. The frequency response indicates deep notches, signifying severe frequency selective fading.
\end{itemize}

These scenarios underscore the necessity for effective equalization techniques to counteract the increasing ISI as the delay spread grows in relation to the symbol period.
These figures visually represent the impact of different delay spreads on the channel response, critical for wireless communication system designers to ensure reliable data transmission.


\subsection*{Bit Error Rate in Different Channel Conditions}

The Bit Error Rate (BER) performance is crucial for assessing the reliability of communication systems under various channel conditions:

\begin{itemize}
    \item For a Gaussian channel (no fading), the BER performance is ideal and serves as a baseline for comparison.
    \item In flat fading channels, the BER is affected by the fading process, typically resulting in worse performance compared to the Gaussian channel.
    \item Frequency-selective channels without equalization can have even more degraded BER performance due to the multipath effects.
\end{itemize}

The BER curves as a function of Signal-to-Noise ratio (S/N) illustrate these differences.

\subsection*{BER in Flat Fading Channels with AWGN}

For flat fading channels with AWGN, the decision variable \( x(m) \) is given by the sum of the transmitted symbol \( c_m \) and noise \( n(m) \):

\begin{equation}
    x(m) = \alpha c_m + n(m)
\end{equation}

The mean error probability \( P_e \) is obtained by averaging the error probability over the channel fading distribution.
\subsection*{BER on Flat Rayleigh Fading Channel}

When the channel experiences Rayleigh fading, the error probability depends on the fading distribution:

\begin{itemize}
    \item The decision variable still contains the transmitted symbol and noise, but each received symbol is affected by a fading coefficient \( \alpha \) that is Rayleigh-distributed.
    \item The BER is calculated by integrating the error probability over the fading distribution, providing insight into the channel's performance under Rayleigh fading conditions.
\end{itemize}

This integration is essential for designing robust modulation and coding schemes that can operate efficiently in fading environments.
These figures and equations form the basis for understanding how different channel conditions affect the BER, which is fundamental for communication system optimization.

\subsection*{BER in Multipath Rayleigh Fading Channels}

For a frequency-selective channel typical of multipath Rayleigh fading, the decision variable \( x(m) \) accounts for ISI:

\begin{equation}
    x(m) = g(0)c_m + \sum_{\substack{k \\ k \neq 0}} g(kT)c_{m-k} + n(m)
\end{equation}

Here:
\begin{itemize}
    \item \( g(kT) \) represents the channel's impulse response at various multiples of the symbol period \( T \), influencing the current and previous symbols.
    \item ISI arises when symbols overlap due to the multipath delay spread, necessitating advanced equalization techniques to mitigate its effects.
\end{itemize}

Without proper countermeasures, such as equalization or the use of orthogonal frequencies, the system's error probability will reach an irreducible error floor. This phenomenon significantly degrades the BER performance as the noise can no longer be considered the only limiting factor—interference between symbols becomes a substantial problem.
This figure shows the BER curve for a frequency-selective channel and illustrates the challenge of achieving low error rates in such environments.e relative motion between the transmitter and receiver.











\subsection*{Time-Varying Channel and Doppler Shift}

In mobile communication, the channel characteristics can vary with time due to the movement of the receiver, resulting in a time-varying channel model:

\begin{equation}
    h(t, \tau) = A_{LS} \sum_{\ell=0}^{N_c-1} \alpha_{\ell}(t) e^{j\phi_{\ell}(t)} \delta(\tau - \tau_{\ell})
\end{equation}

where \( \alpha_{\ell}(t) \) and \( \phi_{\ell}(t) \) are the time-varying amplitude and phase for each path, and \( \tau_{\ell} \) is the delay for the \(\ell\)th path. The channel's large-scale fading \( A_{LS} \) and delays \( \tau_{\ell} \) change more slowly compared to the gains and phases.

The received signal \( y(t) \) is then the convolution of the transmitted signal \( s(t) \) with the time-varying channel impulse response:
\begin{equation}
    y(t) = A_{LS} \sum_{\ell=0}^{N_c-1} \alpha_{\ell}(t) e^{j\phi_{\ell}(t)} s(t - \tau_{\ell})
\end{equation}

\subsection*{Doppler Shift}

Doppler shift is a phenomenon that occurs when a mobile user moves with velocity \( v \) causing a frequency shift in the received signal due to the change in distance over time:

\begin{itemize}
    \item For a sinusoidal signal \( s(t) = \sin(2\pi f_c t) \) transmitted from a stationary source, the received signal after traveling distance \( d \) with velocity \( v \) is:
    \[
    y_Y(t) = \sin\left(2\pi f_c (t - \frac{v t}{c})\right) = \sin\left(2\pi (f_c - f_c \frac{v}{c})t\right)
    \]
    \item The term \( f_c \frac{v}{c} \) represents the Doppler shift which modifies the original frequency \( f_c \).
    \item The speed of light \( c \) is approximately \(3 \times 10^8\) meters per second.
\end{itemize}

This shift must be accounted for in the design of communication systems to maintain the integrity of signal reception under mobility.


\subsection*{Doppler Shift in Mobile Communication}

In mobile communications, the relative velocity between the transmitter and receiver introduces a Doppler shift in the received signal:

\begin{itemize}
    \item The Doppler shift \( f_D \) is the change in frequency due to the motion of the mobile receiver and is given by \( f_D = -\frac{f_cv}{c} \), where \( f_c \) is the carrier frequency, \( v \) is the velocity of the receiver relative to the transmitter, and \( c \) is the speed of light.
    \item For a mobile moving away from the base station, \( v > 0 \), resulting in a negative Doppler shift, while a mobile moving towards the base station, \( v < 0 \), will experience a positive Doppler shift.
    \item The received frequency at a point in time is \( f_{\text{received}} = f_c (1 - \frac{v}{c}) \).
\end{itemize}

\subsection*{Scattering and Doppler Spectrum}

In the presence of scattering, the received signal experiences a Doppler spread, which affects the signal as follows:

\begin{itemize}
    \item Multiple signal paths caused by scattering result in different Doppler shifts for each path, leading to a spectrum of frequencies, known as the Doppler spectrum.
    \item The received signal is then the sum of all scattered waves and can be described as a stochastic process, characterized by autocorrelation and power spectral density.
    \item Each path's Doppler shift depends on the angle \( \theta \) of arrival, with each path having a shift \( f_D = \frac{v}{c} f_c \cos(\theta) \).
\end{itemize}

The Doppler effect is a fundamental consideration in the design and analysis of mobile communication systems as it affects the autocorrelation properties of the received signal and, ultimately, the system performance.








\subsection*{Jake's Doppler Spectrum}

The spectral broadening due to receiver movement is described by the Doppler spread and can be characterized using Jake's Doppler spectrum for a sinusoidal tone:

\begin{equation}
    S(f) = \frac{1}{\pi f_d \sqrt{1 - (\frac{f}{f_d})^2}}
\end{equation}

where:
\begin{itemize}
    \item \( S(f) \) represents the power spectral density.
    \item \( f_d \) is the maximum Doppler shift, which occurs when the mobile receiver is moving directly towards or away from the transmitter.
\end{itemize}

Jake's Doppler spectrum provides insight into the frequency domain characteristics of a time-varying channel.

\subsection*{Time-Varying Channel Analysis}

In a time-varying channel, the Doppler spectrum \( S_{g}(f) \) of a sinusoid on a time-varying channel is the power spectral density:

\begin{equation}
    S_{g}(f) = \pi f_d A \sqrt{1 - (\frac{f}{f_d})^2}
\end{equation}

The coherence time \( T_c \) of the channel, which is the time over which the channel can be assumed uncorrelated, is given by:

\begin{equation}
    T_c = \frac{1}{2 f_d}
\end{equation}

For \( f_d = \frac{v}{c}f_c \), where \( v \) is the relative velocity and \( c \) is the speed of light, \( T_c \) becomes inversely proportional to both the carrier frequency \( f_c \) and the relative velocity.

The time domain autocorrelation function \( \rho(t) \) is related to the Doppler spectrum \( S(f) \) and is used to determine the level of correlation between signals received at different times.

\begin{equation}
    \rho(t) = J_0(2\pi f_d t)
\end{equation}

where \( J_0 \) is the 0th-order Bessel function of the first kind, which approaches zero for \( t \geq T_c \), indicating that the channel is uncorrelated for \( t \geq \frac{1}{2 f_d} \).





\subsection*{Doppler Spectrum and Fading Types}

Doppler spread is crucial in characterizing the impact of the user's movement on the signal's frequency components:

\begin{itemize}
    \item When the baseband signal bandwidth \(B_s\) is much larger than the Doppler spread \(f_d\), Doppler effects are minimal and the channel experiences slow fading.
    \item Conversely, if \(B_s < f_d\), the channel undergoes fast fading and the Doppler spread can significantly distort the received signal, often leading to an irreducible Bit Error Rate (BER) and synchronization issues.
    \item The fading type can also be characterized in terms of symbol duration: a channel is in slow fading if the coherence time \(T_c\) is greater than the symbol duration \(T\), and in fast fading if \(T_c < T\).
\end{itemize}

\subsection*{Fading Channel Example}

Consider an example to illustrate these concepts:

\begin{itemize}
    \item Transmission frequency \(f_c = 2.1 \text{ GHz}\), in a suburban area with a delay spread \( \sigma_{\tau} = 2 \mu s \), to a user moving at \(90 \text{ km/h}\) (\(v = 25 \text{ m/s}\)).
    \item The signal bandwidth is \(B_s = 2 \text{ MHz}\), corresponding to a symbol time \(T \approx 500 \text{ ns}\).
    \item Doppler spread \(f_d = \frac{v}{c}f_c \approx 175 \text{ Hz}\), leading to a coherence time \(T_c \approx 3 \text{ ms}\).
    \item The coherence bandwidth \(B_c = \frac{1}{\sigma_{\tau}} = 500 \text{ kHz}\).
\end{itemize}

This channel would be considered slow fading since \(B_s > f_d\) or \(T < T_c\), and frequency-selective as \(B_s > B_c\), requiring careful consideration in system design to mitigate fading impacts.


\subsection*{Recap of Small-Scale Fading Types}

The types of small-scale fading are characterized by their relative signal bandwidths (\(B_s\)), coherence bandwidth (\(B_c\)), symbol period (\(T\)), and coherence time (\(T_c\)):

\begin{itemize}
    \item \textbf{Flat Fading:}
    \begin{itemize}
        \item Occurs when \(B_s < B_c\) and the delay spread is less than the symbol period.
        \item Described as slow if \(T_c > T\) and fast if \(T_c < T\).
    \end{itemize}
    \item \textbf{Frequency-Selective Fading:}
    \begin{itemize}
        \item Occurs when \(B_s > B_c\) and the delay spread is greater than the symbol period.
        \item Also characterized as slow or fast based on whether \(T_c\) is greater or less than \(T\), respectively.
    \end{itemize}
\end{itemize}

The type of fading has important implications on the design and performance of wireless communication systems:

\begin{itemize}
    \item \textbf{Fast Fading:} Requires adaptive modulation and coding schemes to rapidly adjust to channel variations.
    \item \textbf{Slow Fading:} Allows for more predictable performance and simpler countermeasures.
\end{itemize}
Understanding these fading characteristics is essential for optimizing the design of wireless channels and selecting appropriate mitigation techniques to ensure reliable communication.




\section*{Multi-Carrier and OFDM Technologies}






La tecnologia multi-carrier è impiegata da tutte le modulazioni più recenti e rappresenta la base del livello fisico per trasmissioni LTE, 5G e Wi-Fi. 
L'ampio utilizzo di questa tiplogia di modulazione deriva dalla capacità di ridurre i problemi derivanti dall'utilizzo di un canale frequency selective, i cui problemi sono sempre più evidenti all'aumentare del rate di trasmissione. Inoltre presenta dei benefici anche nell'occupazione spetrale e nella flessibilità di allocazione delle risorse radio ai vari utenti del sistema.
\begin{itemize}
    \item Robustezza contro canali frequecy-selective
    \item Efficienza spettrale
    \item Allocazione flessibile di risorse
\end{itemize}

L'idea base consiste nel suddividere il segnale originale da trasmettere, caratterizzato da una banda ben superiore rispetto alla coherence bandwidth del canale in tanti segnali, ciascuno avente una banda tale da evitare i problemi del canale frequency selective, oveero inferiori alla coherence bandwidth.





\begin{center}
    \includegraphics[width=0.8\textwidth]{imgs/multicarrier.jpg}
\end{center}
 



\[
  B_s > B_c \Rightarrow \text{frequency selective channel}  
\]

\[
    \Delta B = \frac{B_s}{N} < B_c \Rightarrow \text{flat fading channel}
\]
I segnali in cui è stato suddiviso l'originale non saranno affetti da ISI.

Per effettuare una trasmissione parallela dei veri segnali si potrebbe utilizzare filtri passa banda in parallelo, tuttavia nella pratica cioè non è possibile a causa della mancanza di un filtro ideale, la cui rappresentazione in frequenza sarebbe una perfetta rect. Inoltre l'utilizzo di molti filtri sarebbe molto costoso.
Per poter analizzare una modulazione multi-carrier è necessaria una rappresentazione alternativa, ma equivalente, del canale di trasmissione, ottenuto campionando il segnale con frequenza $\frac{1}{T}$. La rappresentazione ottenuta risulta valida solo nella banda del segnale trasmesso, ma non è un problema dato che non si ha interesse nell'analizzare il canali in altri punti.

Considerando l'inviluppo complesso risulta chiaro che la condizione di Nyquist\footnote{\label{nyquist_cond} La condizione di Nyquist garantisce l'assenza di aliasing per segnali campionati, ovvero dato l'intervallo di campionamento $T_s$ e la banda $B$ del segnale da campionare, deve valere $T_s \leq \frac{1}{2B}$, ovvero $f_s \geq 2B$} è rispettata con frequenza di campionamente $\frac{1}{T}$.



\[
    f_s \geq 2 \frac{1}{2T} = \frac{1}{T} \quad \text{frequenza di campionamento} 
\]
\[
    h_{eq}(t) = \sum_{\ell=0}^{L-1} h \left[\ell\right] \delta(t - \ell T) \quad \text{rappresentazione equivalente del canale}
\]

\[
    y(t) = h_{eq}(t) \ast s(t) = \sum_{\ell=0}^{L-1} h\left[\ell\right] s(t - \ell T) \quad \text{inviluppo complesso del segnale ricevuto}
\]

Il segnale ricevuto è identico al segnale ottenuto utilizzando la rappresentazione fisica del canale, ma la forma permette un'analisi più semplice.


\subsection*{Modulazione OFDM}

La modulazione OFDM risulta essere una delle più utilizzate per trasmissioni digitali ed è del tipo multi-carrier, ereditando quindi i vantaggi di tale tipo di modulazione.
Si consideri un blocco $S$ composto da $N$ campioni da trasmettere:
\[
  S = \left\{s\left[0\right], s\left[1\right], \ldots, s\left[N-1\right]\right\}
\]

L'effetto introdotto dal canale genera una componente ISI lato ricevitore:
\[
  y\left[k\right] = \sum_{\ell=0}^{L-1} h\left[\ell\right] s\left[k - \ell\right] = h(0)s\left[k\right] + \sum_{\ell=1}^{L-1} h\left[\ell\right] s\left[k - \ell\right]
\]
Tipicamente quando si studia una modulazione si considera una sequenza infinita di simboli, ma in questo caso se ne considera $N$, quindi gli indici negativi vengono considerati come 0 dato che non esistono.

% create a system of equations
%\[
%  \begin{cases}
%    y(0) = h(0)s(0) \\
%    y(1) = h(0)s(1) + h(1)s(0) \\
%    \vdots \\
%    y(N-1) = h(0)s(N-1) + h(1)s(N-2) + \ldots + h(L-1)s(N-L)
%  \end{cases}
%\]

% inset square brackets
\[
    \begin{cases}
        y[0] = h[0]s[0] \\
        y[1] = h[0]s[1] + h[1]s[0] \\
        \vdots \\
        y[N-1] = h[0]s[N-1] + h[1]s[N-2] + \ldots + h[L-1]s[N-L]
    \end{cases}
\]



L'espressione può essere riscritta in forma matriciale: $\mathbf{y} = \mathbf{H} \mathbf{s}$, con $\mathbf{H} \in \mathbb{C}^{N \times N}$ 

\[ 
\begin{bmatrix} y[0] \\ y[1] \\ \vdots \\ y[N-1] \end{bmatrix} 
= 
\begin{bmatrix}
    h[0] & 0 & \cdots & \cdots & \cdots & \cdots & \cdots & 0 \\
    h[1] & h[0] & \cdots & 0 \\
    \vdots & \vdots & \ddots & \vdots \\
    h[L-1] & h[L-2] & \cdots & h[0] & 0 & \cdots & \cdots & 0 \\
    0 & h[L-1] & \cdots & h[1] & h[0] & 0 & \cdots & 0 \\
    \vdots & \vdots & \ddots & \vdots & \vdots \\
    0 & 0 & \cdots & h[L-1] & h[L-2] & \cdots & h[1] & h[0]
\end{bmatrix}   
\begin{bmatrix} s[0] \\ s[1] \\ \vdots \\ s[N-1] \end{bmatrix}
\]

La matrice $\mathbf{H}$ è detta \textbf{matrice di Toeplitz} e la sua proprietà caratteristica è la presenza del solito elemento lungo le diagonali. Copiando gli ultimi $N_{CP} > L$ campioni del blocco $S$ e appendendoli in testa si ottiene un nuovo blocco con struttura circolare, cioè i primi $N_{CP}$ campioni sono identici agli ultimi $N_{CP}$.
\[
    \overline{S} = \left\{s\left[N - N_{CP} - 1\right], \ldots, s\left[N - 1\right], s\left[0\right], \ldots, s\left[N - 1\right]\right\} \quad \text{blocco cliclico}
\]

Il prefisso appeso in testa può essere utilizzato come elementi con indice negativo nella convuluzione con il canale.

%\[
%    \overline{S}(-1) = S(N - 1)
%\]
%\[
%    \overline{S}(-2) = S(N - 2)
%\]
%\[
%    \vdots
%\]
%\[
%    \overline{S}(-N_{CP}) = S(N - N_{CP})
%\]


\[
    \begin{array}{ll}
        \overline{s}[-1] = s[N - 1] \\
        \overline{s}[-2] = s[N - 2] \\
        \vdots \\
        \overline{s}[-N_{CP}] = s[N - N_{CP}]
    \end{array}
\]



Calcolando nuovamente la convoluzione, adesso ogni campione avrà anche alcune componenti con indice negativo.
Introducendo il prefisso nel blocco tramesso in uscita dal canale si ottiene un vettore $\mathbf{y}$ i cui elementi sono costituiti dalla somma di $L$ termini.
\[
    y[k] = \sum_{\ell=0}^{L-1} h[\ell] \ \overline{s}[k-\ell]
\]
\[
    \mathbf{y} = \mathbf{\overline{H}} \mathbf{s}
\]
Invece di aggiungere elementi al vettore $\mathbf{s}$, ottentendo un vettore $\mathbf{\overline{s}}$, si modifica la matrice $\mathbf{H}$.
\[ 
\begin{bmatrix} y[0] \\ y[1] \\ \vdots \\ y[N-1] \end{bmatrix} 
= 
\begin{bmatrix}
    h[0] & 0 & \cdots & \cdots & \cdots & h[3] & h[2] & h[1] \\
    h[1] & h[0] & \cdots & 0 & \cdots & h[4] & h[3] & h[2] \\
    \vdots & \vdots & \ddots & \vdots & &  & \vdots & \vdots \\
    h[L-1] & h[L-2] & \cdots & h[0] & 0 & \cdots & 0 & 0 \\
    0 & h[L-1] & \cdots & h[1] & h[0] & 0 & \cdots & 0 \\
    \vdots & \vdots & \ddots & \vdots & \vdots & & h[0] & 0 \\
    0 & 0 & \cdots & h[L-1] & h[L-2] & \cdots & h[1] & h[0]
\end{bmatrix}   
\begin{bmatrix} s[0] \\ s[1] \\ \vdots \\ s[N-1] \end{bmatrix}
\]



La nuova matrice risulta ancora essere del tipo Toeplitz, ma in aggiunta è anche \textbf{circolante}, dato che ogni riga è ottenuta tramite uno shift circolare verso destra della riga precedente (vale anche per le colonne, shiftando verso il basso).

Una matrice circolante può essere diagonalizzata, cioè può essere espressa come:
\[
    \overline{\mathbf{H}} = \mathbf{F}^H \mathbf{H} \mathbf{F}
\]




TODO: che vuole dire H(n)?
\[
    \begin{cases*}
        \mathbf{F}: \left[\mathbf{F}\right]_{k, n} = \frac{1}{\sqrt{N}} e^{\frac{-j2\pi kn}{N}} \quad \text{fourier trasform matrix normalizzata} \\
        \mathbf{H}: \left[\mathbf{H}\right]_{n, m} = \begin{cases*}
                                                        \sum_{\ell=0}^{L-1} h(\ell) e^{\frac{-j2\pi \ell n}{N}} & \text{se } $n = m$ \\
                                                        0 & \text{se } $n \neq m$
                                                    \end{cases*}
    \end{cases*}
\]
L'elemento $n$-esimo sulla diagonale risulta avere la trasformata di fourier discreta di fourier del canale (manca solo il termine $\frac{1}{\sqrt{N}}$)


La matrice $\mathbf{F}$ è unitaria, ovvero gode delle proprietà:
\begin{itemize}
    \item $\mathbf{F}^H \mathbf{F} = \mathbf{F} \mathbf{F}^H = \mathbf{I}_N$
    \item $\| \mathbf{F} \| = 1$ 
\end{itemize}

Considerando $\mathbf{Y} = \mathbf{F} \mathbf{y}$ e $\mathbf{S} = \mathbf{F} \mathbf{s}$, ovvero le DFT dei due vettori si ottiene la seguente espresione:
\[
    \mathbf{Y} = \mathbf{F} \mathbf{y} = \mathbf{F} \mathbf{\overline{H}} \mathbf{s} = \mathbf{F} \left(\mathbf{F}^H \mathbf{H} \mathbf{F}\right)\mathbf{S} =  \mathbf{H} \mathbf{F} \mathbf{s} = \mathbf{H} \mathbf{S}
\]
\[
    \Rightarrow \mathbf{Y}_{n,n} = \mathbf{H}_{n,n} \mathbf{S}_{n,n} \quad \text{per } n = 0, 1, \ldots, N-1
\]
Dato che $\mathbf{H}$ è diagonale, il segnale ricevuto sul subcarrier $n$ dipende esclusivamente dal segnale trasmesso sullo stesso subcarrier. Nel dominio della frequenza quindi non abbiamo ISI.


\begin{center}
    \resizebox{\textwidth}{!}{
        \begin{tikzpicture}[
            block/.style={rectangle, draw, minimum height=1cm, minimum width=2.5cm},
            node distance=1cm and 2cm,
            align=center,
            auto
        ]
        \node[block] (BitSource) {Bit source};
        \node[block, right=of BitSource] (SymbolMapping) {Symbol\\mapping};
        \node[block, right=of SymbolMapping] (S2PConverter) {Serial\\to\\parallel\\converter};

        \node[block, right=of S2PConverter] (InverseDFT) {Inverse\\DFT};
        \node[block, right=of InverseDFT] (CpInsertion) {CP\\insertion};
        \node[right=of CpInsertion, inner sep=0pt, minimum size=0pt] (channel) {};

        \node[block, below=of channel] (MultipathChannel) {Multipath\\channel};

        \node[below=of MultipathChannel, inner sep=0pt, minimum size=0pt] (dummy2) {};
        \node[block, left=of dummy2] (CpRemoval) {CP\\removal};
        \node[block, left=of CpRemoval] (P2SConverter) {Serial\\to\\parallel\\converter};
        \node[block, left=of P2SConverter] (DFT) {DFT};
        \node[block, left=of DFT] (SymbolDecision) {Symbol\\decision};
        \node[block, left=of SymbolDecision] (Destination) {Destination};

        \draw[->] (BitSource) -- (SymbolMapping) node[midway,above] {$b[n]$};
        \draw[->] (SymbolMapping) -- (S2PConverter) node[midway,above] {$s[n]$};
        \draw[->] (S2PConverter) -- (InverseDFT) node[midway,above] {$\mathbf{s}$};
        \draw[->] (InverseDFT) -- (CpInsertion) node[midway,above] {$S_k$};

        \draw[-] (CpInsertion) -- (channel) node[midway,above] {$\overline{S}_k$};
        \draw[->] (channel) -- (MultipathChannel) node[midway,right] {};
        \draw[-] (MultipathChannel) -- (dummy2) node[midway,right] {};
        \draw[->] (dummy2) -- (CpRemoval) node[midway,above] {$Y_k$};
%        % TODO: 2 Y uguali?
        \draw[->] (CpRemoval) -- (P2SConverter) node[midway,above] {$\mathbf{Y}$};
        \draw[->] (P2SConverter) -- (DFT) node[midway,above] {$\mathbf{Y}$};
        \draw[->] (DFT) -- (SymbolDecision) node[midway,above] {$\mathbf{y}$};
        \draw[->] (SymbolDecision) -- (Destination) node[midway,above] {$\hat{b}[n]$};
        \end{tikzpicture}
    }
\end{center}

%\draw[dashed, red, thick] ([xshift=-0.5cm,yshift=0.5cm]sampler.north west) rectangle ([xshift=0.5cm,yshift=-0.5cm]filter.south east);

%\node[align=center, red, above right= -1cm and -6cm of filter.south east] (channel-label) {Demodulatore numerico};


I simboli trasmessi sono idealmente generati in frequenza, ovvero sono convoluti prima della trasmissione tramite DFT inversa, quindi per essere recuperati è necessario effettuare la DFT. L'uso del prefisso risulta comunque essenziale affinché le proprietà algebriche sfruttate siano verificate. 
La DFT può essere calcolata tramite FFT, un'operazione molto semplice dato che è costituita da una moltiplicazione matriciale, quindi il costo per la rimozione dell'ISI è molto contenuto. In questo modo è vicino il limite del rate di tramissione raggiungibile, a patto che vi sia una banda sufficientemente ampia.
Il costo da pagare è l'utilizzo aggiuntivo di energia e banda per la trasmissione del prefisso, il quale non contiene alcuna informazione utile.
La denominazione OFDM deriva dal fatto che si tratta di una sorta di modulazione in frequenza, in cui le varie sub-carrier sono ortogonali, ovvero non interferiscono tra di loro. La banda occupata è data dalla somma delle bande occupate dalle varie sub-carrieri, dunque è necessario determinare quali sono tali contributi.

\[
    s[k] = \frac{1}{\sqrt{N}} \sum_{n=0}^{N-1} S_n e^{\frac{j2\pi nk}{N}}, \quad k = 0, \ldots, N-1
\]

Dove $s[k]$ è il segnale trasmesso nel dominio del tempo. Si può notare che in ogni istante ci sono informazioni di ogni simbolo del blocco, questo deriva dal fatto che vi è stata l'operazione di DFT inversa. OFDM tramette i simboli in parallelo sulle varie sub-carrier. 

% TODO: perché BT = 1?

\[
    S_n e^{\frac{j2\pi nk}{N}} = S_n e^{\frac{j2\pi BTnk}{N}} = \underbrace{S_n e^{j2\pi n\Delta f k T_s}}_{\text{Frequency tone con frequenza $\Delta f$}} = \underbrace{S_n e^{j 2 \pi n \Delta f t}}_{\text{segnale analogico campionato}} \bigg|_{t=kT}
\]


















\makeatletter
\pgfkeys{/tikz/semiellipse/.cd,
  width/.initial=2cm,
  height/.initial=1cm,
  fill color/.initial=red,
  fill opacity/.initial=0.5
}
\pgfdeclareshape{semiellipse}{
    % The 'anchor' for positioning the node
    \savedanchor\centerpoint{
        \pgf@x=0pt
        \pgf@y=0pt
    }
    \anchor{center}{\centerpoint}
    \anchor{north}{
        \pgf@x=0pt
        \pgf@y=0.5\ht\pgfnodeparttextbox
    }
    \anchor{south}{
        \pgf@x=0pt
        \pgf@y=-0.5\ht\pgfnodeparttextbox
    }
    \anchor{east}{
        \pgf@x=0.5\wd\pgfnodeparttextbox
        \pgf@y=0pt
    }
    \anchor{west}{
        \pgf@x=-0.5\wd\pgfnodeparttextbox
        \pgf@y=0pt
    }

    % The background path
    \backgroundpath{
        % Get parameters
        \pgfkeysgetvalue{/tikz/semiellipse/width}{\semiw}
        \pgfkeysgetvalue{/tikz/semiellipse/height}{\semih}
        \pgfkeysgetvalue{/tikz/semiellipse/fill color}{\semifillcolor}
        \pgfkeysgetvalue{/tikz/semiellipse/fill opacity}{\semifillopacity}

        % Convert to dimensions
        \pgfmathsetlengthmacro\halfwidth{.5*\semiw}
        \pgfmathsetlengthmacro\halfheight{.5*\semih}
        
        % Draw and fill the semi-ellipse
        \pgfpathmoveto{\pgfpoint{\halfwidth}{0pt}}
        \pgfpatharc{0}{180}{\halfwidth and \halfheight}
        \pgfpathclose % Close the path to form a proper semi-ellipse

        % Fill settings
        \pgfsetfillcolor{\semifillcolor}
        \pgfsetfillopacity{\semifillopacity}
        \pgfusepath{fill,stroke} % Fill and then draw the stroke
    }
}
\makeatother

\resizebox{\textwidth}{!}{
    \begin{tikzpicture}    
        \foreach \i in {0,1,2,3,4} {
            \node[shape=semiellipse, draw, semiellipse/width=1cm, semiellipse/height=1.5cm, semiellipse/fill color=gray, semiellipse/fill opacity=0.4] (se) at (0.5+\i*0.5,0) {};
        }
        \node[shape=semiellipse, draw, semiellipse/width=1cm, semiellipse/height=7.5cm, semiellipse/fill color=blue, semiellipse/fill opacity=0.5] (se) at (3,0) {};

        \foreach \i in {7,8,9, 10, 11, 12, 13, 14, 15, 16, 17, 18} {
            \node[shape=semiellipse, draw, semiellipse/width=1cm, semiellipse/height=5.5cm, semiellipse/fill color=red, semiellipse/fill opacity=0.4] (se) at (\i*0.5,0) {};
        }
        \node[shape=semiellipse, draw, semiellipse/width=1cm, semiellipse/height=7.5cm, semiellipse/fill color=blue, semiellipse/fill opacity=0.5] (se) at (9.5,0) {};

        \foreach \i in {19, 20, 21, 22, 23, 24, 25, 26, 27, 28, 29, 30} {
            \node[shape=semiellipse, draw, semiellipse/width=1cm, semiellipse/height=5.5cm, semiellipse/fill color=red, semiellipse/fill opacity=0.4] (se) at (0.5+\i*0.5,0) {};
        }
        \node[shape=semiellipse, draw, semiellipse/width=1cm, semiellipse/height=7.5cm, semiellipse/fill color=blue, semiellipse/fill opacity=0.5] (se) at (16,0) {};
        \foreach \i in {31, 32, 33, 34, 35, 36, 37, 38, 39, 40, 41, 42} {
            \node[shape=semiellipse, draw, semiellipse/width=1cm, semiellipse/height=5.5cm, semiellipse/fill color=red, semiellipse/fill opacity=0.4] (se) at (1+\i*0.5,0) {};
        }
        \node[shape=semiellipse, draw, semiellipse/width=1cm, semiellipse/height=7.5cm, semiellipse/fill color=blue, semiellipse/fill opacity=0.5] (se) at (22.5,0) {};
        \foreach \i in {43, 44, 45, 46} {
            \node[shape=semiellipse, draw, semiellipse/width=1cm, semiellipse/height=1.5cm, semiellipse/fill color=gray, semiellipse/fill opacity=0.4] (se) at (1.5+\i*0.5,0) {};
        }
        \draw[->] (-1,0) -- (26.0,0) node[right] {Frequency};
        \draw[->] (-1,0) -- (-1,6) node[above] {};
        \begin{scope}[shift={(24, 5)}] % Adjust the position of the legend
            \node[draw, fill=gray, fill opacity=0.4, label=right: Guard sub-carrier] at (0,0) {};
            \node[draw, fill=blue, fill opacity=0.5, label=right:Pilot sub-carrier] at (0,-1) {};
            \node[draw, fill=red, fill opacity=0.4, label=right:Data sub-carrier] at (0,-2) {};
        \end{scope}
    \end{tikzpicture}
}





















La somma di frequency tone genera uno spettro a righe, ogni frequenza genera una delta. In realtà trattandosi di un segnale finito, visto come sinusoide infinita moltiplicato per una rect, lo spettro sarà composto dalla convoluzione fra una delta e una sinc.
Un simbolo OFMD corrisponde alla sovrapposizione di $N$ segnali nell'intervallo $[0, NT_s]$
\[
    s_n(t) = \frac{1}{\sqrt{N}} S(n) e^{j2\pi n \Delta f t}, \quad n = 0, \ldots, N-1 \quad \text{segnali analogici sovrapposti}
\]
\[
    S_{s_n}(f) = \frac{A}{N} \text{sinc}^2 ((f-n\Delta f)N T_s) \quad \text{PSD segnale sull'$n$-esima subcarrier}
\]
Quindi si ottiene che la banda occupata da OFDM è la somma di N funzioni sinc, ciascuna centrata in una sub-carrier differente. In teoria ciò implica che l'occupazione di banda risulti essere infinita, tuttavia la sinc tende a 0 molto velocemente, quindi il contributo può considerarsi limitato ad una certa banda. Inoltre ciò che si può notare è che le varie sinc non interferiscono tra loro nelle sub-carrier, infatti in corrispondenza di tali valori solo una sinc risulterà non essere nulla. Per contenere la banda le sub-carrier agli estremi non sono utilizzate, in modo che al di fuori della banda a disposizione non vi sia un contributo energetico significativo.
\[
    B_{OFDM} \approx N \cdot \frac{1}{NT_s} = \frac{1}{T_s}
\]
Tale relazione vale solo se si impongono virtual carriers per minimizzare il fenomeno indicato come \textbf{out of band radiation} da parte della sinc.
La banda risulta particolarmente ristretta in quanto non ci sono fattori a moltiplicare $\frac{1}{T_s}$, come avviene utilizzando un RRC. Tuttavia parte della banda risulta non sfruttata in quanto è necessario trasmettere sia il prefisso, sia non utilizzare alcune subcarrier agli estremi, dette \textbf{virtual subcarrier}.


Per quanto riguarda il symbol time, in OFDM si ha la relazione:
\[
    T_{OFDM} = T_s(N+N_{CP}), \quad N_{CP} > L
\]  
% TODO: cos'è L?
Dove $L$ non è noto a priori.
Da questo valore si ottiene:
\[
    T_s < \sigma_{\tau} \ll T_{OFDM}, \quad B_s > B_c \gg \Delta f
\]
Ogni sub-carrier può considerare il canale \textbf{flat-fading}. Il canale per poter ricevere correttamente il segnale deve essere stimato, e ciò avviene su speciali sub-carrier detti \textbf{pilot sub-carrier}. Su tali frequenze avviene la trasmissione di simboli noti in modo da ricostruire la risposta del canale. Inoltre per poter ridurre la presenza di rumore la tramissione su tali frequenze avviene con una potenza superiore. In realtà la stima del canale è valida solo sulle frequenze pilotate, per quelle intermedie si effettua un'interpolazione e si ottiene un'approssimazione.
L'efficienza spettrale è ridotta dell'utilizzo di virtual sub-carriers, pilot-subcarriers e cyclic prefix:
\[
    R = \left[N - (N_v + N_p) \right]\frac{1}{(N+N_{CP})T_s} = \frac{N - (N_v + N_p)}{(N+N_{CP})} \frac{1}{T_s}
\]

\paragraph*{OFDM error rate}
Considerando la presenza del rumore il segnale ricevuto avrà una componente aggiuntiva di disturbo
\[
    r(k) = y(k) + n(k)  
\]
Applicando la DFT si ottiene:
\[
    \mathbf{R} = \mathbf{F}\mathbf{r} = \mathbf{F}\mathbf{y} + \mathbf{F}\mathbf{n} = \mathbf{H}\mathbf{S} + \mathbf{N}
\]

\[
    \mathbf{R}(m) = \mathbf{H}(m)\mathbf{S}(m) + \mathbf{N}(m)
\]
Per poter dare un valore all'errore della modulazione è necessario analizzare le statistiche di $N(m)$, ovvero il rumore dopo la DFT.
Data l'unitarietà della matrice $\mathbf{F}$ si ha che, le statistiche del rumore dopo la trasformazione rimangono inalterate.
\[
    \mathbb{E}[\mathbf{N}] = \mathbb{E}[\mathbf{F}\mathbf{n}] = \mathbf{F} \mathbb{E}[\mathbf{n}] = 0
\]  
\[
    \mathbf{R}_{N,N} = \mathbb{E}[\mathbf{N}\mathbf{N}^H] = \mathbb{E}[\mathbf{F}\mathbf{n}\mathbf{n}^H\mathbf{F}^H] = \mathbf{F}\mathbb{E}[\mathbf{n}\mathbf{n}^H]\mathbf{F}^H = \mathbf{F}\mathbf{R}_{n,n}\mathbf{F}^H
\]

Dove $\mathbf{R}_{n,n}$ è l'autocorrelazione del tempo.
Per rimuovere l'effetto introdotto dal canale, ovvero $\mathbf{H}(m)$, il canale è stimato:
\[
    \mathbf{H}(n) = x(n) e^{j\phi(n)}
\]

Se il canale fosse stimato alla perfezione si otterrebbe:
\[
    X(n) = \frac{R(n)}{N(n)} = S(n) + \frac{N(n)e^(-j\phi(n))}{\alpha(n)}
\]


Se il canale è molto attenuato, ovvero $\alpha(n) \approx 0$, il rumore viene amplificato a seguito della divisione, richiedere un SNR superiore.
La fase non ha alcuna implicazione sul rumore, durante i calcoli di $\sigma ^2=2N_0$ infatti sparisce per via del complesso coniugato.
L'espressione della variabile decisionale è confrontabile con quella di un sistema tradizionale
\[
    x(m) = x_m + n(m) \quad \text{variabile tradizionale}
\]

\[
    X(m) = S(m) + N'(m) \quad \text{variabile OFDM}
\]
Questo permette di applicare le solite considerazioni per la probabilità di errore:

\[
    P(e|H(m)) = 2 Q\left( \sqrt{\frac{1}{\sigma^2(m)}}  \right) =  2 Q\left( \sqrt{\frac{\alpha^2(m)}{N_0}}  \right) = 2 Q\left( \sqrt{\frac{E_s \alpha^2(m)}{N_0}}  \right) 
\]
\[
    P(e) = \sum_{m=0}^{N-1} P(e|H(m))P(H(m)) = \frac{2}{N} \sum_{m=0}^{N-1} Q\left( \sqrt{\frac{E_s \alpha^2(m)}{N_0}}  \right)
\]




Multi-carrier modulation techniques, which are foundational in LTE and 5G technologies, offer several advantages:

\begin{itemize}
    \item \textbf{Robustness against multipath fading:} As data rates increase, multipath fading becomes more problematic for single-carrier transmissions. Multi-carrier systems are less susceptible to these effects.
    \item \textbf{Spectral efficiency:} These systems utilize the spectrum more efficiently than single-carrier systems.
    \item \textbf{Flexible resource allocation:} OFDM, a key multi-carrier technology, allows for dynamic assignment of radio resources based on channel conditions.
\end{itemize}

\subsection*{OFDM Technology Explained}

OFDM is a special case of multi-carrier transmission where a single data stream is split across multiple sub-carriers, each with a narrower bandwidth:

\begin{itemize}
    \item The total bandwidth \( B_s \) of the signal is divided among \( N \) subcarriers, each effectively experiencing flat fading if properly dimensioned.
    \item This division into subchannels means the channel appears frequency selective when \( B_s > B_c \), but each subcarrier with bandwidth \( \frac{B_s}{N} < B_c \) sees the channel as flat.
\end{itemize}

The figure should illustrate how OFDM's subcarriers divide the total signal bandwidth, enabling each subcarrier to experience more stable channel conditions and allowing the use of simpler equalizers.

\subsection*{Frequency-Selective Multipath Channel}

In a frequency-selective multipath environment, the channel impulse response is represented as a sum of time-delayed and scaled copies of the transmitted signal:

\begin{equation}
    h(t) = A_{LS} \sum_{\ell=0}^{N_c-1} \alpha_{\ell} e^{j\phi_{\ell}} \delta(t - \tau_{\ell})
\end{equation}

Here, \( A_{LS} \) denotes the large-scale fading, \( \alpha_{\ell} \) and \( \phi_{\ell} \) are the amplitude and phase of the \(\ell\)-th path, and \( \tau_{\ell} \) represents its time delay.

\subsection*{Channel Modeled as a Tapped Delay Line}

The multipath channel can also be described as a tapped delay line, which is a discrete-time model useful for digital signal processing:

\begin{equation}
    h_{eq}(t) = \sum_{\ell=0}^{L-1} h(\ell) \delta(t - \ell T)
\end{equation}

where \( T \) is the symbol period, and \( L \) is the number of discrete paths or taps. Even if \( L \) differs from \( N_c \), the characteristics of the channel remain consistent. The received signal's complex envelope is then:

\begin{equation}
    y(t) = \sum_{m=0}^{N_c-1} \alpha_m e^{j\phi_m} s(t - \tau_m) = \sum_{\ell=0}^{L-1} h(\ell) s(t - \ell T)
\end{equation}

This model simplifies the analysis of multipath channels and is pivotal in designing equalization algorithms to counteract ISI caused by multipath propagation.

\subsection*{OFDM Signal Model with Multipath}

The Orthogonal Frequency Division Multiplexing (OFDM) signal at the receiver in a multipath environment is modeled as follows:

\begin{equation}
    y(k) = \sum_{\ell=0}^{L-1} h(\ell) s(k - \ell)
\end{equation}

where:
\begin{itemize}
    \item \(y(k)\) is the received signal.
    \item \(s(k)\) is the transmitted signal block consisting of \(N\) samples.
    \item \(h(\ell)\) is the channel's impulse response at the \(\ell\)-th tap.
    \item \(L\) is the number of taps, representing the multipath delay spread.
\end{itemize}

For a single block transmission, we consider \(s(k)\) to be zero for negative indices, leading to the received signal for the first and last samples being given by:

\begin{align}
    y(0) &= h(0)s(0) + h(1)s(-1) + \ldots + h(L-1)s(-L+1) \\
    y(N-1) &= h(0)s(N-1) + h(1)s(N-2) + \ldots + h(L-1)s(N-L)
\end{align}

For negative indices, where the transmitted block \(s(k)\) is not defined, the corresponding \(s(k)\) terms are treated as zero. This simplification is practical for computational models and aligns with the cyclic prefix insertion in OFDM systems, which is designed to prevent ISI.

\subsection*{Handling Inter-Symbol Interference in OFDM}

Inter-symbol interference is a significant challenge in multipath conditions. The OFDM system's cyclic prefix is a critical feature to mitigate ISI:

\begin{itemize}
    \item The cyclic prefix length is typically chosen to be at least as long as the maximum multipath delay spread \(L\).
    \item This ensures that \(s(k - \ell)\) for \(\ell > N\) (outside the block) does not interfere with the current block.
\end{itemize}

These models are essential for understanding and mitigating the effects of multipath fading in OFDM-based wireless communication systems.


\subsection*{OFDM Signal Model in Matrix Notation}

The OFDM signal model can be succinctly expressed using matrix notation. The received sample vector \(\mathbf{y}\) is the result of the channel matrix \(\mathbf{H}\) acting on the transmitted sample vector \(\mathbf{s}\):

\begin{equation}
    \mathbf{y} = \mathbf{H} \mathbf{s}
\end{equation}

The channel matrix \(\mathbf{H}\) is a Toeplitz matrix formed by the channel impulse response, implying that elements along any diagonal are equal. This matrix form allows for efficient computations and insights into the structure of the channel.

\subsection*{Cyclic Extension in OFDM}

Cyclic extension, an essential part of the OFDM transmission scheme, involves copying the last \(N_{CP}\) samples of a block and placing them at the start. This circular structure creates a buffer to absorb multipath interference and allows for the use of the Fast Fourier Transform (FFT) in the receiver:

\begin{equation}
    \mathbf{s}' = [s(N - N_{CP} - 1), \ldots, s(N - 1), s(0), \ldots, s(N - 1)]
\end{equation}

Here, \(N_{CP}\) is the cyclic prefix length, and \(s'\) denotes the extended block. With cyclic extension, samples with negative indices in the convolutive sum are wrapped around to take the values of the samples at the end of the block, ensuring that linear convolution appears as circular convolution when processed with an FFT, which simplifies equalization.






\subsection*{OFDM Signal Model with Cyclic Extension}

The received OFDM signal after cyclic extension is represented by:

\begin{equation}
    y(k) = \sum_{\ell=0}^{L-1} h(\ell) s((k - \ell)_N)
\end{equation}

where \((k - \ell)_N\) is a modulo-\(N\) operation to account for the cyclic nature of the extension. The cyclic prefix ensures that samples from the previous block do not interfere, as illustrated by:

\begin{equation}
    y(0) = h(0)s(0) + h(1)s(-1)_N + \ldots + h(L - 1)s(-L + 1)_N
\end{equation}

and in general:

\begin{equation}
    y(k) = h(0)s(k)_N + h(1)s(k - 1)_N + \ldots + h(L - 1)s(k - L + 1)_N
\end{equation}

for \(k = 0, 1, \ldots, N-1\).

\subsection*{Matrix Representation of Cyclically Extended OFDM}

In matrix form, the cyclically extended OFDM signal is expressed as:

\begin{equation}
    \mathbf{y} = \mathbf{H}_{\text{circ}} \mathbf{s}
\end{equation}

The channel matrix \(\mathbf{H}_{\text{circ}}\) is now a circulant matrix, which means each row is a cyclic shift of the previous row. This structure of the matrix captures the essence of the cyclic prefix where:

\begin{itemize}
    \item \(h_{\text{circ}}(0)\) is the first row, 
    \item \(h_{\text{circ}}(1)\) is the second row, shifted one element to the right, and so on.
\end{itemize}

The circulant matrix is crucial in simplifying the signal recovery process in OFDM, allowing for efficient implementation of the FFT in the receiver.

This matrix representation provides a powerful tool for simulating and understanding OFDM systems, offering a clear visualization of how the cyclic prefix protects against ISI.

\subsection*{OFDM Channel Matrix and Diagonalization}

The OFDM channel matrix \(\mathbf{H}_{\text{circ}}\) is a circulant matrix, which possesses a powerful property – it can be diagonalized using the Fourier transform:

\begin{equation}
    \mathbf{\tilde{H}} = \mathbf{F}^\mathsf{H} \mathbf{H}_{\text{circ}} \mathbf{F}
\end{equation}

where:
\begin{itemize}
    \item \(\mathbf{F}\) is the normalized Fourier transform matrix, with elements given by \([F]_{k,n} = \frac{1}{\sqrt{N}} e^{-j2\pi kn / N}\).
    \item \(\mathbf{H}\) is a diagonal matrix representing the channel frequency response, with its elements \(H(n)\) derived from the channel impulse response \(h(\ell)\).
\end{itemize}

The unitary nature of \(\mathbf{F}\) (\(\mathbf{F}^\mathsf{H}\mathbf{F} = \mathbf{I}_N\)) ensures energy preservation during transformation.

\subsection*{Frequency-Domain Signal Representation in OFDM}

By defining the frequency-domain vectors of the transmitted and received signals, we can utilize the properties of the Fourier transform to simplify the signal model:

\begin{equation}
    \mathbf{Y} = \mathbf{F} \mathbf{y}, \quad \mathbf{S} = \mathbf{F} \mathbf{s}
\end{equation}

Therefore, applying the FFT to the received time-domain signal vector:

\begin{equation}
    \mathbf{Y} = \mathbf{F} \mathbf{H}_{\text{circ}} \mathbf{F}^\mathsf{H} \mathbf{S} = \mathbf{\tilde{H}} \mathbf{S}
\end{equation}

Since \(\mathbf{\tilde{H}}\) is diagonal, the received frequency-domain signal on subcarrier \(n\) depends exclusively on the transmitted signal on the same subcarrier, eliminating ISI in the frequency domain:

\begin{equation}
    Y(n) = H(n) S(n)
\end{equation}

This orthogonality is the cornerstone of OFDM, allowing for individual subcarrier equalization without interference from adjacent subcarriers.


\subsection*{OFDM Baseband Transceiver Structure}

The baseband transceiver in an OFDM system involves several key stages of processing:

\begin{enumerate}
    \item \textbf{Serial-to-Parallel Conversion:} A sequence of bits from the source is mapped to symbols, and then arranged into a vector \(\mathbf{s}\) of \(N\) consecutive data symbols.
    \item \textbf{Inverse Discrete Fourier Transform (IDFT):} This vector is transformed from the frequency domain to the time domain using an IDFT to obtain the time-domain vector \(\mathbf{s}'\).
    \item \textbf{Cyclic Prefix Insertion:} To mitigate intersymbol interference, a cyclic prefix of length \(N_{CP}\) is added to create an extended time-domain vector of length \(N + N_{CP}\).
\end{enumerate}

The transmitted signal then passes through the multipath channel, after which the following operations occur in the receiver:

\begin{enumerate}
    \item \textbf{Cyclic Prefix Removal:} The cyclic prefix is removed from the received vector to restore the original block size.
    \item \textbf{Discrete Fourier Transform (DFT):} The vector is converted back to the frequency domain using a DFT.
    \item \textbf{Symbol Decision:} Finally, decisions about the transmitted symbols are made to reconstruct the transmitted bit sequence.
\end{enumerate}

Each step is critical to ensuring the robust transmission and reception of data in OFDM systems, particularly in the presence of multipath propagation.




\subsection*{Time and Frequency Domain Interpretation of OFDM}

In an OFDM system, the conversion between the time and frequency domains is a central concept:

\begin{itemize}
    \item In the \textbf{time domain}, the signal \(s(k)\) is a summation of \(N\) orthogonal subcarriers, each modulated by its corresponding frequency symbol \(S(n)\), represented as:
    \begin{equation}
        s(k) = \frac{1}{N} \sum_{n=0}^{N-1} S(n) e^{\frac{j2\pi nk}{N}}, \quad k = 0, \ldots, N - 1
    \end{equation}
    \item The contribution of the \(n\)-th subcarrier to the \(k\)-th sample in the time domain is the product of the frequency symbol \(S(n)\) and a complex exponential oscillating at a frequency \(n/T\) over the duration of \(N\) samples, inclusive of the cyclic prefix (CP).
\end{itemize}

\textbf{Frequency domain interpretation:}
\begin{itemize}
    \item The symbol duration \(T\) is defined as an integer multiple of the period of all subcarriers. Since the input to the channel is periodic, the output is also periodic.
    \item The received signal on the \(n\)-th subcarrier \(y_n(t)\) can be expressed as:
    \begin{equation}
        y_n(t) = \sum_{\ell=0}^{L-1} h(\ell)S_n(t - \ell T) = S(n) \sum_{\ell=0}^{L-1} h(\ell)e^{-j2\pi n \Delta f (t - \ell T)}
    \end{equation}
    where \(L\) is the number of multipath components, and \(h(\ell)\) is the channel impulse response at delay \(\ell T\).
\end{itemize}

This structure ensures that each subcarrier can be treated independently, allowing for straightforward equalization in the frequency domain and effective mitigation of intersymbol interference (ISI) caused by the channel's multipath nature.



\subsection*{Signal Propagation and Reception in OFDM Systems}

The OFDM signal's journey involves several critical transformations:

\begin{enumerate}
    \setcounter{enumi}{3}
    \item The signal traverses the wireless medium characterized by the channel impulse response \(\mathbf{h} = [h(0), h(1), \ldots, h(L - 1)]\). The received signal \(\mathbf{y}\) can be expressed as the convolution of \(\mathbf{h}\) and the transmitted signal \(\mathbf{s}\), given by \(y(k) = \sum_{\ell=0}^{L-1} h(\ell)s(k - \ell)\).
    \item At the receiver, after discarding the cyclic prefix, the samples are converted to the frequency domain to yield \(\mathbf{Y} = \mathbf{F}\mathbf{y}\), simplifying to \(\mathbf{Y} = \mathbf{F}\mathbf{H}\mathbf{F}^\mathsf{H}\mathbf{s}\) due to the circulant structure of the channel matrix.
\end{enumerate}

\subsection*{OFDM Operation on Multipath Channels}

\begin{itemize}
    \item The total bandwidth of the OFDM signal (\(B_s\)) determines the sampling duration (\(T_s = \frac{1}{B_s}\)) and the OFDM symbol duration (\(T_{\text{OFDM}} = T_s(N + N_{CP})\)), where \(N_{CP}\) is the length of the cyclic prefix.
    \item Subcarrier spacing (\(\Delta f\)) is a function of the total bandwidth and the number of subcarriers (\(\Delta f = \frac{B_s}{N}\)), ensuring that each subcarrier channel experiences flat fading conditions.
    \item By judiciously selecting \(N\), the system can satisfy the conditions \(T_s \ll T_{\text{OFDM}}\) and \(B_s \gg \Delta f\), which is crucial for the robustness against multipath fading characterized by the delay spread \(\sigma_\tau\).
    \item Under these conditions, each subcarrier effectively experiences a flat channel, significantly simplifying the equalization process.
\end{itemize}

By dividing the available bandwidth into multiple subcarriers, OFDM systems can efficiently combat the effects of multipath fading, making it a preferred choice for high data rate wireless communication.

\subsection*{OFDM in WiFi Standards}

WiFi standards utilize Orthogonal Frequency Division Multiplexing (OFDM) as the underlying transmission technique to efficiently handle multipath and frequency-selective fading. The IEEE 802.11a/g/n/ac standards are particularly discussed with regard to their OFDM implementation:

\begin{itemize}
    \item The total bandwidth \( B \) of 20 MHz is divided into \( N = 64 \) sub-carriers, each with a spacing \( \Delta f \) of 312.5 kHz.
    \item For 802.11a/g, there are 48 data subcarriers, 4 pilot subcarriers for tracking the channel, and 12 are null subcarriers including guard bands.
    \item The 802.11n/ac standards utilize a similar structure but with a different allocation of subcarriers, adjusting for more robust transmission and additional features like MIMO.
\end{itemize}

\subsubsection*{OFDM Symbol Timing}

\begin{itemize}
    \item The OFDM symbol duration \( T_{\text{OFDM}} \) includes the cyclic prefix and is given by \( T_{\text{OFDM}} = T_s (N + N_{\text{CP}}) \), where \( T_s \) is the sampling duration and \( N_{\text{CP}} \) represents the number of samples in the cyclic prefix.
    \item Given the subcarrier spacing and the number of subcarriers, the OFDM block structure ensures that the channel can be considered flat across each subcarrier, thereby simplifying equalization.
\end{itemize}

\subsubsection*{Channel Characteristics}

\begin{itemize}
    \item The indoor channel typically exhibits a delay spread \( \sigma_\tau \) less than the OFDM symbol period, thus implying a flat fading channel for each subcarrier.
    \item With mobility, the maximum Doppler shift \( f_d \) is considered, which in turn defines the coherence time \( T_c \) of the channel, influencing how often channel state information should be updated.
\end{itemize}

The use of OFDM in WiFi takes advantage of the robustness against multipath fading and allows for flexible resource allocation through dynamic subcarrier assignments.



\subsection*{OFDM in IEEE 802.11a/g/n/ac WiFi Standards}

The WiFi transmission employs OFDM where the data stream is distributed across multiple subcarriers:

\begin{itemize}
    \item Each subcarrier transmits a new symbol every $T_{\text{OFDM}} = 4 \ \mu s$.
    \item The symbol rate per subcarrier is $\frac{1}{T_{\text{OFDM}}} = 0.25 \times 10^6 \ \text{symbols/second}$.
    \item With $48$ data subcarriers, the total data symbol rate is $48 \times 0.25 \times 10^6 = 12 \times 10^6 \ \text{symbols/second}$, or equivalently, $12$ Mbps.
\end{itemize}

\subsubsection*{Efficiency Considerations}

The insertion of the cyclic prefix and guard subcarriers introduces some efficiency loss:

\begin{itemize}
    \item The cyclic prefix (CP) results in a loss of efficiency given by:
    \[
    \eta_{\text{CP}} = \frac{N_{\text{CP}}}{N + N_{\text{CP}}} = \frac{16}{80} = 20\%
    \]
    where $N_{\text{CP}}$ is the number of cyclic prefix samples and $N$ is the number of OFDM symbols.

    \item Additionally, the inclusion of guard subcarriers leads to further loss of spectral efficiency:
    \[
    \eta_{\text{GS}} = \frac{N_{\text{GS}}}{N} = \frac{16}{64} = 25\%
    \]
    where $N_{\text{GS}}$ is the number of guard subcarriers.
\end{itemize}

The efficiency losses must be balanced with the robustness benefits that CP and guard subcarriers bring, especially in combating multipath fading and inter-symbol interference (ISI).


\textbf{Error rate for OFDM systems}
\begin{itemize}
    \item Considering the presence of noise, the output of the FFT is
    \[ R(n) = Y(n) + N(n) = H(n)S(n) + N(n) \]
    where \( N(n) = Fn \) and the vector \( n \) collects the received noise samples in time, \( n = [n(0), n(1), \ldots, n(N - 1)] \).
    
    \item Due to the properties of the unitary matrix \( F \), the statistics of \( N(n) \) are equivalent to the statistics of the noise samples \( n(k) \)
    \[ n(k) \sim \mathcal{N}(0, \sigma^2) \Rightarrow N(n) \sim \mathcal{N}(0, \sigma^2) \]
    
    \item The decision variable is 
    \[ X(n) = \frac{R(n)}{H(n)} = \frac{S(n) + N(n)}{H(n)} \]
\end{itemize}


\textbf{OFDM error probability}
\begin{itemize}
    \item The decision variable on the \(n\)-th subcarrier is
    \[ X(n) = \frac{R(n)}{H(n)} = \frac{S(n) + N(n)}{H(n)} = S(n) + \frac{N'(n)}{H(n)} \]
    where \( S(n) \) is an information symbol and the noise is \( N'(n) \sim \mathcal{N}\left(0, \frac{\sigma^2}{|H(n)|^2}\right) \).
    
    \item Hence, the symbol error probability will be
    \[ P(e|H(n)) = Q\left(\frac{|H(n)|}{\sigma}\right) = Q\left(\sqrt{\frac{E_s |H(n)|^2}{N_0}}\right) \]
\end{itemize}

\textbf{PAM error probability (\( M = 2 \))}
\begin{itemize}
    \item To compute \( P(e|a(i)) \) we assume that \( x(m) = a(i) + n(m) \) and the probability of error is
    \[ P(e|a(i)) = \text{Pr}\{x(m) \notin \mathcal{Z}|a_m = a(i)\} = Q\left(\frac{d(a(i), 0)}{\sigma}\right) = Q\left(\frac{1}{\sigma}\right) \]
\end{itemize}


\section*{Single-user OFDM}

\begin{itemize}
    \item The wireless channel is divided in \( N \) parallel subchannels, each with a different gain \( H(n) \).
    \item By adapting transmission parameters to the different channel gains allows to exploit the channel \textit{frequency diversity}.
\end{itemize}

\section*{Multi-user OFDM: OFDMA}

\begin{itemize}
    \item OFDM technology can be employed to enforce a multiple access technique: OFDMA
    \begin{itemize}
        \item Assigns an orthogonal subset of the \( N \) available sub-carriers to different users
    \end{itemize}
\end{itemize}


\section*{Multi-user OFDM-FDMA}

\begin{itemize}
    \item Different subcarriers assigned to different users
    \item The fading gain on each subchannel is independent from user to user
    \item Adaptive resource allocation is designed to assign to each user its best subchannels according different optimality criterion
\end{itemize}

\section*{Channel capacity}

\begin{itemize}
    \item We assume that the maximum achievable rate is measured as the Shannon channel capacity
    \item Let \( P(n) \) be the power of the signal transmitted on sub-channel \( n \), the maximum achievable rate is 
    \[ r(n) = \log_2 \left( 1 + \frac{|H(n)|^2 P(n)}{\sigma_o^2} \right) \]
\end{itemize}

\section*{Diversity Techniques in Wireless Communications}



\begin{figure}[ht]
    \centering
    \includegraphics[width=0.675\textwidth]{imgs/diversity_graph.jpg}
\end{figure}


Il fading rappresenta il principale problema nelle comunicazioni radio, sebbene modulazioni come OFDM siano in grado di ridurre l'effetto del multipath fading, tuttavia lo \textbf{slow flat Rayleigh fading} non può essere contrastato nello stesso modo.

\begin{itemize}
    \item \textbf{Slow fading}: coherence time $>$ symbol period (doppler spread)
    \item \textbf{Flat fading}: delay spread $<$ symbol period (multipath time delay spread)
    \item \textbf{Rayleigh fading}: l'attenuazione delle repliche ha una distribuzione di Rayleigh.
\end{itemize}

I metodo principali per ridurre gli effetti di un fading di tali tipologie sono le diversity techniques, ovvero lo sfruttamente di canali con caratteristiche differenti per trasmettere la stessa informazione, aumentando la probabilità che il ricevitore possa ricostruire correttamente il messaggio.
Le principali tecniche di diversity sono:
\begin{itemize}
    \item \textbf{Time diversity}: relative al coherence time. Sfruttano trasmmissioni in slot temporali separati utilizzado anche \textbf{coding} e \textbf{interleaving}. Slow fading channels potrebbero non garantire una diversity sufficiente. Il canale deve variare sufficientemente in maniera veloce.
    \item \textbf{Frequency diversity}: relative alla coherence bandwidth. Sfruttano tramissioni su bande differenti. I flat fading channels potrebbero non garantire una diversity sufficiente.
    \item \textbf{Spatial diversity}: relative alla coherence distance. Sfruttano path di propagazione differenti, as esempio antenne differenti.
\end{itemize}

\subsection*{Time diversity: interleaving and coding}
Il channel coding consiste nell'introdurre dei bit ridondanti insieme a quelli trasmessi per rilevare eventuali errori al ricevitore e migliorare la bit error probability.
La ridondanza è misurata come:
\[
    R = \frac{k}{n} < 1, \quad \begin{cases}
        k \text{ bit contenenti informazione} \\
        n \text{ bit in uscita dall'encoder, contenenti informazione più ridondanza} \\
        n - k \text{ bit di ridondanza}
    \end{cases}
\]

Le operazioni effettuate sfruttano le proprietà matematiche del Galois Field GF(2), su cui sono definite somma (xor) e moltiplicazione (and) per due elementi $\{0, 1\}$.

\begin{figure}[ht]
    \centering
    \includegraphics[width=0.675\textwidth]{imgs/encoder_decoder.jpg}
\end{figure}

Sorgente e destinazione introducono \textbf{encoder} e \textbf{decoder} per la gestione dei bit di ridondanza, si può utilizzare lo schema:


\begin{center}
\begin{tikzpicture} [align=center]
    % Draw the block
    
    \node[draw, rectangle, minimum width=2.5cm, minimum height=1.5cm] (block) {Block \\ coder};
    
    % Draw the input arrow
    \draw[-] (block.west) -- ++(-4,0) node[midway, above] {$k$ information digits};
    
    % Draw the output arrow
    \draw[->] (block.east) -- ++(3,0) node[midway, above] {$n$ encoded digits};
\end{tikzpicture}
\end{center}


\begin{center}
    \begin{tikzpicture}[align=center]
    % Draw the main rectangle
    \draw[thick] (0,0) rectangle (5,1);
    
    % Draw the dividing line
    \draw[thick] (3,0) -- (3,1);
    
    % Labels inside the rectangles
    \node at (1.5,0.5) {Information\\digits};
    \node at (4,0.5) {Parity\\digits};
    
    % Arrows and labels
    \draw[<->] (0,1.25) -- (3,1.25) node[midway, above] {$k$};
    \draw[<->] (3,1.25) -- (5,1.25) node[midway, above] {$n-k$};
    \draw[<->] (0,-0.25) -- (5,-0.25) node[midway, below] {$n$ digit codeword};
\end{tikzpicture}

\end{center}

In questo schema l'intero sistema è visto come una componente che aggiunge un errore al messaggio trasmesso.

\paragraph*{Block code}
Si tratta della tipologia più semplice di coding in cui la coded word tramessa è composta da $n-k$ bit di parità.
L'operazione è rappresentabile come operazione di moltiplicazione fra un vettore (word da trasmettere) e una matrice (\textbf{generator matrix}, definisce il tipo di operazione).

\[
    \mathbf{d} = \mathbf{uG}, \quad \begin{cases}
        \mathbf{d} \text{ coded word} \quad d \in GF(2)^{1 \times n} \\
        \mathbf{u} \text{ word da trasmettere} \quad u \in GF(2)^{1 \times k} \\
        \mathbf{G} \text{ generator matrix} \quad G \in GF(2)^{k \times n}
    \end{cases}
\]

In generale le prime $k$ colonne della matrice $G$ equiavalgono a $I_k$ (matrice identità $k \times k$) e le restanti $n-k$ colonne sono i bit di ridondanza.
In totale si possono ottenere $2^k$ codeword differenti (uscita dell'encoder).
Si parla di coding sistematico quando i bit di informazione sono semplicemente copiati.

\paragraph*{Error detection}

\begin{center}   
    \begin{tikzpicture}
        % Nodes for the transmitter side
        \node[draw, rectangle, minimum width=1.5cm, minimum height=0.65cm] (T1) at (0,4) {Data 1};
        \node[draw, rectangle, minimum width=1.5cm, minimum height=0.65cm] (T2) at (3,4) {Data 2};
        \node[draw, rectangle, minimum width=1.5cm, minimum height=0.65cm] (T3) at (6,4) {Data 2};
        % create a node containing a red cross
        %\node[draw, cross out, red, thick, minimum size=1cm] (cross) at (4,3) {};
        % create a rotate of 45 degrees
        \node[draw, cross out, red, thick, minimum size=0.5cm, rotate=45] (cross) at (4,3) {};
        % Nodes for the receiver side
        \node[draw, rectangle, minimum width=1.5cm, minimum height=0.65cm] (R1) at (2,2) {Data 1};
        \node[draw, rectangle, minimum width=1.5cm, minimum height=0.65cm] (R2) at (5,2) {Data 2};
        \node[draw, rectangle, minimum width=1.5cm, minimum height=0.65cm] (R3) at (8,2) {Data 2};

        % Arrows for data transmission
        \draw[->, thick] (T1) -- (R1);
        \draw[->, thick] (R1) -- (T2) node[midway, above, sloped] {ACK};
        \draw[->, thick] (T2) -- (R2);
        \draw[->, thick] (R2) -- (T3) node[midway, above, sloped, red] {NACK};
        \draw[->, thick] (T3) -- (R3);

        % Arrow for time
        \draw[<->, dashed] (4.75,1.25) -- (7.75,1.25) node[midway, below] {$T_{ARQ} > T_c$};

        % Vertical lines for transmission boundaries
        \draw[dashed] (4.675,0.5) -- (4.675,3.5);
        \draw[dashed] (7.675,0.5) -- (7.675,3.5);
        
        % Labels for transmissions
        \node at (4.675,0) {1\textsuperscript{st} transmission};
        \node at (7.675,0) {2\textsuperscript{nd} transmission};
        
        % Labels for Transmitter and Receiver
        \node[] at (-2,4) {Transmitter};
        \node[] at (-2,2) {Receiver};
    \end{tikzpicture}
\end{center}



Le tecniche di error detection consistono nel confrontare i bit di ridondanza ricevuti con i bit di ridondanza calcolati utilizzando le word ricevute. Se il confronto ha successo si assume che la trasmissione non abbia introdotto errori, altrimenti si rileva un errore nella trasmissione.
In casi di errore il ricevitore può richiedere una nuova trasmission, adottando lo schema \textbf{ARQ} (Automatic Repeat reQuest). In tale schema ad ogni ricezione si risponde con un ACK o un NACK, in base al risultato del confronto. In caso di NACK si procede con una nuova trasmissione. 
Si tratta di una tecnica di time diversity, in quanto la ritrasmissione avviende dopo $T_{ARQ}$, un intervallo temporale superiore al coherence time del canale ($T_{ARQ} > T_c$).
Alcuni ricevitori sono in grado di cambiare i due messaggi ricevuto, incrementando la probabilità di ricostruire l'informazione trasmessa.
Una semplice tecnica di error detection è il \textbf{parity check code}, in cui si aggiunge un bit di parità alla fine della word di 7 bit da trasmettere. Se il numero di bit a 1 è pari, il bit di parità è 0, altrimenti è 1.

\[
    \begin{cases}
        k = 7 \\
        n = 8 
    \end{cases}
    \Rightarrow R = \frac{7}{8},
    \quad \mathbf{G} = \left[I_7, 1_7\right] = 
    \begin{bmatrix}
        1 & 0 & 0 & 0 & 0 & 0 & 0 & 1 \\
        0 & 1 & 0 & 0 & 0 & 0 & 0 & 1 \\
        0 & 0 & 1 & 0 & 0 & 0 & 0 & 1 \\
        0 & 0 & 0 & 1 & 0 & 0 & 0 & 1 \\
        0 & 0 & 0 & 0 & 1 & 0 & 0 & 1 \\
        0 & 0 & 0 & 0 & 0 & 1 & 0 & 1 \\
        0 & 0 & 0 & 0 & 0 & 0 & 1 & 1 \\
    \end{bmatrix}
    \Rightarrow u_7 = \sum_{i=0}^{6} u_i
\]
Il parity bit è calcolato sommando tutti i bit delle word usando l'algebra in GF(2).
Sebbene sia molto semplice questa tecnica, non può essere sempre efficace in quanto può riconoscere unicamente un numero di errori dispari, mentre in caso di numero di errori pari si avrà un bilanciamente degli 1 ed il confronto avrà successo.

\paragraph*{Error correction}
Le tecniche di error correction sono utilizzate per rilevare e correggere errori di trasmissione, senza necessità di richiedere una nuova trasmissione.
Dato un canale si definisce \textbf{capacità} il massimo rate a cui è possibile trasmettere:
\[
    C = B \log_2(1 + \text{SNR}) \quad \text{bit/s}
\]

Ogni trasmissione con rate $R < C$ ed $\epsilon$ arbitratio è possibile determinare un error correction code per cui $P_e < \epsilon$.
Una semplice tecnica di error correction è il \textbf{repetition code}, in cui la word è ripetuta 3 volte. Per stabilire quale sia il bit corretto in casi si incongruenza si adotta una strategia maggioritaria.


\[
    \begin{cases}
        k = 1 \\
        n = 3 
    \end{cases}
    \Rightarrow R = \frac{1}{3},
    \quad \mathbf{G} =
    \begin{bmatrix}
        1 & 1 & 1
    \end{bmatrix}
    \quad 
    \begin{cases}
        u = \begin{bmatrix}0\end{bmatrix} \Rightarrow d = \begin{bmatrix}0 & 0 & 0\end{bmatrix} \\
        \\
        u = \begin{bmatrix}1\end{bmatrix} \Rightarrow d = \begin{bmatrix}1 & 1 & 1\end{bmatrix}
    \end{cases}
\]


Questa tecnica è in grado di correggere un unico errore, tuttavia se utilizzato come error detection può rilevare fino a due errori.
La distanza tra due codeword è calcolata come numero di bit differenti fra le due stringhe, detta anche \textbf{Hamming distance}. Il decoder selezione le word con minima distanza rispetto a quella ricevuta.

\[
    \hat{d} = \text{argmin}_d \left\{\text{distance}(d, \hat{x})\right\}
\]

Dove $\hat{d}$ è la word in uscita dal decoder, $d$ è una word possibile e $\hat{x}$ è la word ricevuta
Gli errori possono far scegliere al decoder la word sbagliata, la capacità di error correction di un block code è tanto maggiore quanto maggiore è la Hamming distance tra le codeword generabili. La bontà del block code è misurabile con la minima distanza tra codewords $d_{min}$.
\[
    d_{min} - 1 \quad \text{Numero massimo di errori rilevabili}
\]

\[
    \left\lfloor \frac{d_{min} - 1}{2} \right\rfloor \quad \text{Numero massimo di errori correggibili}
\]
Maggiore è la distanza di Hamming, maggiore è la ridondanza da aggiungere.
Fissata $R=\frac{k}{n}$, $d_{min}$ sarà più grande al crescere di $k$ e $n$, tuttavia si complica anche il sistema.
\section*{Convolutional code}

I codici convoluzionali, al contrario dei block code, non sono sistematici, si trasmettono infatti solo i bit di parità.
L'encoder utilizza una sliding window per generare $n>1$ bit di parità, combinando vari sottoinsiemi di bit nel campo GF(2), realizzando una sorta di convoluzione.
L'encoder si comporta come $n$ filtri lineari in paraellelo, i parametri che lo costituiscono sono:
\begin{itemize}
    \item $n$: bit generati
    \item $k$: bit di informazione (si considererà sempre $k=1$)
    \item $L$: lunghezza del vincolo, ovvero il numero di parole di input di $k$ bit che concorrono alla generazione degli $n$ bit dvincolo, ovvero il numero di parole di input di $k$ bit che concorrono alla generazione degli $n$ bit di output. Nel caso di $k=1$ sarà la lunghezza dei bit che concorrono alla generazione del codice.
\end{itemize}

La dimensione della finestra corrisponde a $L-K$, quindi si considererà $L-1$. Il codice generato, oltre alla word, è anche funzione dei bit di input precedenti. Ogni bit generato è ottenuto dalla convoluzione in GF(2) con una risposta impulsiva rappresentata da un diverso generatore $g$, di dimensione $kL$.
I bit in uscita sono quindi un flusso continuo e non organizzati in blocchi.

Trattandosi di sistemi con memoria, l'encoder può essere rappresentato come una macchina a stati finiti.
L'output dell'encoder dipende dal bit in input e dalla stato corrente.
L'evoluzione temporale dell'encoder può essere catturata dal \textbf{trellis diagram}, in cui si ha l'evoluzione degli stati in funzione del tempo.
Ogni sequenza, con la corrispondente encoded word, può essere rappresentata come un cammino nel trellis diagram.
Una trasmissione di $N$ codewords implica la trasmissione di $n \cdot N$ bits, ottenuti dalla codifica di $k \cdot N$ input word.
Poiché ogni codeword è funzione anche di $L-1$ input word precedenti, la sequenza può essere codificata considerandola solo nella sua interezza. 
Il decoder dovrà scegliere la sequenza più "vicina" rispetto a quella ricevuta tra tutte le possibili sequenza, ovvero $2^{k \cdot N}$
\[
    \hat{d} = \text{argmin}_d \left\{\text{distance}(d, \hat{x})\right\}
\]
Dove d è una possibile sequenza, $\hat{x}$ è la sequenza ricevuta e $\hat{d}$ è la sequenza in uscita dal decoder.
L'utilizzo nella pratica di codici convoluzionali è stata resa possibile solo dall'introduzione dell'algoritmo di Viterbi, in grado di applicare una decodifica con complessità lineare e non più esponenziale.

\paragraph*{Algoritmo di Viterbi}
L'obiettivo è trovare la sequenza con distanza minima rispetto a quella ricevuta:

\[
    \hat{d} = \text{argmin}_d \left( d_H \left(\tilde{d}, \hat{x}\right) \right)
\]
Le possibili sequenza $\tilde{d}$ sono viste come sequene di $N$ blocchi, ciascuna composta da 
$n$ bit, ovver i bit prodotti a partire dai $k$ bit di informaione in ingresso all'encoder.


\[
    d_H\left(\tilde{d}, \hat{x}\right) = \sum_{j=1}^{N} d_H\left(\tilde{d}_j, \hat{x}_j\right)
\]

Dove $d_H$ è la distanza di Hamming tra due sequenze di bit, $\tilde{d}_j$ è la $j$-esima codeword possibile e $\hat{x}_j$ è la $j$-esima codeword ricevuta.
Ogni sequenza $\tilde{d}$ corrisponde ad una seuqenza di stati $\tilde{S}_0, \ldots, \tilde{S}_N$ nel diagramma a trabocco, ovvero a un determinato path. La $j$-esima uscita dell'encoder, $\tilde{d}_j$, dipende dalla transizione tra gli stati $\tilde{S}_{j-1}$ e $\tilde{S}_j$.
.
.
.
.

% insert python code hello world hre
La distanza di Hamming può essere calcolata in Python come:
\begin{minted}{python3}
def hamming_distance(a: int, b: int) -> int:
    return bin(a ^ b).count('1')
\end{minted}


Dal Trellis diagram possiamo dedurre la funzione di transizione come:
\begin{minted}{python3}

def state_machine(state: int, input_bit: bool) -> Tuple[int, int]:
    match state:
        case 0b01:
            return (0b11, 0b00) if not input_bit else (0b00, 0b10)
        case 0b10:
            return (0b10, 0b01) if not input_bit else (0b01, 0b11)
        case 0b11:
            return (0b01, 0b01) if not input_bit else (0b10, 0b11)
        case 0b00:
            return (0b00, 0b00) if not input_bit else (0b11, 0b10)
        case _:
            raise ValueError("Invalid state")

\end{minted}
Per quanto riguarda invece l'algoritmo di Viterbi, esso può essere definito per sommi capi come:

\begin{minted}{python3}
def viterbi_algorithm(sequence: List[bool]) -> List[bool]:
    chunks: List[int] = bit_pairs_to_integers(sequence)
    matrix: List[List[Optional[ViterbiCell]]] = populate_matrix(chunks)
    final_state = get_last_state(matrix)
    return get_decoded_sequence(matrix, final_state)
\end{minted}

La funzione \textit{populate\_matrix} riempie una matrice con celle di Viterbi aggiornando distanze globali e stati successivi per ogni colonna di input.

\begin{minted}{python3}
def populate_matrix(chunks: List[int]) -> List[List[Optional[ViterbiCell]]]:
    matrix: List[List[Optional[ViterbiCell]]] = empty_matrix(rows=4, cols=len(chunks) + 1)
    matrix[0][0] = ViterbiCell(prev_state=0, global_distance=0, input_bit=False)

    for col in range(len(chunks)):
        for row in range(4):
            current_cell = matrix[row][col]
            if current_cell is None:
                continue
            for input_bit in [False, True]:
                output, next_state = state_machine(row, input_bit)
                new_distance = current_cell.global_distance + hamming_distance(output, chunks[col])
                next_cell = matrix[next_state][col + 1]
                if next_cell is None or next_cell.global_distance > new_distance:
                    matrix[next_state][col + 1] = ViterbiCell(
                        row, new_distance, input_bit
                    )
    return matrix
\end{minted}
La funzione \textit{get\_last\_state} trova lo stato finale con la distanza globale minima nell'ultima colonna della matrice.
\begin{minted}{python3}
def get_last_state(matrix: List[List[Optional[ViterbiCell]]]) -> int:
    min_distance = float('inf')
    final_state = None
    for i in range(4):
        last_row_cell = matrix[i][-1]
        if last_row_cell and last_row_cell.global_distance < min_distance:
            min_distance = last_row_cell.global_distance
            final_state = i
    return final_state
\end{minted}
La funzione \textit{get\_decoded\_sequence} ricostruisce la sequenza decodificata risalendo dalla matrice a partire dallo stato finale.
\begin{minted}{python3}
def get_decoded_sequence(matrix: List[List[Optional[ViterbiCell]]], final_state: int) -> List[bool]:
    sequence_length = len(matrix[0]) - 1
    decoded_sequence: List[bool] = [False] * sequence_length
    
    current_state = final_state
    for i in range(sequence_length - 1, -1, -1):
        cell = matrix[current_state][i + 1]
        decoded_sequence[i] = cell.input_bit
        current_state = cell.prev_state

    return decoded_sequence
\end{minted}


Come esempio consideriamo di voler trasmettere la sequenza di bit $010000$, l'encoder emetterà quindi la sequenza $00 \ 11 \ 10 \ 11 \ 00 \ 00$. Supponendo che il decoder riceva la sequenza $00 \ 10 \ 10 \ 11 \ 00 \ 00$, l'algoritmo di Viterbi sarà in grado di correggere l'errore e restituire la sequenza corretta $010000$.




\begin{center}
    \includegraphics[width=1\textwidth]{imgs/viterbi_example.png}
\end{center}


Ovvero eseguendo l'algoritmo in Python si ottiene la seguente rappresentazione in memoria, stampando il campo \textit{global\_distance}:

\begin{table}[h!]
\centering
\begin{tabular}{|c|c|c|c|c|c|c|c|}
\hline
\textbf{Row} & \textbf{Col 1} & \textbf{Col 2} & \textbf{Col 3} & \textbf{Col 4} & \textbf{Col 5} & \textbf{Col 6} & \textbf{Col 7} \\ \hline
0b00         & 0              & 0              & 1              & 2              & 1              & 1              & 1              \\ \hline
0b10         & None           & 2              & 1              & 2              & 2              & 3              & 3              \\ \hline
0b01         & None           & None           & 2              & 1              & 3              & 3              & 4              \\ \hline
0b11         & None           & None           & 4              & 3              & 3              & 3              & 4              \\ \hline
\end{tabular}
\end{table}

\section*{Interleaving}



\begin{center}
    \resizebox{\textwidth}{!}{
    \begin{tikzpicture}[node distance=1.5cm, auto, >=Stealth, minimum height=1cm, minimum width=1.5cm]
        % Nodes
        \node[draw, rectangle] (S) {S};
        \node[draw, rectangle, right=of S] (Encoder) {Encoder};
        \node[draw, rectangle, right=of Encoder] (Interleaver) {Interleaver};
        \node[draw, rectangle, right=of Interleaver] (Channel) {Channel};
        \node[draw, rectangle, right=of Channel] (Deinterleaver) {Deinterleaver};
        \node[draw, rectangle, right=of Deinterleaver] (Decoder) {Decoder};
        \node[right=of Decoder, inner sep=0pt, minimum size=0pt] (Uhat) {};

        % Connections
        \draw[->] (S) -- node {$u$} (Encoder);
        \draw[->] (Encoder) -- (Interleaver);
        \draw[->] (Interleaver) -- node {$d$} (Channel);
        \draw[->] (Channel) -- node {$x$} (Deinterleaver);
        \draw[->] (Deinterleaver) -- (Decoder);
        \draw[->] (Decoder) -- node {$\hat{u}$} (Uhat);


        % Dashed boxes
        \draw[dashed] ($(S.north west)+(-0.5,0.5)$) rectangle ($(Interleaver.south east)+(0.5,-0.5)$);
        \draw[dashed] ($(Deinterleaver.north west)+(-0.5,0.5)$) rectangle ($(Decoder.south east)+(0.5,-0.5)$);
    \end{tikzpicture}
}
\end{center}
\begin{center}
    \includegraphics[width=0.5\textwidth]{imgs/deep_fade.png}
\end{center}

I codici convoluzionali risultano adatti per canali memoryless con errori randomici, uniformamente distribuiti ed incorrelati. 
Tuttavia un canale fading è tipicamente suggetto a \textbf{bursty errors}, ovvero genera un gruppo di errori consecutivi nel tempo e/o in frequenza. 
Quando il canale è in \textbf{deep fade} ci sono delle dipendenze statistiche tra errori consecutivi.
L'idea dell'interleaving consiste nel far sembrare il canale memoryless dal punto di vista del decoder, de-correlando gli errori provocati dal canale, semplicemente rimescolando i bit prodotti dell'encoder prima di effettuare la trasmissione.
Lato ricevitore prima del decoder si effettuare un'operazione di de-interleaving per ripristinare l'ordine originale.
Il costo delle operazioni di (de-)interleaving è pagato in termini di latenza in quanto sia lato ricevitore che lato trasmettitore è necessario avere un intero blocco di dati prima di poter effettuare le operazioni. 
Esiste un trade-off tra latenza e decorrelazione ottenibile, basata sulla profondità $k$ dell'interleaver, ovvero la dimensione del blocco sul quale si effettua l'operazione di rimescolamento.

Per esempio disponendo una sequenza dentro una matrice, possiamo ottenere un interleaver considerando la trasposta:
\begin{table}[h!]
    \centering
    \begin{tabular}{c}
    
    \begin{tabular}{|c|c|c|c|}
    \hline
    A & B & C & D \\ \hline
    E & F & G & H \\ \hline
    I & J & K & L \\ \hline
    M & N & O & P \\ \hline
    \end{tabular}
    
    \quad $\rightarrow$ \quad
    
    \begin{tabular}{|c|c|c|c|}
    \hline
    A & E & I & M \\ \hline
    B & F & J & N \\ \hline
    C & G & K & O \\ \hline
    D & H & L & P \\ \hline
    \end{tabular}
    
    \end{tabular}
\end{table}
   

\begin{center}
    \resizebox{\textwidth}{!}{
    \begin{tikzpicture}
        \node (input) at (0,0) {$\{A,B,C,D,E,F,G,H,I,J,K,L,M,N,O,P\}$};
        \node[draw, minimum height=2cm, minimum width=3cm, align=center] (interleaver) at (6,0) {Interleaver};
        \node (output) at (12,0) {$\{A,E,I,M,B,F,J,N,C,G,K,O,D,H,L,P\}$};

        \draw[->] (input) -- (interleaver);
        \draw[->] (interleaver) -- (output);
    \end{tikzpicture}
    }
\end{center}



\paragraph*{Turbo code e LDPC (Low Density Parity Check)}
Nella formula della capacità del canale di Shannon\footnote{$C=B\log_2(1+\text{SNR}) \si{b/s}$} il rate $R = \frac{k}{n}$ prevede sia $k$ che $n$ tendenti all'infinito, tuttavia la lunghezza del codice nei sistemi reali è finito, quindi ciò che si ottiene è una performance lontana da quella teorica.
Negli anni sono stati introdotti nuovi codici, sempre più efficienti, tra cui turbo code e LDPC, i quali si avvicinano al limite teorico imposto dalla formula di Shannon.


\begin{center}

\includegraphics[width=0.5\textwidth]{imgs/codes_and_shannon_bound.png}
\end{center}
\paragraph*{Turbo code (error correction)}
\begin{center}
\begin{tikzpicture}

    % Draw blocks
    \node[draw, rectangle] (RSC1) at (3,2) {RSC1};
    % dummy node for alignment
    \node[inner sep=0pt, minimum size=0pt] (dummy) at (3,1) {};
    \node[draw, rectangle] (Interleaver) at (0,0) {Interleaver};
    \node[draw, rectangle] (RSC2) at (3,0) {RSC2};
    
    \draw[->] (0,2) -- (RSC1) node[midway, above] {};
    \draw[->] (RSC1) -- (5,2) node[midway, above] {$d^{(1)} = p_1$};
    \draw[->] (Interleaver) -- (RSC2);
    \draw[->] (RSC2) -- (5,0) node[midway, above]{$d^{(2)} = p_2$}; 
    \draw[->] (-1,1) -- (5,1) node[midway, above, xshift=63, yshift=2] {$d^{(0)} = u$};
    \draw[-] (Interleaver) -- (0, 2) node[midway, left] {}; 
    \node at (-0.2,1.2) {$u$};
    
\end{tikzpicture}
\end{center}
Si tratta di un convolutional code in cui sono utilizzati due encoder in parallelo in grado di generare due sequenze indipendenti, migliorando il processo di decoding grazie alla ridondanza ottenuta e alla diversity, ottenuta dalla presenza di uno stream di dati differente.
Il sistema è composto da tre sequenze di cui due codificate e una inalterata. L'utilizzo dell'interleaver permette di estendere artificialmente la lunghezza della sequenza, avvicinandosi alle ipotesi di Shannon. Inoltre l'interleaver permette di ottenere sequenze indipendenti. 
Il trasmettitore invia quindi tre sequenze, tra cui quella originaria, ottenendo un $R=\frac{1}{3}$. L'idea è che se il canale produce un errore le due sequenze prodotte in maniera differente possono essere unite  per correggere l'errore.
Per quanto riguarda il decoder si utilizzano due decoder in cascata, il primo prevede in ingresso i bit sistematici e la prima sequenza generanza senza interleaving. L'output prodotto invece di essere utilizzato come uscita del sistema è inviato ad un inteleaver, assieme ai bit sistematici. 
Le due sequenze dopo l'interleaving sono mandate in ingresso, assieme alla seconda sequenza di parità, al secondo decoder. L'uscita è infine inviata al deinterleaver. 
Completato il primo ciclo è possibile effettuarne altri, mandando nuovamente in infresso al primo decoder l'uscita dell'interleaver. 
L'uscita dei decoder rappresenta una stima, sempre più accurata, della sequenza trasmessa. Il processo iterativo va avanti finché vi sono modifiche nei bit decifrati.
In generale se il SNR rispetta le condizioni di Shannon l'impiego di turbo codes permette di effettuare la correzione degli errori ed ottenere dei BER molto bassi, tuttavia il prezzo è ancora pagato in termini di latenza.
\begin{center}
    

\begin{tikzpicture}[node distance=1.5cm, auto, >=Stealth, minimum height=1cm, minimum width=1.5cm]

    % Nodes
    \node[draw, rectangle] (D1) at (3,0) {D1};
    \node[draw, rectangle] (I) at (6,0) {I};
    \node[draw, rectangle] (D2) at (9,0) {D2};
    \node[draw, rectangle] (InvI) at (6,2) {Inv(I)};

    % Arrows
    \draw[->] (0,0) -- (2.25, 0) node[midway, above] {Parity1};
    \draw[->] (0,-0.875) -| (D1) node[midway, below] {Systematic};
    \draw[->] (0,-1.75) -| (I) node[midway, below] {Parity2};
    \draw[->] (3,-0.875) -| (5.675, -0.5) node[midway, below] {};
    %\draw[->] (-2,-0.3) -- (2.25, -0.3) node[midway, below] {Parity2};
    \draw[->] (D1) -- (I) node[midway, above] {Le12};
    \draw[->] (I) -- (D2);
    \draw[->] (D2) |- (InvI) node[midway, right] {Le21};
    \draw[->] (InvI) -| (D1);

% Additional Arrows
\draw[->] (6.25, -0.5) -- ++(0,-0.875) -| (D2);

\end{tikzpicture}


\end{center}
\section*{Space diversity}

In generale sia la frequency che time diversity richiedono parecchie risorse, per questo un altro tipo di diversity è spesso sfruttata, semplicemente utilizzando più antenne. senza sacrifici di risorse. Si possono sfruttare due tipologie di gain differenti:
\begin{itemize}
    \item \textbf{Array gain}: si tratta del guadagno di potenza ottenuto utilizzando più antenne rispetto all'utilizzo della singola antenna. Il gain è tanto più alto quanto è alta la correlazione spaziale del canale. Si sfruttano tipicamente antenne direzionali.
    \item \textbf{Diversity gain}: si tratta del guadagno ottenuto combinando i segnali ricevuti dalle varie antenne, considerati incorrelati. Tale gain è massimo quando i segnali sono completamente decorrelati. 
\end{itemize} 

Per assumere incorrelazione tra le varie antenne la loro distanza deve essere almeno la metà della lunghezza d'onda:
\[
    d_c = \frac{\lambda}{2} \quad \text{coherence distance}
\]  

La \textbf{coherence distance} deriva dalle proprietà di time varying del canale, tuttavia come si può osservare non ha dipendenze né temporali né della velocità.

Le carrier frequencies attualmente utilizzate permettono di utilizzare anche antenne nel solito sistema sfruttando al massimo la space diversity. Bisogna comunque tenere in considerazione che ogni antenna ha bisogno della propria RF chain, quindi lo spazio occupato è maggiore rispetto alla grandezza dell'antenna.


\paragraph*{Canale MIMO (Multiple Input Multiple Output)}


Nei classici sistemi con un antenna in trasmission ed un'antenna in ricezione, il canle, se narrow-band può essere descritto da uno scalare complesso con una certa fase e ampiezza.
Nel caso di sistemi con più antenne in trasmission e ricezione tra ogni coppia di antenne vi è un canale differente, quindi l'intero sistema, per quanto riguarda il canale, può essere descritto tramite una mtrice complessa.
\[ 
    \mathbf{H} = 
    \begin{bmatrix}
        h_{11} & h_{12} & \ldots & h_{1M} \\
        h_{21} & h_{22} & \ldots & h_{2M} \\
        \vdots & \vdots & \ddots & \vdots \\
        h_{N1} & h_{N2} & \ldots & h_{NM} \\
    \end{bmatrix}
    , \quad
    \begin{array}{ll}
            \mathbf{H} \in \mathbb{C}^{N \times M} \\
            \mathbf{y} = \mathbf{Hx} 
    \end{array}
\]

Dove $\mathbf{x}$ è il segnale trasmesso, $\mathbf{y}$ è il segnale ricevuto e $\mathbf{H}$ è la matrice del canale. In questo caso, $M$ rappresenta il numero di antenne in trasmissione mentre $N$ il numero di antenne in ricezione.

Nel caso di unica antenna in trasmissione si parla di canale SIMO ed in tal caso $\mathbf{H} \in \mathbb{C}^{N \times 1}$
\[
    x_i(m) = h_i c(m) + n_i(m) \quad i = 1, \ldots, N
\]
Questo rappresenta il segnale ricevuto sull'$i$-esima antenna. Ogni segnale ha il proprio rumore e distorisione da parte del canale.

L'idea più semplice per sfruttare la diversity è scegliere di volta in volta il segnale con attenuzione minore, stimando il canale. Tuttavia esistono tecniche in grado di utilizzare le informazioni provenienti da tutti i segnali ricevuti.

\[
    z(m) = \sum_{i=1}^{N} \left( w_i h_i c(m) + w_i n_i(n) \right)
\]
dove $w_i$ sono i pesi assegnati alle varie antenne.


\[
    \begin{array}{ll} 
        P = \mathbb{E} \left[ \left| \sum_{i=1}^{N} w_i h_i c(m)  \right|^2 \right] \quad & \text{Potenza segnale utile} \\
        P_N = \mathbb{E} \left[ \left| \sum_{i=1}^{N} w_i n_i(m)  \right|^2 \right] \quad & \text{Potenza rumore}
    \end{array}
\]
L'obiettivo è massimizzare il SNR, così definito:
\[
    \text{SNR} = \frac{\sum_{i=1}^{N} w_i h_i}{\sum_{i=1}^{N} w_i} = \frac{A}{\sigma^2}  
\]

\[
    P = \mathbb{E} \left[ \left|  \sum_{i=1}^{N} w_i h_i c(m)  \right| ^2  \right] = \left| \sum_{i=1}^{N} w_i h_i \right|^2 \mathbb{E} \left[ \left| c(m)\right|^2  \right] = \left| \sum_{i=1}^{N} w_i h_i \right|^2 A
\]
Scegliendo accuratamente $w_i$ si può eliminare la parte immaginaria, quindi possiamo riscrivere $P$ come:
\[
    P = \left( \sum_{i=1}^{N} w_i h_i \right)^2 A
\]



\[
    \begin{array}{ll}
            P_N = \mathbb{E} \left[ \left| \sum_{i=1}^{N} w_i n_i(m)  \right|^2 \right] \\
            = \mathbb{E} \left[ \left( \sum_{i=1}^{N} w_i n_i(m) \right)  \left( \sum_{i=1}^{N} w_i n_i(m) \right)^*  \right] \\
            = \mathbb{E} \left[  \sum_{i=1}^{N} \sum_{j=1}^{N} w_i w_j^* n_i n_j^*  \right] \\
            = \sum_{i=1}^{N} \sum_{j=1}^{N} w_i w_j^* \mathbb{E} \left[ n_i n_j^* \right] \\
            = \sum_{i=1}^{N} \sum_{j=1}^{N} w_i w_j^* \sigma^2 \delta_{ij} \\
            = \sum_{i=1}^{N} w_i w_i^* \sigma^2 \\
            = \sum_{i=1}^{N} |w_i|^2 \sigma^2 \\
            = \sigma^2 \sum_{i=1}^{N} |w_i|^2 
    \end{array}
\]

Se si ha un unico canale l'espressione del SNR si riduce al semplice rapporto $hA/\sigma^2$, ovvero la forma valida per un semplice fading channel.
L'espressione completa può essere analizzata sfruttando la disuguaglianza di Schwarz:
\[
    \text{SNR} = \frac{\left(\sum_{i} w_i h_i \right)^2}{\sum_{i} |w_i|^2} \frac{A}{\sigma ^2} \leq \frac{\sum_{i} |h_i|^2 \sum_{j} \left| w_j \right|^2}{\sum_{i} |w_i|^2} \frac{A}{\sigma^2}
\]


Rispettando la condizione $w_i = h_i^*$ si ottiene:
\[
    \text{SNR} = \frac{\left(\sum_{i} \left| h_i \right|^2 \right)^2}{\sum_{i} |h_i|^2} \frac{A}{\sigma ^2} \leq \sum_{i} |h_i|^2 \frac{A}{\sigma^2}
\]


Per quanto riguarda la configurazione con più antenne in trasmissione e singola antenna in ricezione, sistema MISO, si ottiene la stessa espressione con alcuni accorgimenti lato trasmettitore
\[
    x(m) = \sum_{i=1}^{N} h_i y_i(m) + n(m) \quad \text{segnale ricevuto tramite l'unica antenna disponibile}
\]


La differenza sostanziale con il caso SIMO è che i vari segnali sono combinati ``naturalmente" durante la trasmissione e ricevuti quindi già aggregati. 
La scelta dei coefficienti non può essere effettuata lato ricevitore, ma deve essere il trasmettitore a scegliere stimando il canale, lato ricevitore la stima è ovviamente più complessa rispetto a quella che effettuerebbe il ricevitore per proprio conto.
L'operazione effettuata dal trasmettitore è detta \textbf{spatial precoding} ed il simbolo è denominato \textbf{spatial pre-coded symbol}.
\[
    y_i(m) = b_i c(m) \quad \text{simbolo trasmesso sulla $i$-esima antenna}
\]
L'energia spesa dipende anche dal coefficiente $b_i$, l'obiettivo è comunque utilizzare la solita energia usata nel caso di singola antenna.

Il rapporto SNR è massimizzato scegliendo $b_i = \frac{h_i^*}{ \| h \| }$

\[
    x(m) = \sum_{i=1}^{N} h_i b_i c(m) + n(m) = \sum_{i=1}^{N} \frac{h_i h_i^*}{\| \mathbf{h} \|} c(m) + n(m) = \| \mathbf{h} \| c(m) + n(m) = \sum_{i=1}^{N} |h_i|^2 \frac{A}{\sigma^2}
\]

La differenza tra il caso con singola antenna (rx e tx) e il caso con più antenne (MISO/SIMO) è dato dal fattore $\| \mathbf{h} \| = \sum_{i} |h_i|^2$.
Considerando i gain del canale come variabili aleatore con distribuzione di Rayleigh è possibile calcolare la distribuzione della somma come convoluzione delle PDF.
Il risultato dipende dal nnumero di antenne, tuttavia in generale si ottiene un \textbf{array gain}, ovvero uno spostamente del valor medio dei gain verso destra (più potenza). Inoltre il \textbf{diversity gain} si manisgesta riducendo notevolmente la probabilità di ottenere gain molto bassi. Ciò è visibile confrontando gli integrali delle due PDF.
Il caso generale MIMO, ovvero con più antenne sia in trasmissione che ricezione, sfrutta una tecnica denominata \textbf{spatial multiplexing} ed è basata sulla decomposizione ai valori singolari (SVD) della matrice $\mathbf{H}$ che descrive il canale.

\paragraph*{SVD (Singular Value Decomposition)}

Data una matrice $\mathbf{A} \in \mathbb{C}^{N \times M}$, la SVD permette di scrivere la matrice come prodotto di tre matrici:
\[
    \mathbf{A} = \mathbf{U} \mathbf{\Sigma} \mathbf{V}^H
\]
con $\mathbf{U} \in \mathbb{C}^{m \times p}$, $\mathbf{\Sigma} \in \mathbb{C}^{p \times p}$ e $\mathbf{V} \in \mathbb{C}^{n \times o}$,
dove $\mathbf{U}$ e $\mathbf{V}$ sono matrici unitarie, mentre $\mathbf{\Sigma} = diag(\sigma_1, \ldots, \sigma_p)$ è una matrice diagonale con $\sigma_1 \geq \sigma_2 \geq \ldots \geq \sigma_p$. 
\
In un sistema MIMO, se il canale è sufficientemente multipath, la matrice $H$ ha rango pieno, dunque:
\[
    P = rank(H) = \min(N_R, N_T)
\]

La tecnica di \textbf{spatial multiplexing} consiste nell'effettuare sia un pre-coding lato trasmettitore, sia un combining lato ricevitore, dunque untilizzando due pesi differenti. In questo modo è possibile creare un certo numero di canali ortogonali indipendenti, detti \textbf{spatial channels}.
\[
    \begin{array}{ll}
        \mathbf{H} = \mathbf{U} \mathbf{\Sigma} \mathbf{V}^H \quad \text{decomposizione SVF matrice del canale MIMO} \\
        \mathbf{B} = \mathbf{V} \quad \text{matrice pre-coder (Tx)} \\
        \mathbf{W} = \mathbf{U} \quad \text{matrice combiner (Rx)}
    \end{array}
\]

Scegliendo in questo modo le matrici si ottiene:
\[
    \begin{array}{ll}
        \mathbf{z} = \mathbf{W}^H \mathbf{y}  \\
        = \mathbf{W}^H \left( \mathbf{H} \mathbf{x} + \mathbf{n}  \right) \\  
        = \mathbf{W}^H \left( \mathbf{H} \mathbf{B} \mathbf{s} + \mathbf{n} \right) \\
        = \mathbf{W}^H \mathbf{H} \mathbf{B} \mathbf{s} + \mathbf{W}^H \mathbf{n}  \\
        = \mathbf{U}^H \mathbf{H} \mathbf{V} \mathbf{s} + \mathbf{W}^H \mathbf{n} \\
        = \mathbf{\Sigma} \mathbf{s} + \mathbf{W}^H \mathbf{n}
    \end{array}
\]
Sebbene ogni antenna riceva dei segnali sovrapposti, scegliendo le matrici in questa maniera è possibile ottenere dei canali ortogonali che non interferiscono tra loro dato che la matrice $\mathbf{\Sigma}$ è diagonale.

Considerando un sistema $2 \times 2$ per esempio si ottiene:
\[
    \begin{cases*}
        y_1 = h_{11} x_1 + h_{12} x_2 + n_1 \\
        y_2 = h_{21} x_1 + h_{22} x_2 + n_2
    \end{cases*}
    \quad
    \text{segnali sovrapposti su ogni antenna}
\]

L'utilizzo di un canale MIMO può incrementare la capacità del sistema, tuttavia è necessario tenere in considerazione anche i valori singolari, i quali potrebbero comportare un'attenuazione eccessiva.


\[
    \mathbf{z} = \begin{matrix}
        \begin{bmatrix}
            \sigma_1 s_1 + n_1' \\
            \sigma_2 s_2 + n_2'
        \end{bmatrix}
    \end{matrix}
    \quad 
    \text{segnale ottenuto dopo combining lato ricevitore, non è alcuna sovrapposizione}
\]

\section*{Frequency diversity}

La formula di Shannon fornisce il massimo rate sostenibile su un dato canale di trasmissione:
\[
    C = B \log_2 \left( 1 + \text{SNR} \right), \quad \text{SNR} = \frac{\left| H \right|^2 P}{\sigma^2}
\]
Dove $H$ è il gain del canale.
In sistemi reali ci si può solo avvicinare al limite teorico e vi è una forte dipendenza, per quanto riguarda l'efficienza spettrale, del coding rate e modulation order, data dalla dimensione della costellazione di simbolim per cui esiste un trade-off tra capacità di trasmissione ed energia richiesta. Il rate di trasmissione reale può essere approssimato:
\[
    R_b = \log_2 M \  R \frac{1}{T}  \approx m R B
\]

Dove è $M$ è il numeri di simboli nella costenllazione, $R$ è il coding rate e $T$ è il simbol time, si fa l'approssimazione che $\frac{1}{T} \approx B$. 


Per rispettare il limite di capacità si ha:
\[
    R_b < B \quad \Rightarrow \quad mR < \log_2 (1 + \text{SNR})
\] 
Misurando il SNR è possibile adattare i valori di $m$ ed $R$, ad esempio in base alla posizone in una cella.
I valori di $m$ ed $R$ che si possono utilizzare appartengono ad un insieme finito, non è possibile scegliere arbitrariamente.

La qualità del canale in termini di SNR può essere espressa tramite \textbf{radio beared index} (1 - 15).
Utilizzando la modulazione OFDM l'idea sarebbe adattare $m$ ed $R$ per ogni canale, distribuendo la potenza nel modo più efficace possibile per massimizzare il rate di trasmmissione. Questo problema è inidcato con il termine di \textbf{water-filling}, la potenza a disposizione è invece detta \textbf{power budget}. Nella pratica si tratta di un problema di ottimizzazione per il quale sono definiti alcuni vincoli da rispettare:


\[
    \begin{cases}
       \underset{p}{\text{max}} \sum_{n=1}^{N} \log_2(1 + \frac{P_n}{\sigma_n^2}), \quad P_n \geq 0 \quad n = 1, \ldots, N \\
       \sum_{n=1}^{N} P_n = P_0 \quad \text{(power budget)}
    \end{cases}    
\]

Dove $\sigma_n^2 = \frac{\sigma^2}{\left| H_n \right| ^2}$. Il termine $B$ non appare nell'espressione in quanto costante per ogni canale, massimizzare il rate o l'efficienza spettrale genera lo stesso risultato.
\[
    P_n^* = \text{max} \{ 0, \mu - \sigma_n^2    \} = (\mu - \sigma_n^2)^+, \quad n = 1, \ldots, N
\]

Per ottenere dei valori è necessario calcolare $\mu$, scelto in modo da rispettare il vincolo sulla potenza a disposizione.
\[
    \sum_{n=1}^{N} (\mu - \sigma_n^2)^+ = P_0
\]
\[
    \begin{cases}
        f(\mu) = \sum_{n=1}^{N} (\mu - \sigma_n^2)^+ - P_0 \\
        f: \mathbb{R}^n \rightarrow \mathbb{R}
    \end{cases}
\]
La risoluzione dell'equazione garantisce il valore $\mu$ tale per cui i vincoli sono rispettati, è quindi necessario trovare lo zero di una funzione in $\mu$, ad esempio con il metodo della bisettrice. Il metodo della bisettrice può essere applicata dato che la funzione risulta continua e monotona sul dominio.
Il termine water-filling deriva dal fatto che osservando il grafico della distribuzione della potenza, si ha l'impressione che la potenza si distribuisca come l'acqua all'interno di un contenitore con superiore irregolare, fino ad un livello pari a $\mu$.
La distribuzione è ottima e rappresenta la miglior situazione per sfruttare la frequency diversity del canale. In sostanza la distribuzione favorisce i canali con miglior qualità, scartando completamente quelli la cui capacità è troppo bassa.






\section*{Transmissions over Fading Channels}

\begin{itemize}
    \item Signal fading is the main problem in wireless communications.
    \item OFDM is a technique designed to combat the destructive effects of multipath fading.
    \item Slow flat Rayleigh fading is still a big problem.
    \item One of the most effective resources against the effects of channel fading is \textbf{diversity}.
\end{itemize}

\section*{Diversity in Wireless Communications}

\begin{itemize}
    \item \textbf{Diversity} refers to the possibility of improving the reliability of a message by transmitting it over two or more communication channels with different characteristics.
    \item Diversity is a common technique for combatting fading and co-channel interference and avoiding error bursts.
\end{itemize}


\section*{Diversity in Wireless Communication Systems}

\subsection*{Time Diversity:}
Time diversity relates to the \textit{coherence time} of the channel and includes techniques such as transmission over multiple time slots by channel coding plus interleaving. It is particularly effective over very slow fading channels.

\subsection*{Frequency Diversity:}
Frequency diversity relates to the \textit{coherence bandwidth} of the channel. It involves transmission over multiple frequency bands and offers advantages over very flat fading channels.

\subsection*{Spatial Diversity:}
Spatial diversity is associated with the \textit{coherence distance} and involves the transmission and reception employing multiple antennas.

\section*{Time Diversity: Interleaving and Coding}
\begin{itemize}
    \item Initially proposed by Claude Shannon in 1948, channel coding introduces redundancy to detect errors at the receiver and improve the bit error probability.
    \item Redundancy is quantified by the code rate $R = \frac{k}{n}$, where $k$ is the number of input bits, and $n$ is the number of output bits from the encoder.
    \item While initially studied for AWGN channels, channel codes are applicable to fading channels to enhance reliability.
\end{itemize}



\section*{Error Detection Coding}

A straightforward technique in error detection is the use of parity bits. Parity bits are additional bits at the end of a data word used for error detection. They provide a simple way to check the integrity of the data. 

\subsection*{Parity Check Code}
A parity check code is an error detection code where the code rate \( R \) is determined by the ratio of information bits to total bits, including the parity bit(s). 

For example, consider a parity check code with rate \( R = \frac{7}{8} \):
\begin{itemize}
    \item The word is composed of \( k = 7 \) bits.
    \item A single parity bit is added, making \( n = 8 \) bits in total.
    \item The generator matrix \( \mathbf{G} \) for such a code can be expressed as a \( 7 \times 8 \) matrix, where the additional column is used for the parity bit, ensuring that the word always contains an even number of '1's.
    \item The encoded vector \( \mathbf{d} \) is then given by \( \mathbf{d} = \mathbf{uG} \), where \( \mathbf{u} \) is the vector of information bits.
\end{itemize}

The encoded word is an 8-bit vector with the property that it contains an even number of '1's if the original 7-bit word contained an odd number of '1's, and vice versa.

At the decoder side:
\begin{itemize}
    \item The receiver computes the parity check by examining the number of '1's in the received 8-bit word.
    \item If the number of '1's is even, the word is considered error-free.
    \item If the number of '1's is odd, this indicates that an error has occurred, assuming only single-bit errors are possible.
\end{itemize}

Not all errors can be detected with this method, particularly if an even number of bits are in error, as this will not affect the parity check condition.


\section*{Data Retransmission and Channel Capacity}

\subsection*{Data Retransmission}
\begin{itemize}
    \item Automatic Repeat Request (ARQ) is a protocol for error control in data transmission where the receiver sends back an acknowledgment (ACK) for correctly received packets and a negative acknowledgment (NACK) for faulty ones.
    \item When a NACK is received, the transmitter resends the data packet. This method exploits the \textit{time diversity} of the channel by retransmitting the data after a time interval longer than the channel coherence time \( T_c \).
    \item Advanced receivers can combine multiple received messages to improve the chances of successful reception, known as \textit{diversity combining}.
\end{itemize}

\subsection*{Error Correction Coding and Channel Capacity}
\begin{itemize}
    \item Error correction coding is utilized to correct errors introduced by the channel during transmission.
    \item Shannon's theorem establishes that for a communication channel of bandwidth \( B \), the channel capacity \( C \) is the upper bound on the rate at which information can be reliably transmitted over the channel and is given by:
    \[ C = B \log_2 (1 + \text{SNR}) \text{ bits/s}, \]
    where SNR is the Signal-to-Noise Ratio.
    \item If the transmission rate \( R \) is less than the channel capacity \( C \), it is theoretically possible to design an error correction code that makes the probability of error as low as desired, but not zero.
    \item On the contrary, if \( R > C \), it is impossible to guarantee an arbitrarily low error probability.
\end{itemize}

\subsection*{Practical Example}
\begin{itemize}
    \item In a communication system with bandwidth \( B = 1.8 \) MHz and using 16-QAM modulation, if the symbol rate \( R_s = \frac{1}{T} \) is such that \( R_s \cdot B \log_2(1 + \text{SNR}) \) does not exceed \( 40 \) Mbits/s, the transmission is within the channel capacity and is thus feasible.
    \item Otherwise, if the product of symbol rate and channel capacity is exceeded, the transmission would exceed the channel's capability to convey information without errors, leading to an increased error rate.
\end{itemize}






\section*{Block Codes: Repetition Code and Decoding}

\subsection*{Repetition Code}
\begin{itemize}
    \item The simplest form of block codes is the repetition code.
    \item For \( k = 1 \) and \( n = 3 \), the \( 3 \)-repetition code is used, which can be considered as a \( (1,3) \) code.
    \item The generator matrix for the code is \( G = \begin{bmatrix} 1 & 1 & 1 \end{bmatrix} \).
    \item Encoding of a bit \( u \) is done as \( d = uG \), resulting in \( d \) being either \( [0 \, 0 \, 0] \) or \( [1 \, 1 \, 1] \) for \( u = 0 \) or \( u = 1 \), respectively.
\end{itemize}

\subsection*{Decoder Operation}
\begin{itemize}
    \item The decoder uses majority decision decoding.
    \item Given a received bit string \( \hat{d} \), the decoder computes \( u = \hat{d} G^T \) and makes a decision based on the majority of bits.
    \item The rate \( R \) of the repetition code is \( \frac{1}{3} \).
\end{itemize}

\subsection*{Hamming Distance and Error Detection}
\begin{itemize}
    \item The Hamming distance is utilized to measure the distance between two codewords.
    \item For error detection, \( \hat{d} \) is chosen such that the Hamming distance \( d(\hat{d},d) \) is minimized.
    \item The Hamming distance provides a measure of error correction capability and error detection.
\end{itemize}


\section*{Block Codes: Decoder}

\begin{itemize}
    \item \textbf{Error Events:} An error event in block codes occurs when noise causes the received vector \(\hat{x}\) to be closer to a codeword that is different from the transmitted one.
    \item \textbf{Robustness of Codes:} Codes that have a larger Hamming distance between words are more robust against noise and fading compared to codes with a smaller Hamming distance.
    \item \textbf{Hamming Distance:} The Hamming distance is a metric that increases with the length of the code, specifically the parameters \(k\) (number of information bits) and \(n\) (length of the codeword). Unfortunately, the complexity of the receiver that must decode these codes also grows with \(k\).
\end{itemize}

\section*{Implications on Decoder Complexity}

The robustness of error-correcting codes and the ability of a decoder to correct errors without requiring retransmission are essential for efficient communication systems. However, this error correction capability comes at the cost of increased receiver complexity. This complexity is influenced by factors such as the length of the codeword \(n\) and the number of information bits \(k\). As these parameters increase, the computational effort required to decode the received messages also increases, which can impact the processing speed and energy consumption of communication devices.


\section*{Convolutional Codes: Encoder}

\begin{itemize}
    \item The encoder of a convolutional code, denoted as \((n, k, L)\), operates as \(n\) parallel linear filters over the Galois field GF(2).
    \item The parameter \(L\) is the \emph{constraint length} of the encoder, determining how many previous \(k\)-bit input words affect the \(n\)-bit output.
    \item Unlike block codes, convolutional codes have \emph{memory}: each output bit is a function of the current and \(L-1\) previous input bits, introducing temporal correlation in the encoded sequence.
\end{itemize}

\section*{Code Generators for Convolutional Codes}

\begin{itemize}
    \item The impulse response of the \(n\) linear filters defines the convolutional code's generator vectors, each of length \(L\cdot k\).
    \item In GF(2), these impulse responses consist of binary sequences representing the filter taps.
    \item The output codeword bits are computed by convolving the input bits with these generator sequences.
\end{itemize}

\subsection*{Example: Convolutional Code Generation}
Given a single-bit input (\(k=1\)) and two output bits (\(n=2\)), with \(L=3\), the convolutional encoding can be represented as follows:
\begin{align*}
    d^{(i)}_j &= \sum_{l=0}^{2} g_j(l)\cdot u^{(i-l)} \quad \text{for } j = 1,2 \\
    &= g_j(0)\cdot u^{(i)} + g_j(1)\cdot u^{(i-1)} + g_j(2)\cdot u^{(i-2)}
\end{align*}
where \(g_j(l)\) are the elements of the generator vectors, and \(u^{(i)}\) is the \(i\)-th input bit.

\section*{Matrix Representation}
The systematic form of a convolutional encoder can also be represented using a generator matrix \(G\), though the inherent memory aspect of convolutional codes makes this representation more complex than for block codes.

\[
G = 
\begin{bmatrix}
    g_{1,0} & g_{1,1} & \dots & g_{1,L-1} \\
    g_{2,0} & g_{2,1} & \dots & g_{2,L-1} \\
    \vdots  & \vdots  & \ddots & \vdots    \\
    g_{n,0} & g_{n,1} & \dots & g_{n,L-1} \\
\end{bmatrix}
\]


\section*{The Convolutional Code (2,1,3)}

Let's delve deeper into the specifics of the (2,1,3) convolutional code.

\subsection*{Encoding Process}
The generator polynomials for this code are $g_1 = [1\ 1\ 1]$ and $g_2 = [1\ 0\ 1]$, with the code rate $R = \frac{1}{2}$. The codeword bits are calculated using:
\begin{align*}
    d_1^{(i)} &= u^{(i)} + u^{(i-1)} + u^{(i-2)} \\
    d_2^{(i)} &= u^{(i)} + u^{(i-2)}
\end{align*}
where $u^{(i)}$ represents the input bit at time index $i$.

\subsection*{State Diagram Representation}
The encoder can also be represented by a state diagram, illustrating the finite state machine nature of the encoder:

In the diagram, each state transition corresponds to an input bit (above the line) and the resulting encoded output bits (below the line).

\subsection*{Encoding Memory}
It's crucial to note that the convolutional encoder has memory, which is reflected in the state transitions. The current state along with the input bit determines the next state and the output bits.








\section*{Convolutional Codes}

\subsection*{Trellis Diagram}
The trellis diagram for the (2,1,3) encoder is a graphical representation that shows all possible transitions between states at each time step based on the input bit. The red path indicates the sequence of states and outputs for a given input.
\subsection*{Decoder Functionality}
Convolutional code decoders process the received bit stream and attempt to reconstruct the transmitted information sequence. The decoder selects the path through the trellis diagram that best matches the received sequence, which is the path with the minimum Hamming distance to the received sequence.
\section*{Convolutional Codes: The Viterbi Algorithm}

\subsection*{Introduction}
\begin{itemize}
  \item The Viterbi algorithm, presented in 1967 by Andrew Viterbi, revolutionized the decoding of convolutional codes, reducing the complexity from exponential to linear in \( N \), where \( N \) is the length of the encoded sequence.
  \item It operates by evaluating the minimum Hamming distance of all possible paths through a trellis diagram representing the state transitions of the encoder.
\end{itemize}

\subsection*{Algorithm Overview}
The Viterbi algorithm simplifies the decoding process by:
\begin{enumerate}
  \item Identifying and expanding only the most likely paths at each step (also known as `survivor paths').
  \item Computing a `cumulated metric' for each path, which is a running sum of the `branch metrics', measures of how well the path matches the received sequence.
  \item Discarding less likely paths, thus reducing the need to compute the entire trellis, which is \( 2^{kN} \) for a block of \( N \) bits and \( k \) information bits.
  \item Selecting the path with the lowest cumulative metric at the end of the decoding process, representing the most likely transmitted sequence.
\end{enumerate}

\subsection*{Practical Impact}
\begin{itemize}
  \item The Viterbi algorithm is widely used in various communication systems, including those that require real-time decoding, due to its efficiency and performance.
\end{itemize}



\section*{Interleaving in Communication Systems}

Convolutional codes are well-suited for memoryless channels with random error events. They are most effective when errors are uniformly distributed and uncorrelated. However, fading channels often cause errors that are bursty in nature, meaning that errors tend to be correlated over time. 

Interleaving is a technique used to combat the effect of burst errors. It rearranges the order of the transmitted symbols according to a certain pattern. When the sequence passes through a channel that causes burst errors, these errors will be spread out in the deinterleaved sequence at the receiver. This makes the channel seem more like a memoryless channel, which helps standard error-correcting codes to perform error correction more efficiently by decorrelating error events.

An example of block interleaving is shown below, where the sequence of symbols is rearranged before transmission:

\begin{verbatim}
Original sequence: A, B, C, D, E, F, G, H, I, J, K, L, M, N, O, P
Interleaved sequence: A, E, I, M, B, F, J, N, C, G, K, O, D, H, L, P
\end{verbatim}

\section*{Block Interleaver Example}
Consider an interleaver with the following input and output:
\begin{align*}
\text{Input:} & \quad (A,B,C,D,E,F,G,H,I,J,K,L,M,N,O,P) \\
\text{Interleaved Output:} & \quad (A,E,I,M,B,F,J,N,C,G,K,O,D,H,L,P)
\end{align*}

This interleaving strategy ensures that errors affecting consecutive symbols in the original sequence will be spread out in the transmitted sequence, thus reducing the probability of burst errors.



\section*{Interleaving}
Interleaving is a technique applied in communication systems for error correction, particularly effective against burst errors. It works by spreading the coded symbols across the time or frequency domain, which makes bursty error patterns appear random, thereby improving the performance of convolutional codes.

\textbf{Interleaving Considerations:}
\begin{itemize}
    \item It introduces latency in the transmission process.
    \item The depth of the interleaver (\( K \)) is proportional to the de-correlation of errors but also to the introduced latency.
    \item Types of interleaving include block and convolutional (or cross) interleaving.
\end{itemize}

\section*{Turbo Codes and LDPC}
Advanced coding techniques, such as Turbo codes and Low-Density Parity-Check (LDPC) codes, mark significant milestones in channel coding theory. These coding strategies approach the Shannon limit, which is the theoretical maximum data rate of a noisy channel:

\[ C = B \log_2 (1 + \text{SNR}) \text{ bits/s} \]

\begin{itemize}
    \item Turbo codes, introduced in 1993, utilize a feedback loop in the encoding process.
    \item LDPC codes, introduced in 1999, use a sparse parity-check matrix.
    \item Both coding techniques offer near-Shannon limit performance.
\end{itemize}


\section*{Turbo Codes}

Turbo codes are a class of high-performance error correction codes that employ a parallel concatenation of two or more convolutional codes, separated by an interleaver.

\subsection*{Encoding Process}
\begin{enumerate}
    \item Data bits are input to the first encoder to produce a set of parity bits.
    \item The data bits are then interleaved, effectively shuffling the order of the bits, to spread out bursts of errors.
    \item The interleaved bits are input to a second encoder to produce another set of parity bits.
\end{enumerate}

\subsection*{Decoding Process}
Turbo decoding employs an iterative algorithm using two decoders, which pass soft information back and forth between each other.

\begin{enumerate}
    \item The first decoder processes the received data and produces extrinsic information.
    \item This extrinsic information is then de-interleaved and passed as a priori information to the second decoder.
    \item The second decoder refines the information and passes it back to the first decoder after interleaving.
    \item This iterative process continues, improving the estimate of the transmitted data with each iteration.
\end{enumerate}

\subsection*{Performance}
The performance of Turbo codes approaches the Shannon limit, making them highly effective for communication systems where bandwidth efficiency is critical.




\section*{Turbo Codes and Latency}

Turbo codes are a type of error-correcting code that use iterative decoding and interleaving to improve performance. However, this can introduce latency.

\subsection*{Latency in Turbo Codes}
Convolutional codes, and especially Turbo codes, face a trade-off between decoding performance and latency. The iterative process of Turbo decoding and the use of interleavers to randomize bit errors across the transmitted data block both contribute to this latency.

\subsection*{Managing Latency}
\begin{itemize}
    \item At any given Signal-to-Noise Ratio (SNR), system designers must balance the latency caused by the interleaver and the Quality of Service (QoS) required by the application.
    \item For real-time voice communication, which can tolerate a medium to high Bit Error Rate (BER), smaller block sizes (\( K \approx 300 \) bits) are preferred.
    \item For video playback, where a lower BER is essential, mid-range block sizes (\( K \approx 4000 \) bits) are typically used.
    \item For file transfers that can tolerate very low BERs, larger block sizes (\( K \approx 16000 \) bits) are chosen to optimize throughput despite the increased latency.
\end{itemize}






\section*{Spatial Diversity}

Spatial diversity is a form of diversity technique utilized in wireless communications to enhance signal robustness to fading. It involves the use of multiple antennas at the transmitter and/or receiver to create multiple independent channels for the same signal.

\subsection*{Receive Diversity}
\begin{itemize}
    \item Spatial diversity is typically achieved without sacrificing bandwidth, unlike frequency and time diversity which require more resources.
    \item Array gain refers to the power gain achieved by coherently combining the signals from multiple antennas compared to a single antenna case. It increases with the correlation of the spatial channel.
    \item Diversity gain is the improvement in signal-to-noise ratio (SNR) due to the independent fading paths in spatially diverse channels. It is maximized when the spatial channel is uncorrelated.
\end{itemize}

\subsection*{Beamforming}
The diagram also illustrates the concept of beamforming, where the antenna array is used to direct the energy of the transmitted signal in specific directions, enhancing the gain in the desired direction and reducing it in others, thus improving the overall link quality.


As wireless communication systems evolve, there is a trend towards higher carrier frequencies, which allows for the utilization of shorter wavelengths.

\subsubsection*{Uncorrelated Channels and Antenna Spacing}
For the channel to be considered uncorrelated between antenna elements, the spacing \( d_c \) is generally taken as half the wavelength \( \lambda \), i.e., \( d_c = \frac{\lambda}{2} \).

\subsubsection*{Advancements in Telecommunications}
\begin{itemize}
    \item The introduction of higher carrier frequencies leads to shorter wavelengths, enabling denser spatial diversity configurations.
    \item Technologies such as Wi-Fi have progressed from operating at 2.4 GHz (\( \lambda \approx 12.5 \) cm) to incorporating bands around 5 GHz (\( \lambda \approx 6 \) cm).
    \item The 5G networks are working at even higher frequencies like 3.8 GHz (\( \lambda \approx 8 \) cm) and planning for bands up to 52 GHz (\( \lambda \approx 6 \) mm), which allows for a significant increase in the number of antennas and thus potential array gain.
\end{itemize}

\subsection*{Wi-Fi 6 and Beyond}
Wi-Fi 6 (802.11ax) represents a significant step forward, offering enhancements such as:
\begin{itemize}
    \item Improved performance in dense environments.
    \item Higher throughput, claiming up to 40\% increase over previous standards.
    \item Greater network efficiency and extended battery life for connected devices.
\end{itemize}




\section*{SIMO Channel: Receive Diversity}

The receive diversity in a SIMO (Single Input, Multiple Output) channel utilizes multiple antennas at the receiver to improve the signal quality. The decision variable at the \(i\)-th receive antenna is given by:
\begin{equation}
    x_i(m) = h_i c_m + n_i(m),
\end{equation}
where \( h_i \) is the channel gain for the \(i\)-th antenna, \( c_m \) is the transmitted signal, and \( n_i(m) \) is the noise at the \(i\)-th antenna. The received signals \( x_i(m) \) are then combined to form a single decision statistic \( z(m) \), which is a weighted sum of the received signals. 

\subsection*{Combining the Signals}
The optimal combination of the signals can be expressed as:
\begin{equation}
    z(m) = w_1 x_1(m) + w_2 x_2(m) + \ldots + w_N x_N(m),
\end{equation}
where \( w_i \) are the weights chosen to maximize the SNR of the combined signal.

\subsection*{Handwritten Notes}
The handwritten notes from the image state that each channel is independently known, implying that the channel gains \( h_i \) are estimated separately for each receiving antenna. This allows the receiver to adaptively adjust the weights \( w_i \) for each antenna to maximize the overall SNR, which leads to the following optimization problem:
\begin{equation}
    \max_{\mathbf{w}} \frac{|\mathbf{w}^H \mathbf{h}|^2}{\mathbf{w}^H \mathbf{w}},
\end{equation}
where \( \mathbf{h} = [h_1, h_2, \ldots, h_N]^T \) and \( \mathbf{w} = [w_1, w_2, \ldots, w_N]^T \).

The notes mention that \( h_i \) is very large when \( w_i \) is very large, and vice versa, suggesting a proportional relationship between the weight and the channel gain for each antenna to achieve maximum SNR.

% Your figure environment and other LaTeX content goes here.



\section*{Maximal Ratio Combining (MRC) and Transmit Diversity in MISO Channels}

\subsection*{Maximal Ratio Combining (MRC)}
Maximal ratio combining utilizes multiple antenna elements to maximize the received signal strength. By applying the Schwarz inequality, we obtain an upper bound for the weighted sum of the channel gains:

\begin{equation}
\left( \sum_{i=1}^{N} w_i h_i \right)^2 \leq \sum_{i=1}^{N} w_i^2 \sum_{i=1}^{N} h_i^2.
\end{equation}

The optimal weighting coefficients, which maximize the SNR, are proportional to the respective channel gains:

\begin{equation}
w_i = h_i.
\end{equation}

This results in the signal-to-noise ratio (SNR) being:

\begin{equation}
\text{SNR} = \frac{A}{\sigma^2} \sum_{i=1}^{N} h_i^2.
\end{equation}

\subsection*{MISO Channel: Transmit Diversity}
In a MISO (Multiple Input Single Output) channel with \( M > 1 \) antennas at the transmitter and \( N = 1 \) at the receiver, spatial pre-coding is employed. The transmitted signal from the \( j \)-th antenna is given by:

\begin{equation}
y_j(m) = b_j^* c_m,
\end{equation}

where \( b_j \) is the precoding weight for the \( j \)-th transmit antenna, and \( c_m \) is the signal to be transmitted. The received signal is the sum of the signals from all transmit antennas:

\begin{equation}
x(m) = \sum_{j=1}^{M} h_j y_j(m).
\end{equation}





\section*{Maximal Ratio Transmit (MRT) Combining}

In Maximal Ratio Transmit (MRT) Combining, the precoding weight for the \(j\)-th transmit antenna is calculated as \( b_j = \frac{h_j}{\|h\|} \), where \( h \) is the channel vector and \( \|h\| \) is its norm. The transmitted signal at the receiver is a sum of the signals from each antenna weighted by this factor:

\begin{equation}
x(m) = \sum_{j=1}^{N} \frac{h_j}{\sqrt{N}} b_j c_m + n(m) = \sum_{j=1}^{N} \frac{|h_j|^2}{\|h\|} c_m + n(m),
\end{equation}

where \( c_m \) is the signal to be transmitted and \( n(m) \) is the noise. The Signal to Noise Ratio (SNR) is given by:

\begin{equation}
\text{SNR} = \frac{A}{\sigma^2} \sum_{i=1}^{N} |h_i|^2.
\end{equation}

\section*{Channel Gain \( \|h\|^2 \) Distribution for \( D \) Antennas}

For MRT or MRC, the SNR is proportional to the channel power gain. The probability density function (pdf) of \( \|h\|^2 \) depends on the number of antennas \( D \), and is given by:

\begin{equation}
f(\|h\|^2 = g) = \frac{1}{(D-1)!} g^{D-1} e^{-g}.
\end{equation}

\subsection*{Handwritten Notes}
The notes provide the calculation for the optimal weights in MRT and confirm the proportionality to the channel gains. The expectation calculations show that the energy of the signal is conserved after precoding, which is a crucial aspect of MRT. 

\begin{align*}
b_j &= \frac{h_j}{\|h\|}, \\
y_j &= b_j c, \\
\mathbb{E} \left[ \sum |b_j c_m|^2 \right] &= \sum |b_j|^2 A = \frac{\sum |h_j|^2}{\|h\|^2} A = A.
\end{align*}

% Your figure environment and other LaTeX content goes here.
\section*{Maximal Ratio Combining (MRC)}

Maximal Ratio Combining is a diversity technique used in wireless communications to combine multiple received signals into a single improved signal. The main benefits and drawbacks of MRC are:

\subsection*{Pros}
\begin{itemize}
\item Diversity gain: Improves signal quality by combining multiple signals.
\item Array gain: Increased signal strength due to multiple antennas.
\item MRT: No additional processing at the receiver is needed.
\end{itemize}

\subsection*{Cons}
\begin{itemize}
\item MRT: Requires channel knowledge at the transmitter.
\item MRC: Requires some extra processing at the receiver.
\end{itemize}

\section*{MIMO: Spatial Multiplexing}

MIMO technology employs spatial multiplexing to increase the capacity of a wireless channel. Spatial multiplexing is achieved through:

\begin{itemize}
\item The channel is represented as a \((N, M)\)-dimensional matrix.
\item The use of Singular Value Decomposition (SVD) of the channel matrix \( H \) to optimize transmission.
\item Coordinating the precoding weights at the transmitter with the combining weights at the receiver.
\end{itemize}

Assuming an equal number of transmit and receive antennas (\(M = N\)), spatial multiplexing can create \(N\) independent spatial channels, significantly increasing throughput.

\subsection*{Handwritten Notes}
The notes mention the advantages of diversity and array gains, emphasizing the lack of need for additional processing at the receiver in MRT and the requirement for some extra processing at the receiver in MRC. They also note the necessity for channel knowledge at the transmitter in MRT.

% Your figure environment and other LaTeX content goes here.
\section*{Singular Value Decomposition}

For any matrix \( A \in \mathbb{C}^{m \times n} \), it can be decomposed as \( A = U\Sigma V^H \) where:
\begin{itemize}
    \item \( U \in \mathbb{C}^{m \times p} \) and \( V \in \mathbb{C}^{n \times p} \) are unitary matrices.
    \item \( \Sigma \in \mathbb{C}^{p \times p} \) is a diagonal matrix with non-negative real numbers on the diagonal.
    \item \( p = \min(m,n) \) represents the number of singular values of \( A \).
\end{itemize}

In MIMO systems, this decomposition helps to transform the channel into parallel SISO channels, facilitating the design of spatial multiplexing schemes.

\subsection*{Optimal MIMO Scheme: Spatial Multiplexing}
By pre-multiplying the signal vector by \( V \) and post-multiplying the received signal by \( U^H \), we can decouple the MIMO channel into \( p \) independent channels, where \( p \) is the rank of the channel matrix \( H \), and \( H = U\Sigma V^H \).

\subsection*{Handwritten Notes Transcription}
The handwritten notes confirm the properties of the matrices \( U \) and \( V \) being unitary, and \( \Sigma \) being diagonal. They also explain that the received vector \( y \) in a MIMO system can be expressed as \( y = Hx + n \), and after processing with \( U^H \), it simplifies to \( z = \Sigma s + n' \) which represents parallel SISO channels. If \( H \) is square and full rank, the optimal weights are given by the eigenvectors, and the signal can be recovered with maximal efficiency.
\section*{Adaptive Modulation and Coding}

The Shannon capacity formula indicates the maximum rate achievable over a transmission channel, which is given by:
\begin{equation}
    C = B\log_2(1 + \text{SNR})
\end{equation}
where the spectral efficiency is measured in b/s/Hz, \( B \) is the bandwidth, and SNR (Signal-to-Noise Ratio) is defined for channel gain \( H \) as:
\begin{equation}
    \text{SNR} = \frac{|H|^2P}{\sigma^2}
\end{equation}

In a practical system, spectral efficiency depends on the modulation order and the coding rate. Given the symbol timing \( T \) and symbol rate \( R_s = \frac{1}{T} \approx B \), for a modulation order \( M \) and a coding rate \( R \), the bit rate \( R_b \) is expressed as:
\begin{equation}
    R_b = \log_2 M \frac{R}{T} \approx mRB
\end{equation}

As \( R_b < C \), the modulation order and the coding rate are bounded by:
\begin{equation}
    mR < \log_2(1 + \text{SNR})
\end{equation}

\subsection*{Handwritten Notes}
The notes detail the process of transmission, highlighting the mapping of information bits through coding to symbols. They express the relationship between the coded bits \( m \times \text{symbol} \) and information bits \( m \times R \), resulting in a formula for calculating \( K \), the number of information bits:
\begin{equation}
    K = R_m \cdot m = \frac{k}{n} m
\end{equation}






\section*{Adaptive Modulation and Coding}

Adaptive modulation and coding (AMC) is a method used in wireless communication systems where the modulation format and coding rate are dynamically adapted to the prevailing channel conditions. This technique aims to maximize spectral efficiency and link reliability. In practice, AMC is implemented by directly mapping the Channel Quality Indicator (CQI) to a specific Modulation and Coding Scheme (MCS).

\begin{table}[htbp]
\centering
\caption{Adaptive Modulation and Coding Schemes}
\begin{tabular}{|c|c|c|c|c|}
\hline
Radio Bearer Index & Name & Modulation & Channel Coding Rate & Bearer Efficiency (bits/symbol) \\
\hline
1 & QPSK \(1/12\) & QPSK & 0.0761719 & 0.1523 \\
2 & QPSK \(1/9\) & QPSK & \ldots & \ldots \\
\ldots & \ldots & \ldots & \ldots & \ldots \\
15 & 64QAM \(11/12\) & 64QAM & 0.925781 & 5.5547 \\
\hline
\end{tabular}
\end{table}

Multicarrier transmissions over selective channels experience a set of parallel channels with diverse channel gains. This frequency diversity can be exploited by adapting the modulation format and the coding rate to the quality of the channel.

The bearer efficiency for each MCS is calculated by:
\begin{equation}
\text{Efficiency} = \log_2(M) \cdot \text{Coding Rate}
\end{equation}

Where \( M \) is the modulation order, and the coding rate is a fraction representing the number of useful bits out of the total bits transmitted. The higher the modulation order and the coding rate, the greater the spectral efficiency, given by:
\begin{equation}
\text{Spectral Efficiency} = B \cdot \text{Efficiency}
\end{equation}

However, the modulation order and the coding rate are constrained by the SNR to ensure reliable transmission.





\section*{Optimal Power Distribution}

The optimal power distribution in a multi-channel system can be described by the water-filling algorithm. The objective is to maximize the total rate over $N$ channels under a total power constraint $P_0$. Each channel has a corresponding noise level $\sigma_n^2$, and the water-filling algorithm is utilized to allocate power across these channels efficiently.

\subsection*{Formulation}
The problem can be formulated as:
\begin{equation}
\begin{aligned}
& \underset{P}{\text{maximize}}
& & \sum_{n=1}^{N} \log_2 \left(1 + \frac{P_n}{\sigma_n^2}\right) \\
& \text{subject to}
& & P_n \geq 0, \; n = 1, \ldots, N, \\
&&& \sum_{n=1}^{N} P_n = P_0,
\end{aligned}
\end{equation}
where $\sigma_n^2 = \frac{\sigma^2}{|H_n|^2}$ and $H_n$ is the channel gain.

\subsection*{Solution using Lagrange Multipliers}
The solution to the power allocation problem utilizes Lagrange multipliers:
\begin{equation}
P_n^* = \left(\mu - \frac{\sigma_n^2}{|H_n|^2}\right)^+
\end{equation}
where $\mu$ is the water level determined by the power constraint and $(x)^+ = \max(x, 0)$.

The Lagrangian is given by:
\begin{equation}
\mathcal{L}(P, \lambda) = \sum_{n=1}^{N} \log_2 \left(1 + \frac{P_n}{\sigma_n^2}\right) + \lambda \left(P_0 - \sum_{n=1}^{N} P_n\right)
\end{equation}

And the Karush-Kuhn-Tucker (KKT) conditions provide the optimal power levels for each channel.

\subsection*{Handwritten Notes Clarification}
From the handwritten notes, we have additional clarifications on the water-filling algorithm and its implications:
\begin{itemize}
    \item The channel SNR is a factor in determining the distribution of power.
    \item When a channel's power is zero, it is excluded from the rate calculation, reflecting the water-filling analogy.
    \item The KKT conditions lead to the equation for the optimal power allocation, considering the noise level and the channel gain.
\end{itemize}





\section*{Waterfilling: Computing the Water Level \(\mu\)}

The water level \(\mu\) in the water-filling algorithm is critical as it determines the power allocation across the channels. It is calculated by solving the equation that ensures the total power used equals the power constraint \(P_0\).

\subsection*{The Nonlinear Equation}
The value of \(\mu\) satisfies the equation:
\begin{equation}
\sum_{n=1}^{N} (\mu - \sigma_n^2)^+ = P_0,
\end{equation}
where \(\sigma_n^2\) is the noise power in the \(n\)-th channel and \((x)^+\) is the positive part of \(x\).

\subsection*{Bisection Method for Solving for \(\mu\)}
To find \(\mu\), we apply the bisection method which is an iterative procedure to find roots of continuous functions.

\subsubsection*{The Function}
We define the function for the bisection method:
\begin{equation}
f(\mu) = \sum_{n=1}^{N} (\mu - \sigma_n^2)^+ - P_0,
\end{equation}
which is a nonlinear function of \(\mu\).

\subsubsection*{The Iterative Process}
The bisection method starts with two initial points \(a_1\) and \(b_1\) such that \(f(a_1) \leq 0\) and \(f(b_1) \geq 0\). It then proceeds iteratively to narrow down the interval containing the root by evaluating the function at the midpoint and selecting the subinterval where a sign change occurs.

\subsubsection*{The Algorithm}
\begin{align*}
&\text{while } |b - a| > \text{tolerance} \\
&\quad c = \frac{a + b}{2} \\
&\quad \text{if } f(c) = 0 \text{ or } (b - a)/2 < \text{tol} \text{ then} \\
&\quad \quad \text{root} = c \\
&\quad \text{end if} \\
&\quad \text{if } \text{sign}(f(c)) = \text{sign}(f(a)) \text{ then} \\
&\quad \quad a = c \\
&\quad \text{else} \\
&\quad \quad b = c \\
&\quad \text{end if} \\
&\text{end while}
\end{align*}

The iteration continues until the interval is sufficiently small, and the midpoint \(c\) is an approximation to the root.


\section*{The Bisection Method for Waterfilling Solution}

The bisection method is an efficient algorithm to find the root of a function, which in the context of the waterfilling algorithm is used to compute the water level \(\mu\). 

The main principle is to iteratively reduce the search interval by half and select the subinterval where the sign of \(f\) changes.

\subsection*{Bisection Algorithm Steps}
Let's denote \(a_k\) and \(b_k\) as the lower and upper bounds of the interval at the \(k\)-th iteration, and \(f\) as our function of interest. The algorithm proceeds as follows:

\begin{itemize}
\item Compute the midpoint \(c_k = \frac{a_k + b_k}{2}\).
\item Evaluate the function at the midpoint: \(f(c_k)\).
\item If \(f(c_k)\) is very small or zero, \(c_k\) is the root and the algorithm stops.
\item If \(f(c_k)\) has the same sign as \(f(a_k)\), then set \(a_{k+1} = c_k\) and \(b_{k+1} = b_k\).
\item Otherwise, set \(a_{k+1} = a_k\) and \(b_{k+1} = c_k\).
\end{itemize}

The algorithm stops when the interval \( |b_k - a_k| \) is sufficiently small.

The figure represents the waterfilling method applied to an OFDM channel, where each channel gain can be visualized as a 'depth', and the water level \(\mu\) is set to equalize the total power distributed across the channels.

\section*{Waterfilling Example: Optimal Power Allocation}

The waterfilling algorithm is a strategy used in communication systems to distribute power across various channels to maximize the total communication rate.

\subsection*{Algorithm Overview}
The waterfilling algorithm allocates more power to channels with better channel-to-noise ratios, akin to 'filling' the channels with water up to a certain 'water level' \(\mu\).

\subsection*{Mathematical Formulation}
The power allocated to the \(n\)-th channel \(P_n\) is determined by:

\[
P_n = \left( \mu - \sigma_{n}^{2} \right)^{+}
\]

where \(\sigma_{n}^{2}\) represents the normalized noise in the \(n\)-th channel, and \((x)^{+}\) denotes the positive part of \(x\). The water level \(\mu\) is found such that the total power constraint is satisfied:

\[
\sum_{n=1}^{N} \left( \mu - \sigma_{n}^{2} \right)^{+} = P_0
\]

This ensures that the power allocated does not exceed the total available power \(P_0\).

In the figure, the blue bars represent the power allocated to each channel, while the red line indicates the channel normalized noise. The power is allocated more to the channels where the blue bar exceeds the red line, i.e., the channel-to-noise ratio is higher.

\subsection*{Achieved Rate vs Power}
The graph below illustrates the achieved communication rate as a function of power. The waterfilling strategy outperforms uniform power allocation strategies, especially on channels with better conditions, leading to an increased average rate per unit of bandwidth.

% Include any additional graphs or figures here.
\section*{LTE (Long Term Evolution)}
Lo standard LTE è il primo ad adottare una modulazione OFDM, con i seguenti parametri:
\[
    \begin{array}{ll}
        f_s = \frac{1}{T_s} = 30.72 \text{ MHz} & \text{(banda totale a disposizione, idealmente)} \\
        N = 2048 & \text{(numero canali)} \\
        \Delta f = \frac{f_s}{N} = 15 \text{ kHz} & \text{(banda per canali)} \\
    \end{array}    
\]
L'allocazione delle risorse di banda agli utenti è indicata come OFDMA, ovvero FDMA basata su OFDM, per cui la banda assegnata è un multiplo intero della banda di un canale. L'allocazione minima per ciascun utente è un blocco, composto da 12 subcarrier, quindi 180 kHz.
Inoltre l'allocazione delle risorse richiede anche l'assegnamente di slot temporali, ovvero il tempo entro cui i blocchi di banda sono utilizzabili. L'unità minima è un slot composta da 7 blocchi OFDM (simboli) della durata di 0.5 ms.
\begin{itemize}
    \item slot temporale: 0.5 ms (7 simboli OFDM) 
    \item blocco di banda: 180 kHz (12 subcarrier)
\end{itemize}
Delle 2048 subcarrier solo 1200 sono utilizzate, le altre rappresentano virtual subcarrier.
Sebbene possa sembrare un numero molto alto di frequenze non utilizzate, la scelta deriva dall'utilizzo di un $f_s$ molto alto, tuttavia lo spettro realmente occupato è circa 20 MHz.
Inoltre un'altra porzione di spettro è utilizzata per informazioni di controllo e sincronizzazione, riducendo la banda disponibile di 4 MHz.
\[
    B_{av} = B_{\text{raw}} - 4 \text{ MHz} = 1200 \cdot 15 \text{ kHz} - 4 \text{ MHz} = 14 \text{ MHz}
\]

Per quanto riguarda il coding rate e modulation order la scelta avvviene in base alla qualità del link, valutata in termini di CQI (Channel Quality Indicator). Le modulazioni possibili sono quelle comprese tra 4-QAM e 256-QAM, mentre il range di coding rate varia da 0.0762 a 0.9258.
Il massimo rate su un canale SISO può quindi essere calcolato:
\[
    R_{1 \times 1} - 0.9258 \cdot 8 \cdot 14 \text{ MHz} \approx 100 \text{ Mbps}
\]
Lo standard prevedere il supporto per canali MIMO in diverse configurazioni, fino a 4$\times$4, per cui si ottiene un rate potenzialmente quadruplicato:
\[
    R_{4 \times 4} = 4 \cdot R_{1 \times 1} = 400 \text{ Mbps}
\]
    
I terminali più avanzati possono affrefare più banda da 20 MHz ciascuna, incrementando notevolemnte il rate raggiungibile, fino a 2 Gbps. Da notare che nella realtà non è possibile raggiungere questi rate in quanto sarebbe richiesto un canale in ottime condizioni, tipicamente solo in prossimità delle antenne, ed inoltre tutte le risorse dovrebbero essere allocato ad un singolo utente. Infine bisogna considerare che spesso gli operatori impongnono dei limiti della performance ottenibile.













\section*{LTE Overview}

\textbf{Long Term Evolution (LTE)} represents the fourth generation (\textbf{4G}) of wireless communication standards, preceded by:
\begin{itemize}
    \item \textbf{1G}: Analog FDMA-based systems
    \item \textbf{2G}: GSM, the first digital standard supporting data rates up to 9.6 kb/s
    \item \textbf{3G}: UMTS (CDMA)
    \item \textbf{5G}: NR, currently being deployed
\end{itemize}
LTE is primarily utilized for data transmission due to its flexibility and high capacity.

\section*{Physical Layer LTE Numerology}

\begin{itemize}
    \item LTE employs \textbf{Orthogonal Frequency-Division Multiplexing (OFDM)} as its core technology, similar to Wi-Fi systems.
    \item \textbf{Sampling time} (\(f_s\)) is 30.72 MHz, which is a critical factor in determining the FFT size and subcarrier bandwidth.
    \item \textbf{FFT size} (\(N\)) is 2048, playing a role in the frequency resolution of the system.
    \item \textbf{Subcarrier bandwidth} (\(\Delta f\)) is defined as \(\frac{30.72 \text{ MHz}}{2048} = 15 \text{ kHz}\), dictating the spacing between individual OFDM subcarriers.
    \item An \textbf{LTE slot} consists of 7 OFDM symbols, with a total duration of 0.5 ms, thus determining the time-domain structure of LTE transmission.
\end{itemize}

The figure below illustrates a \textbf{resource block} in LTE, showing the subcarrier spacing and the time-frequency structure of the LTE slot and resource blocks.

% Insert image of LTE numerology

\section*{Key Mathematical Relationships}

Given the LTE numerology, we can define some key relationships:
\begin{align*}
    \text{Subcarrier spacing} (\Delta f) &= \frac{f_s}{N} \\
    \text{Slot duration} &= 0.5 \text{ ms (for 7 OFDM symbols)} \\
    \text{Resource block width} &= 180 \text{ kHz} = 12 \text{ subcarriers} \times 15 \text{ kHz}
\end{align*}

These relationships are crucial for understanding LTE's capacity to handle various data rates and modulations, balancing the need for speed with the practical limitations of the wireless medium.



\section*{Physical Layer LTE Numerology and Peak Data Rate}

LTE employs \textbf{Orthogonal Frequency-Division Multiple Access (OFDMA)} for its multiple access technique. This involves allocating groups of subcarriers to different users, which enables efficient spectrum utilization.

The \textbf{minimum allocation unit} in LTE is a \textbf{resource block (RB)}. Each RB:
\begin{itemize}
    \item Contains \(12\) subcarriers
    \item Spans \(180\) kHz in frequency for the duration of a slot (\(0.5\) ms)
\end{itemize}

Regarding the peak data rate for high-end mobile phones:
\begin{itemize}
    \item These often fall in the category list that supports very high data rates (e.g., 19-20 category).
    \item The \textbf{raw bandwidth} calculation is as follows:
          \begin{align*}
              B_{\text{raw}} &= \text{Number of RBs} \times \text{Subcarrier Spacing} \\
                             &= 100 \times 180\text{ kHz} \\
                             &= 18\text{ MHz}
          \end{align*}
    \item Considering control and synchronization overheads, the \textbf{available bandwidth} \(B_{av}\) is:
          \begin{align*}
              B_{av} &= B_{\text{raw}} - 4\text{ MHz for control and synchronization} \\
                     &= 14\text{ MHz}
          \end{align*}
\end{itemize}

\subsection*{Additional Notes}

\begin{itemize}
    \item In practice, \(1200\) of the \(2048\) available subcarriers are used, which accounts for the data rate calculations.
    \item A total of \(848\) are left as guard subcarriers, which are not used for transmission to avoid interference.
    \item The effective channel utilization and data rate can be significantly influenced by the actual number of subcarriers allocated for user data.
\end{itemize}

% An illustration or table showing LTE numerology and data rate could be included here
% \includegraphics{path_to_image}







\section*{5G New Radio (NR) and Latest Chipsets}

\subsection*{5G New Radio (NR)}
5G NR is the latest global wireless standard after 1G, 2G, 3G, and 4G networks, enabling a new kind of network designed to connect virtually everyone and everything together including machines, objects, and devices.

\begin{itemize}
    \item It is characterized by \textbf{Enhanced Mobile Broadband (eMBB)} that aims to deliver peak data rates in the range of gigabytes per second.
    \item It supports a variety of services from high-quality video streaming and virtual reality to reliable low latency communications required for autonomous vehicles.
    \item \textbf{Massive Machine-Type Communications (mMTC)} and \textbf{Ultra-Reliable Low Latency Communications (URLLC)} are central to 5G capabilities, supporting the wide-scale IoT and mission-critical applications, respectively.
\end{itemize}

\subsection*{Most Recent 5G Chip}
The \textbf{Qualcomm\textsuperscript{\textregistered} Snapdragon X65} is the latest 5G chipset that includes:
\begin{itemize}
    \item Support for 5G mmWave and sub-6 GHz frequencies with a peak download speed of 10 Gbps.
    \item 3GPP Release 16 support and a 5G mmWave-sub6 aggregation for expanded range and bandwidth.
    \item Advanced features such as Qualcomm\textsuperscript{\textregistered} Smart Transmit technology, Qualcomm\textsuperscript{\textregistered} 545 mmWave Antenna Module for enhanced signal quality, and AI-based signal boost functionalities.
\end{itemize}

\subsection*{Chipset Specifications}
\begin{itemize}
    \item The modem supports dynamic spectrum sharing (DSS), SA (standalone), NSA (non-standalone) modes, and global multi-SIM.
    \item Designed for compatibility across various technologies such as GSM, CDMA, WCDMA, and others to ensure broad spectrum usage and coverage.
\end{itemize}


\end{document}



